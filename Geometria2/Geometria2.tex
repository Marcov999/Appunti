\documentclass{article}
\usepackage{mstyle}

\title{Appunti di Geometria 2 \\ anno accademico 2019/2020}
\date{}
\author{Marco Vergamini \and Alessio Marchetti}

\begin{document}
\maketitle
\newpage
\tableofcontents
\newpage

\section{Introduzione}
Questi appunti sono basati sul corso di Geometria 2 tenuto dai professori
Roberto Frigerio e Jacopo Gandini nell'anno accademico 2019/2020. Sono dati per
buoni i contenuti dei corsi del primo anno, in particolare Analisi 1 e Geometria
1. Verranno omesse o soltanto hintate le dimostrazioni più semplici, ma si
consiglia comunque di provare a svolgerle per conto proprio. Ogni tanto sarà
commesso qualche abuso di notazione, facendo comunque in modo che il significato
sia reso chiaro dal contesto. Inoltre, la notazione verrà alleggerita man mano,
per evitare inutili ripetizioni e appesantimenti nella lettura.


\section{Spazi metrici e spazi topologici}
\subsection{Spazi metrici}
\begin{defn}
	Uno \textsc{spazio metrico} è una coppia $(X, d)$, $X$ insieme,
	${d: X \times X \rightarrow \mathbb{R}}$ t.c. per ogni ${x, y, z \in X}$
    \begin{nlist}
    	\item $d(x, y) \ge 0$ e $d(x, y)=0 \Leftrightarrow x=y$;
    	\item $d(x, y)=d(y, x)$;
    	\item $d(x, z) \le d(x, y)+d(y,z)$ (disuguaglianza triangolare).
    \end{nlist}
	In tal caso $d$ si dice \textsc{distanza} o \textsc{metrica}.
\end{defn}

\begin{ex}
    Ecco alcuni esempi di distanze:
    \begin{itemize}
        \item La distanza $d_1$ su $\mathbb{R}^n$, che alla coppia di elementi
        ${x=(x_1, \dots, x_n)}, {y=(y_1, \dots, y_n)}$ associa il numero
        $${\displaystyle d_1(x, y)=\sum_{i=1}^n |x_i-y_i|};$$
        \item la distanza $d_2$ (o $d_E$, distanza euclidea) su $\mathbb{R}^n$,
        $$\displaystyle d_2(x, y)=\sqrt{\sum_{i=1}^n (x_i-y_i)^2};$$
        \item la distanza $d_{\infty}$ su $\mathbb{R}^n$, ${\displaystyle
        d_{\infty}(x, y)=\sup_{i=1, \dots, n} \{ |x_i-y_i| \}}$;
        \item
        la distanza discreta su un generico insieme $x$, $d(x, y)=1$ se $x \not=
        y$ e $0$ altrimenti;
        \item le distanze $d_1$, $d_2$, $d_{\infty}$ sullo spazio delle funzioni
        continue da $[0, 1]$ in $\mathbb{R}$, definite rispettivamente
        $${\displaystyle d_1(f, g)=\int_0^1 |f(t)-g(t)| dt},$$
        $${\displaystyle d_2(f, g)=\sqrt{\int_0^1 (f(t)-g(t))^2 dt}},$$
        $${\displaystyle d_{\infty}(f, g)=\sup_{t \in [0, 1]} |f(t)-g(t)|}.$$
    \end{itemize}
\end{ex}

\begin{defn}
	$f:(X, d) \rightarrow (Y, d')$ viene detto \textsc{embedding isometrico} se
	$d'(f(x_1), f(x_2))=d(x_1, x_2)$ per ogni $x_1, x_2 \in X$.
\end{defn}

\begin{oss}
    Valgono i seguenti fatti:
    \begin{itemize}
        \item l'identità è un embedding isometrico;
        \item composizione di embedding isometrici è un embedding isometrico;
        \item se un embedding isometrico $f$ è biettivo, anche $f^{-1}$ è un
        embedding isometrico e $f$ si dice \textsc{isometria};
        \item un embedding isometrico è sempre iniettivo, dunque è un'isometria
        se e solo se è suriettivo;
        \item se $(X, d)$ è fissato, l'insieme delle isometrie da $X$ in sé è un
        gruppo con la composizione, chiamato $Isom(X,d)$.
    \end{itemize}
\end{oss}



\subsection{Continuità in spazi metrici}

\begin{defn}
    Dati $p \in X, R>0$, ${B(p, R):=\{ x \in X \mid d(p, x)<R\}}$ è detta la
    \textsc{palla aperta} di centro $p$ e raggio $R$.
\end{defn}

\begin{defn}
    ${f:(X, d) \rightarrow (Y, d')}$ è \textit{continua in $x_0$} se ${\forall
    \epsilon > 0 \; \exists \; \delta >0}\\\ \tc {f(B(x_0, \delta)) \subseteq
    B(f(x_0), \epsilon)}$, cioè ${f^{-1}(B(f(x_0), \epsilon)) \supseteq B(x_0,
    \delta)}$.
\end{defn}

\begin{defn}
    ${f(X, d) \rightarrow (Y, d')}$ è detta \textsc{continua} se è continua in
    ogni ${x_0 \in X}$.
\end{defn}

\begin{oss}
    Gli embedding isometrici sono continui.
\end{oss}

\begin{defn}
    Sia $X$ uno spazio metrico, $A \subseteq X$ è detto \textsc{aperto} se per
    ogni $x \in A$ esiste ${R>0 \tc B(x, R) \subseteq A}$.
\end{defn}

\begin{ftt}
    Le palle aperte sono aperti. Per dimostrarlo, sfruttare la disuguaglianza
    triangolare.
\end{ftt}

\begin{thm} \label{thm:cont_inv}
    $f(X, d) \rightarrow (Y, d')$ è continua se e solo se per ogni aperto $A$ di
    $Y$ $f^{-1}(A)$ è aperto di $X$.
\end{thm}

\begin{proof}
    Supponiamo $f$ continua. Prendiamo $x \in f^{-1}(A)$, allora si ha che $f(x)
    \in A$, ma dato che $A$ è aperto esiste una palla aperta di centro $f(x)$
    $B_Y$ t.c. $B_Y \subseteq A$, da cui $f^{-1}(B_Y) \subseteq f^{-1}(A)$.
    Usando la definizione di continuità, detto $\epsilon$ il raggio di $B_Y$,
    scegliendo il $\delta$ corrispondente come raggio di una palla di centro
    $x$, sia essa $B_X$, si ha che $B_X \subseteq f^{-1}(B_Y) \subseteq
    f^{-1}(A)$, perciò abbiamo trovato una palla centrata in $x$ tutta contenuta
    in $f^{-1}(A)$ e questo prova che è un aperto.

    Viceversa, supponiamo che le controimmagini di aperti siano a loro volta
    aperti. Dati $x_0 \in X$ e $\epsilon >0$ si ha che $B(f(x_0), \epsilon)$ è
    un aperto di $Y$, perciò la sua controimmagine è un aperto, ma allora per
    definizione di aperto esiste $\delta>0$ tale che $B(x_0, \delta) \subseteq
    f^{-1}(B(f(x_0), \epsilon))$ e questo prova che $f$ è continua.
\end{proof}

\begin{oss}
    La continuità di una funzione dipende quindi solo indirettamente dalla
    metrica, mentre è direttamente collegata agli aperti generati dalla metrica
    stessa. Segue facilmente che due metriche che generano gli stessi aperti
    portano anche alla stessa famiglia di funzioni continue. Diamo dunque la
    seguente definizione.
\end{oss}

\begin{defn}
    Due distanze $d, d'$ su un insieme $X$ si dicono \textsc{topologicamente
    equivalenti} se inducono la stessa famiglia di aperti.
\end{defn}

\begin{lm}
    Siano $d, d'$ distanze su $X$ t.c. esiste $k \ge 1$ t.c. per ogni $x, y \in
    X$ valga $d(x, y)/k \le d'(x, y) \le k \cdot d(x, y)$. Allora $d$ e $d'$
    sono topologicamente equivalenti.
\end{lm}

\begin{proof}
    Sia $A$ un aperto indotto da $d'$ e $x_0 \in A$. Per definizione di aperto
    esiste $R>0$ tale che $B_{d'}(x_0, R) \subseteq A$. Considero $B_d(x_0,
    R/k)$. Dato un elemento $x \in B_d(x_0, R/k)$ ho che $d'(x_0, x) \le k \cdot
    d(x_0, x)<k \cdot R/k=R$, dunque $x \in B_{d'}(x_0, R)$. Allora $B_d(x_0,
    R/k) \subseteq B_{d'}(x_0, R/k) \subseteq A$ e dunque $A$ è anche un aperto
    indotto da $d$. Per la simmetria delle disuguaglianze nelle ipotesi si
    dimostra anche l'opposto, perciò gli aperti di $d$ e $d'$ sono gli stessi,
    come voluto.
\end{proof}

\begin{cor}
    $d_1, d_2, d_{\infty}$ sono topologicamente equivalenti su $\mathbb{R}^n$.
\end{cor}

\begin{center}
\pagestyle{empty}
\begin{tikzpicture}[line cap=round,line join=round,
    >=triangle 45,x=2.0cm,y=2.0cm]
    \draw[->,color=black] (-1.72,0) -- (1.78,0);
    \foreach \x in {-1,1}
    \draw[shift={(\x,0)},color=black] (0pt,2pt) -- (0pt,-2pt);
    \draw[->,color=black] (0,-1.4) -- (0,1.44);
    \foreach \y in {-1,1}
    \draw[shift={(0,\y)},color=black] (2pt,0pt) -- (-2pt,0pt);
    \clip(-1.72,-1.2) rectangle (1.78,1.24);
    \draw(0,0) circle (2cm);
    \draw (-1,0)-- (0,1);
    \draw (0,1)-- (1,0);
    \draw (1,0)-- (0,-1);
    \draw (0,-1)-- (-1,0);
    \draw (-1,1)-- (1,1);
    \draw (1,1)-- (1,-1);
    \draw (1,-1)-- (-1,-1);
    \draw (-1,1)-- (-1,-1);
\end{tikzpicture}

Nell'immagine sono rappresentate, nell'ordine dalla più interna alla più
esterna, le palle aperte centrate nell'origine e di raggio $1$ rispettivamente
nelle metriche $d_1, d_2, d_{\infty}$.
\end{center}

\begin{proof}
    Per AM-QM si ha che $$\displaystyle \frac{\sum_{i=1}^n |x_i-y_i|}{n} \le
    \sqrt{\frac{\sum_{i=1}^n (x_i-y_i)^2}{n}},$$ da cui $d_1(x, y) \le \sqrt{n}
    \cdot d_2(x, y)$. Stimando tutti i termini con il massimo otteniamo che
    $$\displaystyle \sqrt{\sum_{i=1}^n (x_i-y_i)^2} \le \sqrt{\sum_{i=1}^n
    \sup_{j=1, \dots, n} \{(x_j-y_j)^2\}}=\sqrt{n} \cdot \sup_{j=1, \dots, n} \{
    |x_j-y_j| \},$$ da cui $d_2(x, y) \le \sqrt{n} \cdot d_{\infty} (x, y).$
    Infine è ovvio che $$\displaystyle \sup_{i=1, \dots, n} |x_i-y_i| \le
    \sum_{i=1}^n |x_i-y_i|$$, da cui $d_{\infty}(x, y) \le d_1(x, y).$
    Scegliendo $k=\sqrt{n}$ è ora sufficiente applicare il lemma.
\end{proof}

Nella dimostrazione del corollario era importante che lo spazio fosse di
dimensione finita. Nello spazio delle funzioni continue da $[0, 1]$ a
$\mathbb{R}$ le tre distanze non sono topologicamente equivalenti.

\subsection{Spazi topologici}

\begin{defn}
    Uno \textsc{spazio topologico} è una coppia $(X, \tau),\; {\tau \subseteq
    \mathcal{P}(X)}$, t.c.
    \begin{nlist}
        \item $\emptyset,\ X \in \tau$;
        \item $A_1,\ A_2 \in \tau \Rightarrow A_1 \cap A_2 \in \tau$;
        \item se $I$ è un insieme e $A_i \in \tau \, \forall i \in I$, allora
        $\displaystyle \bigcup_{i \in I} A_i \in \tau$.
    \end{nlist}
    Allora $\tau$ si dice \textsc{topologia} di $X$ e gli elementi della
    topologia sono detti \textsc{aperti} di $\tau$.
\end{defn}

\begin{prop}
    Se $(X, d)$ è uno spazio metrico, gli aperti rispetto a $d$ definiscono una
    topologia.
\end{prop}

\begin{defn}
    Uno spazio topologico $(X, \tau)$ è detto \textsc{metrizzabile} se $\tau$ è
    indotta da una distanza su $X$.
\end{defn}

\begin{defn}
    Sia $(X, \tau)$ uno spazio topologico, $C \subseteq X$ è \textsc{chiuso} se
    $X \setminus C$ è aperto.
\end{defn}

\begin{oss}
\begin{nlist}
\item possono esistere insiemi né aperti né chiusi;
\item $\emptyset$ e $X$ sono chiusi;
\item unione finita di chiusi è chiusa;
\item intersezione arbitraria di chiusi è chiusa.
\end{nlist}
\end{oss}

\begin{ex}
    Ecco alcuni esempi di spazi topologici:
    \begin{itemize}
        \item tutte le topologie indotte da una metrica;
        \item la topologia discreta, cioè $\tau=\mathcal{P}(X)$, indotta dalla
        distanza discreta;
        \item la topologia indiscreta, cioè $\tau=\{ \emptyset, X \}$;
        \item la topologia cofinita, dove gli aperti sono l'insieme vuoto più
        tutti e soli gli insiemi il cui complementare è un insieme finito.
    \end{itemize}
\end{ex}

\begin{defn}
	Dato uno spazio topologico $X$ e un insieme $B \subseteq X$, si chiama
	\textit{parte interna} di $B$, e la si indica con $B^{\circ}$, il più grande
	aperto contenuto in $B$. Analogamente, la \textit{chiusura} di $B$, indicata
	con $\overline{B}$, è il più piccolo chiuso che contiene $B$. Definiamo
	infine la \textit{frontiera} (o \textit{bordo}) di un insieme come l'insieme
	$\partial B= \overline{B} \setminus B^{\circ}$.
\end{defn}

Notiamo che parte interna e chiusura sono ben definite: per la prima basta
prendere l'unione di tutti gli aperti contenuti in $B$ (alla peggio c'è solo il
vuoto, che è contenuto in ogni insieme), per la seconda si prende l'intersezione
di tutti i chiusi che lo contengono (alla peggio c'è solo $X$). Per la stabilità
degli aperti per unione arbitraria e dei chiusi per intersezione arbitraria, gli
insiemi così ottenuti sono ancora un aperto e un chiuso e sono rispettivamente
il più grande aperto contenuto e il più piccolo chiuso che contiene per come
sono stati costruiti. Infine, è banale mostrare che la parte interna di un
aperto è l'aperto stesso e la chiusura di un chiuso è il chiuso stesso.

\begin{ftt}
	$X= B^{\circ} \sqcup\ \partial B\ \sqcup (X
	\setminus B)^{\circ}$ dove con $\sqcup$ si indica l'unione disgiunta.
\end{ftt}

\begin{proof}
	Per definizione $B^{\circ} \cup\ \partial B=\overline{B}$ e i due insiemi
	sono disgiunti. Inoltre, essendo $\overline B$ il più piccolo chiuso che
	contiene $B$, il suo complementare dev'essere il più grande aperto disgiunto
	da $B$, cioè il più grande aperto contenuto in $X \setminus B$, che è la
	parte interna di quest'ultimo. La tesi segue facilmente.
\end{proof}

\subsection{Continuità in spazi topologici}

\begin{defn}
    $f: (X, \tau) \rightarrow (Y, \tau')$ si dice \textsc{continua} se
    ${f^{-1}(A) \in \tau,}\; {\forall A \in \tau'}$, cioè una funzione \`e
    continua se la controimmagine di aperti \`e aperta.
\end{defn}

Notiamo che per il teorema \ref{thm:cont_inv}, le due
definizioni di funzione continua sono equivalenti per uno spazio metrico se si
considera la topologia indotta dalla metrica.

\begin{thm}
    \begin{nlist}
        \item L'identità è una funzione continua;
        \item composizione di funzioni continue è una funzione continua.
    \end{nlist}
\end{thm}

\begin{defn}
    Una funzione $f: X \rightarrow Y$ si dice \textsc{omeomorfismo} se $f$ è
    continua e esiste una funzione $g:Y\rightarrow X$ continua tale che $f \circ
    g = Id_Y$ e $g \circ f = Id_x$. Cio\`e $f$ \`e continua, bigettiva e con
    inversa continua.
\end{defn}

\begin{oss}
	\begin{nlist}
	\item Composizione di omeomorfismi è un omeomerfismo; due spazi legati da un
	omeomorfismo si dicomo \textsc{omeomorfi} e essere omeomorfi è una relazione
	di equivalenza;
	\item l'insieme degli omeomorfismi da $(X, \tau)$ in sé è un gruppo;
	\item se $f:X \rightarrow Y$ è continua e bigettiva non è detto che sia un
	omeomorfismo, cioè $f^{-1}$ può non essere continua.
    \marginpar{\warningsign}
\end{nlist}
\end{oss}

\begin{ex}
	Siano $\tau_E$ la topologia euclidea, $\tau_C$ la cofinita, $\tau_D$ la
	discreta e $\tau_I$ l'indiscreta. Le seguenti mappe sono dunque continue:
    \begin{itemize}
	\item $Id:(\mathbb{R}, \tau_D) \rightarrow (\mathbb{R}, \tau_E)$,
    \item $Id:(\mathbb{R}, \tau_E) \rightarrow (\mathbb{R}, \tau_C)$,
    \item $Id:(\mathbb{R}, \tau_C) \rightarrow (\mathbb{R}, \tau_I)$.
    \end{itemize}
    Nessuna delle inverse è però continua. Più in generale, $Id: (X, \tau)
    \rightarrow (X, \sigma)$ con $\sigma \subsetneq \tau$ è continua ma
    l'inversa no.
\end{ex}

\begin{exc}
	Il seguente esercizio è frutto di una domanda fatta da uno studente a
	lezione e potrebbe essere più difficile di altri esercizi del corso.

    \marginpar{\warningsign}
	Trovare un esempio (o dimostrare che non esiste) di uno spazio topologico
	$X$ e una funzione $f: (X, \tau) \rightarrow (X, \tau)$ continua e bigettiva
	con inversa non continua.

    Stando a quanto dice Frigerio, probabilmente tale funzione esiste,
    euristicamente perché non c'è un modo facile di dimostrare il contrario.
\end{exc}

Introduciamo adesso un concetto che permetterà di caratterizzare la continuità
in spazi topologici in modo analogo a quanto fatto per gli spazi metrici.

\begin{defn}
	Sia $(X, \tau)$ uno spazio topologico fissato e $x_0 \in X$. Un insieme $U
	\subseteq X$ è un \textsc{intorno} di $x_0$ se $x_0 \in
	U^{\circ}$, o equivalentemente se esiste $V$ aperto con $x_0 \in V
	\subseteq U$. L'insieme degli intorni di $x_0$ si denota con
	$\mathcal{I}(x_0)$.
\end{defn}

\begin{defn}
	$f:(X, \tau) \rightarrow (Y, \tau')$ è detta \textit{continua in $x_0$} se
	per ogni intorno $U$ di $f(x_0)$ esiste un intorno $V$ di $x_0$ t.c. $f(V)
	\subseteq U$.
\end{defn}

Appare dunque intuitivo il seguente risultato.

\begin{thm}
	$f$ è continua $\Leftrightarrow$ è continua in ogni $x_0 \in X$.
\end{thm}

\begin{proof}
	($\implies$) Supponiamo $f$ continua e $x_0 \in X$, sia inoltre $U$ un
	intorno di $f(x_0)$. Per definizione di intorno esiste un aperto $A$ con
	$f(x_0) \in A \subseteq U$, perciò $x_0 \in f^{-1}(A) \subseteq f^{-1}(U)$,
	ma poiché $f$ è continua abbiamo che $f^{-1}(A)$ è ancora un aperto, perciò
	ponendo $V=f^{-1}(U)$ abbiamo che $V$ è un intorno di $x_0$ t.c. $f(V)
	\subseteq U$, che è quello che volevamo.

	($\Leftarrow$) Supponiamo $f$ continua in ogni punto di $X$ e sia $A$ un
	insieme aperto in $Y$. Per ogni $x \in f^{-1}(A)$, $A$ è un intorno di
	$f(x)$. Ma dato che $f$ è continua in ogni punto di $X$, esiste un intorno
	$V_x$ di $x$ t.c. $x \in V_x \subseteq f^{-1}(A)$. Per definizione di
	intorno, ciò significa che esiste un aperto $A_x$ di $X$ t.c. $x \in A_x
	\subseteq V \subseteq f^{-1}(A)$. Dunque dev'essere $\displaystyle
	f^{-1}(A)= \bigcup_{x \in f^{-1}(A)} A_x$,
	quindi $f^{-1}(A)$ è un aperto in $X$ per ogni $A$ aperto in $Y$, il che
	equivale a dire che $f$ è continua.
\end{proof}

\subsection{Ordinamento fra topologie, basi e prebasi}

Vogliamo mettere un ordinamento parziale sulle topologie di un certo insieme
fissato.

\begin{defn}
    Dato un insieme $X$ su cui sono definite le topologie $\tau$ e $\tau'$, si
    dice che $\tau$ \`e \textsc{meno fine} di $\tau'$ se $\tau \subseteq \tau'$,
    cioè ogni aperto di $\tau$ è anche aperto di $\tau'$. $\tau'$ si dice
    \textsc{più fine} di $\tau$.
\end{defn}

\begin{oss}
    Equivalentemente alla definizione sopra, si pu\`o dire che $\tau$ \`e meno
    fine di $\tau'$ se e solo se ${Id:(X,\tau')\rightarrow(X, \tau)}$ \`e
    continua.
\end{oss}

Quando $\tau$ è meno fine di $\tau'$ scriveremo $\tau < \tau'$. Si noti che
dalla definizione ogni topologia è meno fine di se stessa, cioè $\tau < \tau'
\; \forall \tau$.

\begin{ex}
	$\tau_I < \tau_C < \tau_E < \tau_D$, \, $\tau_I < \tau < \tau_D \; \forall
	\tau$.
\end{ex}

\begin{lm}
	Intersezione arbitraria di topologie su $X$ è ancora una topologia su $X$.
\end{lm}

\begin{proof}
    Siano $\tau_i,\ i \in I$ topologie su $X$. Verifichiamo che $
    \tau=\bigcap_{i \in I} \tau_i$ soddisfa gli assiomi di topologia.

    \begin{nlist}
        \item $\emptyset, X \in \tau_i \, \forall i \in I \Rightarrow \emptyset,
        X \in \tau$.

        \item $A_1, A_2 \in \tau \Rightarrow A_1, A_2 \in \tau_i \, \forall i
        \in I \Rightarrow A_1 \cap A_2 \in \tau_i \, \forall i \in I \Rightarrow
        A_1 \cap A_2 \in \tau$.

        \item Siano $A_j, j \in J$ insiemi che stanno in $\tau$. \\
        $\displaystyle A_j \in \tau \, \forall j \in J \Rightarrow A_j \in
        \tau_i \, \forall i \in I, j \in J \Rightarrow {\bigcup_{j \in J} A_j
        \in \tau_i \, \forall i \in I} \Rightarrow {\bigcup_{j \in J} A_j \in
        \tau}$.
    \end{nlist}
\end{proof}

\begin{cor}
	Data una famiglia $\tau_i, i \in I$ di topologie su $X$, esiste la più fine
	tra le topologie meno fini di ogni $\tau_i$: è $\displaystyle \bigcap_{i \in
	I} \tau_i$.
\end{cor}

\begin{cor}
	Sia $X$ un insieme, $S \subseteq \mathcal{P}(X)$, allora esiste la topologia
	meno fine tra quelle che contengono $S$. Tale topologia si dice
	\textit{generata} da $S$ e $S$ si dice \textsc{prebase} della topologia. Se
	$\Omega= \{ \tau \text{ topologia } |\; S \in \tau \}$ (che è non vuoto
	perché contiene almeno la topologia discreta), la topologia cercata è
	$\displaystyle \bigcap_{\tau \in \Omega} \tau$.
\end{cor}

\begin{defn}
	sia $(X, \tau)$ uno spazio topologico fissato, una \textsc{base} di $\tau$ è
	un insieme $\mathcal{B} \subseteq \tau$ t.c. $\forall A \in \tau, \exists
	B_i \in \mathcal{B}, i \in I$ t.c. $A= \bigcup_{i \in I} B_i$. Ovvero
	$\mathcal{B}$ \`e una base se ogni aperto di $\tau$ pu\`o essere scritto
	come unione qualunque di elementi di $\mathcal{B}$.
\end{defn}

\begin{ex}
	Se $X$ è uno spazio metrico, una base della topologia indotta sono le palle.
\end{ex}

\begin{defn} \label{N2}
	$(X, \tau)$ si dice \textit{a base numerabile} (o che soddisfa il
	\textsc{secondo assioma di numerabilità}) se ammette una base numerabile.
\end{defn}

\begin{prop} \label{prop:base}
	Sia $X$ un insieme senza topologia, $\mathcal{B} \subseteq \mathcal{P}(X)$ è
	base di una topologia su $X$ $\Leftrightarrow$ valgono le seguenti:
	\begin{nlist}
		\item \label{i_prop} $\displaystyle X=\bigcup_{B \in \mathcal{B}} B$;
		\item $\forall A, A' \in \mathcal{B}, \exists B_i \in \mathcal{B}, i \in
		I$ t.c. $A \cap A'= \bigcup_{i \in I} B_i$. Cio\`e ogni intersezione di
		una coppia di elementi di $\mathcal{B}$ pu\`o essere scritta come unione
		di elementi di $\mathcal{B}$.
	\end{nlist}
\end{prop}

\begin{proof} \label{prop:preb}
    ($\Rightarrow$) Ovviamente, se $\mathcal{B}$ è la base di una topologia su
    $X$, l'insieme $X$ deve essere unione di elementi di $\mathcal{B}$, inoltre
    tutti gli elementi di $\mathcal{B}$ devono essere sottoinsiemi di $X$, da
    cui discende (i).

	$A, A' \in \mathcal{B} \Rightarrow A, A' \in \tau \Rightarrow A \cap A' \in
	\tau$, per cui $A \cap A'$ deve poter essere esprimibile come unione di
	elementi di $\mathcal{B}$, che è l'affermazione (ii).

	($\Leftarrow$) Dobbiamo mostrare che l'insieme $\tau$ di tutte le possibili
	unioni di elementi di $\mathcal{B}$ soddisfa gli assiomi di topologia.

	Chiaramente $\emptyset \in \tau$ come unione di un insieme vuoto di elementi
	di $\mathcal{B}$ e $X \in \tau$ per (ii).  Se faccio l'unione arbitraria di
	insiemi ottenuti come unione di elementi di $\mathcal{B}$ ottengo ovviamente
	un insieme che è unione di elementi di $\mathcal{B}$.

	Infine, $\displaystyle A, A' \in \tau \Rightarrow A=\bigcup_{i \in I} B_i,
	A'=\bigcup_{j \in J} B_j$ con $B_i, B_j \in \mathcal{B} \, \forall i \in I,
	j \in J$. Allora $\displaystyle A \cap A'= \left(\bigcup_{i \in I} B_i
	\right) \cap \left(\bigcup_{j \in J} B_j \right)=\bigcup_{i \in I, j \in J}
	(B_i \cap B_j)$, ma tutti i $B_i$ e $B_j$ stanno in $\mathcal{B}$, dunque
	per (ii) tutti i $B_i \cap B_j$ sono rappresentabili come unione di elementi
	di $\mathcal{B}$, perciò anche la loro unione, che è proprio $A \cap A'$,
	può essere scritta in quel modo e quindi sta in $\tau$.
\end{proof}

\begin{prop}
	Siano $X$ un insieme e $S \subseteq \mathcal{P}(X)$ la prebase di una
	topologia $\tau$ su $X$. Allora:
	\begin{nlist}
		\item le intersezioni finite di elementi di $S \cup \{X\}$ sono una base
		di $\tau$;
		\item $A \in \tau$ $\Leftrightarrow$ $A$ è unione arbitraria di
		intersezioni finite di elementi di $S \cup \{X\}$.
	\end{nlist}
\end{prop}

\begin{proof}
	Sicuramente, poiché $\tau$ è generata da $S$, $S \subseteq \tau$ e quindi
	anche tutte le intersezioni finite di elementi di $S$ e le unioni arbitrarie
	di tali inersezioni devono stare in $\tau$. Se mostriamo che sono
	sufficienti a definire una topologia, abbiamo finito.

    Chiaramente il vuoto è l'intersezione di un insieme vuoto di insiemi e $X$
    c'è perché lo abbiamo aggiunto a mano.

    Unione arbitraria di unioni arbitrarie di elementi di un insieme è ancora
    unione arbitraria di elementi di tale insieme.

    Siano ora $\displaystyle A_1= \bigcup_{i \in I} B_i,\ A_2=\bigcup_{j \in J}
    B_j$ con tutti
    i $B_i,\ B_j$ intersezioni finite di elementi di $S$. Allora
    $$A_1 \cap A_2 =
    \left( \bigcup_{i \in I} B_i \right) \cap \left(\bigcup_{j \in J} B_j
    \right)= \bigcup_{i \in I, j \in J} (B_i \cap B_j),$$
     ma  intersezione di due intersezioni finite di elementi di $S$ è ancora
     un'intersezione finita di elementi di $S$, perciò $A_1 \cap A_2$ è ancora
     un'unione di intersezioni finite di elementi di $S$. Questo basta per
     dimostrare (i) e (ii) è una semplice riformulazione.
\end{proof}

\begin{prop}
	Sia $f: (X, \tau) \rightarrow (Y, \tau')$ e $S,\ \mathcal{B}$
	rispettivamente una prebase e una base di $\tau'$. Allora sono equivalenti:
	\begin{nlist}
		\item $f$ è continua;
		\item $f^{-1}(A)$ è aperto per ogni $A \in S$;
		\item $f^{-1}(A)$ è aperto per ogni $A \in \mathcal{B}$.
	\end{nlist}
\end{prop}

\begin{proof}
	Poiché ogni base è una prebase, ((i) $\Leftrightarrow$ (ii)) $\Rightarrow$
	((i) $\Leftrightarrow$ (iii)).

	(i) $\Rightarrow$ (ii) è ovvia, perciò resta da dimostrare (ii)
	$\Rightarrow$ (i).

	Sia $A$ un aperto di $\tau'$. Per la proposizione \ref{prop:preb}, $A$ si
	può scrivere come unione di intersezioni finite di elementi di $S$. Poiché
	la controimmagine di un'unione è l'unione delle controimmagini e la
	controimmagine di un'intersezione è l'intersezione delle controimmagini, la
	controimmagine di $A$ è unione di intersezioni finite di controimmagini di
	elementi di $S$, ma queste controimmagini sono aperte, perciò un'unione di
	loro intersezioni finite è ancora aperta, perciò $f^{-1}(A)$ è aperto in $X$
	per ogni $A$ aperto in $Y$, e questo equivale a dire che $f$ è continua.
\end{proof}

\subsection{Assiomi di numerabilità}

Nella definizione \ref{N2} abbiamo stabilito quando uno spazio topologico
soddisfa il secondo assioma di numerabilità. Vediamo gli altri due.

\begin{defn}
	$Y \subseteq X$ si dice \textsc{denso} se $\overline{Y}=X$.
\end{defn}

\begin{oss}
	$Y$ è denso $\Leftrightarrow$ $Y \cap A \not=\emptyset$ per ogni $A$ aperto
	non vuoto.
\end{oss}

\begin{defn} \label{N3}
	$X$ si dice \textsc{separabile} (o che soddisfa il \textsc{terzo assioma di
	numerabilità}) se ammette un sottoinsieme denso numerabile.
\end{defn}

\begin{prop} \label{N2impN3}
	Se $X$ è a base numerabile, $X$ è separabile.
\end{prop}

\begin{proof}
	Sia $\{ B_i, {i \in \mathbb{N}}\}$ la base numerabile di $X$ e scegliamo per
	ogni $i \in \mathbb{N}$ un elemento $x_i \in B_i$. Vogliamo mostrare che $\{
	x_i, i \in \mathbb{N} \}$ è un sottoinsieme denso numerabile di $X$.

	Ovviamente è numerabile. Per dimostrare che è denso, mostriamo che interseca
	ogni aperto non vuoto.

	Sia dunque $A$ un aperto non vuoto di $X$, allora, dato che i $B_i$ formano
	una base, $A$ è esprimibile come unione di alcuni di essi, in particolare
	esiste $i_0$ t.c. $B_{i_0} \subseteq A$ e perciò $x_{i_0} \in B_{i_0}
	\Rightarrow x_{i_0} \in A$, dunque l'intersezione tra il nostro insieme e un
	aperto non vuoto non è mai vuota, come voluto.
\end{proof}

\begin{prop} \label{metr-num}
	Se $(X, \tau)$ è metrizzabile, $X$ è separabile $\Leftrightarrow$ è a base
	numerabile.
\end{prop}

\begin{proof}
	($\Rightarrow$) Mostriamo che per ogni aperto $A$ e per ogni $x \in A$,
	esiste una palla di centro un elemento del sottoinsieme denso numerabile e
	raggio un razionale positivo tutta contenuta in $A$. Allora tali palle
	formano una base numerabile.

	Sicuramente esiste una palla di centro $x$ e raggio $R \in \mathbb{R}^+$
	tutta contenuta in $A$. Consideriamo la palla di centro $x$ e raggio $R/3$,
	che è contenuta nella prima. Questa palla è in particolare un insieme
	aperto, dunque ha un elemento $y$ in comune con il sottoinsieme denso
	numerabile. Scegliendo un razionale $R/3<r<2 \cdot R/3$ si ottiene che la
	palla di centro $y$ e raggio $r$ è tutta contenuta nella palla di centro $x$
	e raggio $R$, dunque è tutta contenuta in $A$, e contiene $x$, come voluto.

	L'altra freccia discende dalla proposizione \ref{N2impN3}.
\end{proof}

\begin{defn}
	Un \textsc{sistema fondamentale di intorni} per $x_0$ è una famiglia
	$\mathcal{F} \subseteq \mathcal{I}(x_0)$ t.c. $\forall U \in
	\mathcal{I}(x_0) \, \exists V \in \mathcal{F}$ con $V \subseteq U$.
\end{defn}

\begin{ex} \label{1/n}
	Se $X$ è uno spazio metrico, le palle centrate in $x_0$ do raggio $1/n$ al
	variare di $n$ intero positivo sono un sistema fondamentale di intorni. In
	particolare, sono un sistema fondamentale di intorni numerabile.
\end{ex}

\begin{defn} \label{N1}
	$X$ soddisfa il \textsc{primo assioma di numerabilità} se ogni $x_0 \in X$
	ha un sistema fondamentale di intorni numerabile.
\end{defn}

\begin{ftt}
	Se $X$ è metrizzabile $X$ soddisfa il primo assioma di numerabilità. Ciò
	discende direttamente dall'esempio \ref{1/n}.
\end{ftt}

\begin{prop} \label{N2power}
	Il secondo assioma di numerabilità implica il primo e il terzo assioma di
	numerabilità. Discende dalla proposizione \ref{metr-num} che in spazi
	metrici vale anche che il terzo assioma di numerabilità implica il secondo
	assioma di numerabilità.
\end{prop}

\begin{proof}
	Supponiamo che $X$ soddisfi il secondo assioma di numerabilità. Per la
	proposizione \ref{N2impN3} otteniamo subito che soddisfa il terzo. Vogliamo
	adesso mostrare che soddisfa anche il primo. Sia $\mathcal{B}$ una base
	numerabile. Dato $x \in X$, consideriamo l'insieme ${\{ B \in \mathcal{B}\
	|\ x \in B \}}$. Essendo un sottoinsieme di $\mathcal{B}$ è sicuramente
	numerabile e contiene solo insiemi aperti, cioè intorni di $x$. Mostriamo
	che è un sistema fondamentale di intorni per $x$.

	Consideriamo un intorno $U$ di $x$. Questo ha un sottoinsieme aperto $V
	\subseteq U$ contenente $x$, dunque $V$ è esprimibile come unione di
	elementi di $\mathcal{B}$ e ce ne dev'essere uno che contiene $x$, che sta
	quindi nel nostro insieme ed è contenuto in $V$ e quindi in $U$. Dunque, per
	ogni intorno $U$ di $x$ troviamo un elemento del nostro insieme di intorni
	di $x$ contenuto in $U$, che è quello che dovevamo dimostrare.
\end{proof}

\begin{prop}
	\begin{nlist}
		\item $A \subseteq X$ è aperto se e solo se è intorno di ogni suo punto;
		\item sia $C \subseteq X$ un insieme generico, allora $x \in
		\overline{C}$ se e solo se ogni intorno di $x$ interseca $C$.
	\end{nlist}
\end{prop}

\begin{proof}
    \begin{nlist}
        \item Sia $A$ un insieme che \`e intorno di ogni suo punto, cio\`e per
        ogni $x \in A$ esiste un aperto $V_x \subseteq A$ che lo contiene.
        Allora vale che $\displaystyle A = \bigcup_{x \in A} V_x$. Poiché $A$
        \`e unione di aperti\`e aperto.

        Viceversa sia $A$ aperto. Allora ogni suo punto \`e interno ad esso.

        \item Si ha che $\overline{C}$ \`e il complementare di $(X\setminus
        C)^\circ$. Dunque $x$ appartiene a $\overline{C}$ se e solo se non \`e
        interno a $X \setminus C$, cioè se e solo se ogni suo intorno $U$ non
        \`e tutto contenuto nel complementare di $C$. Questo accade quando ogni
        intorno di $X$ interseca $C$ in almeno un punto.
    \end{nlist}
\end{proof}

\begin{ex}
	La retta di Sorgenfrey. Su $X=\mathbb{R}$ consideriamo il seguente insieme:
	$${\mathcal{B}=\{\,[a, b)\ |\ a<b\}}.$$
    Allora valgono le seguenti:
	\begin{nlist}
		\item $\mathcal{B}$ è base di una toplogia $\tau$;
		\item $\tau$ \`e pi\`u fine della topologia euclidea;
		\item $\tau$ è separabile;
		\item $\tau$ non è a base numerabile;
		\item $\tau$ non è metrizzabile;
		\item $\tau$ è primo numerabile.
	\end{nlist}
\end{ex}

\begin{proof}
    \begin{nlist}
        \item Verifichiamo utilizando la proposizione \ref{prop:base}. Il primo
        punto vale in quanto $\mathbb{R}$ \`e coperto da intervalli del tipo
        $\left[-n, n\right)$ con $n$ naturale. Inoltre dati i due intervalli
        $[a,b)$ e $[c,d)$, distinguo:
        \begin{itemize}
            \item $b\leq c$. Allora l'itersezione \`e vuota.
            \item $c < b$ e $b \leq d$. Allora l'itersezione \`e $[c,b)$.
            \item $c < b$ e $b > d$. Allora l'itersezione \`e $[c,d)$.
        \end{itemize}
        In ogni caso vale anche il secondo punto.

        \item Voglio mostrare che ogni aperto di $\tau$ \`e anche aperto di
        $\tau_E$. Sia allora $(a, b)$ un aperto della base degli intervalli
        aperti della topologia euclidea. Si considerino gli aperti in $\tau$
        della forma $[a+\frac{1}{n},b)$ con $n$ naturale positivo. L'unione di
        tutti questi \`e allora $(a,b)$.

        \item Si noti che $\mathbb{Q}$ \`e un denso numerabile.

        \item Sia $\mathcal{D}$ una base di $\tau$. Si noti che ogni intervallo
        del tipo $[x,x+1)$, con $x$ reale, pu\`o essere scritto come unione
        degli elementi della base solo se in $\mathcal{D}$ c'\`e un elemento $D$
        tale che $x \in D \subseteq [x, x+1)$, da cui $\inf(D) = x$. Allora la
        mappa
        \begin{align*}
            \phi:\mathcal{D}&\to \mathbb{R}\\
            D & \mapsto \inf(D)
        \end{align*}
        deve essere surgettiva. Dunque la base scelta non pu\`o essere
        numerabile.

        \item Semplice conseguenza dei punti precedenti e della proposizione
        \ref{metr-num}.

				\item Basta notare che per ogni $x \in \mathbb{R}$ l'insieme $\{
				[x-1/n, x+1/n) | n \in \mathbb{Z}^+\}$ è una famiglia di intorni
				numerabile per $x$.
    \end{nlist}
\end{proof}

\begin{ex}
	Topologia di Zariski. Sia $\mathbb{K}$ un campo. Definiamo una topologia
	$\tau_Z$ in $\mathbb{K}^n$ in cui i chiusi sono tutti e soli gli insiemi i
	cui elementi si annullano in tutti i polinomi di una famiglia arbitraria
	(non vuota) $\mathcal{F} \subseteq \mathbb{K}[x_1, \dots, x_n]$ di polinomi
	a $n$ variabili con coefficienti in $\mathbb{K}$. Dimostreremo che $\tau_Z$
	è effettivamente una topologia, che è meno fine di quella euclidea quando
	$\mathbb{K}=\mathbb{R}$. Inoltre, quando $n=1$ e per $\mathbb{K}$ generico,
	$\tau_Z$ coincide con la topologia cofinita.
\end{ex}

\begin{proof}
	Dimostreremo che i chiusi soddisfano le proprietà di topologia (per i
	chiusi, ovviamente), per passaggio al complementare si può concludere che
	$\tau_Z$ è una topologia.

	Ovviamente il vuoto è chiuso perché la proprietà dei suoi elementi di
	annullarsi in un qualunque insieme di polinomi è sempre vera a vuoto, mentre
	$\mathbb{K}^n$ è chiuso perché tutti gli elementi si annullano nella
	famiglia formata dal solo polinomio nullo.

	Siano ora $C_1, C_2$ due chiusi i cui elementi si annullano, per
	definizione, nei polinomi rispettivamente delle famiglie $\mathcal{F}_1,
	\mathcal{F}_2$. Consideriamo la famiglia $\mathcal{F}_{1, 2}=$ \\ $=\{ f_1
	\cdot f_2 | f_1 \in \mathcal{F}_1, f_2 \in \mathcal{F}_2 \}$. Mostriamo che
	l'insieme dei punti che si annullano nei polinomi di $\mathcal{F}_{1, 2}$ è
	proprio $C_1 \cup C_2$. Ovviamente se $x \in C_1 \cup C_2$ possiamo dire
	senza perdita di generalità che $x$ si annulla in tutti i polinomi in
	$\mathcal{F}_1$, e quindi banalmente anche in tutti i polinomi di
	$\mathcal{F}_{1, 2}$. D'altro canto, se $x$ si annulla in tutti i polinomi
	di $\mathcal{F}_{1, 2}$, si deve annullare almeno o in tutti i polinomi di
	$\mathcal{F}_1$ o in tutti i polinomi di $\mathcal{F}_2$. Se per assurdo
	così non fosse, esisterebbero $f_1 \in \mathcal{F}_1, f_2 \in \mathcal{F}_2$
	t.c. $f_1(x) \not= 0 \not= f_2(x)$ e, poiché siamo in un campo, si avrebbe
	$f_1(x) \cdot f_2(x)=0$, assurdo. Dunque $x \in C_1 \cup C_2$.

	Consideriamo adesso dei chiusi $C_i, i \in I$ definiti dalle famiglie
	$\mathcal{F}_i$. Mostriamo che $\displaystyle C=\bigcap_{i \in I} C_i$ è
	l'insieme dei punti che si annullano nei polinomi di $\displaystyle
	\mathcal{F}_I=\bigcup_{i \in I} \mathcal{F}_i$. Se $x \in C$ allora $x \in
	C_i \, \forall i \in I$ e dunque si annulla in tutti i polinomi di
	$\mathcal{F}_i$ per ogni $i$, quindi si annulla in tutti i polinomi della
	loro unione, che è proprio $\mathcal{F}_I$. Viceversa, se $x$ si annulla in
	tutti i polinomi di $\mathcal{F}_I$ allora si annulla in tutti i polinomi di
	ogni suo sottoinsieme, in particolare in tutti i polinomi di $\mathcal{F}_i
	\, \forall i \in I$, quindi $x \in C_i$ per ogni $i$ da cui $x \in C$.

	Poniamo ora $\mathbb{K}=\mathbb{R}$. Poiché i polinomi in $\mathbb{R}[x_1,
	\dots, x_n]$ sono funzioni continue per la topologia euclidea, la
	controimmagine di un aperto è a sua volta un aperto. Per passaggio al
	complementare la controimmagine di un chiuso è a sua volta un chiuso. Ma
	allora, sia $\mathcal{F}$ una famiglia di polinomi, ho che, essendo $\{ 0
	\}$ chiuso in $\tau_E$ di $\mathbb{R}$, $f^{-1}(0)$ (qui si intende la
	controimmagine) è chiuso in $\tau_E$ di $\mathbb{R}^n$ per ogni $f \in
	\mathcal{F}$. Ma l'insieme dei punti che si annullano in tutti i polinomi di
	$\mathcal{F}$ può essere descritto come $\displaystyle \bigcap_{f \in
	\mathcal{F}} f^{-1}(0)$, che essendo intersezione di chiusi è chiuso in
	$\tau_E$, quindi tutti i chiusi di $\tau_Z$ sono chiusi in $\tau_E$, per
	passaggio al complementare la stessa cosa con gli aperti e dunque $\tau_Z <
	\tau_E$.

	Sia adesso $\mathbb{K}$ generico e $n=1$. Sia $p$ un generico polinomio in
	$\mathbb{R}[x]$ e $\overline{\mathbb{K}}$ la chiusura algebrica di
	$\mathbb{K}$. Per il teorema fondamentale dell'algebra, $p$ ha al più
	$\deg{p}$ zeri in $\overline{\mathbb{K}}$, quindi a maggior ragione ne ha al
	più un numero finito in $\mathbb{K}$. Dunque i punti che si annullano in
	tutti i polinomi di una generica famiglia di polinomi sono finiti (limitati
	dal grado di un qualsiasi polinomio della famiglia), dunque i chiusi sono
	tutti finiti.
	Considerando invece un insieme finito $\{ x_1, \dots, x_m\} \subseteq
	\mathbb{K}$, esso si annulla in tutti i polinomi dell'ideale generato da
	$(x-x_1) \cdot \ldots \cdot (x-x_m)$, dunque è vero anche il viceversa, cioè
	che tutti i finiti sono chiusi, quindi in questo caso $\tau_Z$ coincide con
	la topologia euclidea.
\end{proof}

\begin{ftt}
	Se $X$ non è numerabile, la topologia cofinita su $X$ non soddisfa alcun
	assioma di numerabilità. %Secondo me il terzo lo soddisfa per qualunque
	                           % insieme finito
\end{ftt}

\begin{proof}
	Per la proposizione \ref{N2power}, se dimostriamo che non soddisfa il primo
	otteniamo anche che non soddisfa il secondo. Sia dunque per assurdo $x_0 \in
	X$ e $\mathcal{F}$ un suo sistema fondamentale di intorni numerabile.
	Notiamo che per ogni $x \in X, x\not=x_0$ l'insieme $X \setminus \{ x\}$ è
	un aperto, dunque esiste un intorno di $x_0$ tutto contenuto in esso, cioè
	che non contiene $x$. Notiamo, per come è definita la topologia cofinita,
	che gli intorni, essendo sovrainsiemi di insiemi aperti, sono a loro volta
	aperti, cioè il loro complementare è finito. Ma dato che per ogni $x \in X
	\setminus \{ x_0\}$ esiste un intorno di $x_0$ che non contiene $x$, posso
	scrivere $X \{ x_0\}$, che è ancora un insieme numerabile, come l'unione dei
	complementari degli insiemi in $\mathcal{F}$, cioè un'unione numerabile di
	insiemi finiti, che è numerabile, da cui l'assurdo.

	Per quanto riguarda il terzo assioma, secondo me ho capito male mentre ero a
	lezione, perché mi sembra che qualunque sottoinsieme infinito di $X$ debba
	necessariamente intersecare in qualche punto il complementare di un insieme
	finito, e quindi anche un insieme infinito di cardinalità numerabile avrebbe
	intersezione non nulla con tutti gli aperti e sarebbe di conseguenza un
	denso numerabile.
\end{proof}

Procediamo adesso a caratterizzare, negli insiemi che soddisfano il primo
assioma di numerabilità, aperti, chiusi e continuità tramite successioni.
Diamo prima una definizione.

\begin{defn}
	$l \in X$ è detto \textit{limite} della successione $(a_n)_{n \in
	\mathbb{N}}$ se per ogni intorno $U$ di $l$ esiste $n_0 \in \mathbb{N}$ t.c.
	per ogni $n \ge n_0$ si ha che $a_n \in U$.
\end{defn}

Passiamo dunque alle varia caratterizzazioni. Nei due (tre?) enunciati seguenti,
$X$ sarà sempre uno spazio topologico primo numerabile.

%come si definiscono gli aperti per successioni?

\begin{prop}
	Un sottoinsieme $C \subseteq X$ è chiuso $\Leftrightarrow$ è chiuso per
	successioni, cioè per ogni successione $(a_k)$ di elementi di $C$ che
	converge a un limite $l$, anche $l \in C$.
\end{prop}

\begin{proof}
	Notiamo prima il seguente fatto: se $U_k, \, n \in \mathbb{N}$ è un sistema
	fondamentale di intorni numerabile per $x$, definiamo $V_k=U_0 \cap U_1 \cap
	\dots \cap U_k$. Allora $V_k, \, k \in \mathbb{N}$ è un sistema fondamentale
	di intorni numerabile t.c. $V_{k+1} \subseteq V_k$.

	($\implies$) Supponiamo $C$ chiuso e sia $(a_k)$ una successione di elementi
	di $C$ con limite $l$. Per assurdo, $l \not\in C$. Allora $l \in X \setminus
	C$, che è un insieme aperto, in particolare $X \setminus C$ è un intorno di
	$l$. Esiste dunque, per definizione di limite, un $k_0 \in \mathbb{N}$ t.c.
	per ogni $k \ge k_0$ si abbia che $a_k \in X \setminus C$, ma $a_k \in C$
	per ogni $k \in \mathbb{N}$,
	assurdo. Questa freccia vale in tutti gli spazi topologici.

	($\Leftarrow$) Supponiamo adesso che per ogni successione $(a_k)$ di
	elementi di $C$ avente limite $l$ si abbia $l \in C$. Consideriamo un
	elemento $x \in X \setminus C$. Vogliamo mostrare che esiste un intorno
	$U_x$ di $x$ tutto contenuto in $X \setminus C$. Se così non fosse, per ogni
	intorno di $x$ esisterebbe un elemento di $C$ in esso contenuto. Poiché $X$
	è primo numerabile, consideriamo dunque un sistema fondamentale di intorni
	numerabile di $x$, sia esso $V_k, \, k \in \mathbb{N}$, e lo prendiamo t.c.
	$V_{k+1} \subseteq V_k$. Adesso scegliamo per ogni $k$ un elemento $a_k \in
	V_k$ t.c. $a_k \in C$, che esiste per l'ipotesi assurda. Notiamo che alcuni
	di questi elementi possono essere uguali, non ha importanza. Dato che
	abbiamo preso un sistema di intorni fondamentale, per ogni intorno $U$ di
	$x$ esiste un $k_0$ t.c. $V_{k_0} \subseteq U$. Ma poiché $V_k \subseteq V_
	{k_0}$ per ogni $k \ge k_0$ (facile conseguenza di $V_{k+1} \subseteq V_k$),
	ho che $a_k \in V_k \subseteq V_{k_0} \subseteq U$ per ogni $k \ge k_0$.
	Riassumendo, per ogni $U$ intorno di $x$ esiste un $k_0$ t.c. per ogni $k
	\ge k_0$ si ha $a_k \in U$, ma questo per definizione significa che $x$ è il
	limite di $(a_k)$, assurdo poiché $a_k \in C$ per ogni $k$ mentre $x \in X
	\setminus C$, contro l'ipotesi iniziale che per tutte le successioni in $C$
	aventi limite anche il limite è in $C$. Dunque per ogni $x \in X \setminus
	C$ esiste un intorno $U_x \subseteq X \setminus C$, da cui si ha che esiste
	un aperto $A_x$ con $x \in A_x \subseteq U_x \subseteq X \setminus C$. Ma
	allora, $\displaystyle X \setminus C=\bigcup_{x \in X \setminus C} A_x$ che
	un'unione di aperti, perciò $X \setminus C$ è aperto e di conseguenza $C$ è
	chiuso.
\end{proof}

\begin{prop}
	Sia $Y$ uno spazio topologico (che non deve necessariamente soddisfare il
	primo assioma di numerabilità). Una funzione $f:X \rightarrow Y$ è continua
	in $\bar{x}$ $\Leftrightarrow$ per ogni successione$(x_n)_{n \in
	\mathbb{N}}$ convergente a $\bar{x}$ la successione $(f(x_n))_{n \in
	\mathbb{N}}$ converge a $f(\bar(x))$.
\end{prop}

\begin{proof}
	($\implies$) Supponiamo che $f$ sia continua in $\bar{x}$ e consideriamo una
	generica successione $(x_n)$ convergente a $\bar{x}$. Prendiamo un intorno
	$U$ di $f(\bar{x})$ in $Y$. Dato che $f$ è continua in $\bar{x}$, esiste un
	intorno $V$ di $\bar{x}$ in $X$ t.c. $f(V) \subseteq U$. Prendiamo, per
	ipotesi di convergenza, $n_0 \in \mathbb{N}$ t.c. $a_n \in V$ per ogni $n
	\ge n_0$. Allora si ha anche, sempre per ogni $n \ge n_0$, $f(x_n) \in U$,
	da cui otteniamo che $(f(x_n))$ converge a $f(\bar{x})$. Questa freccia vale
	per $X$ spazio topologico qualsiasi.

	($\Leftarrow$) Supponiamo adesso che per ogni successione convergente in $X$
	la successione immagine converga all'immagine del limite in $Y$. Per
	assurdo, $f$ non è continua in $\bar{x}$. Allora deve esistere un intorno
	$U$ di $f(\bar{x})$ t.c. per ogni intorno $V$ di $\bar{x}$ si ha $f(V) \not
	\subseteq U$. Prendiamo un sistema fondamentale di intorni numerabile di
	$\bar{x}$, sia esso $V_n, \, n \in \mathbb{N}$, e come nella dimostrazione
	precedente lo prendiamo t.c. $V_{n+1} \subseteq V_n$ per ogni $n$.

	Prendiamo, per ogni $n$, un elemento $x_n \in V_n$ t.c. $f(x_n) \not\in U$,
	che esiste per l'ipotesi assurda. Allora si mostra, come nella dimostrazione
	precedente, che la successione $(x_n)$ tende a $\bar{x}$, ma le loro
	immagini $f(x_n)$ sono tutte fuori dallo stesso intorno $U$ di $f(\bar{x})$,
	dunque la successione $(f(x_n))$ non tende a $f(\bar{x})$, assurdo per
	ipotesi.
\end{proof}

\subsection{Sottospazi topologici}

\begin{defn}
    Sia $(X, \tau)$ uno spazio topologico, $Y \subseteq X$ un sottoinsieme,
    cio\`e esiste una mappa iniettiva $i: Y \hooklongrightarrow X$ di
    inclusione. La topologia ristretta su $Y$ da $\tau$ \`e la topologia meno
    fine che rende $i$ continua.
\end{defn}

\begin{oss}
    Per avere $i$ continua, serve che per ogni $U$ aperto di $\tau$, si abbia
    $i^{-1}(U)$ aperto nella topologia ristretta $\tau \restrict{Y}$. Poiché
    $i^{-1}(U) = U \cap Y$, si ha in effetti che per ogni aperto $U$ di $\tau$,
    $U \cap Y \in \tau \restrict{Y}$.

    Siccome $\sigma = \{U \cap Y \;|\; U \in \tau\}$ \`e una topologia su $Y$
    che rende continua $i$ e ${\tau\restrict{Y} \subseteq \sigma}$, deve valere
    l'uguaglianza in quanto $\tau\restrict{Y}$ \`e la meno fine con tali
    proprietà.
\end{oss}

\begin{oss}
    Vale anche che se $\mathcal{B}$ \`e base per $\tau$, allora
    \[
    \mathcal{B'} = \{B \cap Y \;|\; B \in \mathcal{B}\}
    \]
    \`e una base per $\tau\restrict{Y}$. La dimostrazione \`e analoga a quella
    fatta per le topologie.
\end{oss}

\begin{defn}
    Sia $(X, d)$ uno spazio metrico e $Y \subseteq X$ un sottoinsieme generico.
    La distanza $d$ pu\`o essere ristretta a $Y \times Y$. In tal caso
    $d\restrict{Y \times Y}$, che verr\`a indicata anche come $d\restrict{Y}$
    \`e una distanza su X.
\end{defn}

\begin{prop}
    Nel setting della definizione precedente, siano $\tau$ la topologia indotta
    da $d$ su $X$, $\tau\restrict{Y}$ la restrizione di $\tau$ a $Y$, e $\sigma$
    la topologia indotta da $d\restrict{Y}$ su $Y$. Allora $\tau\restrict{Y} =
    \sigma$.
\end{prop}

\begin{proof}
    Siano $\mathcal{B}$ e $\mathcal{D}$ rispettivamente basi per
    $\tau\restrict{y}$ e $\sigma$. Cio\`e:
    \begin{align*}
        \mathcal{B}\ =&\ \{Y\ \cap\ \{x \in X \tc d(x, x_0) < r,\quad x_0 \in
        X,\ r \in \mathbb{R}^+\}\}\\
        =&\ \{y \in Y \tc d(y, x_0) < r,\quad x_0 \in X,\ r \in \mathbb{R}^+\}
    \end{align*}
    \[
        \mathcal{D}\ =\ \{y \in Y \tc d(y, y_0) < r,\quad y_0 \in Y,\ r \in
        \mathbb{R}^+\}
    \]
    Chiaramente $\mathcal{D} \subseteq \mathcal{B}$, quindi anche $\sigma
    \subseteq \tau\restrict{Y}$.

    Inoltre, poiché le palle sono aperte, preso un $B \in \mathcal{B}$, per ogni
    $y \in B \cap Y$ esiste un $r_y>0$ tale che $D_y = B(y,r_y)$ sia contenuto
    in $B$. per cui si ha che
    \[
        B = \bigcup_{y \in B \cap Y} D_y.
    \]
    Ma ogni $D_y$ \`e un elemento della base $\mathcal{D}$, quindi ogni aperto
    di $\tau\restrict{Y}$ \`e un aperto di $\sigma$, concludendo l'ultima
    inclusione.
\end{proof}

\begin{defn}
    Siano $(X, \tau)$ spazio topologico, $Y \subseteq X$. Allora $Y$ si dice
    discreto in $X$ se $\tau\restrict{Y} = \mathcal{P}(Y)$, cioè la topologia
    ristretta a $Y$ \`e quella discreta.
\end{defn}

\begin{oss}
    Se $Y$ \`e discreto in $X$, allora per ogni $y \in Y$ esiste un aperto $U$
    di $X$ tale che $U \cap Y = \{y\}$.
\end{oss}

\begin{thm}
    \emph{Proprietà universale delle immersioni.} Sia $X$ spazio topologico e $Y
    \subseteq X$ con $i: Y \hooklongrightarrow X$ mappa di immersione. Allora
    per ogni spazio topologico $Z$ e per ogni funzione $g: Z \longrightarrow Y$,
    si ha che $f$ \`e continua se e solo se $i \circ f$ \`e continua.
\end{thm}

\begin{proof}
    Poiché $i$ \`e continua per la definizione della topologia su $Y$, e poiché
    la composizione di continue \`e continua, si ha che se $f$ \`e continua,
    anche $i \circ f$ lo \`e.

    Per l'altra implicazione, supponiamo $i \circ f$ continua e sia $A$ un
    aperto di $Y$. Allora esiste $U$ aperto di $X$ tale che ${U\cap Y = A}$, e
    vale ${i^{-1}(U) = A}$. Allora $(i\circ f)^{-1}(U) = f^{-1}(i^{-1}(U)) =
    f^{-1}(A)$ \`e aperto in $Z$.
\end{proof}

\begin{thm}
    \emph{Universalit\`a della proprietà universale.} La proprietà universale
    delle immersioni caratterizza in modo unico la topologia ristretta. Cio\`e
    dato $(X, \tau)$ spazio topologico con $Y \subseteq X$ e immersione ${i: Y
    \hooklongrightarrow X}$, $\tau\restrict{Y}$ \`e l'unica topologia su $Y$ che
    rispetta la proprietà universale.
\end{thm}

\begin{proof}
    Sia $\sigma$ una topologia su $Y$ che rispetta la propriet\`a universale.
    \begin{nlist}
        \item Prendo come $Z$ lo spazio $(Y, \tau\restrict{Y})$ e come $f$
        l'identità su $Y$. Il diagramma della proprietà universale \`e allora il
        seguente:

        \begin{center}\begin{tikzcd}
            (Y, \tau\restrict{Y}) \arrow[r, "\id"] \arrow[rd, "g=i"]
            & (Y, \sigma) \arrow[d, "i", hookrightarrow]\\
            & (X,\tau)
        \end{tikzcd}\end{center}

        Per definizione $g$ \`e continua, $i$ \`e continua, quindi $\id$ \`e
        continua. Allora $\tau\restrict{Y} \subseteq \sigma$.

        \item Questa volata prendo come $Z$ lo spazio $(Y, \sigma)$ mantenendo
        come $f$ l'identità. Il diagramma risulta essere:

        \begin{center}\begin{tikzcd}
            (Y, \sigma) \arrow[r, "\id"] \arrow[rd, "g=i"]
            & (Y, \sigma) \arrow[d, "i", hookrightarrow]\\
            & (X,\tau)
        \end{tikzcd}\end{center}

        Per definizione $\id$ \`e continua, quindi lo \`e anche $g=i$ per la
        propriet\`a universale. Allora $\sigma$ rende continua $i$, e dunque
        vale $\sigma \subseteq \tau\restrict{Y}$ per la definizione della
        restrizione di $\tau$.
    \end{nlist}
\end{proof}
\end{document}
