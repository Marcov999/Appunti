\documentclass{article}
\usepackage[italian]{babel}
\usepackage[T1]{fontenc}
\usepackage[utf8]{inputenc}
\usepackage{amsmath}
\usepackage{amsthm}
\usepackage{amssymb}
\usepackage{pgf,tikz}
\usepackage{mstyle}
\usepackage{hyperref}

\usetikzlibrary{arrows}

\title{Appunti di Geometria 2 \\ anno accademico 2019/2020}
\date{}
\author{Marco Vergamini \& Alessio Marchetti}

\begin{document}
\maketitle
\newpage
\tableofcontents
\newpage

\section{Introduzione}
Questi appunti sono basati sul corso di Geometria 2 tenuto dai professori
Roberto Frigerio e Jacopo Gandini nell'anno accademico 2019/2020. Sono dati per
buoni i contenuti dei corsi del primo anno, in particolare Analisi 1 e Geometria
1. Verranno omesse o soltanto hintate le dimostrazioni più semplici, ma si
consiglia comunque di provare a svolgerle per conto proprio. Ogni tanto sarà
commesso qualche abuso di notazione, facendo comunque in modo che il significato
sia reso chiaro dal contesto. Inoltre, la notazione verrà alleggerita man mano,
per evitare inutili ripetizioni e appesantimenti nella lettura.


\section{Spazi metrici e spazi topologici}
\subsection{Spazi metrici}
\begin{defn}
	Uno \textsc{spazio metrico} è una coppia $(X, d)$, $X$ insieme,
	${d: X \times X \rightarrow \mathbb{R}}$ t.c. per ogni ${x, y, z \in X}$
    \begin{nlist}
	\item $d(x, y) \ge 0$ e $d(x, y)=0 \Leftrightarrow x=y$;
	\item $d(x, y)=d(y, x)$;
	\item $d(x, z) \le d(x, y)+d(y,z)$ (disuguaglianza triangolare).
    \end{nlist}
	In tal caso $d$ si dice \textsc{distanza} o \textsc{metrica}.
\end{defn}

\begin{ex}
Ecco alcuni esempi di distanze:
\begin{itemize}
\item La distanza $d_1$ su $\mathbb{R}^n$, che alla coppia di elementi
${x=(x_1, \dots, x_n)}, {y=(y_1, \dots, y_n)}$
 associa il numero
 $${\displaystyle d_1(x, y)=\sum_{i=1}^n |x_i-y_i|};$$
\item la distanza $d_2$ (o $d_E$, distanza
euclidea) su $\mathbb{R}^n$,
$$\displaystyle d_2(x, y)=\sqrt{\sum_{i=1}^n (x_i-y_i)^2};$$
\item la distanza $d_{\infty}$ su $\mathbb{R}^n$,
${\displaystyle d_{\infty}(x, y)=\sup_{i=1, \dots, n} \{ |x_i-y_i| \}}$;
\item
la distanza discreta su un generico insieme $x$, $d(x, y)=1$ se $x \not= y$ e
$0$ altrimenti;
\item le distanze $d_1$, $d_2$, $d_{\infty}$ sullo spazio delle funzioni
continue da $[0, 1]$ in $\mathbb{R}$, definite rispettivamente
$${\displaystyle d_1(f, g)=\int_0^1 |f(t)-g(t)| dt},$$
$${\displaystyle d_2(f, g)=\sqrt{\int_0^1 (f(t)-g(t))^2 dt}},$$
$${\displaystyle d_{\infty}(f, g)=\sup_{t \in [0, 1]} |f(t)-g(t)|}.$$
\end{itemize}
\end{ex}

\begin{defn}
	$f:(X, d) \rightarrow (Y, d')$ viene detto \textsc{embedding isometrico} se
	$d'(f(x_1), f(x_2))=d(x_1, x_2)$ per ogni $x_1, x_2 \in X$.
\end{defn}

\begin{oss}
Valgono i seguenti fatti:
\begin{itemize}
\item l'identità è un embedding isometrico;
\item composizione di embedding isometrici è un embedding isometrico;
\item se un embedding isometrico $f$ è biettivo, anche $f^{-1}$ è un embedding
isometrico e $f$ si dice \textsc{isometria};
\item un embedding isometrico è sempre iniettivo, dunque è un'isometria se e
solo se è suriettivo;
\item se $(X, d)$ è fissato, l'insieme delle isometrie da $X$ in sé è un gruppo
con la composizione, chiamato $Isom(X,d)$.
\end{itemize}
\end{oss}



\subsection{Continuità in spazi metrici}
\begin{defn}
Dati $p \in X, R>0$, ${B(p, R):=\{ x \in X \mid d(p, x)<R\}}$ è detta la
\textsc{palla aperta} di centro $p$ e raggio $R$.
\end{defn}

\begin{defn}
${f:(X, d) \rightarrow (Y, d')}$ è \textsc{continua in $x_0$} se
${\forall \epsilon > 0 \; \exists \; \delta >0}\\\ \tc {f(B(x_0, \delta))
\subseteq B(f(x_0), \epsilon)}$, cioè ${f^{-1}(B(f(x_0), \epsilon)) \supseteq
B(x_0, \delta)}$.
\end{defn}

\begin{defn}
${f(X, d) \rightarrow (Y, d')}$ è detta \textsc{continua} se è continua in ogni
${x_0 \in X}$.
\end{defn}

\begin{oss}
Gli embedding isometrici sono continui.
\end{oss}

\begin{defn}
Sia $X$ uno spazio metrico, $A \subseteq X$ è detto \textsc{aperto} se per ogni
$x \in A$ esiste ${R>0 \tc B(x, R) \subseteq A}$.
\end{defn}

\begin{ftt}
Le palle aperte sono aperti. Per dimostrarlo, sfruttare la disuguaglianza
triangolare.
\end{ftt}

\begin{thm} \label{thm:cont_inv}
$f(X, d) \rightarrow (Y, d')$ è continua se e solo se per ogni aperto $A$ di $Y$
$f^{-1}(A)$ è aperto di $X$.
\end{thm}

\begin{proof}
Supponiamo $f$ continua. Prendiamo $x \in f^{-1}(A)$, allora si ha che $f(x) \in
A$, ma dato che $A$ è aperto esiste una palla aperta di centro $f(x)$ $B_Y$ t.c.
$B_Y \subseteq A$, da cui $f^{-1}(B_Y) \subseteq f^{-1}(A)$. Usando la
definizione di continuità, detto $\epsilon$ il raggio di $B_Y$, scegliendo il
$\delta$ corrispondente come raggio di una palla di centro $x$, sia essa $B_X$,
si ha che $B_X \subseteq f^{-1}(B_Y) \subseteq f^{-1}(A)$, perciò abbiamo
trovato una palla centrata in $x$ tutta contenuta in $f^{-1}(A)$ e questo prova
che è un aperto.

Viceversa, supponiamo che le controimmagini di aperti siano a loro volta aperti.
Dati $x_0 \in X$ e $\epsilon >0$ si ha che $B(f(x_0), \epsilon)$ è un aperto di
$Y$, perciò la sua controimmagine è un aperto, ma allora per definizione di
aperto esiste $\delta>0$ tale che $B(x_0, \delta) \subseteq f^{-1}(B(f(x_0),
\epsilon))$ e questo prova che $f$ è continua.
\end{proof}

\begin{oss}
La continuità di una funzione dipende quindi solo indirettamente dalla metrica,
mentre è direttamente collegata agli aperti generati dalla metrica stessa. Segue
facilmente che due metriche che generano gli stessi aperti portano anche alla
stessa famiglia di funzioni continue. Diamo dunque la seguente definizione.
\end{oss}

\begin{defn}
Due distanze $d, d'$ su un insieme $X$ si dicono \textsc{topologicamente
equivalenti} se inducono la stessa famiglia di aperti.
\end{defn}

\begin{lm}
Siano $d, d'$ distanze su $X$ t.c. esiste $k \ge 1$ t.c. per ogni $x, y \in X$
valga $d(x, y)/k \le d'(x, y) \le k \cdot d(x, y)$. Allora $d$ e $d'$ sono
topologicamente equivalenti.
\end{lm}

\begin{proof}
Sia $A$ un aperto indotto da $d'$ e $x_0 \in A$. Per definizione di aperto
esiste $R>0$ tale che $B_{d'}(x_0, R) \subseteq A$. Considero $B_d(x_0, R/k)$.
Dato un elemento $x \in B_d(x_0, R/k)$ ho che $d'(x_0, x) \le k \cdot d(x_0,
x)<k \cdot R/k=R$, dunque $x \in B_{d'}(x_0, R)$. Allora $B_d(x_0, R/k)
\subseteq B_{d'}(x_0, R/k) \subseteq A$ e dunque $A$ è anche un aperto indotto
da $d$. Per la simmetria delle disuguaglianze nelle ipotesi si dimostra anche
l'opposto, perciò gli aperti di $d$ e $d'$ sono gli stessi, come voluto.
\end{proof}

\begin{cor}
$d_1, d_2, d_{\infty}$ sono topologicamente equivalenti su $\mathbb{R}^n$.
\end{cor}

\begin{center}
\pagestyle{empty}
\begin{tikzpicture}[line cap=round,line join=round,>=triangle 45,x=2.0cm,y=2.0cm]
    \draw[->,color=black] (-1.72,0) -- (1.78,0);
    \foreach \x in {-1,1}
    \draw[shift={(\x,0)},color=black] (0pt,2pt) -- (0pt,-2pt);
    \draw[->,color=black] (0,-1.4) -- (0,1.44);
    \foreach \y in {-1,1}
    \draw[shift={(0,\y)},color=black] (2pt,0pt) -- (-2pt,0pt);
    \clip(-1.72,-1.2) rectangle (1.78,1.24);
    \draw(0,0) circle (2cm);
    \draw (-1,0)-- (0,1);
    \draw (0,1)-- (1,0);
    \draw (1,0)-- (0,-1);
    \draw (0,-1)-- (-1,0);
    \draw (-1,1)-- (1,1);
    \draw (1,1)-- (1,-1);
    \draw (1,-1)-- (-1,-1);
    \draw (-1,1)-- (-1,-1);
\end{tikzpicture}

Nell'immagine sono rappresentate, nell'ordine dalla più interna alla più
esterna, le palle aperte centrate nell'origine e di raggio $1$ rispettivamente
nelle metriche $d_1, d_2, d_{\infty}$.
\end{center}

\begin{proof}
Per AM-QM si ha che
$$\displaystyle \frac{\sum_{i=1}^n |x_i-y_i|}{n} \le \sqrt{\frac{\sum_{i=1}^n
(x_i-y_i)^2}{n}},$$
da cui $d_1(x, y) \le \sqrt{n} \cdot d_2(x, y)$. Stimando tutti i termini con il
massimo otteniamo che
$$\displaystyle \sqrt{\sum_{i=1}^n (x_i-y_i)^2} \le
\sqrt{\sum_{i=1}^n \sup_{j=1, \dots, n} \{(x_j-y_j)^2\}}=\sqrt{n} \cdot
\sup_{j=1, \dots, n} \{ |x_j-y_j| \},$$
da cui
$d_2(x, y) \le \sqrt{n} \cdot d_{\infty} (x, y).$
 Infine è ovvio che
 $$\displaystyle \sup_{i=1, \dots, n} |x_i-y_i| \le
 \sum_{i=1}^n |x_i-y_i|$$, da cui $d_{\infty}(x, y) \le d_1(x, y).$
Scegliendo $k=\sqrt{n}$ è ora sufficiente applicare il lemma.
\end{proof}

Nella dimostrazione del corollario era importante che lo spazio fosse di
dimensione finita. Nello spazio delle funzioni continue da $[0, 1]$ a
$\mathbb{R}$ le tre distanze non sono topologicamente equivalenti.

\subsection{Spazi topologici}

\begin{defn}
Uno \textsc{spazio topologico} è una coppia $(X, \tau),\quad {\tau \subseteq
\mathcal{P}(X)}$, t.c.
\begin{nlist}
\item $\emptyset, X \in \tau$;
\item $A_1, A_2 \in \tau \Rightarrow A_1 \cap A_2 \in \tau$;
\item se $I$ è un insieme e $A_i \in \tau \, \forall i \in I$, allora
$\displaystyle \bigcup_{i \in I} A_i \in \tau$.
\end{nlist}
$\tau$ si dice \textsc{topologia} di $X$ e gli elementi della topologia sono
detti \textsc{aperti} di $\tau$.
\end{defn}

\begin{prop}
Se $(X, d)$ è uno spazio metrico gli aperti rispetto a $d$ definiscono una
topologia.
\end{prop}

\begin{defn}
Uno spazio topologico $(X, \tau)$ è detto \textsc{metrizzabile} se $\tau$ è
indotta da una distanza su $X$.
\end{defn}

\begin{defn}
Sia $(X, \tau)$ uno spazio topologico, $C \subseteq X$ è \textsc{chiuso} se $X
\setminus C$ è aperto.
\end{defn}

\begin{oss}
\begin{nlist}
\item possono esistere insiemi né aperti né chiusi;
\item $\emptyset$ e $X$ sono chiusi;
\item unione finita di chiusi è chiusa;
\item intersezione arbitraria di chiusi è chiusa.
\end{nlist}
\end{oss}

\begin{ex}
Ecco alcuni esempi di spazi topologici:
\begin{itemize}
\item tutte le topologie indotte da una metrica;
\item la topologia discreta, cioè $\tau=\mathcal{P}(X)$, indotta dalla distanza
discreta;
\item la topologia indiscreta, cioè $\tau=\{ \emptyset, X \}$;
\item la topologia cofinita, dove gli aperti sono l'insieme vuoto più tutti e
soli gli insiemi il cui complementare è un insieme finito.
\end{itemize}
\end{ex}

\begin{defn}
	Dato uno spazio topologico $X$ e un insieme $B \subseteq X$, si chiama \textit{parte interna} di $B$, e la si indica con $\stackrel{\circ}{B}$, il più grande aperto contenuto in $B$. Analogamente, la \textit{chiusura} di $B$, indicata con $\overline{B}$, è il più piccolo chiuso che contiene $B$. Definiamo infine la \textit{frontiera} (o \textit{bordo}) di un insieme come l'insieme $\partial B= \overline{B} \setminus \stackrel{\circ}{B}$.
\end{defn}

Notiamo che parte interna e chiusura sono ben definite: per la prima basta prendere l'unione di tutti gli aperti contenuti in $B$ (alla peggio c'è solo il vuoto, che è contenuto in ogni insieme), per la seconda si prende l'intersezione di tutti i chiusi che lo contengono (alla peggio c'è solo $X$). Per la stabilità degli aperti per unione arbitraria e dei chiusi per intersezione arbitraria, gli insiemi così ottenuti sono ancora un aperto e un chiuso e sono rispettivamente il più grande aperto contenuto e il più piccolo chiuso che contiene per come sono stati costruiti.

\begin{ftt}
	$X= \stackrel{\circ}{B} \sqcup \partial B \sqcup \stackrel{\circ}{(X \setminus B)}$ dove con $\sqcup$ si indica l'unione disgiunta.
\end{ftt}

\begin{proof}
	Per definizione $\stackrel{\circ}{B} \cup \partial B=\overline{B}$ e i due insiemi sono disgiunti. Inoltre, essendo $\overline B$ il più piccolo chiuso che contiene $B$, il suo complementare dev'essere il più grande aperto disgiunto da $B$, cioè il più grande aperto contenuto in $X \setminus B$, che è la parte interna di quest'ultimo. La tesi segue facilmente.
\end{proof}

\subsection{Continuità in spazi topologici}

\begin{defn}
$f: (X, \tau) \rightarrow (Y, \tau')$ si dice \textsc{continua} se ${f^{-1}(A)
\in \tau,}\; {\forall A \in \tau'}$.
\end{defn}

Notiamo che per il teorema \ref{thm:cont_inv}, le due
definizioni di funzione continua sono equivalenti per uno spazio metrico se si
considera la topologia indotta dalla metrica.

\begin{thm}
\begin{nlist}
\item L'identità è una funzione continua;
\item composizione di funzioni continue è una funzione continua.
\end{nlist}
\end{thm}

\begin{defn}
Una funzione $f: X \rightarrow Y$ si dice \textsc{omeomorfismo} se $f$ è
continua e esiste una funzione $g:Y\rightarrow X$ continua tale che $f \circ g =
Id_Y$ e $g \circ f = Id_x$. Cio\`e $f$ \`e continua, bigettiva e con inversa
continua.
\end{defn}

\begin{oss}
	\begin{nlist}
	\item Composizione di omeomorfismi è un omeomerfismo; due spazi legati da un omeomorfismo si dicomo \textsc{omeomorfi} e essere omeomorfi è una relazione di equivalenza;
	\item l'insieme degli omeomorfismi da $(X, \tau)$ in sé è un gruppo;
	\item \warningsign\quad se $f:X \rightarrow Y$ è continua e bigettiva non è detto che sia un omeomorfismo, cioè $f^{-1}$ può non essere continua.
\end{nlist}
\end{oss}

\begin{ex}
	Siano $\tau_E$ la topologia euclidea, $\tau_C$ la cofinita, $\tau_D$ la discreta e $\tau_I$ l'indiscreta. Le seguenti mappe sono dunque continue: \\
	$Id:(\mathbb{R}, \tau_D) \rightarrow (\mathbb{R}, \tau_E)$, $Id:(\mathbb{R}, \tau_E) \rightarrow (\mathbb{R}, \tau_C)$, $Id:(\mathbb{R}, \tau_C) \rightarrow (\mathbb{R}, \tau_I)$. Nessuna delle inverse è però continua. Più in generale, $Id: (X, \tau) \rightarrow (X, \sigma)$ con $\sigma \subsetneq \tau$ è continua ma l'inversa no.
\end{ex}

\subsection{Ordinamento fra topologie, basi e prebasi}

Vogliamo mettere un ordinamento parziale sulle topologie di un certo insieme fissato.

\begin{defn}
Dato un insieme $X$ su cui sono definite le topologie $\tau$ e $\tau'$, si dice che $\tau$ \`e \textit{meno fine} di $\tau'$ se $\tau \subseteq \tau'$, cioè ogni aperto di $\tau$ è anche aperto di $\tau'$. $\tau'$ si dice \textit{più fine} di $\tau$.
\end{defn}

\begin{oss}
Equivalentemente alla definizione sopra, si pu\`o dire che $\tau$ \`e meno fine di $\tau'$ se e solo se ${Id:(X,\tau')\rightarrow(X, \tau)}$ \`e continua.
\end{oss}

Quando $\tau$ è meno fine di $\tau'$ scriveremo $\tau < \tau'$. Si noti che dalla definizione ogni topologia è meno fine di se stessa, cioè $\tau < \tau' \, \forall \tau$.

\begin{ex}
	$\tau_I < \tau_C < \tau_E < \tau_D$, \, $\tau_I < \tau < \tau_D \, \forall \tau$.
\end{ex}

\begin{lm}
	Intersezione arbitraria di topologie su $X$ è ancora una topologia su $X$.
\end{lm}

\begin{proof}
	Siano $\tau_i, i \in I$ topologie su $X$. Verifichiamo che $\displaystyle \tau=\bigcap_{i \in I} \tau_i$ soddisfi gli assiomi di topologia. \\
	$\emptyset, X \in \tau_i \, \forall i \in I \Rightarrow \emptyset, X \in \tau$. \\
	$A_1, A_2 \in \tau \Rightarrow A_1, A_2 \in \tau_i \, \forall i \in I \Rightarrow A_1 \cap A_2 \in \tau_i \, \forall i \in I \Rightarrow A_1 \cap A_2 \in \tau$. \\
	Siano $A_j, j \in J$ insiemi che stanno in $\tau$. \\
	$\displaystyle A_j \in \tau \, \forall j \in J \Rightarrow A_j \in \tau_i \, \forall i \in I, j \in J \Rightarrow \bigcup_{j \in J} A_j \in \tau_i \, \forall i \in I \Rightarrow \bigcup_{j \in J} A_j \in \tau$.
\end{proof}

\begin{cor}
	Data una famiglia $\tau_i, i \in I$ di topologie su $X$, esiste la più fine tra le topologie meno fini di ogni $\tau_i$: è $\displaystyle \bigcap_{i \in I} \tau_i$.
\end{cor}

\begin{cor}
	Sia $X$ un insieme, $S \subseteq \mathcal{P}(X)$, allora esiste la topologia meno fine tra quelle che contengono $S$. Tale topologia si dice \textit{generata} da $S$ e $S$ si dice \textsc{prebase} della topologia. Se $\Omega= \{ \tau \text{ topologia } | S \in \tau \}$ (che è non vuoto perché contiene almeno la topologia discreta), la topologia cercata è $\displaystyle \bigcap_{\tau \in \Omega} \tau$.
\end{cor}

\begin{defn}
	sia $(X, \tau)$ uno spazio topologico fissato, una \textsc{base} di $\tau$ è un insieme $\mathcal{B} \subseteq \tau$ t.c. $\forall A \in \tau, \exists B_i \in \mathcal{B}, i \in I$ t.c. $\displaystyle A= \bigcup_{i \in I} B_i$.
\end{defn}

\begin{ex}
	Se $X$ è uno spazio metrico, una base della topologia indotta sono le palle.
\end{ex}

\begin{defn} \label{N2}
	$(X, \tau)$ si dice \textit{a base numerabile} (o che soddisfa il \textsc{secondo assioma di numerabilità}) se ammette una base numerabile.
\end{defn}

\begin{prop} \label{base}
	Sia $X$ un insieme senza topologia, $\mathcal{B} \subseteq \mathcal{P}(X)$ è base di una topologia su $X$ $\Leftrightarrow$ valgono le seguenti: \\
	\begin{nlist}
		\item $\displaystyle X=\bigcup_{B \in \mathcal{B}} B$;
		\item $\forall A, A' \in \mathcal{B}, \exists B_i \in \mathcal{B}, i \in I$ t.c. $\displaystyle A \cap A'= \bigcup_{i \in I} B_i$.
	\end{nlist}
\end{prop}

\begin{proof}
	($\Rightarrow$) Ovviamente, se $\mathcal{B}$ è la base di una topologia su $X$, l'insieme $X$ deve essere unione di elementi di $\mathcal{B}$, inoltre tutti gli elementi di $\mathcal{B}$ devono essere sottoinsiemi di $X$, da cui discende (i). \\
	$A, A' \in \mathcal{B} \Rightarrow A, A' \in \tau \Rightarrow A \cap A' \in \tau$, per cui $A \cap A'$ deve poter essere esprimibile come unione di elementi di $\mathcal{B}$, che è l'affermazione (ii). \\
	($\Leftarrow$) Dobbiamo mostrare che l'insieme $\tau$ di tutte le possibili unioni di elementi di $\mathcal{B}$ soddisfa gli assiomi di topologia. \\
	Chiaramente $\emptyset \in \tau$ come unione di un insieme vuoto di elementi di $\mathcal{B}$ e $X \in \tau$ per (ii). \\
	Se faccio l'unione arbitraria di insiemi ottenuti come unione di elementi di $\mathcal{B}$ ottengo ovviamente un insieme che è unione di elementi di $\mathcal{B}$. \\
	Infine, $\displaystyle A, A' \in \tau \Rightarrow A=\bigcup_{i \in I} B_i, A'=\bigcup_{j \in J} B_j$ con $B_i, B_j \in \mathcal{B} \, \forall i \in I, j \in J$. Allora $\displaystyle A \cap A'= \left(\bigcup_{i \in I} B_i \right) \cap \left(\bigcup_{j \in J} B_j \right)=\bigcup_{i \in I, j \in J} (B_i \cap B_j)$, ma tutti i $B_i$ e $B_j$ stanno in $\mathcal{B}$, dunque per (ii) tutti i $B_i \cap B_j$ sono rappresentabili come unione di elementi di $\mathcal{B}$, perciò anche la loro unione, che è proprio $A \cap A'$, può essere scritta in quel modo e quindi sta in $\tau$.
\end{proof}

\begin{prop} \label{prop:preb}
	Siano $X$ un insieme e $S \subseteq \mathcal{P}(X)$ la prebase di una topologia $\tau$ su $X$. Allora:
	\begin{nlist}
		\item le intersezioni finite di elementi di $S \cup \{X\}$ sono una base di $\tau$;
		\item $A \in \tau$ $\Leftrightarrow$ $A$ è unione arbitraria di intersezioni finite di elementi di $S \cup \{X\}$.
	\end{nlist}
\end{prop}

\begin{proof}
	Sicuramente, poiché $\tau$ è generata da $S$, $S \subseteq \tau$ e quindi anche tutte le intersezioni finite di elementi di $S$ e le unioni arbitrarie di tali inersezioni devono stare in $\tau$. Se mostriamo che sono sufficienti a definire una topologia, abbiamo finito. \\
	Chiaramente il vuoto è l'intersezione di un insieme vuoto di insiemi e $X$ c'è perché lo abbiamo aggiunto a mano. \\
	Unione arbitraria di unioni arbitrarie di elementi di un insieme è ancora unione arbitraria di elementi di tale insieme. \\
	Siano ora $\displaystyle A_1= \bigcup_{i \in I} B_i, A_2=\bigcup_{j \in J} B_j$ con tutti i $B_i, B_j$ intersezioni finite di elementi di $S$. Allora $\displaystyle A_1 \cap A_2 = \left( \bigcup_{i \in I} B_i \right) \cap \left(\bigcup_{j \in J} B_j \right)= \bigcup_{i \in I, j \in J} (B_i \cap B_j)$, ma  intersezione di due intersezioni finite di elementi di $S$ è ancora un'intersezione finita di elementi di $S$, perciò $A_1 \cap A_2$ è ancora un'unione di intersezioni finite di elementi di $S$. Questo basta per dimostrare (i) e (ii) è una semplice riformulazione.
\end{proof}

\begin{exc}
	\warningsign\quad Il seguente esercizio è frutto di una domanda fatta da uno studente a lezione e potrebbe essere più difficile di altri esercizi del corso. \\
	Trovare un esempio (o dimostrare che non esiste) di uno spazio topologico $X$ e una funzione $f: (X, \tau) \rightarrow (X, \tau)$ continua e bigettiva con inversa non continua. \\
	Stando a quanto dice Frigerio, probabilmente tale funzione esiste, euristicamente perché non c'è un modo facile di dimostrare il contrario.
\end{exc}

\begin{prop}
	Sia $f: (X, \tau) \rightarrow (Y, \tau')$ e $S, \mathcal{B}$ rispettivamente una prebase e una base di $\tau'$. Allora sono equivalenti:
	\begin{nlist}
		\item $f$ è continua;
		\item $f^{-1}(A)$ è aperto per ogni $A \in S$;
		\item $f^{-1}(A)$ è aperto per ogni $A \in \mathcal{B}$.
	\end{nlist}
\end{prop}

\begin{proof}
	Poiché ogni base è una prebase, ((i) $\Leftrightarrow$ (ii)) $\Rightarrow$ ((i) $\Leftrightarrow$ (iii)). \\
	(i) $\Rightarrow$ (ii) è ovvia, perciò resta da dimostrare (ii) $\Rightarrow$ (i). \\
	Sia $A$ un aperto di $\tau'$. Per la proposizione \ref{prop:preb} $A$ si può scrivere come unione di intersezioni finite di elementi di $S$. Poiché la controimmagine di un'unione è l'unione delle controimmagini e la controimmagine di un'intersezione è l'intersezione delle controimmagini, la controimmagine di $A$ è unione di intersezioni finite di controimmagini di elementi di $S$, ma queste controimmagini sono aperte, perciò un'unione di loro intersezioni finite è ancora aperta, perciò $f^{-1}(A)$ è aperto in $X$ per ogni $A$ aperto in $Y$, e questo equivale a dire che $f$ è continua.
\end{proof}

\subsection{Assiomi di numerabilità}

Nella definizione \ref{N2} abbiamo stabilito quando uno spazio topologico soddisfa il secondo assioma di numerabilità. Vediamo gli altri due.

\begin{defn}
	$Y \subseteq X$ si dice \textsc{denso} se $\overline{Y}=X$.
\end{defn}

\begin{oss}
	$Y$ è denso $\Leftrightarrow$ $Y \cap A \not=\emptyset$ per ogni $A$ aperto non vuoto.
\end{oss}

\begin{defn} \label{N3}
	$X$ si dice \textsc{separabile} (o che soddisfa il \textsc{terzo assioma di numerabilità}) se ammette un sottoinsieme denso numerabile.
\end{defn}

\begin{prop} \label{N2impN3}
	Se $X$ è a base numerabile, $X$ è separabile.
\end{prop}

\begin{proof}
	Sia $\{ B_i, {i \in \mathbb{N}}\}$ la base numerabile di $X$ e scegliamo per ogni $i \in \mathbb{N}$ un elemento $x_i \in B_i$. Vogliamo mostrare che $\{ x_i, i \in \mathbb{N} \}$ è un sottoinsieme denso numerabile di $X$. \\
	Ovviamente è numerabile. Per dimostrare che è denso, mostriamo che interseca ogni aperto non vuoto. \\
	Sia dunque $A$ un aperto non vuoto di $X$, allora, dato che i $B_i$ formano una base, $A$ è esprimibile come unione di alcuni di essi, in particolare esiste $i_0$ t.c. $B_{i_0} \subseteq A$ e perciò $x_{i_0} \in B_{i_0} \Rightarrow x_{i_0} \in A$, dunque l'intersezione tra il nostro insieme e un aperto non vuoto non è mai vuota, come voluto.
\end{proof}

\begin{prop} \label{metr-num}
	Se $(X, \tau)$ è metrizzabile, $X$ è separabile $\Leftrightarrow$ è a base numerabile.
\end{prop}

\begin{proof}
	($\Rightarrow$) Mostriamo che per ogni aperto $A$ e per ogni $x \in A$, esiste una palla di centro un elemento del sottoinsieme denso numerabile e raggio un razionale positivo tutta contenuta in $A$. Allora tali palle formano una base numerabile. \\
	Sicuramente esiste una palla di centro $x$ e raggio $R \in \mathbb{R}^+$ tutta contenuta in $A$. Consideriamo la palla di centro $x$ e raggio $R/3$, che è contenuta nella prima. Questa palla è in particolare un insieme aperto, dunque ha un elemento $y$ in comune con il sottoinsieme denso numerabile. Scegliendo un razionale $R/3<r<2 \cdot R/3$ si ottiene che la palla di centro $y$ e raggio $r$ è tutta contenuta nella palla di centro $x$ e raggio $R$, dunque è tutta contenuta in $A$, e contiene $x$, come voluto. \\
	L'altra freccia discende dalla proposizione \ref{N2impN3}.
\end{proof}

\begin{defn}
	Sia $(X, \tau)$ uno spazio topologico fissato e $x_0 \in X$. Un insieme $U \subseteq X$ è un \textsc{intorno} di $x_0$ se $x_0 \in \stackrel{\circ}{U}$, o equivalentemente se esiste $V$ aperto con $x_0 \in V \subseteq U$. L'insieme degli intorni di $x_0$ si denota con $\mathcal{I}(x_0)$.
\end{defn}

\begin{defn}
	Un \textsc{sistema fondamentale di intorni} per $x_0$ è una famiglia $\mathcal{F} \subseteq \mathcal{I}(x_0)$ t.c. $\forall U \in \mathcal{I(x_0)} \, \exists V \in \mathcal{F}$ con $V \subseteq U$.
\end{defn}

\begin{ex} \label{1/n}
	Se $X$ è uno spazio metrico, le palle centrate in $x_0$ do raggio $1/n$ al variare di $n$ intero positivo sono un sistema fondamentale di intorni. In particolare, sono un sistema fondamentale di intorni numerabile.
\end{ex}

\begin{defn} \label{N1}
	$X$ soddisfa il \textsc{primo assioma di numerabilità} se ogni $x_0 \in X$ ha un sistema fondamentale di intorni numerabile.
\end{defn}

\begin{ftt}
	Se $X$ è metrizzabile $X$ soddisfa il primo assioma di numerabilità. Ciò discende direttamente dall'esempio \ref{1/n}.
\end{ftt}

\begin{prop}
	Il secondo assioma di numerabilità implica il primo e il terzo assioma di numerabilità. Discende dalla proposizione \ref{metr-num} che in spazi metrici vale anche che il terzo assioma di numerabilità implica il secondo assioma di numerabilità.
\end{prop}

\begin{proof}
	Supponiamo che $X$ soddisfi il secondo assioma di numerabilità. Per la proposizione \ref{N2impN3} otteniamo subito che soddisfa il terzo. Vogliamo adesso mostrare che soddisfa anche il primo. \\
	Sia $\mathcal{B}$ una base numerabile. Dato $x \in X$, consideriamo l'insieme $\{ B \in \mathcal{B} | x \in B \}$. Essendo un sottoinsieme di $\mathcal{B}$ è sicuramente numerabile e contiene solo insiemi aperti, cioè intorni di $x$. Mostriamo che è un sistema fondamentale di intorni per $x$. \\
	Consideriamo un intorno $U$ di $x$. Questo ha un sottoinsieme aperto $V \subseteq U$ contenente $x$, dunque $V$ è esprimibile come unione di elementi di $\mathcal{B}$ e ce ne dev'essere uno che contiene $x$, che sta quindi nel nostro insieme ed è contenuto in $V$ e quindi in $U$. Dunque, per ogni intorno $U$ di $x$ troviamo un elemento del nostro insieme di intorni di $x$ contenuto in $U$, che è quello che dovevamo dimostrare.
\end{proof}

\begin{prop}
	\begin{nlist}
		\item $A \subseteq X$ è aperto se e solo se è intorno di ogni suo punto;
		\item sia $C \subseteq X$ un insieme generico, allora $x \in \overline{C}$ se e solo se ogni intorno di $X$ interseca $C$.
	\end{nlist}
\end{prop}

\begin{proof}
	Inserire dimostrazione.
\end{proof}

\begin{ex}
	La retta di Sorgenfrey. Su $X=\mathbb{R}$ consideriamo il seguente insieme: $\mathcal{B}=\{[a, b) | a<b\}$. Valgono le seguenti:
	\begin{nlist}
		\item $\mathcal{B}$ è base di una toplogia $\tau$;
		\item $\tau$ fine della topologia euclidea;
		\item $\tau$ è separabile;
		\item $\tau$ non è a base numerabile;
		\item $\tau$ non è metrizzabile.
	\end{nlist}
\end{ex}

\begin{proof}
	Inserire dimostrazione.
\end{proof}

\end{document}
