\documentclass{article}
\usepackage[italian]{babel}
\usepackage[T1]{fontenc}
\usepackage[utf8]{inputenc}
\usepackage{amsmath}
\usepackage{amsthm}
\usepackage{amssymb}
\usepackage{pgf,tikz}
\usepackage{mstyle}
\usepackage{hyperref}

\usetikzlibrary{arrows}

\title{Appunti di Geometria 2 \\ anno accademico 2019/2020}
\date{}
\author{Marco Vergamini \& Alessio Marchetti}

\begin{document}
\maketitle
\newpage
\tableofcontents
\newpage

\section{Introduzione}
Questi appunti sono basati sul corso di Geometria 2 tenuto dai professori
Roberto Frigerio e Jacopo Gandini nell'anno accademico 2019/2020. Sono dati per
buoni i contenuti dei corsi del primo anno, in particolare Analisi 1 e Geometria
1. Verranno omesse o soltanto hintate le dimostrazioni più semplici, ma si
consiglia comunque di provare a svolgerle per conto proprio. Ogni tanto sarà
commesso qualche abuso di notazione, facendo comunque in modo che il significato
sia reso chiaro dal contesto. Inoltre, la notazione verrà alleggerita man mano,
per evitare inutili ripetizioni e appesantimenti nella lettura.


\section{Spazi metrici e spazi topologici}
\subsection{Spazi metrici}
\begin{defn}
	Uno \textsc{spazio metrico} è una coppia $(X, d)$, $X$ insieme,
	${d: X \times X \rightarrow \mathbb{R}}$ t.c. per ogni ${x, y, z \in X}$
    \begin{nlist}
	\item $d(x, y) \ge 0$ e $d(x, y)=0 \Leftrightarrow x=y$;
	\item $d(x, y)=d(y, x)$;
	\item $d(x, z) \le d(x, y)+d(y,z)$ (disuguaglianza triangolare).
    \end{nlist}
	In tal caso $d$ si dice \textsc{distanza} o \textsc{metrica}.
\end{defn}

\begin{ex}
Ecco alcuni esempi di distanze:
\begin{itemize}
\item La distanza $d_1$ su $\mathbb{R}^n$, che alla coppia di elementi
${x=(x_1, \dots, x_n)}, {y=(y_1, \dots, y_n)}$
 associa il numero
 $${\displaystyle d_1(x, y)=\sum_{i=1}^n |x_i-y_i|};$$
\item la distanza $d_2$ (o $d_E$, distanza
euclidea) su $\mathbb{R}^n$,
$$\displaystyle d_2(x, y)=\sqrt{\sum_{i=1}^n (x_i-y_i)^2};$$
\item la distanza $d_{\infty}$ su $\mathbb{R}^n$,
${\displaystyle d_{\infty}(x, y)=\sup_{i=1, \dots, n} \{ |x_i-y_i| \}}$;
\item
la distanza discreta su un generico insieme $x$, $d(x, y)=1$ se $x \not= y$ e
$0$ altrimenti;
\item le distanze $d_1$, $d_2$, $d_{\infty}$ sullo spazio delle funzioni
continue da $[0, 1]$ in $\mathbb{R}$, definite rispettivamente
$${\displaystyle d_1(f, g)=\int_0^1 |f(t)-g(t)| dt},$$
$${\displaystyle d_2(f, g)=\sqrt{\int_0^1 (f(t)-g(t))^2 dt}},$$
$${\displaystyle d_{\infty}(f, g)=\sup_{t \in [0, 1]} |f(t)-g(t)|}.$$
\end{itemize}
\end{ex}

\begin{defn}
	$f:(X, d) \rightarrow (Y, d')$ viene detto \textsc{embedding isometrico} se
	$d'(f(x_1), f(x_2))=d(x_1, x_2)$ per ogni $x_1, x_2 \in X$.
\end{defn}

\begin{oss}
Valgono i seguenti fatti:
\begin{itemize}
\item l'identità è un embedding isometrico;
\item composizione di embedding isometrici è un embedding isometrico;
\item se un embedding isometrico $f$ è biettivo, anche $f^{-1}$ è un embedding
isometrico e $f$ si dice \textsc{isometria};
\item un embedding isometrico è sempre iniettivo, dunque è un'isometria se e
solo se è suriettivo;
\item se $(X, d)$ è fissato, l'insieme delle isometrie da $X$ in sé è un gruppo
con la composizione, chiamato $Isom(X,d)$.
\end{itemize}
\end{oss}



\subsection{Continuità in spazi metrici}
\begin{defn}
Dati $p \in X, R>0$, ${B(p, R):=\{ x \in X \mid d(p, x)<R\}}$ è detta la
\textsc{palla aperta} di centro $p$ e raggio $R$.
\end{defn}

\begin{defn}
${f:(X, d) \rightarrow (Y, d')}$ è \textsc{continua in $x_0$} se
${\forall \epsilon > 0 \; \exists \; \delta >0}\\\ \tc {f(B(x_0, \delta))
\subseteq B(f(x_0), \epsilon)}$, cioè ${f^{-1}(B(f(x_0), \epsilon)) \supseteq
B(x_0, \delta)}$.
\end{defn}

\begin{defn}
${f(X, d) \rightarrow (Y, d')}$ è detta \textsc{continua} se è continua in ogni
${x_0 \in X}$.
\end{defn}

\begin{oss}
Gli embedding isometrici sono continui.
\end{oss}

\begin{defn}
Sia $X$ uno spazio metrico, $A \subseteq X$ è detto \textsc{aperto} se per ogni
$x \in A$ esiste ${R>0 \tc B(x, R) \subseteq A}$.
\end{defn}

\begin{ftt}
Le palle aperte sono aperti. Per dimostrarlo, sfruttare la disuguaglianza
triangolare.
\end{ftt}

\begin{thm} \label{thm:cont_inv}
$f(X, d) \rightarrow (Y, d')$ è continua se e solo se per ogni aperto $A$ di $Y$
$f^{-1}(A)$ è aperto di $X$.
\end{thm}

\begin{proof}
Supponiamo $f$ continua. Prendiamo $x \in f^{-1}(A)$, allora si ha che $f(x) \in
A$, ma dato che $A$ è aperto esiste una palla aperta di centro $f(x)$ $B_Y$ t.c.
$B_Y \subseteq A$, da cui $f^{-1}(B_Y) \subseteq f^{-1}(A)$. Usando la
definizione di continuità, detto $\epsilon$ il raggio di $B_Y$, scegliendo il
$\delta$ corrispondente come raggio di una palla di centro $x$, sia essa $B_X$,
si ha che $B_X \subseteq f^{-1}(B_Y) \subseteq f^{-1}(A)$, perciò abbiamo
trovato una palla centrata in $x$ tutta contenuta in $f^{-1}(A)$ e questo prova
che è un aperto.

Viceversa, supponiamo che le controimmagini di aperti siano a loro volta aperti.
Dati $x_0 \in X$ e $\epsilon >0$ si ha che $B(f(x_0), \epsilon)$ è un aperto di
$Y$, perciò la sua controimmagine è un aperto, ma allora per definizione di
aperto esiste $\delta>0$ tale che $B(x_0, \delta) \subseteq f^{-1}(B(f(x_0),
\epsilon))$ e questo prova che $f$ è continua.
\end{proof}

\begin{oss}
La continuità di una funzione dipende quindi solo indirettamente dalla metrica,
mentre è direttamente collegata agli aperti generati dalla metrica stessa. Segue
facilmente che due metriche che generano gli stessi aperti portano anche alla
stessa famiglia di funzioni continue. Diamo dunque la seguente definizione.
\end{oss}

\begin{defn}
Due distanze $d, d'$ su un insieme $X$ si dicono \textsc{topologicamente
equivalenti} se inducono la stessa famiglia di aperti.
\end{defn}

\begin{lm}
Siano $d, d'$ distanze su $X$ t.c. esiste $k \ge 1$ t.c. per ogni $x, y \in X$
valga $d(x, y)/k \le d'(x, y) \le k \cdot d(x, y)$. Allora $d$ e $d'$ sono
topologicamente equivalenti.
\end{lm}

\begin{proof}
Sia $A$ un aperto indotto da $d'$ e $x_0 \in A$. Per definizione di aperto
esiste $R>0$ tale che $B_{d'}(x_0, R) \subseteq A$. Considero $B_d(x_0, R/k)$.
Dato un elemento $x \in B_d(x_0, R/k)$ ho che $d'(x_0, x) \le k \cdot d(x_0,
x)<k \cdot R/k=R$, dunque $x \in B_{d'}(x_0, R)$. Allora $B_d(x_0, R/k)
\subseteq B_{d'}(x_0, R/k) \subseteq A$ e dunque $A$ è anche un aperto indotto
da $d$. Per la simmetria delle disuguaglianze nelle ipotesi si dimostra anche
l'opposto, perciò gli aperti di $d$ e $d'$ sono gli stessi, come voluto.
\end{proof}

\begin{cor}
$d_1, d_2, d_{\infty}$ sono topologicamente equivalenti su $\mathbb{R}^n$.
\end{cor}

\begin{center}
\pagestyle{empty}
\begin{tikzpicture}[line cap=round,line join=round,
    >=triangle 45,x=2.0cm,y=2.0cm]
    \draw[->,color=black] (-1.72,0) -- (1.78,0);
    \foreach \x in {-1,1}
    \draw[shift={(\x,0)},color=black] (0pt,2pt) -- (0pt,-2pt);
    \draw[->,color=black] (0,-1.4) -- (0,1.44);
    \foreach \y in {-1,1}
    \draw[shift={(0,\y)},color=black] (2pt,0pt) -- (-2pt,0pt);
    \clip(-1.72,-1.2) rectangle (1.78,1.24);
    \draw(0,0) circle (2cm);
    \draw (-1,0)-- (0,1);
    \draw (0,1)-- (1,0);
    \draw (1,0)-- (0,-1);
    \draw (0,-1)-- (-1,0);
    \draw (-1,1)-- (1,1);
    \draw (1,1)-- (1,-1);
    \draw (1,-1)-- (-1,-1);
    \draw (-1,1)-- (-1,-1);
\end{tikzpicture}

Nell'immagine sono rappresentate, nell'ordine dalla più interna alla più
esterna, le palle aperte centrate nell'origine e di raggio $1$ rispettivamente
nelle metriche $d_1, d_2, d_{\infty}$.
\end{center}

\begin{proof}
Per AM-QM si ha che
$$\displaystyle \frac{\sum_{i=1}^n |x_i-y_i|}{n} \le \sqrt{\frac{\sum_{i=1}^n
(x_i-y_i)^2}{n}},$$
da cui $d_1(x, y) \le \sqrt{n} \cdot d_2(x, y)$. Stimando tutti i termini con il
massimo otteniamo che
$$\displaystyle \sqrt{\sum_{i=1}^n (x_i-y_i)^2} \le
\sqrt{\sum_{i=1}^n \sup_{j=1, \dots, n} \{(x_j-y_j)^2\}}=\sqrt{n} \cdot
\sup_{j=1, \dots, n} \{ |x_j-y_j| \},$$
da cui
$d_2(x, y) \le \sqrt{n} \cdot d_{\infty} (x, y).$
 Infine è ovvio che
 $$\displaystyle \sup_{i=1, \dots, n} |x_i-y_i| \le
 \sum_{i=1}^n |x_i-y_i|$$, da cui $d_{\infty}(x, y) \le d_1(x, y).$
Scegliendo $k=\sqrt{n}$ è ora sufficiente applicare il lemma.
\end{proof}

Nella dimostrazione del corollario era importante che lo spazio fosse di
dimensione finita. Nello spazio delle funzioni continue da $[0, 1]$ a
$\mathbb{R}$ le tre distanze non sono topologicamente equivalenti.

\subsection{Spazi topologici}

\begin{defn}
Uno \textsc{spazio topologico} è una coppia $(X, \tau),\; {\tau \subseteq
\mathcal{P}(X)}$, t.c.
\begin{nlist}
\item $\emptyset,\ X \in \tau$;
\item $A_1,\ A_2 \in \tau \Rightarrow A_1 \cap A_2 \in \tau$;
\item se $I$ è un insieme e $A_i \in \tau \, \forall i \in I$, allora
$\displaystyle \bigcup_{i \in I} A_i \in \tau$.
\end{nlist}
Allora $\tau$ si dice \textsc{topologia} di $X$ e gli elementi della topologia
sono detti \textsc{aperti} di $\tau$.
\end{defn}

\begin{prop}
Se $(X, d)$ è uno spazio metrico, gli aperti rispetto a $d$ definiscono una
topologia.
\end{prop}

\begin{defn}
Uno spazio topologico $(X, \tau)$ è detto \textsc{metrizzabile} se $\tau$ è
indotta da una distanza su $X$.
\end{defn}

\begin{defn}
Sia $(X, \tau)$ uno spazio topologico, $C \subseteq X$ è \textsc{chiuso} se $X
\setminus C$ è aperto.
\end{defn}

\begin{oss}
\begin{nlist}
\item possono esistere insiemi né aperti né chiusi;
\item $\emptyset$ e $X$ sono chiusi;
\item unione finita di chiusi è chiusa;
\item intersezione arbitraria di chiusi è chiusa.
\end{nlist}
\end{oss}

\begin{ex}
Ecco alcuni esempi di spazi topologici:
\begin{itemize}
\item tutte le topologie indotte da una metrica;
\item la topologia discreta, cioè $\tau=\mathcal{P}(X)$, indotta dalla distanza
discreta;
\item la topologia indiscreta, cioè $\tau=\{ \emptyset, X \}$;
\item la topologia cofinita, dove gli aperti sono l'insieme vuoto più tutti e
soli gli insiemi il cui complementare è un insieme finito.
\end{itemize}
\end{ex}

\subsection{Continuità in spazi topologici}

\begin{defn}
$f: (X, \tau) \rightarrow (Y, \tau')$ si dice \textsc{continua} se ${f^{-1}(A)
\in \tau,}\; {\forall A \in \tau'}$, cioè una funzione \`e continua se la
controimmagine di aperti \`e aperta.
\end{defn}

Notiamo che per il teorema \ref{thm:cont_inv}, le due
definizioni di funzione continua sono equivalenti per uno spazio metrico se si
considera la topologia indotta dalla metrica.

\begin{thm}
\begin{nlist}
\item L'identità è una funzione continua;
\item composizione di funzioni continue è una funzione continua.
\end{nlist}
\end{thm}

\begin{defn}
Una funzione $f: X \rightarrow Y$ si dice \textsc{omeomorfismo} se $f$ è
continua e esiste una funzione $g:Y\rightarrow X$ continua tale che $f \circ g =
Id_Y$ e $g \circ f = Id_x$. Cio\`e $f$ \`e continua, bigettiva e con inversa
continua.
\end{defn}

\begin{oss}
	\begin{nlist}
	\item Composizione di omeomorfismi è un omeomerfismo; due spazi legati da un
	omeomorfismo si dicomo \textsc{omeomorfi} e essere omeomorfi è una relazione
	di equivalenza;
	\item l'insieme degli omeomorfismi da $(X, \tau)$ in sé è un gruppo;
	\item se $f:X \rightarrow Y$ è continua e bigettiva non è detto che sia un
	omeomorfismo, cioè $f^{-1}$ può non essere continua.
    \marginpar{\warningsign}
\end{nlist}
\end{oss}

\begin{ex}
	Siano $\tau_E$ la topologia euclidea, $\tau_C$ la cofinita, $\tau_D$ la
	discreta e $\tau_I$ l'indiscreta. Le seguenti mappe sono dunque continue:
    \begin{itemize}
	\item $Id:(\mathbb{R}, \tau_D) \rightarrow (\mathbb{R}, \tau_E)$,
    \item $Id:(\mathbb{R}, \tau_E) \rightarrow (\mathbb{R}, \tau_C)$,
    \item $Id:(\mathbb{R}, \tau_C) \rightarrow (\mathbb{R}, \tau_I)$.
    \end{itemize}
    Nessuna delle inverse è però continua. Più in generale, $Id: (X, \tau)
    \rightarrow (X, \sigma)$ con $\sigma \subsetneq \tau$ è continua ma
    l'inversa no.
\end{ex}

\subsection{Ordinamento fra topologie, basi e prebasi}

Vogliamo mettere un ordinamento parziale sulle topologie di un certo insieme
fissato.

\begin{defn}
Dato un insieme $X$ su cui sono definite le topologie $\tau$ e $\tau'$, si dice
che $\tau$ \`e \textsc{meno fine} di $\tau'$ se $\tau \subseteq \tau'$, cioè
ogni aperto di $\tau$ è anche aperto di $\tau'$. $\tau'$ si dice \textsc{più
fine} di $\tau$.
\end{defn}

\begin{oss}
Equivalentemente alla definizione sopra, si pu\`o dire che $\tau$ \`e meno fine
di $\tau'$ se e solo se ${Id:(X,\tau')\rightarrow(X, \tau)}$ \`e continua.
\end{oss}

Quando $\tau$ è meno fine di $\tau'$ scriveremo $\tau < \tau'$. Si noti che
dalla definizione ogni topologia è meno fine di se stessa, cioè $\tau < \tau'
\; \forall \tau$.

\begin{ex}
	$\tau_I < \tau_C < \tau_E < \tau_D$, \, $\tau_I < \tau < \tau_D \; \forall
	\tau$.
\end{ex}

\begin{lm}
	Intersezione arbitraria di topologie su $X$ è ancora una topologia su $X$.
\end{lm}

\begin{proof}
Siano $\tau_i,\ i \in I$ topologie su $X$.
Verifichiamo che $ \tau=\bigcap_{i \in I} \tau_i$ soddisfa gli assiomi di
topologia.

\begin{nlist}
\item $\emptyset, X \in \tau_i \, \forall i \in I \Rightarrow \emptyset, X \in
\tau$.
\item $A_1, A_2 \in \tau \Rightarrow A_1, A_2 \in \tau_i \, \forall i
\in I \Rightarrow A_1 \cap A_2 \in \tau_i \, \forall i \in I \Rightarrow A_1
\cap A_2 \in \tau$.
\item Siano $A_j, j \in J$ insiemi che stanno in $\tau$. \\
$\displaystyle A_j \in \tau \, \forall j \in J \Rightarrow A_j \in \tau_i \,
\forall i \in I, j \in J \Rightarrow {\bigcup_{j \in J} A_j \in \tau_i \,
\forall i \in I} \Rightarrow {\bigcup_{j \in J} A_j \in \tau}$.
\end{nlist}
\end{proof}

\begin{cor}
	Data una famiglia $\tau_i, i \in I$ di topologie su $X$, esiste la più fine
	tra le topologie meno fini di ogni $\tau_i$: è $\displaystyle \bigcap_{i \in
	I} \tau_i$.
\end{cor}

\begin{cor}
	Sia $X$ un insieme, $S \subseteq \mathcal{P}(X)$, allora esiste la topologia
	meno fine tra quelle che contengono $S$. Tale topologia si dice
	\textit{generata} da $S$ e $S$ si dice \textsc{prebase} della topologia. Se
	$\Omega= \{ \tau \text{ topologia } |\; S \in \tau \}$ (che è non vuoto
	perché contiene almeno la topologia discreta), la topologia cercata è
	$\displaystyle \bigcap_{\tau \in \Omega} \tau$.
\end{cor}

\begin{defn}
	sia $(X, \tau)$ uno spazio topologico fissato, una \textsc{base} di $\tau$ è
	un insieme $\mathcal{B} \subseteq \tau$ t.c. $\forall A \in \tau, \exists
	B_i \in \mathcal{B}, i \in I$ t.c. $A= \bigcup_{i \in I} B_i$. Ovvero
	$\mathcal{B}$ \`e una base se ogni aperto di $\tau$ pu\`o essere scritto
	come unione qualunque di elementi di $\mathcal{B}$.
\end{defn}

\begin{ex}
	Se $X$ è uno spazio metrico, una base della topologia indotta sono le palle.
\end{ex}

\begin{defn}
	$(X, \tau)$ si dice \textit{a base numerabile} (o che soddisfa il
	\textsc{secondo assioma di numerabilità}) se ammette una base numerabile.
\end{defn}

\begin{prop}
	Sia $X$ un insieme senza topologia, $\mathcal{B} \subseteq \mathcal{P}(X)$ è
	base di una topologia su $X$ $\Leftrightarrow$ valgono le seguenti:
	\begin{nlist}
		\item \label{i_prop} $\displaystyle X=\bigcup_{B \in \mathcal{B}} B$;
		\item $\forall A, A' \in \mathcal{B}, \exists B_i \in \mathcal{B}, i \in
		I$ t.c. $A \cap A'= \bigcup_{i \in I} B_i$. Cio\`e ogni intersezione di
		una coppia di elementi di $\mathcal{B}$ pu\`o essere scritta come unione
		di elementi di $\mathcal{B}$.
	\end{nlist}
\end{prop}

\begin{proof} \label{base_car}
Sia $\tau$ l'insieme di tutte le possibili unioni di elementi di $\mathcal{B}$.
Si vuole dimostrare che che $\tau$ \`e una topologia su $X$.
\begin{nlist}
\item L'unione di zero elementi \`e l'insieme vuoto, dunque $\emptyset  \in
\tau$. Inoltre \ref{i_prop} dice che $X \in \tau$.
\item Per associativit\`a dell'unione insiemistica, l'unione di unioni di aperti
di $\mathcal{B}$ appartiene ancora a $\tau$. Pi\`u precisamente, siano $A_i,\ i
\in I$ elementi di $\tau$. Allora, per ogni $i$, $A_i = \bigcup_{j \in J_i}
B_j$, con i $B_j \in \mathcal{B}$. Si pu\`o anche scrivere
$$
\bigcup_{i \in I}A_i = \bigcup_{i \in I}\bigcup_{j \in J_i}B_j =
\bigcup_{j \in J}B_j
$$
dove $J = \bigcup_{i \in I}J_i$.
\item Dall'ipotesi (ii) discende direttamente il fatto che $\tau$ \`e chiuso per
intersezione finita, e questo conclude la dimostrazione.
\end{nlist}
\end{proof}

\begin{prop}
	Siano $X$ un insieme e $S \subseteq \mathcal{P}(X)$ la prebase di una
	topologia $\tau$ su $X$. Allora:
	\begin{nlist}
		\item le intersezioni finite di elementi di $S \cup \{X\}$ sono una base
		di $\tau$;
		\item $A \in \tau$ $\Leftrightarrow$ $A$ è unione arbitraria di
		intersezioni finite di elementi di $S \cup \{X\}$.
	\end{nlist}
\end{prop}

\begin{proof}
La parte (i) $\Leftrightarrow$ (ii) viene dalla definizione di base.

Osservando la proposizione \ref{base_car}, si vede che l'insieme delle
intersezioni finite degli elementi $S \cup {X}$ \`e una base di una topologia.
Per dire che tale topologia \`e proprio $\tau$ basta mostrare che comunque le
intersezioni finite di $S$ devono appartenere a tutte le topologie che
contengono $S$, e quindi anche alla loro intersezione. Lo stesso vale per
l'unione.
\end{proof}

\end{document}
