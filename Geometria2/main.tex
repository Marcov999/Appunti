\documentclass{article}
\usepackage{mstyle}

\title{Appunti di Geometria 2 \\ anno accademico 2019/2020}
\date{}
\author{Marco Vergamini \and Alessio Marchetti \and Luigi Traino}

\begin{document}
\maketitle
\newpage
\tableofcontents
\newpage


\section{Introduzione}
Questi appunti sono basati sul corso di Geometria 2 tenuto dai professori
Roberto Frigerio e Jacopo Gandini nell'anno accademico 2019/2020. Sono dati per
buoni i contenuti dei corsi del primo anno, in particolare Analisi 1 e Geometria
1. Come corequisiti si richiedono nozioni insegnate nei corsi di Analisi 2 e Algebra 1. Verranno omesse o soltanto hintate le dimostrazioni più semplici, ma si consiglia comunque di provare a svolgerle per conto proprio. Ogni tanto sarà
commesso qualche abuso di notazione, facendo comunque in modo che il significato
sia reso chiaro dal contesto. Inoltre, la notazione verrà alleggerita man mano,
per evitare inutili ripetizioni e appesantimenti nella lettura. Si ricorda anche
che questi appunti sono scritti a più mani, non sempre subito dopo le lezioni,
non sempre con appunti completi, ecc\dots. Spesso saranno rivisti, verranno
aggiunte cose che mancavano perché c'era poco tempo (o voglia\dots), potrebbero
mancare argomenti più o meno marginali\dots insomma, non è un libro di testo per
il corso, ma vuole essere un valido supporto per aiutare gli studenti che
seguono il corso. Speriamo di essere riusciti in questo intento.



\section{Spazi metrici e spazi topologici}

\subsection{Spazi metrici}
\begin{defn}
	Uno \textsc{spazio metrico} è una coppia $(X, d)$, $X$ insieme,
	${d: X \times X \rightarrow \mathbb{R}}$ t.c. per ogni ${x, y, z \in X}$
    \begin{nlist}
    	\item $d(x, y) \ge 0$ e $d(x, y)=0 \Leftrightarrow x=y$;
    	\item $d(x, y)=d(y, x)$;
    	\item $d(x, z) \le d(x, y)+d(y,z)$ (disuguaglianza triangolare).
    \end{nlist}
	In tal caso $d$ si dice \textsc{distanza} o \textsc{metrica}.
\end{defn}

\begin{ex}
    Ecco alcuni esempi di distanze:
    \begin{itemize}
        \item La distanza $d_1$ su $\mathbb{R}^n$, che alla coppia di elementi
        ${x=(x_1, \dots, x_n)}, {y=(y_1, \dots, y_n)}$ associa il numero
        $${\displaystyle d_1(x, y)=\sum_{i=1}^n |x_i-y_i|};$$
        \item la distanza $d_2$ (o $d_E$, distanza euclidea) su $\mathbb{R}^n$,
        $$\displaystyle d_2(x, y)=\sqrt{\sum_{i=1}^n (x_i-y_i)^2};$$
        \item la distanza $d_{\infty}$ su $\mathbb{R}^n$, ${\displaystyle
        d_{\infty}(x, y)=\sup_{i=1, \dots, n} \{ |x_i-y_i| \}}$;
        \item
        la distanza discreta su un generico insieme $x$, $d(x, y)=1$ se $x \not=
        y$ e $0$ altrimenti;
        \item le distanze $d_1$, $d_2$, $d_{\infty}$ sullo spazio delle funzioni
        continue da $[0, 1]$ in $\mathbb{R}$, definite rispettivamente
        $${\displaystyle d_1(f, g)=\int_0^1 |f(t)-g(t)| dt},$$
        $${\displaystyle d_2(f, g)=\sqrt{\int_0^1 (f(t)-g(t))^2 dt}},$$
        $${\displaystyle d_{\infty}(f, g)=\sup_{t \in [0, 1]} |f(t)-g(t)|}.$$
    \end{itemize}
\end{ex}

\begin{defn}
	$f:(X, d) \rightarrow (Y, d')$ viene detto \textsc{embedding isometrico} se
	$d'(f(x_1), f(x_2))=d(x_1, x_2)$ per ogni $x_1, x_2 \in X$.
\end{defn}

\begin{oss}
    Valgono i seguenti fatti:
    \begin{itemize}
        \item l'identità è un embedding isometrico;
        \item composizione di embedding isometrici è un embedding isometrico;
        \item se un embedding isometrico $f$ è biettivo, anche $f^{-1}$ è un
        embedding isometrico e $f$ si dice \textsc{isometria};
        \item un embedding isometrico è sempre iniettivo, dunque è un'isometria
        se e solo se è suriettivo;
        \item se $(X, d)$ è fissato, l'insieme delle isometrie da $X$ in sé è un
        gruppo con la composizione, chiamato $\Isom (X,d)$.
    \end{itemize}
\end{oss}

\begin{defn}
    Sia $V$ uno spazio vettoriale. Una norma su $V$ \`e una funzione ${\norm{\cdot}\colon V \longrightarrow [0, \infty)}$ che rispetta le seguenti propriet\`a, per ogni $x, y \in V$, $\lambda \in \mathbb{R}$:
    \begin{nlist}
        \item $\norm{x} = 0$ se e solo se $x=0$;
        \item $\norm{\lambda x} = |\lambda|\norm{x}$;
        \item $\norm{x+y} \leq \norm{x}+\norm{y}$.
    \end{nlist}
    Ogni norma induce una distanza $d$ sullo spazio definita come ${d(x,y)=\norm{x-y}}$.
\end{defn}

\begin{thm}
Le norme su uno spazio vettoriale reale di dimensione finita inducono distanze topologicamente equivalenti.
\end{thm}
\begin{proof}
    Sia $n$ la dimensione dello spazio. Poich\`e lo ogni spazio reale di tale dimensione \`e isomorfo a $\mathbb{R}^n$, assumo senza perdita di generalit\`a che $V=\mathbb{R}^n$. Sia allora $\{e_1, \dots, e_n\}$ la base canonica. Sia inoltre $M = \max_{i = 1, \dots, n}\norm{e_i}$. Si ha dunque che
    \[
        \norm{x}-\norm{y} \leq \norm{x-y} \leq M \sum_i |x_i-y_i| = M d_1(x,y)
    \]
    Ho ottenuto allora che $\norm{\cdot}$ \`e Lipschitziana nella distanza $d_1$, e dunque \`e anche continua. Poich\`e $d_1$ e $d_2$ sono equivalenti, la norma \`e continua anche con $d_2$. Allora ammette massimo e minimo sui compatti, e in particolare sulla sfera unitaria. Siano essi rispettivamente $M$ e $m$. Per ogni vettore $v$ dello spazio, vale allora
    \[
        \norm{v} = \norm{\norm{v}_2 \frac{v}{\norm{v}_2}} = \norm{v}_2 \norm{\frac{v}{\norm{v}_2}}
    \]
    che porta a
    \[
        m\norm{v}_2 \leq \norm{v} \leq M\norm{v}_2
    \]
    Passando alle distanze indotte si ottiene l'equivalenza desiderata.
\end{proof}


\subsection{Continuità in spazi metrici}
\begin{defn}
    Dati $p \in X, R>0$, ${B(p, R):=\{ x \in X \mid d(p, x)<R\}}$ è detta la
    \textsc{palla aperta} di centro $p$ e raggio $R$.
\end{defn}

\begin{defn}
    ${f:(X, d) \rightarrow (Y, d')}$ è \textit{continua in $x_0$} se ${\forall
    \epsilon > 0 \; \exists \; \delta >0}\\\ \tc {f(B(x_0, \delta)) \subseteq
    B(f(x_0), \epsilon)}$, cioè ${f^{-1}(B(f(x_0), \epsilon)) \supseteq B(x_0,
    \delta)}$.
\end{defn}

\begin{defn}
    ${f(X, d) \rightarrow (Y, d')}$ è detta \textsc{continua} se è continua in
    ogni ${x_0 \in X}$.
\end{defn}

\begin{oss}
    Gli embedding isometrici sono continui.
\end{oss}

\begin{defn}
    Sia $X$ uno spazio metrico, $A \subseteq X$ è detto \textsc{aperto} se per
    ogni $x \in A$ esiste ${R>0 \tc B(x, R) \subseteq A}$.
\end{defn}

\begin{ftt}
    Le palle aperte sono aperti. Per dimostrarlo, sfruttare la disuguaglianza
    triangolare.
\end{ftt}

\begin{thm} \label{thm:cont_inv}
    $f(X, d) \rightarrow (Y, d')$ è continua se e solo se per ogni aperto $A$ di
    $Y$ $f^{-1}(A)$ è aperto di $X$.
\end{thm}

\begin{proof}
    Supponiamo $f$ continua. Prendiamo $x \in f^{-1}(A)$, allora si ha che $f(x)
    \in A$, ma dato che $A$ è aperto esiste una palla aperta $B_Y$ di centro
    $f(x)$
     t.c. $B_Y \subseteq A$, da cui $f^{-1}(B_Y) \subseteq f^{-1}(A)$.
    Usando la definizione di continuità, detto $\epsilon$ il raggio di $B_Y$,
    scegliendo il $\delta$ corrispondente come raggio di una palla di centro
    $x$, sia essa $B_X$, si ha che $B_X \subseteq f^{-1}(B_Y) \subseteq
    f^{-1}(A)$, perciò abbiamo trovato una palla centrata in $x$ tutta contenuta
    in $f^{-1}(A)$ e questo prova che è un aperto.

    Viceversa, supponiamo che le controimmagini di aperti siano a loro volta
    aperti. Dati $x_0 \in X$ e $\epsilon >0$ si ha che $B(f(x_0), \epsilon)$ è
    un aperto di $Y$, perciò la sua controimmagine è un aperto, ma allora per
    definizione di aperto esiste $\delta>0$ tale che $B(x_0, \delta) \subseteq
    f^{-1}(B(f(x_0), \epsilon))$ e questo prova che $f$ è continua.
\end{proof}

\begin{oss}
    La continuità di una funzione dipende quindi solo indirettamente dalla
    metrica, mentre è direttamente collegata agli aperti generati dalla metrica
    stessa. Segue facilmente che due metriche che generano gli stessi aperti
    portano anche alla stessa famiglia di funzioni continue. Diamo dunque la
    seguente definizione.
\end{oss}

\begin{defn}
    Due distanze $d, d'$ su un insieme $X$ si dicono \textsc{topologicamente
    equivalenti} se inducono la stessa famiglia di aperti.
\end{defn}

\begin{lm}
    Siano $d, d'$ distanze su $X$ t.c. esiste $k \ge 1$ t.c. per ogni $x, y \in
    X$ valga $d(x, y)/k \le d'(x, y) \le k \cdot d(x, y)$. Allora $d$ e $d'$
    sono topologicamente equivalenti.
\end{lm}

\begin{proof}
    Sia $A$ un aperto indotto da $d'$ e $x_0 \in A$. Per definizione di aperto
    esiste $R>0$ tale che $B_{d'}(x_0, R) \subseteq A$. Considero $B_d(x_0,
    R/k)$. Dato un elemento $x \in B_d(x_0, R/k)$ ho che $d'(x_0, x) \le k \cdot
    d(x_0, x)<k \cdot R/k=R$, dunque $x \in B_{d'}(x_0, R)$. Allora $B_d(x_0,
    R/k) \subseteq B_{d'}(x_0, R/k) \subseteq A$ e dunque $A$ è anche un aperto
    indotto da $d$. Per la simmetria delle disuguaglianze nelle ipotesi si
    dimostra anche l'opposto, perciò gli aperti di $d$ e $d'$ sono gli stessi,
    come voluto.
\end{proof}

\begin{cor}
    $d_1, d_2, d_{\infty}$ sono topologicamente equivalenti su $\mathbb{R}^n$.
\end{cor}

\begin{center}
\pagestyle{empty}
\begin{tikzpicture}[line cap=round,line join=round,
    >=triangle 45,x=2.0cm,y=2.0cm]
    \draw[->,color=black] (-1.72,0) -- (1.78,0);
    \foreach \x in {-1,1}
    \draw[shift={(\x,0)},color=black] (0pt,2pt) -- (0pt,-2pt);
    \draw[->,color=black] (0,-1.4) -- (0,1.44);
    \foreach \y in {-1,1}
    \draw[shift={(0,\y)},color=black] (2pt,0pt) -- (-2pt,0pt);
    \clip(-1.72,-1.2) rectangle (1.78,1.24);
    \draw(0,0) circle (2cm);
    \draw (-1,0)-- (0,1);
    \draw (0,1)-- (1,0);
    \draw (1,0)-- (0,-1);
    \draw (0,-1)-- (-1,0);
    \draw (-1,1)-- (1,1);
    \draw (1,1)-- (1,-1);
    \draw (1,-1)-- (-1,-1);
    \draw (-1,1)-- (-1,-1);
\end{tikzpicture}

Nell'immagine sono rappresentate, nell'ordine dalla più interna alla più
esterna, le palle aperte centrate nell'origine e di raggio $1$ rispettivamente
nelle metriche $d_1, d_2, d_{\infty}$.
\end{center}

\begin{proof}
    Per AM-QM si ha che $$\displaystyle \frac{\sum_{i=1}^n |x_i-y_i|}{n} \le
    \sqrt{\frac{\sum_{i=1}^n (x_i-y_i)^2}{n}},$$ da cui $d_1(x, y) \le \sqrt{n}
    \cdot d_2(x, y)$. Stimando tutti i termini con il massimo otteniamo che
    $$\displaystyle \sqrt{\sum_{i=1}^n (x_i-y_i)^2} \le \sqrt{\sum_{i=1}^n
    \sup_{j=1, \dots, n} \{(x_j-y_j)^2\}}=\sqrt{n} \cdot \sup_{j=1, \dots, n} \{
    |x_j-y_j| \},$$ da cui $d_2(x, y) \le \sqrt{n} \cdot d_{\infty} (x, y).$
    Infine è ovvio che $$\displaystyle \sup_{i=1, \dots, n} |x_i-y_i| \le
    \sum_{i=1}^n |x_i-y_i|$$, da cui $d_{\infty}(x, y) \le d_1(x, y).$
    Scegliendo $k=\sqrt{n}$ è ora sufficiente applicare il lemma.
\end{proof}

Nella dimostrazione del corollario era importante che lo spazio fosse di
dimensione finita. Nello spazio delle funzioni continue da $[0, 1]$ a
$\mathbb{R}$ le tre distanze non sono topologicamente equivalenti.


\subsection{Spazi topologici}
\begin{defn}
    Uno \textsc{spazio topologico} è una coppia $(X, \tau),\; {\tau \subseteq
    \mathcal{P}(X)}$, t.c.
    \begin{nlist}
        \item $\emptyset,\ X \in \tau$;
        \item $A_1,\ A_2 \in \tau \Rightarrow A_1 \cap A_2 \in \tau$;
        \item se $I$ è un insieme e $A_i \in \tau \, \forall i \in I$, allora
        $\displaystyle \bigcup_{i \in I} A_i \in \tau$.
    \end{nlist}
    Allora $\tau$ si dice \textsc{topologia} di $X$ e gli elementi della
    topologia sono detti \textsc{aperti} di $\tau$.
\end{defn}

\begin{prop}
    Se $(X, d)$ è uno spazio metrico, gli aperti rispetto a $d$ definiscono una
    topologia.
\end{prop}

\begin{defn}
    Uno spazio topologico $(X, \tau)$ è detto \textsc{metrizzabile} se $\tau$ è
    indotta da una distanza su $X$.
\end{defn}

\begin{defn}
    Sia $(X, \tau)$ uno spazio topologico, $C \subseteq X$ è \textsc{chiuso} se
    $X \setminus C$ è aperto.
\end{defn}

\begin{oss}
\begin{nlist}
\item possono esistere insiemi né aperti né chiusi;
\item $\emptyset$ e $X$ sono chiusi;
\item unione finita di chiusi è chiusa;
\item intersezione arbitraria di chiusi è chiusa.
\end{nlist}
\end{oss}

\begin{ex}
    Ecco alcuni esempi di spazi topologici:
    \begin{itemize}
        \item tutte le topologie indotte da una metrica;
        \item la topologia discreta, cioè $\tau=\mathcal{P}(X)$, indotta dalla
        distanza discreta;
        \item la topologia indiscreta, cioè $\tau=\{ \emptyset, X \}$;
        \item la topologia cofinita, dove gli aperti sono l'insieme vuoto più
        tutti e soli gli insiemi il cui complementare è un insieme finito.
    \end{itemize}
\end{ex}

\begin{defn}
	Dato uno spazio topologico $X$ e un insieme $B \subseteq X$, si chiama
	\textit{parte interna} di $B$, e la si indica con $B^{\circ}$, il più grande
	aperto contenuto in $B$. Analogamente, la \textit{chiusura} di $B$, indicata
	con $\overline{B}$, è il più piccolo chiuso che contiene $B$. Definiamo
	infine la \textit{frontiera} (o \textit{bordo}) di un insieme come l'insieme
	$\partial B= \overline{B} \setminus B^{\circ}$.
\end{defn}

Notiamo che parte interna e chiusura sono ben definite: per la prima basta
prendere l'unione di tutti gli aperti contenuti in $B$ (alla peggio c'è solo il
vuoto, che è contenuto in ogni insieme), per la seconda si prende l'intersezione
di tutti i chiusi che lo contengono (alla peggio c'è solo $X$). Per la stabilità
degli aperti per unione arbitraria e dei chiusi per intersezione arbitraria, gli
insiemi così ottenuti sono ancora un aperto e un chiuso e sono rispettivamente
il più grande aperto contenuto e il più piccolo chiuso che contiene per come
sono stati costruiti. Infine, è banale mostrare che la parte interna di un
aperto è l'aperto stesso e la chiusura di un chiuso è il chiuso stesso.

\begin{ftt}
	$X= B^{\circ} \sqcup\ \partial B\ \sqcup (X
	\setminus B)^{\circ}$ dove con $\sqcup$ si indica l'unione disgiunta.
\end{ftt}

\begin{proof}
	Per definizione $B^{\circ} \cup\ \partial B=\overline{B}$ e i due insiemi
	sono disgiunti. Inoltre, essendo $\overline B$ il più piccolo chiuso che
	contiene $B$, il suo complementare dev'essere il più grande aperto disgiunto
	da $B$, cioè il più grande aperto contenuto in $X \setminus B$, che è la
	parte interna di quest'ultimo. La tesi segue facilmente.
\end{proof}

\begin{exc}
  Sono lasciati come semplice esercizio alcuni fatti su parte interna e chiusura:
  \begin{nlist}
    \item $\overline{A \cup B}=\overline{A} \cup \overline{B}$, ma $\displaystyle \bigcup_{q \in \mathbb{Q}} \overline{\{q\}}=\mathbb{Q}, \overline{\bigcup_{q \in \mathbb{Q}}\{q\}}=\mathbb{R}$, quindi l'uguaglianza non vale in generale per unioni infinite;
    \item $\overline{A \cap B} \subseteq \overline{A} \cap \overline{B}$, ma prendendo $A=\mathbb{Q}, B=\mathbb{R} \setminus \mathbb{Q}$ otteniamo che può non valere l'uguaglianza;
    \item gli analoghi per la parte interna (attenzione: unione e intersezione si scambiano);
    \item $\left(\overline{(\overline{A})^{\circ}}\right)^{\circ}=(\overline{A})^{\circ}$ e $(\overline{\overline{A^{\circ}})^{\circ}}=\overline{A^{\circ}}$;
    \item trovare un esempio di $X$ spazio topologico e $A \subseteq X$ t.c. $A, \overline{A}, (\overline{A})^{\circ}, \overline{(\overline{A})^{\circ}}, A^{\circ}, \overline{A^{\circ}}, (\overline{A^{\circ}})^{\circ}$ sono tutti diversi.
  \end{nlist}
\end{exc}


\subsection{Continuità in spazi topologici}
\begin{defn}
    $f: (X, \tau) \rightarrow (Y, \tau')$ si dice \textsc{continua} se
    ${f^{-1}(A) \in \tau,}\; {\forall A \in \tau'}$, cioè una funzione \`e
    continua se la controimmagine di aperti \`e aperta.
\end{defn}

Notiamo che per il teorema \ref{thm:cont_inv}, le due
definizioni di funzione continua sono equivalenti per uno spazio metrico se si
considera la topologia indotta dalla metrica.

\begin{thm}
    \begin{nlist}
        \item L'identità è una funzione continua;
        \item composizione di funzioni continue è una funzione continua.
    \end{nlist}
\end{thm}

\begin{defn}
    Una funzione $f: X \rightarrow Y$ si dice \textsc{omeomorfismo} se $f$ è
    continua e esiste una funzione $g:Y\rightarrow X$ continua tale che $f \circ
    g = Id_Y$ e $g \circ f = Id_x$. Cio\`e $f$ \`e continua, bigettiva e con
    inversa continua.
\end{defn}

\begin{oss}
	\begin{nlist}
	\item Composizione di omeomorfismi è un omeomorfismo; due spazi legati da un
	omeomorfismo si dicomo \textsc{omeomorfi} e essere omeomorfi è una relazione
	di equivalenza;
	\item l'insieme degli omeomorfismi da $(X, \tau)$ in sé è un gruppo;
	\item se $f:X \rightarrow Y$ è continua e bigettiva non è detto che sia un
	omeomorfismo, cioè $f^{-1}$ può non essere continua.
    \marginpar{\warningsign}
\end{nlist}
\end{oss}

\begin{ex}
	Siano $\tau_E$ la topologia euclidea, $\tau_C$ la cofinita, $\tau_D$ la
	discreta e $\tau_I$ l'indiscreta. Le seguenti mappe sono dunque continue:
    \begin{itemize}
	\item $Id:(\mathbb{R}, \tau_D) \rightarrow (\mathbb{R}, \tau_E)$,
    \item $Id:(\mathbb{R}, \tau_E) \rightarrow (\mathbb{R}, \tau_C)$,
    \item $Id:(\mathbb{R}, \tau_C) \rightarrow (\mathbb{R}, \tau_I)$.
    \end{itemize}
    Nessuna delle inverse è però continua. Più in generale, $Id: (X, \tau)
    \rightarrow (X, \sigma)$ con $\sigma \subsetneq \tau$ è continua ma
    l'inversa no.
\end{ex}

\begin{exc}
	Il seguente esercizio è frutto di una domanda fatta da uno studente a
	lezione e potrebbe essere più difficile di altri esercizi del corso.

    \marginpar{\warningsign}
	Trovare un esempio (o dimostrare che non esiste) di uno spazio topologico
	$X$ e una funzione $f: (X, \tau) \rightarrow (X, \tau)$ continua e bigettiva
	con inversa non continua.

    Stando a quanto dice Frigerio, probabilmente tale funzione esiste,
    euristicamente perché non c'è un modo facile di dimostrare il contrario.
\end{exc}

\begin{sol}
	La seguente soluzione è un esempio mostrato a lezione da Gandini. Vedi te a
	volte il caso.

	Mettiamo su $\mathbb{Z}$ la seguente topologia: $A \subseteq \mathbb{Z}$ è
	aperto se $A=\mathbb{Z}$ o $A \subseteq \mathbb{N}$. È semplice verificare
	che è una topologia. Allora la funzione $f: \mathbb{Z} \rightarrow
	\mathbb{Z}$ t.c. $f(n)=n-1$ è l'esempio cercato. Banalmente è bigettiva, è
	semplice verificare che è continua ma l'inversa no.
\end{sol}

Introduciamo adesso un concetto che permetterà di caratterizzare la continuità
in spazi topologici in modo analogo a quanto fatto per gli spazi metrici.

\begin{defn}
	Sia $(X, \tau)$ uno spazio topologico fissato e $x_0 \in X$. Un insieme $U
	\subseteq X$ è un \textsc{intorno} di $x_0$ se $x_0 \in
	U^{\circ}$, o equivalentemente se esiste $V$ aperto con $x_0 \in V
	\subseteq U$. L'insieme degli intorni di $x_0$ si denota con
	$\mathcal{I}(x_0)$.
\end{defn}

\begin{defn}
	$f:(X, \tau) \rightarrow (Y, \tau')$ è detta \textit{continua in $x_0$} se
	per ogni intorno $U$ di $f(x_0)$ esiste un intorno $V$ di $x_0$ t.c. $f(V)
	\subseteq U$.
\end{defn}

Appare dunque intuitivo il seguente risultato.

\begin{thm}
	$f$ è continua $\Leftrightarrow$ è continua in ogni $x_0 \in X$.
\end{thm}

\begin{proof}
	($\implies$) Supponiamo $f$ continua e $x_0 \in X$, sia inoltre $U$ un
	intorno di $f(x_0)$. Per definizione di intorno esiste un aperto $A$ con
	$f(x_0) \in A \subseteq U$, perciò $x_0 \in f^{-1}(A) \subseteq f^{-1}(U)$,
	ma poiché $f$ è continua abbiamo che $f^{-1}(A)$ è ancora un aperto, perciò
	ponendo $V=f^{-1}(U)$ abbiamo che $V$ è un intorno di $x_0$ t.c. $f(V)
	\subseteq U$, che è quello che volevamo.

	($\Leftarrow$) Supponiamo $f$ continua in ogni punto di $X$ e sia $A$ un
	insieme aperto in $Y$. Per ogni $x \in f^{-1}(A)$, $A$ è un intorno di
	$f(x)$. Ma dato che $f$ è continua in ogni punto di $X$, esiste un intorno
	$V_x$ di $x$ t.c. $x \in V_x \subseteq f^{-1}(A)$. Per definizione di
	intorno, ciò significa che esiste un aperto $A_x$ di $X$ t.c. $x \in A_x
	\subseteq V \subseteq f^{-1}(A)$. Dunque dev'essere $\displaystyle
	f^{-1}(A)= \bigcup_{x \in f^{-1}(A)} A_x$,
	quindi $f^{-1}(A)$ è un aperto in $X$ per ogni $A$ aperto in $Y$, il che
	equivale a dire che $f$ è continua.
\end{proof}


\subsection{Ordinamento fra topologie, basi e prebasi}
Vogliamo mettere un ordinamento parziale sulle topologie di un certo insieme
fissato.

\begin{defn}
    Dato un insieme $X$ su cui sono definite le topologie $\tau$ e $\tau'$, si
    dice che $\tau$ \`e \textsc{meno fine} di $\tau'$ se $\tau \subseteq \tau'$,
    cioè ogni aperto di $\tau$ è anche aperto di $\tau'$. $\tau'$ si dice
    \textsc{più fine} di $\tau$.
\end{defn}

\begin{oss}
    Equivalentemente alla definizione sopra, si pu\`o dire che $\tau$ \`e meno
    fine di $\tau'$ se e solo se ${Id:(X,\tau')\rightarrow(X, \tau)}$ \`e
    continua.
\end{oss}

Quando $\tau$ è meno fine di $\tau'$ scriveremo $\tau < \tau'$. Si noti che
dalla definizione ogni topologia è meno fine di se stessa, cioè $\tau < \tau'
\; \forall \tau$.

\begin{ex}
	$\tau_I < \tau_C < \tau_E < \tau_D$, \, $\tau_I < \tau < \tau_D \; \forall
	\tau$.
\end{ex}

\begin{lm}
	Intersezione arbitraria di topologie su $X$ è ancora una topologia su $X$.
\end{lm}

\begin{proof}
    Siano $\tau_i,\ i \in I$ topologie su $X$. Verifichiamo che $
    \tau=\bigcap_{i \in I} \tau_i$ soddisfa gli assiomi di topologia.

    \begin{nlist}
        \item $\emptyset, X \in \tau_i \, \forall i \in I \Rightarrow \emptyset,
        X \in \tau$.

        \item $A_1, A_2 \in \tau \Rightarrow A_1, A_2 \in \tau_i \, \forall i
        \in I \Rightarrow A_1 \cap A_2 \in \tau_i \, \forall i \in I \Rightarrow
        A_1 \cap A_2 \in \tau$.

        \item Siano $A_j, j \in J$ insiemi che stanno in $\tau$. \\
        $\displaystyle A_j \in \tau \, \forall j \in J \Rightarrow A_j \in
        \tau_i \, \forall i \in I, j \in J \Rightarrow {\bigcup_{j \in J} A_j
        \in \tau_i \, \forall i \in I} \Rightarrow {\bigcup_{j \in J} A_j \in
        \tau}$.
    \end{nlist}
\end{proof}

\begin{cor}
	Data una famiglia $\tau_i, i \in I$ di topologie su $X$, esiste la più fine
	tra le topologie meno fini di ogni $\tau_i$: è $\displaystyle \bigcap_{i \in
	I} \tau_i$.
\end{cor}

\begin{cor}
	Sia $X$ un insieme, $S \subseteq \mathcal{P}(X)$, allora esiste la topologia
	meno fine tra quelle che contengono $S$. Tale topologia si dice
	\textit{generata} da $S$ e $S$ si dice \textsc{prebase} della topologia. Se
	$\Omega= \{ \tau \text{ topologia } |\; S \in \tau \}$ (che è non vuoto
	perché contiene almeno la topologia discreta), la topologia cercata è
	$\displaystyle \bigcap_{\tau \in \Omega} \tau$.
\end{cor}

\begin{defn}
	sia $(X, \tau)$ uno spazio topologico fissato, una \textsc{base} di $\tau$ è
	un insieme $\mathcal{B} \subseteq \tau$ t.c. $\forall A \in \tau, \exists
	B_i \in \mathcal{B}, i \in I$ t.c. $A= \bigcup_{i \in I} B_i$. Ovvero
	$\mathcal{B}$ \`e una base se ogni aperto di $\tau$ pu\`o essere scritto
	come unione qualunque di elementi di $\mathcal{B}$.
\end{defn}

\begin{ex}
	Se $X$ è uno spazio metrico, una base della topologia indotta sono le palle.
\end{ex}

\begin{defn} \label{N2}
	$(X, \tau)$ si dice \textit{a base numerabile} (o che soddisfa il
	\textsc{secondo assioma di numerabilità}) se ammette una base numerabile.
\end{defn}

\begin{prop} \label{prop:base}
	Sia $X$ un insieme senza topologia, $\mathcal{B} \subseteq \mathcal{P}(X)$ è
	base di una topologia su $X$ $\Leftrightarrow$ valgono le seguenti:
	\begin{nlist}
		\item \label{i_prop} $\displaystyle X=\bigcup_{B \in \mathcal{B}} B$;
		\item $\forall A, A' \in \mathcal{B}, \exists B_i \in \mathcal{B}, i \in
		I$ t.c. $A \cap A'= \bigcup_{i \in I} B_i$. Cio\`e ogni intersezione di
		una coppia di elementi di $\mathcal{B}$ pu\`o essere scritta come unione
		di elementi di $\mathcal{B}$.
	\end{nlist}
\end{prop}

\begin{proof} \label{prop:preb}
    ($\Rightarrow$) Ovviamente, se $\mathcal{B}$ è la base di una topologia su
    $X$, l'insieme $X$ deve essere unione di elementi di $\mathcal{B}$, inoltre
    tutti gli elementi di $\mathcal{B}$ devono essere sottoinsiemi di $X$, da
    cui discende (i).

	$A, A' \in \mathcal{B} \Rightarrow A, A' \in \tau \Rightarrow A \cap A' \in
	\tau$, per cui $A \cap A'$ deve poter essere esprimibile come unione di
	elementi di $\mathcal{B}$, che è l'affermazione (ii).

	($\Leftarrow$) Dobbiamo mostrare che l'insieme $\tau$ di tutte le possibili
	unioni di elementi di $\mathcal{B}$ soddisfa gli assiomi di topologia.

	Chiaramente $\emptyset \in \tau$ come unione di un insieme vuoto di elementi
	di $\mathcal{B}$ e $X \in \tau$ per (ii).  Se faccio l'unione arbitraria di
	insiemi ottenuti come unione di elementi di $\mathcal{B}$ ottengo ovviamente
	un insieme che è unione di elementi di $\mathcal{B}$.

	Infine, $\displaystyle A, A' \in \tau \Rightarrow A=\bigcup_{i \in I} B_i,
	A'=\bigcup_{j \in J} B_j$ con $B_i, B_j \in \mathcal{B} \, \forall i \in I,
	j \in J$. Allora $\displaystyle A \cap A'= \left(\bigcup_{i \in I} B_i
	\right) \cap \left(\bigcup_{j \in J} B_j \right)=\bigcup_{i \in I, j \in J}
	(B_i \cap B_j)$, ma tutti i $B_i$ e $B_j$ stanno in $\mathcal{B}$, dunque
	per (ii) tutti i $B_i \cap B_j$ sono rappresentabili come unione di elementi
	di $\mathcal{B}$, perciò anche la loro unione, che è proprio $A \cap A'$,
	può essere scritta in quel modo e quindi sta in $\tau$.
\end{proof}

\begin{prop}
	Siano $X$ un insieme e $S \subseteq \mathcal{P}(X)$ la prebase di una
	topologia $\tau$ su $X$. Allora:
	\begin{nlist}
		\item le intersezioni finite di elementi di $S \cup \{X\}$ sono una base
		di $\tau$;
		\item $A \in \tau$ $\Leftrightarrow$ $A$ è unione arbitraria di
		intersezioni finite di elementi di $S \cup \{X\}$.
	\end{nlist}
\end{prop}

\begin{proof}
	Sicuramente, poiché $\tau$ è generata da $S$, $S \subseteq \tau$ e quindi
	anche tutte le intersezioni finite di elementi di $S$ e le unioni arbitrarie
	di tali intersezioni devono stare in $\tau$. Se mostriamo che sono
	sufficienti a definire una topologia, abbiamo finito.

    Chiaramente il vuoto è l'intersezione di un insieme vuoto di insiemi e $X$
    c'è perché lo abbiamo aggiunto a mano.

    Unione arbitraria di unioni arbitrarie di elementi di un insieme è ancora
    unione arbitraria di elementi di tale insieme.

    Siano ora $\displaystyle A_1= \bigcup_{i \in I} B_i,\ A_2=\bigcup_{j \in J}
    B_j$ con tutti
    i $B_i,\ B_j$ intersezioni finite di elementi di $S$. Allora
    $$A_1 \cap A_2 =
    \left( \bigcup_{i \in I} B_i \right) \cap \left(\bigcup_{j \in J} B_j
    \right)= \bigcup_{i \in I, j \in J} (B_i \cap B_j),$$
     ma  intersezione di due intersezioni finite di elementi di $S$ è ancora
     un'intersezione finita di elementi di $S$, perciò $A_1 \cap A_2$ è ancora
     un'unione di intersezioni finite di elementi di $S$. Questo basta per
     dimostrare (i) e (ii) è una semplice riformulazione.
\end{proof}

\begin{prop}
	Sia $f: (X, \tau) \rightarrow (Y, \tau')$ e $S,\ \mathcal{B}$
	rispettivamente una prebase e una base di $\tau'$. Allora sono equivalenti:
	\begin{nlist}
		\item $f$ è continua;
		\item $f^{-1}(A)$ è aperto per ogni $A \in S$;
		\item $f^{-1}(A)$ è aperto per ogni $A \in \mathcal{B}$.
	\end{nlist}
\end{prop}

\begin{proof}
	Poiché ogni base è una prebase, ((i) $\Leftrightarrow$ (ii)) $\Rightarrow$
	((i) $\Leftrightarrow$ (iii)).

	(i) $\Rightarrow$ (ii) è ovvia, perciò resta da dimostrare (ii)
	$\Rightarrow$ (i).

	Sia $A$ un aperto di $\tau'$. Per la proposizione \ref{prop:preb}, $A$ si
	può scrivere come unione di intersezioni finite di elementi di $S$. Poiché
	la controimmagine di un'unione è l'unione delle controimmagini e la
	controimmagine di un'intersezione è l'intersezione delle controimmagini, la
	controimmagine di $A$ è unione di intersezioni finite di controimmagini di
	elementi di $S$, ma queste controimmagini sono aperte, perciò un'unione di
	loro intersezioni finite è ancora aperta, perciò $f^{-1}(A)$ è aperto in $X$
	per ogni $A$ aperto in $Y$, e questo equivale a dire che $f$ è continua.
\end{proof}


\subsection{Assiomi di numerabilità}
Nella definizione \ref{N2} abbiamo stabilito quando uno spazio topologico
soddisfa il secondo assioma di numerabilità. Vediamo gli altri due.

\begin{defn}
	$Y \subseteq X$ si dice \textsc{denso} se $\overline{Y}=X$.
\end{defn}

\begin{oss}
	$Y$ è denso $\Leftrightarrow$ $Y \cap A \not=\emptyset$ per ogni $A$ aperto
	non vuoto.
\end{oss}

\begin{defn} \label{N3}
	$X$ si dice \textsc{separabile} (o che soddisfa il \textsc{terzo assioma di
	numerabilità}) se ammette un sottoinsieme denso numerabile.
\end{defn}

\begin{prop} \label{N2impN3}
	Se $X$ è a base numerabile, $X$ è separabile.
\end{prop}

\begin{proof}
	Sia $\{ B_i, {i \in \mathbb{N}}\}$ la base numerabile di $X$ e scegliamo per
	ogni $i \in \mathbb{N}$ un elemento $x_i \in B_i$. Vogliamo mostrare che $\{
	x_i, i \in \mathbb{N} \}$ è un sottoinsieme denso numerabile di $X$.

	Ovviamente è numerabile. Per dimostrare che è denso, mostriamo che interseca
	ogni aperto non vuoto.

	Sia dunque $A$ un aperto non vuoto di $X$, allora, dato che i $B_i$ formano
	una base, $A$ è esprimibile come unione di alcuni di essi, in particolare
	esiste $i_0$ t.c. $B_{i_0} \subseteq A$ e perciò $x_{i_0} \in B_{i_0}
	\Rightarrow x_{i_0} \in A$, dunque l'intersezione tra il nostro insieme e un
	aperto non vuoto non è mai vuota, come voluto.
\end{proof}

\begin{prop} \label{metr-num}
	Se $(X, \tau)$ è metrizzabile, $X$ è separabile $\Leftrightarrow$ è a base
	numerabile.
\end{prop}

\begin{proof}
	($\Rightarrow$) Mostriamo che per ogni aperto $A$ e per ogni $x \in A$,
	esiste una palla di centro un elemento del sottoinsieme denso numerabile e
	raggio un razionale positivo tutta contenuta in $A$. Allora tali palle
	formano una base numerabile.

	Sicuramente esiste una palla di centro $x$ e raggio $R \in \mathbb{R}^+$
	tutta contenuta in $A$. Consideriamo la palla di centro $x$ e raggio $R/3$,
	che è contenuta nella prima. Questa palla è in particolare un insieme
	aperto, dunque ha un elemento $y$ in comune con il sottoinsieme denso
	numerabile. Scegliendo un razionale $R/3<r<2 \cdot R/3$ si ottiene che la
	palla di centro $y$ e raggio $r$ è tutta contenuta nella palla di centro $x$
	e raggio $R$, dunque è tutta contenuta in $A$, e contiene $x$, come voluto.

	L'altra freccia discende dalla proposizione \ref{N2impN3}.
\end{proof}

\begin{defn}
	Un \textsc{sistema fondamentale di intorni} per $x_0$ è una famiglia
	$\mathcal{F} \subseteq \mathcal{I}(x_0)$ t.c. $\forall U \in
	\mathcal{I}(x_0) \, \exists V \in \mathcal{F}$ con $V \subseteq U$.
\end{defn}

\begin{ex} \label{1/n}
	Se $X$ è uno spazio metrico, le palle centrate in $x_0$ do raggio $1/n$ al
	variare di $n$ intero positivo sono un sistema fondamentale di intorni. In
	particolare, sono un sistema fondamentale di intorni numerabile.
\end{ex}

\begin{defn} \label{N1}
	$X$ soddisfa il \textsc{primo assioma di numerabilità} se ogni $x_0 \in X$
	ha un sistema fondamentale di intorni numerabile.
\end{defn}

\begin{ftt}
	Se $X$ è metrizzabile $X$ soddisfa il primo assioma di numerabilità. Ciò
	discende direttamente dall'esempio \ref{1/n}.
\end{ftt}

\begin{prop} \label{N2power}
	Il secondo assioma di numerabilità implica il primo e il terzo assioma di
	numerabilità. Discende dalla proposizione \ref{metr-num} che in spazi
	metrici vale anche che il terzo assioma di numerabilità implica il secondo
	assioma di numerabilità.
\end{prop}

\begin{proof}
	Supponiamo che $X$ soddisfi il secondo assioma di numerabilità. Per la
	proposizione \ref{N2impN3} otteniamo subito che soddisfa il terzo. Vogliamo
	adesso mostrare che soddisfa anche il primo. Sia $\mathcal{B}$ una base
	numerabile. Dato $x \in X$, consideriamo l'insieme ${\{ B \in \mathcal{B}\
	|\ x \in B \}}$. Essendo un sottoinsieme di $\mathcal{B}$ è sicuramente
	numerabile e contiene solo insiemi aperti, cioè intorni di $x$. Mostriamo
	che è un sistema fondamentale di intorni per $x$.

	Consideriamo un intorno $U$ di $x$. Questo ha un sottoinsieme aperto $V
	\subseteq U$ contenente $x$, dunque $V$ è esprimibile come unione di
	elementi di $\mathcal{B}$ e ce ne dev'essere uno che contiene $x$, che sta
	quindi nel nostro insieme ed è contenuto in $V$ e quindi in $U$. Dunque, per
	ogni intorno $U$ di $x$ troviamo un elemento del nostro insieme di intorni
	di $x$ contenuto in $U$, che è quello che dovevamo dimostrare.
\end{proof}

\begin{prop}
	\begin{nlist}
		\item $A \subseteq X$ è aperto se e solo se è intorno di ogni suo punto;
		\item sia $C \subseteq X$ un insieme generico, allora $x \in
		\overline{C}$ se e solo se ogni intorno di $x$ interseca $C$.
	\end{nlist}
\end{prop}

\begin{proof}
    \begin{nlist}
        \item Sia $A$ un insieme che \`e intorno di ogni suo punto, cio\`e per
        ogni $x \in A$ esiste un aperto $V_x \subseteq A$ che lo contiene.
        Allora vale che $\displaystyle A = \bigcup_{x \in A} V_x$. Poiché $A$
        \`e unione di aperti\`e aperto.

        Viceversa sia $A$ aperto. Allora ogni suo punto \`e interno ad esso.

        \item Si ha che $\overline{C}$ \`e il complementare di $(X\setminus
        C)^\circ$. Dunque $x$ appartiene a $\overline{C}$ se e solo se non \`e
        interno a $X \setminus C$, cioè se e solo se ogni suo intorno $U$ non
        \`e tutto contenuto nel complementare di $C$. Questo accade quando ogni
        intorno di $X$ interseca $C$ in almeno un punto.
    \end{nlist}
\end{proof}

\begin{ex} \label{Sorgenfrey-retta}
	La retta di Sorgenfrey. Su $X=\mathbb{R}$ consideriamo il seguente insieme:
	$${\mathcal{B}=\{\,[a, b)\ |\ a<b\}}.$$
    Allora valgono le seguenti:
	\begin{nlist}
		\item $\mathcal{B}$ è base di una topologia $\tau$;
		\item $\tau$ \`e pi\`u fine della topologia euclidea;
		\item $\tau$ è separabile;
		\item $\tau$ non è a base numerabile;
		\item $\tau$ non è metrizzabile;
		\item $\tau$ è primo numerabile.
	\end{nlist}
\end{ex}

\begin{proof}
    \begin{nlist}
        \item Verifichiamo utilizando la proposizione \ref{prop:base}. Il primo
        punto vale in quanto $\mathbb{R}$ \`e coperto da intervalli del tipo
        $\left[-n, n\right)$ con $n$ naturale. Inoltre dati i due intervalli
        $[a,b)$ e $[c,d)$, distinguo:
        \begin{itemize}
            \item $b\leq c$. Allora l'intersezione \`e vuota.
            \item $c < b$ e $b \leq d$. Allora l'intersezione \`e $[c,b)$.
            \item $c < b$ e $b > d$. Allora l'intersezione \`e $[c,d)$.
        \end{itemize}
        In ogni caso vale anche il secondo punto.

        \item Voglio mostrare che ogni aperto di $\tau$ \`e anche aperto di
        $\tau_E$. Sia allora $(a, b)$ un aperto della base degli intervalli
        aperti della topologia euclidea. Si considerino gli aperti in $\tau$
        della forma $[a+\frac{1}{n},b)$ con $n$ naturale positivo. L'unione di
        tutti questi \`e allora $(a,b)$.

        \item Si noti che $\mathbb{Q}$ \`e un denso numerabile.

        \item Sia $\mathcal{D}$ una base di $\tau$. Si noti che ogni intervallo
        del tipo $[x,x+1)$, con $x$ reale, pu\`o essere scritto come unione
        degli elementi della base solo se in $\mathcal{D}$ c'\`e un elemento $D$
        tale che $x \in D \subseteq [x, x+1)$, da cui $\inf(D) = x$. Allora la
        mappa
        \begin{align*}
            \phi:\mathcal{D}&\to \mathbb{R}\\
            D & \mapsto \inf(D)
        \end{align*}
        deve essere surgettiva. Dunque la base scelta non pu\`o essere
        numerabile.

        \item Semplice conseguenza dei punti precedenti e della proposizione
        \ref{metr-num}.

				\item Basta notare che per ogni $x \in \mathbb{R}$ l'insieme $\{
				[x-1/n, x+1/n) | n \in \mathbb{Z}^+\}$ è una famiglia di intorni
				numerabile per $x$.
    \end{nlist}
\end{proof}

\begin{ex}
	Topologia di Zariski. Sia $\mathbb{K}$ un campo. Definiamo una topologia
	$\tau_Z$ in $\mathbb{K}^n$ in cui i chiusi sono tutti e soli gli insiemi i
	cui elementi si annullano in tutti i polinomi di una famiglia arbitraria
	(non vuota) $\mathcal{F} \subseteq \mathbb{K}[x_1, \dots, x_n]$ di polinomi
	a $n$ variabili con coefficienti in $\mathbb{K}$. Dimostreremo che $\tau_Z$
	è effettivamente una topologia, che è meno fine di quella euclidea quando
	$\mathbb{K}=\mathbb{R}$. Inoltre, quando $n=1$ e per $\mathbb{K}$ generico,
	$\tau_Z$ coincide con la topologia cofinita.
\end{ex}

\begin{proof}
	Dimostreremo che i chiusi soddisfano le proprietà di topologia (per i
	chiusi, ovviamente), per passaggio al complementare si può concludere che
	$\tau_Z$ è una topologia.

	Ovviamente il vuoto è chiuso perché la proprietà dei suoi elementi di
	annullarsi in un qualunque insieme di polinomi è sempre vera a vuoto, mentre
	$\mathbb{K}^n$ è chiuso perché tutti gli elementi si annullano nella
	famiglia formata dal solo polinomio nullo.

	Siano ora $C_1, C_2$ due chiusi i cui elementi si annullano, per
	definizione, nei polinomi rispettivamente delle famiglie $\mathcal{F}_1,
	\mathcal{F}_2$. Consideriamo la famiglia $\mathcal{F}_{1, 2}=$ \\ $=\{ f_1
	\cdot f_2 | f_1 \in \mathcal{F}_1, f_2 \in \mathcal{F}_2 \}$. Mostriamo che
	l'insieme dei punti che si annullano nei polinomi di $\mathcal{F}_{1, 2}$ è
	proprio $C_1 \cup C_2$. Ovviamente se $x \in C_1 \cup C_2$ possiamo dire
	senza perdita di generalità che $x$ si annulla in tutti i polinomi in
	$\mathcal{F}_1$, e quindi banalmente anche in tutti i polinomi di
	$\mathcal{F}_{1, 2}$. D'altro canto, se $x$ si annulla in tutti i polinomi
	di $\mathcal{F}_{1, 2}$, si deve annullare almeno o in tutti i polinomi di
	$\mathcal{F}_1$ o in tutti i polinomi di $\mathcal{F}_2$. Se per assurdo
	così non fosse, esisterebbero $f_1 \in \mathcal{F}_1, f_2 \in \mathcal{F}_2$
	t.c. $f_1(x) \not= 0 \not= f_2(x)$ e, poiché siamo in un campo, si avrebbe
	$f_1(x) \cdot f_2(x)=0$, assurdo. Dunque $x \in C_1 \cup C_2$.

	Consideriamo adesso dei chiusi $C_i, i \in I$ definiti dalle famiglie
	$\mathcal{F}_i$. Mostriamo che $\displaystyle C=\bigcap_{i \in I} C_i$ è
	l'insieme dei punti che si annullano nei polinomi di $\displaystyle
	\mathcal{F}_I=\bigcup_{i \in I} \mathcal{F}_i$. Se $x \in C$ allora $x \in
	C_i \, \forall i \in I$ e dunque si annulla in tutti i polinomi di
	$\mathcal{F}_i$ per ogni $i$, quindi si annulla in tutti i polinomi della
	loro unione, che è proprio $\mathcal{F}_I$. Viceversa, se $x$ si annulla in
	tutti i polinomi di $\mathcal{F}_I$ allora si annulla in tutti i polinomi di
	ogni suo sottoinsieme, in particolare in tutti i polinomi di $\mathcal{F}_i
	\, \forall i \in I$, quindi $x \in C_i$ per ogni $i$ da cui $x \in C$.

	Poniamo ora $\mathbb{K}=\mathbb{R}$. Poiché i polinomi in $\mathbb{R}[x_1,
	\dots, x_n]$ sono funzioni continue per la topologia euclidea, la
	controimmagine di un aperto è a sua volta un aperto. Per passaggio al
	complementare la controimmagine di un chiuso è a sua volta un chiuso. Ma
	allora, sia $\mathcal{F}$ una famiglia di polinomi, ho che, essendo $\{ 0
	\}$ chiuso in $\tau_E$ di $\mathbb{R}$, $f^{-1}(0)$ (qui si intende la
	controimmagine) è chiuso in $\tau_E$ di $\mathbb{R}^n$ per ogni $f \in
	\mathcal{F}$. Ma l'insieme dei punti che si annullano in tutti i polinomi di
	$\mathcal{F}$ può essere descritto come $\displaystyle \bigcap_{f \in
	\mathcal{F}} f^{-1}(0)$, che essendo intersezione di chiusi è chiuso in
	$\tau_E$, quindi tutti i chiusi di $\tau_Z$ sono chiusi in $\tau_E$, per
	passaggio al complementare la stessa cosa con gli aperti e dunque $\tau_Z <
	\tau_E$.

	Sia adesso $\mathbb{K}$ generico e $n=1$. Sia $p$ un generico polinomio in
	$\mathbb{R}[x]$ e $\overline{\mathbb{K}}$ la chiusura algebrica di
	$\mathbb{K}$. Per il teorema fondamentale dell'algebra, $p$ ha al più
	$\deg{p}$ zeri in $\overline{\mathbb{K}}$, quindi a maggior ragione ne ha al
	più un numero finito in $\mathbb{K}$. Dunque i punti che si annullano in
	tutti i polinomi di una generica famiglia di polinomi sono finiti (limitati
	dal grado di un qualsiasi polinomio della famiglia), dunque i chiusi sono
	tutti finiti.
	Considerando invece un insieme finito $\{ x_1, \dots, x_m\} \subseteq
	\mathbb{K}$, esso si annulla in tutti i polinomi dell'ideale generato da
	$(x-x_1) \cdot \ldots \cdot (x-x_m)$, dunque è vero anche il viceversa, cioè
	che tutti i finiti sono chiusi, quindi in questo caso $\tau_Z$ coincide con
	la topologia euclidea.
\end{proof}

\begin{ftt}
	Se $X$ non è numerabile, la topologia cofinita su $X$ non soddisfa alcun
	assioma di numerabilità. %Secondo me il terzo lo soddisfa per qualunque
	                           % insieme finito
\end{ftt}

\begin{proof}
	Per la proposizione \ref{N2power}, se dimostriamo che non soddisfa il primo
	otteniamo anche che non soddisfa il secondo. Sia dunque per assurdo $x_0 \in
	X$ e $\mathcal{F}$ un suo sistema fondamentale di intorni numerabile.
	Notiamo che per ogni $x \in X, x\not=x_0$ l'insieme $X \setminus \{ x\}$ è
	un aperto, dunque esiste un intorno di $x_0$ tutto contenuto in esso, cioè
	che non contiene $x$. Notiamo, per come è definita la topologia cofinita,
	che gli intorni, essendo sovrainsiemi di insiemi aperti, sono a loro volta
	aperti, cioè il loro complementare è finito. Ma dato che per ogni $x \in X
	\setminus \{ x_0\}$ esiste un intorno di $x_0$ che non contiene $x$, posso
	scrivere $X \{ x_0\}$, che è ancora un insieme numerabile, come l'unione dei
	complementari degli insiemi in $\mathcal{F}$, cioè un'unione numerabile di
	insiemi finiti, che è numerabile, da cui l'assurdo.

	Per quanto riguarda il terzo assioma, secondo me ho capito male mentre ero a
	lezione, perché mi sembra che qualunque sottoinsieme infinito di $X$ debba
	necessariamente intersecare in qualche punto il complementare di un insieme
	finito, e quindi anche un insieme infinito di cardinalità numerabile avrebbe
	intersezione non nulla con tutti gli aperti e sarebbe di conseguenza un
	denso numerabile.
\end{proof}

Procediamo adesso a caratterizzare, negli insiemi che soddisfano il primo
assioma di numerabilità, aperti, chiusi e continuità tramite successioni.
Diamo prima una definizione.

\begin{defn}
	$l \in X$ è detto \textit{limite} della successione $(a_n)_{n \in
	\mathbb{N}}$ se per ogni intorno $U$ di $l$ esiste $n_0 \in \mathbb{N}$ t.c.
	per ogni $n \ge n_0$ si ha che $a_n \in U$.
\end{defn}

Passiamo dunque alle varia caratterizzazioni. Nei tre enunciati seguenti,
$X$ sarà sempre uno spazio topologico primo numerabile.

\begin{prop}
	Un sottoinsieme $A \subseteq X$ è aperto $\Leftrightarrow$ è aperto per
	successioni (si dice aperto per successioni?), cioè se per ogni successione
	$(a_k)$ di elementi di $x$ che converge a un limite $l \in A$ esiste $k_0$
	t.c. per ogni $k \ge k_0$ si ha $a_k \in A$.
\end{prop}

\begin{proof}
	Notiamo prima il seguente fatto: se $U_k, \, n \in \mathbb{N}$ è un sistema
	fondamentale di intorni numerabile per $x$, definiamo $V_k=U_0 \cap U_1 \cap
	\dots \cap U_k$. Allora $V_k, \, k \in \mathbb{N}$ è un sistema fondamentale
	di intorni numerabile tale che $V_{k+1} \subseteq V_k$.

	($\implies$) Sia $A$ un aperto e $(a_k)$ una successione avente limite $l
	\in A$. Poiché $A$ stesso è un intorno di $l$, si conclude che esiste $k_0$
	t.c. $a_k \in A$ per ogni $k \ge k_0$ dalla definizione di limite.

	($\Leftarrow$) Sia $A$ un insieme aperto per successioni e prendiamo $x \in
	A$. Vogliamo mostrare che esiste un intorno $U_x$ di $x$ tutto contenuto in
	$A$. Se per assurdo così non fosse, allora ogni intorno di $x$ avrebbe un
	elemento non contenuto in $A$. Consideriamo adesso un sistema fondamentale
	di intorni numerabile $V_k, k \in \mathbb{N}$, e lo prendiamo t.c. $V_{k+1}
	\subseteq V_k$. Per ciascuno di questi intorni, prendiamo un elemento $a_k
	\in V_k$ t.c. $a_k \not\in A$, che esiste per ipotesi assurda (alcuni
	elementi possono anche ripetersi, non ha importanza). Allora avremmo, dato
	che il sistema di intorni è fondamentale e che $V_{k'} \subseteq V_k$ se
	$k'>k$
	(segue da $V_{k+1} \subseteq V_k$), che la successione $(a_k)$ sta
	definitivamente in ogni intorno di $x$ e dunque ha limite $x$, ma è tutta
	fuori da $A$, assurdo perché $A$ è aperto per successioni. Allora per ogni
	$x \in A$ esiste un intorno $U_x$ t.c. $U_x \subseteq A$, ma questo ci dice
	anche che per ogni $x \in A$ esiste un aperto $A_x$ t.c. $x \in A_x
	\subseteq U_x \subseteq A$, da cui otteniamo che $\displaystyle A=\bigcup_{x
	\in A} A_x$ è un'unione di aperti e dunque è aperto.
\end{proof}

\begin{prop} \label{chiusoxsucc}
	Un sottoinsieme $C \subseteq X$ è chiuso $\Leftrightarrow$ è chiuso per
	successioni, cioè per ogni successione $(a_k)$ di elementi di $C$ che
	converge a un limite $l$, anche $l \in C$.
\end{prop}

\begin{proof}
	($\implies$) Supponiamo $C$ chiuso e sia $(a_k)$ una successione di elementi
	di $C$ con limite $l$. Per assurdo, $l \not\in C$. Allora $l \in X \setminus
	C$, che è un insieme aperto, in particolare $X \setminus C$ è un intorno di
	$l$. Esiste dunque, per definizione di limite, un $k_0 \in \mathbb{N}$ t.c.
	per ogni $k \ge k_0$ si abbia che $a_k \in X \setminus C$, ma $a_k \in C$
	per ogni $k \in \mathbb{N}$,
	assurdo. Questa freccia vale in tutti gli spazi topologici.

	($\Leftarrow$) Supponiamo adesso che per ogni successione $(a_k)$ di
	elementi di $C$ avente limite $l$ si abbia $l \in C$. Consideriamo un
	elemento $x \in X \setminus C$. Vogliamo mostrare che esiste un intorno
	$U_x$ di $x$ tutto contenuto in $X \setminus C$. Se così non fosse, per ogni
	intorno di $x$ esisterebbe un elemento di $C$ in esso contenuto. Poiché $X$
	è primo numerabile, consideriamo dunque un sistema fondamentale di intorni
	numerabile di $x$, sia esso $V_k, \, k \in \mathbb{N}$, e come nella
	dimostrazione precedente lo prendiamo t.c.
	$V_{k+1} \subseteq V_k$. Adesso scegliamo per ogni $k$ un elemento $a_k \in
	V_k$ t.c. $a_k \in C$, che esiste per l'ipotesi assurda. Notiamo che alcuni
	di questi elementi possono essere uguali, non ha importanza. Dato che
	abbiamo preso un sistema di intorni fondamentale, per ogni intorno $U$ di
	$x$ esiste un $k_0$ t.c. $V_{k_0} \subseteq U$. Ma poiché $V_k \subseteq V_
	{k_0}$ per ogni $k \ge k_0$ (facile conseguenza di $V_{k+1} \subseteq V_k$),
	ho che $a_k \in V_k \subseteq V_{k_0} \subseteq U$ per ogni $k \ge k_0$.
	Riassumendo, per ogni $U$ intorno di $x$ esiste un $k_0$ t.c. per ogni $k
	\ge k_0$ si ha $a_k \in U$, ma questo per definizione significa che $x$ è il
	limite di $(a_k)$, assurdo poiché $a_k \in C$ per ogni $k$ mentre $x \in X
	\setminus C$, contro l'ipotesi iniziale che per tutte le successioni in $C$
	aventi limite anche il limite è in $C$. Dunque per ogni $x \in X \setminus
	C$ esiste un intorno $U_x \subseteq X \setminus C$, da cui si ha che esiste
	un aperto $A_x$ con $x \in A_x \subseteq U_x \subseteq X \setminus C$. Ma
	allora, $\displaystyle X \setminus C=\bigcup_{x \in X \setminus C} A_x$ che
	un'unione di aperti, perciò $X \setminus C$ è aperto e di conseguenza $C$ è
	chiuso.
\end{proof}

\begin{prop}
	Sia $Y$ uno spazio topologico (che non deve necessariamente soddisfare il
	primo assioma di numerabilità). Una funzione $f:X \rightarrow Y$ è continua
	in $\bar{x}$ $\Leftrightarrow$ per ogni successione$(x_n)_{n \in
	\mathbb{N}}$ convergente a $\bar{x}$ la successione $(f(x_n))_{n \in
	\mathbb{N}}$ converge a $f(\bar(x))$.
\end{prop}

\begin{proof}
	($\implies$) Supponiamo che $f$ sia continua in $\bar{x}$ e consideriamo una
	generica successione $(x_n)$ convergente a $\bar{x}$. Prendiamo un intorno
	$U$ di $f(\bar{x})$ in $Y$. Dato che $f$ è continua in $\bar{x}$, esiste un
	intorno $V$ di $\bar{x}$ in $X$ t.c. $f(V) \subseteq U$. Prendiamo, per
	ipotesi di convergenza, $n_0 \in \mathbb{N}$ t.c. $a_n \in V$ per ogni $n
	\ge n_0$. Allora si ha anche, sempre per ogni $n \ge n_0$, $f(x_n) \in U$,
	da cui otteniamo che $(f(x_n))$ converge a $f(\bar{x})$. Questa freccia vale
	per $X$ spazio topologico qualsiasi.

	($\Leftarrow$) Supponiamo adesso che per ogni successione convergente in $X$
	la successione immagine converga all'immagine del limite in $Y$. Per
	assurdo, $f$ non è continua in $\bar{x}$. Allora deve esistere un intorno
	$U$ di $f(\bar{x})$ t.c. per ogni intorno $V$ di $\bar{x}$ si ha $f(V) \not
	\subseteq U$. Prendiamo un sistema fondamentale di intorni numerabile di
	$\bar{x}$, sia esso $V_n, \, n \in \mathbb{N}$, e come nella dimostrazione
	precedente lo prendiamo t.c. $V_{n+1} \subseteq V_n$ per ogni $n$.

	Prendiamo, per ogni $n$, un elemento $x_n \in V_n$ t.c. $f(x_n) \not\in U$,
	che esiste per l'ipotesi assurda. Allora si mostra, come nella dimostrazione
	precedente, che la successione $(x_n)$ tende a $\bar{x}$, ma le loro
	immagini $f(x_n)$ sono tutte fuori dallo stesso intorno $U$ di $f(\bar{x})$,
	dunque la successione $(f(x_n))$ non tende a $f(\bar{x})$, assurdo per
	ipotesi.
\end{proof}


\subsection{Sottospazi topologici}
\begin{defn}
    Sia $(X, \tau)$ uno spazio topologico, $Y \subseteq X$ un sottoinsieme,
    cio\`e esiste una mappa iniettiva $i: Y \hooklongrightarrow X$ di
    inclusione. La topologia ristretta su $Y$ da $\tau$ \`e la topologia meno
    fine che rende $i$ continua.
\end{defn}

\begin{oss}
    Per avere $i$ continua, serve che per ogni $U$ aperto di $\tau$, si abbia
    $i^{-1}(U)$ aperto nella topologia ristretta $\tau \restrict{Y}$. Poiché
    $i^{-1}(U) = U \cap Y$, si ha in effetti che per ogni aperto $U$ di $\tau$,
    $U \cap Y \in \tau \restrict{Y}$.

    Siccome $\sigma = \{U \cap Y \;|\; U \in \tau\}$ \`e una topologia su $Y$
    che rende continua $i$ e ${\tau\restrict{Y} \subseteq \sigma}$, deve valere
    l'uguaglianza in quanto $\tau\restrict{Y}$ \`e la meno fine con tali
    proprietà.
\end{oss}

\begin{oss}
    Vale anche che se $\mathcal{B}$ \`e base per $\tau$, allora
    \[
    \mathcal{B'} = \{B \cap Y \;|\; B \in \mathcal{B}\}
    \]
    \`e una base per $\tau\restrict{Y}$. La dimostrazione \`e analoga a quella
    fatta per le topologie.
\end{oss}

\begin{defn}
    Sia $(X, d)$ uno spazio metrico e $Y \subseteq X$ un sottoinsieme generico.
    La distanza $d$ pu\`o essere ristretta a $Y \times Y$. In tal caso
    $d\restrict{Y \times Y}$, che verr\`a indicata anche come $d\restrict{Y}$
    \`e una distanza su X.
\end{defn}

\begin{prop}
    Nel setting della definizione precedente, siano $\tau$ la topologia indotta
    da $d$ su $X$, $\tau\restrict{Y}$ la restrizione di $\tau$ a $Y$, e $\sigma$
    la topologia indotta da $d\restrict{Y}$ su $Y$. Allora $\tau\restrict{Y} =
    \sigma$.
\end{prop}

\begin{proof}
    Siano $\mathcal{B}$ e $\mathcal{D}$ rispettivamente basi per
    $\tau\restrict{y}$ e $\sigma$. Cio\`e:
    \begin{align*}
        \mathcal{B}\ =&\ \{Y\ \cap\ \{x \in X \tc d(x, x_0) < r,\quad x_0 \in
        X,\ r \in \mathbb{R}^+\}\}\\
        =&\ \{y \in Y \tc d(y, x_0) < r,\quad x_0 \in X,\ r \in \mathbb{R}^+\}
    \end{align*}
    \[
        \mathcal{D}\ =\ \{y \in Y \tc d(y, y_0) < r,\quad y_0 \in Y,\ r \in
        \mathbb{R}^+\}
    \]
    Chiaramente $\mathcal{D} \subseteq \mathcal{B}$, quindi anche $\sigma
    \subseteq \tau\restrict{Y}$.

    Inoltre, poiché le palle sono aperte, preso un $B \in \mathcal{B}$, per ogni
    $y \in B \cap Y$ esiste un $r_y>0$ tale che $D_y = B(y,r_y)$ sia contenuto
    in $B$. per cui si ha che
    \[
        B = \bigcup_{y \in B \cap Y} D_y.
    \]
    Ma ogni $D_y$ \`e un elemento della base $\mathcal{D}$, quindi ogni aperto
    di $\tau\restrict{Y}$ \`e un aperto di $\sigma$, concludendo l'ultima
    inclusione.
\end{proof}

\begin{defn}
    Siano $(X, \tau)$ spazio topologico, $Y \subseteq X$. Allora $Y$ si dice
    discreto in $X$ se $\tau\restrict{Y} = \mathcal{P}(Y)$, cioè la topologia
    ristretta a $Y$ \`e quella discreta.
\end{defn}

\begin{oss}
    Se $Y$ \`e discreto in $X$, allora per ogni $y \in Y$ esiste un aperto $U$
    di $X$ tale che $U \cap Y = \{y\}$.
\end{oss}

\begin{thm}
    \emph{Proprietà universale delle immersioni.} Sia $X$ spazio topologico e $Y
    \subseteq X$ con $i: Y \hooklongrightarrow X$ mappa di immersione. Allora
    per ogni spazio topologico $Z$ e per ogni funzione $g: Z \longrightarrow Y$,
    si ha che $f$ \`e continua se e solo se $i \circ f$ \`e continua.
\end{thm}

\begin{proof}
    Poiché $i$ \`e continua per la definizione della topologia su $Y$, e poiché
    la composizione di continue \`e continua, si ha che se $f$ \`e continua,
    anche $i \circ f$ lo \`e.

    Per l'altra implicazione, supponiamo $i \circ f$ continua e sia $A$ un
    aperto di $Y$. Allora esiste $U$ aperto di $X$ tale che ${U\cap Y = A}$, e
    vale ${i^{-1}(U) = A}$. Allora $(i\circ f)^{-1}(U) = f^{-1}(i^{-1}(U)) =
    f^{-1}(A)$ \`e aperto in $Z$.
\end{proof}

\begin{thm}
    \emph{Universalit\`a della proprietà universale.} La proprietà universale
    delle immersioni caratterizza in modo unico la topologia ristretta. Cio\`e
    dato $(X, \tau)$ spazio topologico con $Y \subseteq X$ e immersione ${i: Y
    \hooklongrightarrow X}$, $\tau\restrict{Y}$ \`e l'unica topologia su $Y$ che
    rispetta la proprietà universale.
\end{thm}

\begin{proof}
    Sia $\sigma$ una topologia su $Y$ che rispetta la propriet\`a universale.
    \begin{nlist}
        \item Prendo come $Z$ lo spazio $(Y, \tau\restrict{Y})$ e come $f$
        l'identità su $Y$. Il diagramma della proprietà universale \`e allora il
        seguente:

        \begin{center}\begin{tikzcd}
            (Y, \tau\restrict{Y}) \arrow[r, "\id"] \arrow[rd, "g=i"]
            & (Y, \sigma) \arrow[d, "i", hookrightarrow]\\
            & (X,\tau)
        \end{tikzcd}\end{center}

        Per definizione $g$ \`e continua, $i$ \`e continua, quindi $\id$ \`e
        continua. Allora $\tau\restrict{Y} \subseteq \sigma$.

        \item Questa volata prendo come $Z$ lo spazio $(Y, \sigma)$ mantenendo
        come $f$ l'identità. Il diagramma risulta essere:

        \begin{center}\begin{tikzcd}
            (Y, \sigma) \arrow[r, "\id"] \arrow[rd, "g=i"]
            & (Y, \sigma) \arrow[d, "i", hookrightarrow]\\
            & (X,\tau)
        \end{tikzcd}\end{center}

        Per definizione $\id$ \`e continua, quindi lo \`e anche $g=i$ per la
        propriet\`a universale. Allora $\sigma$ rende continua $i$, e dunque
        vale $\sigma \subseteq \tau\restrict{Y}$ per la definizione della
        restrizione di $\tau$.
    \end{nlist}
\end{proof}
\begin{defn}
    Sia $f:X\longrightarrow Y$ una funzione. Essa si dice \textsc{aperta} se
    manda aperti in aperti e \textsc{chiusa} se manda chiusi in chiusi.
\end{defn}
% TODO: dimostrazione?

\begin{oss}
    Sia $f$ come sopra continua e bigettiva. Allora essa \`e un omeomorfismo se
    e solo se \`e aperta, e se solo se \`e chiusa.
\end{oss}

\begin{ex}
    Si consideri l'insieme $\mathbb{Z}$ con la topologia $\tau =
    \mathcal{P}(\mathbb{N})\cup\mathbb{Z}$. Allora $f: n \longmapsto n-1$ \`e
    bigettiva, continua, non aperta e non omeomerfismo.
\end{ex}

\begin{defn}
    Sia $f:X \hooklongrightarrow Y$ una funzione iniettiva continua. Allora $f$
    si dice \textsc{immersione topologica} se per ogni $A \subseteq X$ aperto
    esiste un aperto $U \subseteq Y$ tale che $A = f^{-1}(U)$.
\end{defn}

\begin{oss}
    Sia $f: X \longrightarrow Y$ continua. Allora $f$ \`e chiusa e iniettiva se
    e solo se \`e un'immersione chiusa, se e solo se f \`e immersione con $f(X)$
    chiuso. Idem con gli aperti.
\end{oss}

\begin{oss}
    Sia $Y \subseteq X$ un sottospazio. Allora:
    \begin{nlist}
        \item se $X$ \`e N1 allora $Y$ \`e N1;

        \item idem con N2;

        \item se $X$ e separabile e metrizzabile allora $Y$ \`e separabile.
    \end{nlist}
\end{oss}

\begin{proof}
    I primi punti sono ovvi. Per quanto riguarda l'ultimo, se $X$ \`e
    metrizzabile e separabile, \`e N1. Dunque anche $Y$ \`e metrizzabile e N1, e
    quindi anche separabile.
\end{proof}

\begin{ex} \label{Sorgenfrey-piano}
    \emph{Il piano di Sorgenfrey}. Si prenda lo spazio $\mathbb{R}^2$ dotato
    della topologia generata dai rettangoli del tipo ${[a,b) \times [c.d)}$.
    Esso \`e chiamto il piano di Sorgenfrey. Si consideri il sottospazio della
    retta $y=-x$. La topologia indotta \`e la discreta, infatti per ogni punto
    $(k, -k)$, il rettangolo ${[k, k+1)\times[-k, 1-k)}$ interseca la retta solo
    nel punto considerato. Dunque si \`e trovato uno spazio separabile
    (prendendo per esempio $\mathbb{Q}^2$ come denso numerabile) che ha un
    sottospazio non separabile.
\end{ex}


\section{Prodotti e quozienti topologici}

\subsection{Topologie prodotto}
\begin{defn} 
	Sia $\{X_\alpha\}_{\alpha \in A}$ una famiglia di spazi topologici. Allora
	il \textsc{prodotto cartesiano} della famiglia è:
	$$X:=\prod_{\alpha \in A} X_\alpha = \{f:A \longrightarrow \bigcup _{\alpha
	\in A} X_\alpha  \mid f(\alpha)\in X_\alpha, \forall \alpha \in A\} $$
	Se $A=\{1, \dots ,n\}$, allora:
	$$\prod_{i=1}^n X_i=X_1 \times \cdots \times X_n \ni (x_1, \dots ,x_n)
	\qquad x_i:=f(i)$$
\end{defn}
\begin{defn}
	La \textsc{topologia prodotto} su $X=\begin{matrix} \prod_{i\alpha \in A}
	X_\alpha \end{matrix}	$ è la topologia meno fine su $X$ che rende
	continue tutte le proiezioni $p_\alpha : X \longrightarrow X_\alpha$.
\end{defn}
\begin{oss}
	$X$ viene naturalmente con proiezioni:
	$$p_\alpha :X \longrightarrow X_\alpha \qquad f \longmapsto f(\alpha) \qquad
	\forall \alpha \in A$$
	Inoltre questa topologia è ben definita poiché l'intersezione di topologie è
	una topologia.
\end{oss}
\begin{prop}
	Una base per la topologia prodotto su $X$ è data da:
	$$\mathcal{B}=\left \{\prod_{\alpha \in A} U_\alpha \mid U_\alpha \subset
	X_\alpha \text{ aperto }, U_\alpha =X_\alpha \text{ tranne al più 		un
	numero finito di }\alpha \right \}$$
\end{prop}
\begin{proof}
	Sia $U_\alpha \subset X_\alpha$ aperto. Allora:
	$$p_\alpha ^{-1} (U_\alpha)=\prod_{\beta \in A} V_\beta \qquad
	\text{dove}\qquad V_\beta =\begin{cases}U_\alpha & \text{se }\beta \alpha \\
	X_\beta & \text{se } \beta \ne \alpha \end{cases}$$
	Dunque le proiezioni $p_\alpha$ sono continue se e solo se tutte le
	preimmagini di tale forma sono aperte nella topologia prodotto.
    Quindi:
	$$p_\alpha ^{-1} (U_\alpha) \subset X \text{ aperto } \qquad \forall \text{
	aperto } U_\alpha \subset X_\alpha$$
	Siano $U_{\alpha _1} \subset X_{\alpha _1} , \dots , U_{\alpha _n} \subset
	X_{\alpha _n}$ aperti, con $\alpha _1, \dots, \alpha _n \in A$. Allora:
	$$X \supset \bigcap _{i=1}^n p_{\alpha _I}^{-1}(U_{\alpha _i}) = \prod
	_{\alpha \in A} V_\alpha \qquad \text{dove}\qquad V_\alpha =
	\begin{cases}U_{\alpha _i} & \text{se }\alpha =\alpha  _i \\ X_\alpha &
	\text{se } \alpha \ne \alpha _i \end{cases}$$
	Dunque ogni elemento di $\mathcal{B}$ è aperto nella topologia prodotto su
	$X$ (infatti l'intersezione di aperti è un aperto). Se $ \mathcal{B}$ è
	base di una topologia abbiamo finito, in quanto per definizione la topologia
	prodotto è la meno fine che rende continue tutte le $p_\alpha$.
	Altrimenti:
	$$\left( \prod_{\alpha \in A} U_\alpha \right) \bigcap \left( \prod_{\alpha
	\in A} V_\alpha \right)=\prod _{\alpha \in A} (U_\alpha \cap V_\alpha)$$
	Dunque le intersezioni di elementi di $\mathcal{B}$ sono ancora in
	$\mathcal{B}$, che quindi è chiuso rispetto alle intersezioni finite, e
	perciò è una base.
\end{proof}
\begin{oss}
	Supponiamo $\mathcal{B}_\alpha$ base della topologia di $X_\alpha$.
	Definiamo:
	$$\mathcal{B}'=\left \{ \prod_{\alpha \in A} B_\alpha \mid B_\alpha \in
	\mathcal{B}_\alpha, B_\alpha =X_\alpha \text{ tranne al più un numero
	finito di } \alpha \right \}$$
	Allora $\mathcal{B}'$ è una base per la topologia prodotto in $X$.
\end{oss}
\begin{cor}
	Supponiamo $A$ numerabile e $\mathcal{B}_\alpha$ base numerabile per
	$X_\alpha$ per ogni $\alpha$. Allora $\mathcal{B}'$ è una base numerabile
	per $X$.
\end{cor}
\begin{proof}
	Per esercizio.
\end{proof}
\begin{cor}
	(analogo al precedente) Supponiamo $A$ numerabile e che tutti gli $X_\alpha$
	siano primo-numerabili. Allora la topologia prodotto è	primo-numerabile.
\end{cor}
\begin{oss}
	In generale, se $A$ non è numerabile i due corollari sono falsi.
\end{oss}
\begin{ex}
	Prendiamo $\mathbb{R}$ con la topologia euclidea, e consideriamo:
	$$\mathbb{R}^n=\underbrace{\mathbb{R} \times \cdots \times 	\mathbb{R}}_
	{n\text{ volte}}$$
	Esso eredita una topologia prodotto che coincide con la topologia euclidea
	su $\mathbb{R}^n$. In particolare, la topologia euclidea ha per base le
	palle aperte. La topologia prodotto ha per base i parallelepipedi retti
	aperti:
	$$(a_1,b_1)\times \cdots \times (a_n,b_n)$$
	In effetti tali parallelepipedi sono aperti nella topologia euclidea (e
	viceversa).
\end{ex}


\subsection{Proprietà delle topologie prodotto}
\begin{prop}
	$p_\alpha :X \longrightarrow X_\alpha$ è un'applicazione aperta $\forall
	\alpha \in A$ (sempre prendendo $X=\begin{matrix} \prod_{\alpha \in A}
	X_\alpha \end{matrix}$).
\end{prop}
\begin{proof}
	Sia $\mathcal{B}$ la base di $X$, allora $p_\alpha$ è aperta se e solo se
	$p_\alpha (B)$ aperto $\forall B \in \mathcal{B}$.
	Supponiamo $B=\begin{matrix} \prod_{\alpha \in A} U_\alpha \end{matrix}$ con
	$U_\alpha \subset X_\alpha$ aperto. Allora $p_\alpha (B)=U_\alpha$ è
	aperto per definizione.
\end{proof}
\begin{oss}
	Le proiezioni $p_\alpha$ in generale non sono chiuse.
\end{oss}
\begin{ex}
	Prendiamo $p_1 :\mathbb{R}^2=\mathbb{R}\times \mathbb{R} \longrightarrow
	\mathbb{R}$ e sia $Z$ l'iperbole equilatera, ovvero $Z=\{xy \mid
	xy=1\}$. Allora $Z$ è chiuso in quanto $Z=p^{-1}(0)$ con $p=xy-1$, $p$
	continua e $0 \in \mathbb{R}$ è chiuso.  Invece, $p_1(\{xy \mid xy=1
	\})=\mathbb{R} \smallsetminus \{0\}$ non è chiuso perché $\{0\}$ non è
	aperto.
\end{ex}
\begin{prop}
	Dato $\alpha \in A$ fissiamo $x_\beta \in X_\beta$ $\forall \beta \ne
	\alpha$, e definiamo $X \supset X(\alpha)=\{f \in X \mid
	f(\beta)=x_\beta\}$. Allora la restrizione:
	$$p_\alpha \restrict {X(\alpha)} :X(\alpha) \longrightarrow X_\alpha$$
	è un omeomorfismo.
\end{prop}
\begin{proof}
	\begin{nlist}
    	\item $p_\alpha \restrict{X(\alpha)}$ è continua perche è restrizione di
    	$p_\alpha$, che è continua.
    	\item $p_\alpha \restrict{X(\alpha)}$ è bigettiva.
    	\item $p_\alpha \restrict{X(\alpha)}$ è aperta: gli aperti di
    	$X(\alpha)$ sono:
    	$$\prod _{\beta \in A} U_\beta \cap X(\alpha)$$
    	dove
    	$$\prod_{\beta \in A} U_\beta=U:=\{f \in X(\alpha) \mid f(\alpha) \in
    	U_\alpha \}$$
    \end{nlist}
    D'altra parte $p_\alpha \restrict{X(\alpha)}(U)=p_\alpha (U)=U_\alpha$, che
    è aperto in $X_\alpha$.
\end{proof}
\begin{prop}(\emph{Proprietà universale della topologia prodotto})\\
	La topologia prodotto ha la seguente proprietà: dato $Z$ spazio topologico e
	$f:Z \longrightarrow X$ funzione arbitraria, allora $f$ è continua se e
	solo se $p_\alpha \circ f$ è continua. Il diagramma risulta essere:
	\begin{center}\begin{tikzcd}
            & X \arrow[d, "p_\alpha"]\\
            Z \arrow[ru, "f"] \arrow[r, "p_\alpha \circ f"'] & X_\alpha
     \end{tikzcd}\end{center}
\end{prop}
\begin{proof}
	($\Rightarrow$) La composizione di funzioni continue è una funzione
	continua.\\
	($\Leftarrow$) Sia $U \subset X$ aperto. Vediamo se $f^{-1}(U)$ aperto. Ci
	basta vederlo per $U \in \mathcal{B}$, quindi:
	$$U=\prod_{\alpha \in A} U_\alpha \qquad \qquad \text{con }U_\alpha \subset
	X_\alpha \text{ aperto}$$
	Inoltre, $\exists A_0 \subset A$ finito tale che $U_\alpha =X_\alpha$
	($\forall \alpha \notin A_0$). D'altra parte:
	$$f^{-1}(U)=f^{-1}\left(\prod_{\alpha \in A} U_\alpha\right)=f^{-1}
	\left(\bigcap _{\alpha \in A} p_\alpha ^{-1}(U_\alpha) \right)=
	\bigcap_{\alpha \in A} (p_\alpha \circ f)^{-1}(U_\alpha)$$
	Dunque $f^{-1}(U)$ è un'intersezione finita di aperti.
\end{proof}
\begin{thm}
	La topologia prodotto è univocamente caratterizzata dalla proprietà
	universale.\\
	\underline{Equivalentemente}: Se $\tau _X$ è una topologia in $X$ con la
	proprietà:\\
	$\forall Z$ spazio topologico, $f:Z \longrightarrow X$, allora $f$ è
	continua se e solo se $p_\alpha \circ f$ è continua.
\end{thm}
\begin{proof}
	Abbiamo visto che la topologia prodotto soddisfa la proprietà. Siano $\tau
	_X$ una topologia che soddisfa la proprietà, $\tau _\alpha$ la topologia
	in $X_\alpha$ e $\tau _{pr}$ la topologia prodotto su $X$. Prendiamo
	$Z=(X,\tau _{pr})$, e otteniamo il diagramma:
		\begin{center}\begin{tikzcd}[column sep=small]
			& (X,\tau _X) \arrow[rd, "p_\alpha"] & \\
			(X,\tau _{pr})	\arrow[ru, "\id _X"] \arrow[rr, "p_\alpha"'] & &
			(X_\alpha,\tau _\alpha)
    		\end{tikzcd}\end{center}
    	Abbiamo mostrato che $p_\alpha :(X,\tau _{pr}) \longrightarrow
    	(X_\alpha, \tau _\alpha)$ è continua. Allora per la proprietà è continua
    	anche $\id _X$, quindi $\tau _X \subset \tau _{pr}$, ovvero $\tau _X <
    	\tau _{pr}$. Viceversa, per la minimalità di $\tau _{pr}$, basta
    	vedere che $p_\alpha :(X,\tau _X) \longrightarrow (X_\alpha, \tau
    	_\alpha)$ è continua $\forall \alpha \in A$. Prendiamo $Z=(X,\tau _X)$ e
    	$f=\id _X$. Allora otteniamo il diagramma:
    		\begin{center}\begin{tikzcd}
         	& (X,\tau _X) \arrow[dd, "p_\alpha"]\\
         	(X,\tau _X) \arrow[ru, "\id _X"] \arrow[rd, "p_\alpha"'] & \\
         	& (X_\alpha,\tau _\alpha)
         \end{tikzcd}\end{center}
     Da cui possiamo osservare che $\id _X$ è continua per la proprietà
     universale, e anche $p_\alpha$ lo è.
\end{proof}

Diamo adesso alcune informazioni generali che verranno dimostrate più avanti,
dopodiché vedremo alcune proprietà dei prodotti topologici nel caso degli spazi
metrici.

\begin{nlist}
\item Il prodotto di una quantità numerabile di spazi metrici è metrizzabile. In
generale, è falso se la quantità non è numerabile.
\item Il prodotto di una quantità al più continua di spazi separabili è
separabile.
\end{nlist}
\begin{prop}
	Sia $(X,d)$ spazio metrico. Allora $\bar{d}:X \times X \longrightarrow
	\mathbb{R}$ definita come:
	$$\bar{d}(x,y)=\min \{d(x,y),1\}$$
	è una distanza topologicamente equivalente a $d$. Dunque la topologia di uno
	spazio metrizzabile è indotta da una distanza $\le 1$.
\end{prop}
\begin{proof}
	Verifichiamo che $\bar{d}$ è una distanza. \\
	\begin{nlist}
	\item $\bar{d}(x,y) \ge 0 \quad \forall x,y \quad \text{e} \quad
	\bar{d}(x,y)=0 \Leftrightarrow x=y$ è ovvio dalla definizione.
	\item Anche la simmetria di $\bar{d}$ è ovvia dalla definizione
	\item Vediamo che $\bar{d}(x,z) \le \bar{d}(x,y)+\bar{d}(y,z) \quad \forall
	x,y,z$. Se almeno una tra $\bar{d}(x,y)$ e $\bar{d}(y,z)$ è uguale a 1
	la tesi è ovvia ($\bar{d}(x,z) \le 1$). Altrimenti:
	$$\bar{d}(x,z) \le d(x,z) \le d(x,y)+ d(y,z) =\bar{d}(x,y)+\bar{d}(y,z)$$
	\end{nlist}
	Dunque $\bar{d}$ è una distanza. Come base della topologia associata ad una
	distanza si possono prendere le palle di raggio $R$, al	variare di $R<1$.
	Ma $\forall x \in X, \forall R<1, B_d(x,R)=B_{\bar{d}}(x,R)$, perciò le
	topologie indotte coincidono.
\end{proof}
\begin{defn}
	$f:(X,d) \longrightarrow (Y,d')$ di dice $K$-\textsc{Lipschitz} se $\forall
	K>0,\forall x_1,x_2 \in X$:
	$$d'(f(x_1),f(x_2)) \le Kd(x_1,x_2)$$
	Inoltre, poiché $f(B(x,\frac{\varepsilon}{k}) \subseteq
	B(f(x),\varepsilon)$, una funzione $K$-Lipschitz è continua.
\end{defn}
\begin{thm}
Sia $\{(X_i,d_i)\}_{i \in \mathbb{N}}$ una famiglia di spazi metrici. Allora
$X=\prod _{i \in \mathbb{N}} X_i$ è metrizzabile.
\end{thm}
\begin{proof}
	Costruiamo $d:X \times X \longrightarrow \mathbb{R}$ distanza che induce
	$\tau _{pr}$ (la topologia prodotto). $\forall i \in \mathbb{N}		$ posso
	supporre $d_i \le 1$. Denotiamo con $(x_i)_{i \in \mathbb{N}}$ i punti di
	$X$, dove $x_i \in X_i$, $\forall i \in \mathbb{N}$. 				Poniamo:
	$$d(x,y)=\sum _{i=0}^\infty 2^{-i} \cdot d_i(x_i,y_i)$$
	che è $<+\infty$ poichè $d_i \le 1$. È facile verificare che $d$ è una
	distanza. Sia ora $\tau _d$ la topologia indotta, e mostriamo che $		\tau
	_{pr} =\tau _d$.\\
	Se $\pi _i:X \longrightarrow X_i$ è la proiezione su $X_i$, allora:
	$$d_i(\pi _i(x),\pi _i(y))=d_i(x_i,y_i)=2^i(2^{-i} \cdot d_i(x_i,y_i)) \le
	2^i d(x,y)$$
	Quindi $\pi _i$ è $2^i$-Lipschitz, dunque è continua. Allora ogni proiezione
	è continua rispetto a $\tau _d$, quindi $\tau _d >\tau _{pr}$. 		Vediamo
	adesso l'inclusione opposta ($\tau _d < \tau _{pr}$).\\
	Basta osservare che ogni palla di $d$ è aperta in $\tau _{pr}$. Sia
	$B=B_d(x,\varepsilon) \subseteq X$ e sia $y \in B$. Allora $\exists
	\delta >0$ tale che $B(y,\delta) \subseteq B$. Dobbiamo allora mostrare che
	$\exists U$ aperto di $\tau _{pr}$ con $y \in U \subseteq B_d(y,\delta)$.
	Sia quindi $n_0 \in \mathbb{N}$ tale che $\sum _{i=n_0+1}^\infty 2^{-i} <
	\delta /2$. Poniamo:
	$$U=\bigcap _{i=0}^{n_0} \pi _i ^{-1} \left( B\left(
	y_i,\dfrac{\delta}{4}\right) \right) =B_{d_0}\left( y_0,
	\dfrac{\delta}{4}\right) \times \cdots		\times B_{d_n}\left(
	y_n,\dfrac{\delta}{4}\right) \times X_{n_0+1} \times \cdots \times X_n
	\times \cdots$$
	Se $z \in U, \quad d_i(x_i,y_i) < \dfrac{\delta}{4} \quad \forall i \le
	n_0$, allora:
	$$d(y,z)=\sum _{i=0}^\infty 2^{-i} d_i(y_i,z_i)=\sum _{i=0}^{n_0} 2^{-i}
	d_i(y_i,z_i)+\sum _{i=n_0+1}^\infty 2^{-i} d_i(y_i,z_i)<$$
	$$<\dfrac{\delta}{4}\left( \sum _{i=0}^{n_0}2^{-i} \right) +\sum
	_{i=n_0+1}^\infty 2^{-i} < \dfrac{\delta}{4} \cdot
	2+\dfrac{\delta}{2}=\delta		$$
	Dunque $U \subseteq B(y,\delta)$.
\end{proof}
\begin{oss}
	In realtà potevamo usare una qualsiasi serie convergente a termini positivi
	al posto di $\sum _{i=0}^\infty 2^{-i}$.
\end{oss}
\begin{oss}
	Genericamente, se il prodotto più che numerabile di spazi metrici non è
	primo-numerabile allora non è metrizzabile.
\end{oss}
\begin{prop}(\emph{Prodotti finiti})
	Se $(X,d)$ e $(Y,d')$ sono spazi metrici, allora $X \times Y$ è
	metrizzabile, e ha topologia indotta da una qualsiasi delle seguenti
	distanze:
	\begin{nlist}
	\item $d_\infty ((x_1,y_1),(x_2,y_2))=\max \{d(x_1,x_2),d'(y_1,y_2)\}$
	\item $d_2 ((x_1,y_1),(x_2,y_2))=\sqrt{d(x_1,x_2)^2+d'(y_1,y_2)^2}$
	\item $d_1 ((x_1,y_1),(x_2,y_2))=d(x_1,x_2)+d'(y_1,y_2)$
	\end{nlist}
\end{prop}
\begin{proof}
	Si può verificare che $d_1,d_2,d_\infty$ sono equivalenti, come già visto su
	$\mathbb{R}^n$. Inoltre, è facile vedere che inducono la topologia
	prodotto.
\end{proof}

\begin{thm}
    (Lemma di Zorn). Dato un insieme $A$ parzialmente ordinato non vuoto, se ogni catena di $A$ ammette maggiorante, allora $A$ ammette almeno un elemento massimale.
\end{thm}
\begin{oss}
    Il lemma di Zorn \`e equivalente all'assioma della scelta, e la sua dimostrazione \`e ben al di fuori dagli scopi del corso.
\end{oss}
\begin{thm}
    (di Alexander). Sia $X$ uno spazio topologico e sia $\mathcal{P}$ una sua prebase. Se da ogni ricoprimento di $X$ con elementi di $\mathcal{P}$ si pu\`o estrarre un sottoricoprimento finito, allora $X$ \`e compatto.
\end{thm}
\begin{proof}
    Per asurdo, si $X$ non compatto. Definiamo
    \[
        \Omega = \{\mathcal{F} \text{ ricoprimento di $X$ da cui non si pu\`o estrarre un sottoricoprimento finito}\}
    \]
    Ordiniamo $\Omega$ con l'inclusione tra insiemi. Voglio applicare il lemma di Zorn a $\Omega$ per dire che contiene un massimale. Per l'ipotesi dell'assurdo $\Omega$ non \`e vuoto. Sia allora $C$ una catena,
    \[
        C = \{U_i\}_{i\in I}
    \]
    con tutti gli $U_i$ ricoprimenti aperti. Definisco
    \[
        U = \bigcup_{i\in I}U_i
    \]
    Chiaramente $U$ \`e un maggiorante di $C$. Voglio far vedere che $U \subseteq\Omega$. Allora per assurdo siano $A_1, \dots A_n$ aperti di $U$  che ricoprono $X$. Ogni $A_j$ appartiene ad un certo $U_{i_j}$. Poich\'e $C$ \`e una catena, tutti gli $U_{i_j}$ sono ``inscatolati'', e quindi vi \`e un $U_s$ fra questi che contiene tutti gli $A_j$. Ma questo non \`e possibile perch\'e gli $U$ erano scelti tali che non si poteva estrarre un sottoricoprimento finito.

    Quindi posso effettivamente applicare Zorn e ottenere un ricoprimento $Z$ massimale.

    Se mostro che $\mathcal{P}\cap Z$ \`e ricoprimento di $X$ ho l'assurdo che cercavo perch\'e da quella intesezione non posso estrarre alcun sottoricoprimento finito (per come \`e costruito $Z$).

    Sia $x\in X$, $Z = \{Z_i\}_{i\in I}$. Allora esiste $i$ per cui $x\in Z_i$. Dunque esiste un $B$ nella base associata a $\mathcal{P}$ per cui $x\in B\subseteq Z_i$. Posso quindi prendere $P_1, \dots, P_n$ in $\mathcal{P}$ tali che la loro intersezione sia $B$. Vorrei trovare un $P_j$ che sta anche in $Z$. Per assurdo supponiamo che nessun $P_j$ sia in $Z$. Allora considero ${\{P_j \} \cup Z}$, che per massimalit\`a di $Z$ non appartiene a $\Omega$.
    Allora potrei scegliere un ricoprimento finito $P_j, V_{j,1}, \dots, V_{j.m}$, con i $V_{j,i}$ in $Z$. Ma quindi avrei
    \[
        X = \left(\bigcup_{j=1}^nP_j\right)\cup \bigcup_{j=1}^n\bigcup_{s=1}^m V_{j,s} \subseteq Z_i \cup \bigcup_{j=1}^n\bigcup_{s=1}^m V_{j,s} = X.
    \]
    Questo \`e assurdo in quanto ho appena trovato un sottoricoprimento finito di $Z$.
    Quindi ho finito, osservando che $P_j\in \mathcal{P}\cap Z$.
\end{proof}

\begin{thm}
    (di Tychonoff). Siano $X_i$ con $i\in I$ spazi compatti. allora
    \[
        X=\prod_{i\in I}X_i
    \]
    \`e compatto.
\end{thm}
\begin{proof}
    Per il teorema di Alexandroff, mi basta verificare la compattezza con ricoprimenti di aperti presi da una prebase del prodotto. Scelgo la prebase canonica
    \[
        U = \bigcup_{i\in I}A_i;\quad A_i = \{\pi_i^{-1}(D)\ | \ D \in \mathcal{D}_i\}
    \]
    dove i $\pi_i$ sono le proiezioni sulle componenti e dove ogni $\mathcal{D}_i$ \`e un ricoprimento aperto di $X_i$. 
\end{proof}


\subsection{Assiomi di separazione}
In un generico spazio topologico, vorremmo usare gli aperti per dire in qualche modo che punti o insiemi (che prenderemo chiusi) disgiunti sono separati anche a livello della topologia. Per questo richiediamo che gli spazi soddisfino alcune condizioni, i cosiddetti \textsc{assiomi di separazione}. Negli enunciati degli assiomi $X$ è uno spazio topologico con una topologia data. Ecco i primi due.

(T1) $\forall \, x, y \in X, x \not=y$ esistono aperti $U, V \subseteq X$ con $x \in U \setminus V, y \in V \setminus U$;

(T2) $\forall \, x, y \in X, x \not=y$ esistono aperti $U, V \subseteq X$ con $U \cap V= \emptyset, x \in U, y \in V$.

\begin{defn} Se $X$ gode di T2 è detto \textsc{spazio di Hausdorff}, o spazio topologico \textit{separato}.
\end{defn}

Chiaramente T2 $\implies$ T1.

\begin{prop}
  Uno spazio topologico $X$ è T1 $\Leftrightarrow$ i punti sono chiusi.
\end{prop}

\begin{proof}
  ($\implies$) Supponiamo $X$ sia T1 e sia $x \in X$. Per ogni $y \in X, y \not=x$ esistono $U_y, V_y$ aperti che soddisfano le condizioni imposte da T1. Ma allora $x \not \in V_y, y \in V_y$, da cui $\displaystyle X \setminus \{x\}=\bigcup_{y \in X, y \not=x} V_x$, essendo unione di aperti è aperto, perciò il suo complementare, $\{x\}$, è chiuso, come voluto.

  ($\Leftarrow$) Supponiamo adesso che i punti di $X$ siano chiusi. Allora, dati $x, y \in X, \\ x \not= y$, scegliendo gli aperti $U=X \setminus \{y\}, V=X \setminus \{x\}$ si ha che questi soddisfano le condizioni di T1.
\end{proof}

\begin{oss}
  Segue facilmente dalla proposizione sopra che in uno spazio T1 tutti gli insiemi cofiniti sono aperti. D'altra parte, se tutti i cofiniti sono aperti i punti sono chiusi. Dunque, se abbiamo uno spazio topologico generico $(X, \tau)$, possiamo affermare che esso è T1 $\Leftrightarrow$ $\tau$ è più fine della topologia cofinita.
\end{oss}

\begin{prop}
  $X$ è di Hausdorff $\Leftrightarrow$ $\Delta_X \subseteq X \times X$ è chiuso, dove $\Delta_x=\{ (x, x) | x \in X \}$ è la \textit{diagonale di $X$}.
\end{prop}

\begin{proof}
  ($\implies$) Sia $(x, y) \in X \times X \setminus \Delta_X$. Allora $x \not= y$, dunque esistono due aperti disgiunti $U_x, V_y$ t.c. $x \in U_x, y \in U_y$, da cui $(x, y) \in U_x \times V_y$, che è un aperto della topologia prodotto. Dunque $\displaystyle X \times X \setminus \Delta_X=\bigcup_{(x, y) \in X \times X \setminus \Delta_X} U_x \times V_y$ è unione di aperti, dunque è aperto e il complementare, $\Delta_X$, è chiuso.

  ($\Leftarrow$) Supponiamo adesso $\Delta_X$ chiuso. Dati due punti $x, y \in X, x \not= y$ la coppia $(x, y)$ sta nel complementare di $\Delta_X$, che è aperto. Esiste allora un elemento della base della topologia prodotto su $X \times X$ tutto contenuto in $X \times X \setminus \Delta_X$ che contiene $(x, y)$. Ma questo è della forma $U \times V$ per qualche $U$ e $V$ aperti con $x \in U, y \in V$ e dato che non interseca $\Delta_X$, significa che non possiamo trovare lo stesso elemento in entrambi gli insiemi, cioè $U \cap V= \emptyset$. Segue allora che $X$ è di Hausdorff.
\end{proof}

\begin{ex}
  Un esempio di spazio di Hausdorff è un qualsiasi spazio metrizzabile. Lasciamo come semplice esercizio la dimostrazione che metrizzabile implica Hausdorff.

  Un esempio, invece, di uno spazio T1 ma non T2, è un qualsiasi spazio infinito con la topologia cofinita. Lasciamo anche in questo caso la verifica di quanto appena detto come semplice esercizio.
\end{ex}

\begin{prop}
  Sottospazi e prodotti arbitrari di spazi topologici T1 (rispettivamente T2) sono ancora T1 (rispettivamente T2).
\end{prop}

\begin{proof}
  Per quanto riguarda i sottospazi, basta prendere l'intersezione degli aperti in questione con il sottospazio stesso ed è semplice verificare che rispettano ancora le condizioni, sia per T1 che per T2.

  Per quanto riguarda i prodotti, ricordando la loro definizione come insiemi di funzioni, se due funzioni differiscono per almeno un elemento allora basta prendere gli aperti dati dall'assioma del caso (T1 o T2) per lo spazio topologico relativo a quell'elemento e tutto lo spazio per tutti gli altri elementi.
\end{proof}

\begin{prop}
  Questa proposizione segue dal fatto che se abbiamo più aperti allora di sicuro possiamo trovarne che rispettano le condizioni degli assiomi, anzi, magari ne troviamo anche più di prima. Sia dunque $X$ un insieme e $\tau_1< \tau_2$ topologie su $X$, se $\tau_1$ è T1 (rispettivamente T2), allora anche $\tau_2$ è T1 (rispettivamente T2).
\end{prop}

Vediamo ora alcune applicazioni di questi primi due assiomi di separazione.


\subsection{Quozienti topologici}
\begin{defn}
    Siano $X$ uno spazio topologico e $\sim$ una relazione di equivalenza
    definita su $X$. Vengono allora indotti un insieme $Y = \faktor{X}{\sim}$ e
    una proiezione $\pi\colon X\longrightarrow Y$ che manda ogni punto nella sua
    classe di equivalenza. La topologia quoziente definita su $Y$ \`e la
    topologia pi\`u fine che rende $\pi$ continua.
\end{defn}

\begin{defn}
    Data la proiezione $\pi$ come sopra, un sottoinsieme $Z \subseteq X$ si dice
    saturo se \`e unione di classi di equivalenza per $\sim$. Cio\`e ${Z =
    \pi^{-1}\pi(Z)}$.
\end{defn}

\begin{oss}
    Nel setting di prima, detta $\tau_Y$ la topologia quoziente, si ha che
    \[
    A\in \tau_Y \Longleftrightarrow
    \pi^{-1}(A) \text{\ aperto} \Longleftrightarrow
    A = \pi(B) \text{\ con $B$ aperto saturo}.
    \]
\end{oss}

\begin{thm}
    \emph{Propriet\`a universale del quoziente.} Siano $X$ un insieme con una relazione di equivalenza $\sim$ e una proiezione $\pi$. Allora la topologia quoziente \`e l'unica topologia per cui per ogni spazio topologico $Z$ e funzione ${f\colon \faktor{X}{\sim} \longrightarrow Z}$, $f$ \`e continua se e solo se $f \circ \pi$ \`e continua.
\end{thm}
\begin{proof}
    Il seguente diagramma \`e commutativo.

    \begin{center}\begin{tikzcd}
        X \arrow[dd, "\pi", two heads] \arrow[rd, "f\circ \pi"] &   \\
        & Z \\
            \faktor{X}{\sim} \arrow[ru, "f"] &
    \end{tikzcd}\end{center}

    Come prima cosa verifichiamo che la topologia quoziente ha la propriet\`a universale. Un'implicazione \`e facile: se $f$ e $\pi$ sono continue, anche la loro composizione lo \`e. Viceversa sia $f\circ \pi$ continua, e si prenda un aperto in Z. Allora $(f \circ \pi)^{-1}(A) = (\pi^{-1} \circ f^{-1})(A)$ \`e aperto, e dunque anche $f^{-1}(A)$ \`e aperto.
    % TODO: forse qui bisogna spiegarlo meglio.

    Mostriamo ora che se una certa topologia $\sigma$ gode della propriet\`a universale, allora \`e uguale alla topologia quoziente $\tau$.

    Si prenda $Z = (\faktor{X}{\sim}, \tau)$ e $f$ l'identit\`a. Allora poich\'e $\pi$ \`e continua con la topologia $\tau$, si ha che
    \[
        \id\colon \faktor{X}{\sim}, \sigma \longrightarrow
        \faktor{X}{\sim}, \tau
    \]
    \`e dunque $\sigma$ \`e pi\`u fine di $\tau$.

    Viceversa, prendendo per $Z$ lo stesso insieme ma dotato della topologia $\sigma$ e $f = \id$, si ottiene che $\pi$ \`e continua anche con $\sigma$. Per definizione di topologia quoziente, allora $\tau$ \`e pi\`u fine di $\sigma$.
\end{proof}

\begin{defn}
    Una funzione $f\colon X \longrightarrow Y$ continua e surgettiva si chiama identificazione quando per ogni $A \subseteq Y$, $A$ \`e aperto se e solo se $f^{-1}(A)$ \`e aperto.
\end{defn}

\begin{oss}
    Sia $f$ una bigezione continua. Sono equivalenti:
    \begin{nlist}
        \item $f$ \`e immersione;
        \item $f$ \`e omeomorfismo;
        \item $f$ \`e identificazione.
    \end{nlist}
\end{oss}

\begin{oss}
    Sia $f$ continua surgettiva. Allora:
    \begin{nlist}
        \item Se $f$ \`e aperta, \`e identificazione.
        \item Se $f$ \`e chiusa, \`e identificazione.
        \item Ogni altra implicazione \`e abusiva.
    \end{nlist}
\end{oss}

\begin{prop}
    Sia $f\colon X \longrightarrow Y$ un'identificazione. Allora la funzione $\bar{f}$ definita dal seguente diagramma commutativo \`e un omeomorfismo.

    \begin{center}
        \begin{tikzcd}
            X \arrow[d, "\pi"'] \arrow[r, "f"]     & Y \\
            \faktor{X}{\sim} \arrow[ru, "\bar{f}"'] &
        \end{tikzcd}
    \end{center}
\end{prop}
\begin{proof}
    Per come \`e definita $\bar{f}$, essa \`e ben definita e bigettiva. Poich\'e $f$ \`e continua, per la propriet\`a universale, anche $\bar{f}$ \`e continua. Voglio mostrare che $\bar{f}$ \`e aperta. Sia allora $A$ un aperto di $\faktor{X}{\sim}$. Si ha che $\bar{f}(A)=(f\circ \pi^{-1})(A)$ \`e aperto se e solo se ${ (f^{-1} \circ f \circ \pi ^{-1})(A)}$ \`e aperto perch\'e $f$ \`e identificazione. Poich\'e $\pi$ \`e la proiezione sulla topologia quoziente, $A$ \`e aperto.
\end{proof}

\begin{oss}
    Si ha che $\bar{f}$ \`e un omeomorfismo se e solo se $f$ \`e un'identificazione.
\end{oss}

\begin{defn}
    Sia $A \subseteq X$. Si definisce il quoziente
    \[
    \faktor{X}{A} = \faktor{X}{\sim}
    \]
    dove $x\sim x'$ se e solo se $x = x'$ oppure $x,x'\in A$. Cio\`e in questo quoziente $A$ ``collassa'' in un punto.
\end{defn}

\begin{ex}
    L'intervallo $[0, 1]$ quozientato sul suo sottoinsieme $\{0,1\}$ \`e isomorfo alla circonferenza $S^1$.
\end{ex}
\begin{proof}
    % TODO: inserire dimostrazione.
\end{proof}


\subsection{Quozienti per azioni di gruppi}
Consideriamo $X$ spazio topologico e $G$ gruppo. Diciamo che $G$ agisce su $X$ tramite omeomorfismi, cioè:
$$G \longrightarrow \text{Homeo}(X)$$
è un omomorfismo di gruppi. Consideriamo allora $\faktor{X}{G}$ con la relazione di equivalenza di essere nella stessa orbita, cioè:
$$x \sim y \Longleftrightarrow \exists g \in G: g \cdot x=y$$
In altre parole, le classi di equivalenza sono le orbite di $G$ in $X$.
\begin{ex}
$\mathbb{Z}$ agisce su $\mathbb{R}$ per traslazione:
$$n \cdot x=n+x$$
Più in generale, $\mathbb{Q}$ agisce su $\mathbb{R}$ allo stesso modo.
\end{ex}

\begin{oss}(\emph{Osservazione notazionale})
Il quoziente $\faktor{\mathbb{R}}{\mathbb{Z}}$ può indicare due cose ben distinte:
\begin{nlist}
\item $\mathbb{Z} \subset \mathbb{R}$ sottoinsieme $\Longrightarrow \faktor{\mathbb{R}}{\mathbb{Z}}$ bouquet infinito numerabile di circonferenze
\item $\mathbb{Z}$ agisce su $\mathbb{R}$ come nell'esempio sopra $\Longrightarrow S^1$
\end{nlist}
\end{oss}

\begin{prop}
$\mathbb{Z}$ agisce su $\mathbb{R}$ come nell'esempio sopra $\Longrightarrow S^1$
\end{prop}
\begin{proof}
Consideriamo:
\begin{align*}
f:\mathbb{R} &\longrightarrow S^1 \\
t &\longmapsto (\cos(2\pi t), \sin (2\pi t))
\end{align*}
Se vediamo $f$ identificazione, allora abbiamo finito in quanto le fibre di $f$ sono proprio le orbite di $\mathbb{Z}$. Sappiamo inoltre che $f$ è continua e surgettiva, vediamo allora che è aperta. Ci basta guardare $f(I)$ con $I \subset \mathbb{R}$ intervallo aperto. Sia $a \in \mathbb{R}$ e osserviamo che, preso $I=(a,a+1)$ si ha che:
$$f \restrict {(a,a+1)}:(a,a+1) \longrightarrow S^1 \smallsetminus \{f(a)\}$$
è invertibile con inversa continua. Dunque $f \restrict {I}$ è un omeomorfismo. Segue allora che $f \restrict {I}$ è aperta per ogni aperto $A$. Infatti se $A$ è contenuto in un intervallo $I$ di ampiezza 1 allora $f(A)$ è aperto per l'osservazione. Altrimenti, più in generale, possiamo scrivere $A= \bigcup _{j \in J} A_j$ dove $A_j$ è un aperto contenuto in intervalli di ampiezza 1. Allora:
$$f(A)=f\left(\bigcup A_j\right)=\bigcup f(A_j)$$
che è aperto in quanto unione di aperti.
\end{proof}

\begin{prop}
Sia $G$ gruppo che agisce su $X$ spazio topologico (per omeomorfismi). Allora la proiezione al quoziente $\pi:X \longrightarrow \faktor {X}{G}$ è aperta. Inoltre, se $G$ è finito, $\pi$ è anche chiusa.
\end{prop}
\begin{proof}
Se $U \subset X$ è un aperto, allora $\pi (U)=\pi \left(\bigcup _{g \in G} gU\right)$ è un aperto saturo. Inoltre, poiché $G$ agisce tramite omeomorfismi, $gU$ è aperto $\forall g \in G$. Se $G$ è finito, il ragionamento è analogo per i chiusi:
$$C \subset X \text{ chiuso } \Longrightarrow f(C)=f\left(\bigcup _{g \in G} gC\right) \Longrightarrow f(C) \text{ chiuso}$$
\end{proof}

\begin{oss}
Se $G$ è infinito, in generale non è vero che $\pi$ è chiusa.
\end{oss}

\begin{oss}
Siano $f_i:X_i \longrightarrow Y_i$ identificazioni per $i=1,2$. Allora l'applicazione:
$$f_1 \times f_2:X_1 \times X_2 \longrightarrow Y_1 \times Y_2$$
è continua e surgettiva, ma in generale non è un'identificazione. Inoltre il prodotto cartesiano di identificazioni aperte è un'identificazione aperta.
\end{oss}

\begin{prop}
Sia $X$ spazio di Hausdorff e $G$ gruppo che agisce su $X$ tramite omeomorfismi. Sia $K=\{(x,gx) \mid x \in X, g \in G \} \subset X \times X$. Allora $\faktor {X}{G}$ è di Hausdorff se e solo se $K$ è chiuso in $X \times X$.
\end{prop}
\begin{proof}
In altre parole, possiamo riformulare l'enunciato come:
$$\faktor{X}{G} \text{ Hausdorff } \Longleftrightarrow \Delta_{\faktor{X}{G}} \subset \faktor{X}{G} \times \faktor{X}{G} \text{ chiusa}$$
Sia $\pi:X \longrightarrow \faktor{X}{G}$ la proiezione, allora $\pi$ è un'identificazione aperta. Ma allora $\pi \times \pi$ è a sua volta identificazione aperta. Dunque $\Delta _{\faktor{X}{G}}$ è chiusa se e solo se $(\pi \times \pi)^{-1}(\Delta _{\faktor{X}{G}}) \subset X \times X$ è chiuso. D'altra parte $(\pi \times \pi)^{-1}(\Delta _{\faktor{X}{G}})=K$.
\end{proof}

\section{Connessioni e compattezza}

\subsection{Ricoprimenti}
Sia $X$ spazio topologico e $\mathcal{U} \subset \mathcal{P}(X)$.

\begin{defn}
$\mathcal{U}$ è un \textsc{ricoprimento di $X$} se $X=\bigcup _{U \in \mathcal{U}}U$. Se tutti gli $U \in \mathcal{U}$ sono aperti [chiusi] allora diremo che $\mathcal{U}$ è un \textsc{ricoprimento aperto [ricoprimento chiuso]}.
\end{defn}

\begin{defn}
$\mathcal{U} \subset \mathcal{P}(X)$ è detto \textsc{famiglia localmente finita} se $\forall x \in X, \exists V \in I(x)$ tale che $V \cap U \neq \emptyset$ solamente per un numero finito di $U \in \mathcal{U}$.
\end{defn}

\begin{ex}
$\mathbb{R}=\bigcup _{n \in \mathbb{Z}}[n,n+1]$ è un ricoprimento chiuso localmente finito.
\end{ex}

\begin{defn}
$\mathcal{U}$ ricoprimento di $X$ è detto \textsc{fondamentale} se, dato $A \subset X$, allora:
\begin{align*}
A \text{ aperto in }X &\Longleftrightarrow A \cap U \text{ aperto in }U \qquad &&\forall U \in \mathcal{U} \\
(\underline{\text{Equivalentemente}} \qquad C \text{ chiuso in }X &\Longleftrightarrow C \cap U \text{ chiuso in }U \qquad &&\forall U \in \mathcal{U})
\end{align*}
\end{defn}

\begin{ex}
\begin{nlist}
\item $\mathcal{U}$ ricoprimento aperto $\Longrightarrow \mathcal{U}$ ricoprimento fondamentale.
\item $\mathbb{R}=\bigcup _{x \in \mathbb{R}} \{x\}$ è un ricoprimento chiuso. D'altra parte però, $A \cap \{x\}$ è aperto in $\{x\}$ $\forall x \in \mathbb{R}$, quindi il ricoprimento non è fondamentale.
\end{nlist}
\end{ex}

\begin{prop}
Sia $\mathcal{U}$ ricoprimento fondamentale di $X$, e sia $f:X \longrightarrow Y$ funzione tra spazi topologici. Allora $f$ è continua se e solo se $f \restrict {U}$ è continua $\forall U \in \mathcal{U}$.
\end{prop}
\begin{proof}

\end{proof}

Enunciamo ora un lemma che ci permetterà di dimostrare un teorema particolarmente importante.

\begin{lm}
Un'unione localmente finita di chiusi è chiusa.
\end{lm}
\begin{proof}
Sia $\{C_i\}_{i \in I}$ una famiglia localmente finita di chiusi. $\forall x \in X$, $\exists U_x \subseteq X$ aperto, con $x \in U_x$ che interseca solo un numero finito di $C_i$. Sia $\mathcal{U}=\{U_x\}_{x \in X}$ il ricoprimento aperto così definito. $\mathcal{U}$ è fondamentale, per cui basta vedere che $\left(\bigcup _{i \in I} C_i\right) \cap U_x$ è chiuso in $U_x$, $\forall x \in X$. Ma, fissato $x$, esistono $i_1, \dots ,i_n$ tali che:
$$U_x \cap \left(\bigcup _{i \in I} C_i\right) = U_x \cap (C_{i_1} \cup \cdots \cup C_{i_n})$$
che è chiuso in $U_x$ in quanto $C_{i_1} \cup \cdots \cup C_{i_n}$ è unione finita di chiusi.
\end{proof}

\begin{cor}
In generale, se $\{Y_i\}_{i \in I}$ è una famiglia localmente finita, allora:
$$\overline{\bigcup _{i \in I} Y_i}=\bigcup _{i \in I} \overline{Y_i}$$
\end{cor}
\begin{proof}
$$Y_{i_0} \subseteq \bigcup _{i \in I} Y_i \Longrightarrow \overline{Y_{i_0}} \subseteq \overline{\bigcup _{i \in I} Y_i} \quad \forall i_0 \Longrightarrow \bigcup _{i \in I} \overline{Y_i} \subseteq \overline{\bigcup _{i \in I} Y_i}$$
Per il viceversa, si mostra prima che se gli $Y_i$ sono localmente finiti, anche gli $\overline{Y_i}$ lo sono. Dunque, per il lemma appena visto, $\bigcup _{i \in I} \overline{Y_i}$ è un chiuso che contiene $\bigcup _{i \in I} Y_i$. Per cui:
$$\overline{\bigcup _{i \in I} Y_i} \subseteq \bigcup _{i \in I} \overline{Y_i}$$
\end{proof}

Possiamo adesso enunciare e dimostrare il seguente teorema:
\begin{thm}
Un ricoprimento chiuso localmente finito è fondamentale.
\end{thm}
\begin{proof}
Sia $\{C_i\} _{i \in I}$ un ricoprimento chiuso localmente finito, e sia $Z \subseteq X$ tale che $Z \cap C_i$ sia chiuso in $C_i$, $\forall i \in I$. Poiché $C_i$ è chiuso, e un chiuso di un chiuso di $X$ è chiuso in $X$, allora $Z \cap C_i$ è chiuso in $X$ $\forall i \in I$. Dunque la famiglia $\{Z \cap C_i \} _{i \in I}$ è una famiglia localmente finita di chiusi. Allora, per il lemma, $Z=\bigcup _{i \in I} (Z \cap C_i)$ è chiuso in $X$, da cui la tesi.
\end{proof}

\subsection{Connessioni}
\begin{defn}
Uno spazio topologico $X$ si dice \textsc{sconnesso} se vale una delle seguenti condizioni (equivalenti):
\begin{nlist}
\item $X=A \sqcup B$ (cioè $X=A \cup B$ con $A \cap B = \emptyset$) con $A,B$ aperti non vuoti. Allora in particolare $A$ e $B$ sono anche chiusi, in quanto sono uno il complementare dell'altro.
\item $X=A \sqcup B$ con $A,B$ chiusi non vuoti
\item $\exists A \subseteq X, A \neq \emptyset, A \neq X, A$ è sia aperto che chiuso
\end{nlist}
\end{defn}

\begin{defn}
$X$ si dice \textsc{connesso} se non è sconnesso. In altre parole, se: \\ $\forall A \subseteq X, A \neq \emptyset, A$ aperto e chiuso, allora $A=X$.
\end{defn}

\begin{ex}
$\mathbb{R} \smallsetminus \{0\}$ è sconnesso in quanto unione degli aperti $(-\infty,0)$ e $(0,+\infty)$.
\end{ex}

\begin{thm}
$[0,1] \subset \mathbb{R}$ è connesso.
\end{thm}
\begin{proof}
Siano $A,B$ aperti non vuoti di [0,1], con $[0,1]=A \sqcup B$. Posso supporre $0 \in A$. Dunque, se $t_0=\inf B$ (esiste poiché $B$ è non vuoto e limitato inferiormente), allora $t_0 \ge \varepsilon >0$, perché altrimenti avrei $B \cap [0,\varepsilon) \neq \emptyset$, che contraddice le ipotesi. $B$ è chiuso ($A$ e $B$ sono entrambi aperti e chiusi), quindi $t_0 \in B$. Ma $B$ è anche aperto e $t_0 >0$, per cui $\exists \delta >0$ con $(t_0-\delta, t_0] \subseteq B$, e ciò contraddice il fatto che $t_0=\inf B$.
\end{proof}

\begin{defn}
$X$ si dice \textsc{connesso per archi} se $\forall x_0,x_1 \in X$, \\
$\exists \alpha:[0,1] \longrightarrow X$ continua tale che $\alpha (0)=x_0, \alpha (1)=x_1$.
\end{defn}

\subsection{Compattezza}
\begin{defn}
  Uno spazio topologico $X$ è \textsc{compatto} se ogni suo ricoprimento aperto ammette un sottoricoprimento finito, cioè se dato $\mathcal{U}=\{U_i\}_{i \in I}$ t.c. $\displaystyle X=\bigcup_{i \in I} U_i$ esistono $i_0, \dots, i_n$ t.c. $X=U_{i_0} \cup \dots \cup U_{i_n}$. Un sottospazio $Y \subseteq X$ è compatto se lo è con la topologia di sottospazio.
\end{defn}

\begin{defn}
  Uno spazio metrico $X$ è \textsc{limitato} se esistono $X_0 \in X$ e $R>0$ t.c. $X=B(x_0, R)$ o equivalentemente se $diam(X)=\sup{\{d(x, y), x, y \in X\}}<+\infty$.
\end{defn}

\begin{lm}
  Se $X$ è metrico compatto, allora $X$ è limitato.
\end{lm}

\begin{proof}
  Scelto $x_0 \in X$, $U_n=B(x_0, n), n \in \mathbb{N}$ è un ricoprimento aperto di $X$. Esistono allora $i_0, \dots, i_n$ t.c. $X=U_{i_0} \cup \dots \cup U_{i_n}$. Poniamo allora $R=\max\{i_0, \dots, i_n\}$ e otteniamo $X=B(x_0, R)$.
\end{proof}

\begin{cor}
  $\mathbb{R}$ non è compatto.
\end{cor}

\begin{oss}
  La compattezza è invariante per omeomorfismo, dunque $(-1, 1)$ non è compatto in quanto omeomorfo a $\mathbb{R}$ tramite $x \longmapsto \tan{\frac{\pi}{2}x}$, con inversa $y \longmapsto \frac{2}{\pi}\tan^{-1}{y}$. Segue, per omeomorfismo affine con $(-1, 1)$, che nessun intervallo aperto $(a, b)$ è compatto.
\end{oss}

\begin{cor}
  Metrico e limitato non implica compatto.
\end{cor}

\begin{thm}
  $[0, 1]$ è compatto.
\end{thm}

\begin{proof}
  Sia $\mathcal{U}=\{U_i\}_{i \in I}$ un ricoprimento aperto di $[0, 1]$ e consideriamo $t_0=\sup{A}$ dove $A=\{t \in [0, 1] \text{ t.c. c'è un sottoricoprimento finito che ricopre } [0, t]\}$.
  Chiaramente l'insieme è non vuoto e contiene elementi maggiori di $0$, infatti c'è un aperto $U_{i_0}$ che contiene $0$ e dunque contiene un intervallo della forma $[0, \epsilon), \epsilon>0$: qualunque numero maggiore di $0$ e minore di $\epsilon$ sta dunque in $A$. Il $\sup$ di $A$ è anche un massimo: infatti esiste un aperto $U_{i_{t_0}}$ che contiene $t_0$, dunque contiene un intorno della forma $(t_0-\delta, t_0]$.
  Prendendo allora un sottoricoprimento finito che ricopre $[0, t_0-\delta/2]$ è aggiungendoci $U_{i_{t_0}}$ otteniamo un sottoricoprimento finito che ricopre $[0, t_0]$, da cui $t_0 \in A$ e dunque è un massimo.
  Supponiamo per assurdo che $t_0 \not=1$, cioè $0<t_0<1$, allora sempre in $U_{i_{t_0}}$ c'è un intorno della forma $[t_0, t_1)$ con $t_0<t_1<1$, quindi tutti i punti in $(t_0, t_1)$ stanno in $A$, contro la massimalità di $t_0$, assurdo.
\end{proof}

\begin{thm}
  $f:X \rightarrow Y$ continua, $X$ compatto $\Rightarrow$ $f(X)$ compatto.
\end{thm}

\begin{proof}
  Sia $\{U_i\}$ un ricoprimento aperto di $f(X)$ con gli $U_i$ aperti di $f(X)$, allora esistono $\{V_i\}$ aperti di $Y$ t.c. $U_i=f(X) \cap V_i$.
  Allora $\{f^{-1}(U_i)\}$ è un ricoprimento di $X$ e $f^{-1}(U_i)=f^{-1}(f(X) \cap V_i)=f^{-1}(f(X)) \cap f^{-1}(V_i)=X \cap f^{-1}(V_i)$ (il penultimo uguale è vero perché $f^{-1}(V_i) \subseteq f^{-1}(f(X))$), dunque per la continuità di $f$ è un'intersezione di un aperto con tutto, cioè è aperto, dunque $\{f^{-1}(U_i)\}$ è un ricoprimento aperto di $X$, ne estraiamo un sottoricoprimento, prendiamo le immagini degli aperti così trovati e questi sono un sottoricoprimento finito di $f(X)$ estratto da $\{U_i\}$.
\end{proof}

\begin{ftt}
  I seguenti sono fatti banali:
  \begin{nlist}
    \item ogni spazio finito è compatto;
    \item unione finita di spazi compatti è compatta.
  \end{nlist}
\end{ftt}

\begin{thm} \label{ccc}
  $X$ compatto, $Y \subseteq X$ chiuso $\implies$ $Y$ compatto.
\end{thm}

\begin{proof}
  Sia $\{U_i\}$ un ricoprimento di $Y$ con gli $U_i$ aperti di $Y$, allora per ogni $i$ esiste un $V_i$ aperto di $X$ t.c. $U_i=Y \cap V_i$ e $\{V_i\}$ è ancora un ricoprimento di $Y$. Notiamo che anche $W=X \setminus Y$ è un aperto e dunque $\{V_i\} \cup \{W\}$ è un ricoprimento aperto di $X$.
  Allora, essendo $X$ compatto, esiste un sottoricoprimento finito $W, V_{i_0}, \dots V_{i_n}$ (a priori, $W$ potrebbe non essere necessario, ma aggiungerlo non cambia niente). Essendo $W \cap Y=\emptyset$ dev'essere che $V_{i_0}, \dots, V_{i_n}$ ricoprono $Y$, dunque i rispettivi $U_{i_0}, \dots, U_{i_n}$ sono il sottoricoprimento finito cercato.
\end{proof}

\begin{oss}
  Abbiamo visto (e useremo ancora) che $Y \subseteq X$ è compatto se e solo se per ogni famiglia $\{U_i\}_{i \in I}$ di aperti di $X$ t.c. $\displaystyle Y \subseteq \bigcup_{i \in I} U_i$ esiste $\mathcal{I} \subseteq I$ finito con $\displaystyle Y \subseteq \bigcup_{i \in \mathcal{I}} U_i$.
\end{oss}

\begin{ex}
  Sia $X$ un insieme con la topologia cofinita, allora ogni sottoinsieme di $X$ è compatto. Infatti, sia $Y \subseteq X$ e supponiamo $\displaystyle Y \subseteq \bigcup_{i \in I} U_i$, $U_i$ aperti di $X$. Se $Y=\emptyset$ allora è compatto, altrimenti esiste $y_0 \in Y$ e $y_0 \in U_{i_0}$ per qualche $i_0 \in I$.
  Poiché $X \setminus U_{i_0}$ è finito, anche $Y \setminus U_{i_0}$ è finito, diciamo $Y \setminus U_{i_0}=\{y_1, \dots, y_n\}$, allora per ogni $j=1, \dots, n$ esiste $i_j$ con $y_j \in U_{i_j}$ e dunque $Y \subseteq U_{i_0} \cup U_{i_1} \dots \cup U_{i_n}$.

  Ad esempio, se $X=\mathbb{Z}$ con la topologia cofinita, $\mathbb{N} \subseteq \mathbb{Z}$ è compatto ma non chiuso.
\end{ex}

\begin{thm} \label{chc}
  Sia $X$ spazio topologico T2, $Y \subseteq X$, $Y$ compatto. Allora $Y$ è chiuso.
\end{thm}

\begin{proof}
  Dimostriamo che $X \setminus Y$ è aperto. Sia $x \in X \setminus Y$. Dato che $X$ è T2, per ogni $y \in Y$ esistono due aperti disgiunti $U_y, V_y$ t.c. $x \in U_y, y \in V_y$.
  $\{V_y\}_{y \in Y}$ è un ricoprimento aperto di $Y$, che è compatto, dunque esiste un sottoricoprimento finito $V_{y_1}, \dots V_{y_n}$. Notiamo allora che $V=V_{y_1} \cup \dots \cup V_{y_n}, U=U_{y_1} \cap \dots \cap U_{y_n}$ sono aperti disgiunti t.c. $Y \subseteq V, x \in U$.
  In particolare, $U$ è un intorno aperto di $x$ disgiunto da $Y$ e, dato che lo possiamo trovare per ogni $x \in X \setminus Y$, possiamo concludere che $X \setminus Y$ è aperto e dunque $Y$ è chiuso.
\end{proof}

\begin{lm} \label{ch->r}
  $X$ compatto T2 $\implies$ $X$ regolare.
\end{lm}

\begin{proof}
  T2 $\implies$ T1, dobbiamo dimostrare T3. Siano $Y \subseteq X$ chiuso, $x_0 \in X \setminus Y$. Dal teorema \ref{ccc} $Y$ è compatto, prendendo gli aperti $U, V$ della dimostrazione del teorema \ref{chc} abbiamo esattamente quello che ci serve per soddisfare T3.
\end{proof}

\begin{thm} \label{ch->n}
  $X$ compatto T2 $\implies$ $X$ normale.
\end{thm}

\begin{proof}
  T2 $\implies$ T1, dobbiamo dimostrare T4. Siano $C$ e $D$ due chiusi disgiunti in $X$. Dal lemma \ref{ch->r} sappiamo che $X$ è regolare, dunque per ogni $x \in C$ esistono $U_x, V_x$ aperti disgiunti $x \in U_x, D \subseteq V_x$. $\{U_x\}_{x \in C}$ è un ricoprimento aperto di $C$, che per il teorema \ref{ccc} è compatto, dunque esiste un sottoricoprimento finito $U_{x_1}, \dots, U_{x_n}$.
  Allora prendendo $U=U_{x_1} \cup \dots \cup U_{x_n}, V=V_{x_1} \cap \dots \cap V_{x_n}$ abbiamo che sono due aperti disgiunti t.c. $C \subseteq U, D \subseteq V$, che è esattamente quello che ci serve per dimostrare T3.
\end{proof}

\begin{prop}
  $X$ T2, $C, D \subseteq X$ compatti e disgiunti, allora esistono $U, V$ aperti disgiunti con $C \subseteq U, D \subseteq V$.
\end{prop}

\begin{proof}
  Dato che sfrutta le stesse idee delle dimostrazioni precedenti, è lasciata come esercizio al lettore.
\end{proof}

\begin{thm}
  $X$ spazio topologico, $Y_i, i \in I$ famiglia di sottoinsiemi chiusi con $Y_{i_0}$ compatto per qualche $i_0 \in I$. Se per ogni $J \subseteq I$ finito vale che $\displaystyle \bigcap_{i \in J} Y_i \not = \emptyset$ (\textsc{proprietà dell'intersezione finita}), allora $\displaystyle \bigcap_{i \in I} Y_i \not=\emptyset$.
\end{thm}

\begin{proof}
  Per ogni $i \in I, i \not=i_0$, sia $Z_i=Y_{i_0} \cap Y_i$. $Z_i$ è un chiuso di $Y_{i_0}$, dunque $W_i=Y_{i_0} \setminus Z_i=Y_{i_0} \setminus Y_i$ è un aperto di $Y_{i_0}$. Quindi, se per assurdo $\displaystyle \bigcap_{i \in I} Y_i=\emptyset$, avremmo che $\{W_i, i \in I\}$ sarebbe un ricoprimento aperto di $Y_{i_0}$.
  Allora, per compattezza di $Y_{i_0}$, esiste un sottoricoprimento finito $W_{i_1}, \dots, W_{i_n}$,
  cioè $Y_{i_0}=W_{i_1} \cup \dots \cup W_{i_n}=(Y_{i_0} \setminus Y_{i_1}) \cup \dots \cup (Y_{i_0} \setminus Y_{i_n})=Y_{i_0} \setminus (Y_{i_1} \cap \dots \cap Y_{i_n}) \implies$
  $\implies Y_{i_0} \cap Y_{i_1} \cap \dots \cap Y_{i_n}=\emptyset$, assurdo.
\end{proof}

\begin{ex}
  Serve un $Y_i$ compatto: $X=\mathbb{R}, Y_n=[n, +\infty)$ è un controesempio. \marginpar\warningsign
\end{ex}

\begin{cor}
  Se $Y_n, n \in \mathbb{N}$ è una famiglia di sottoinsiemi non vuoti di $X$ con $Y_0$ compatto, $Y_i$ chiuso per ogni $i \in \mathbb{N}$, $Y_{n+1} \subseteq Y_n$ per ogni $n \in \mathbb{N}$, allora $\displaystyle \bigcap_{n \in \mathbb{N}} Y_n \not= \emptyset$.
\end{cor}

\begin{lm} \label{comp_base}
  Sia $X$ spazio topologico e $\mathcal{B}$ una base di $X$. Se ogni ricoprimento di $X$ con aperti di $\mathcal{B}$ ammette un sottoricoprimento finito, allora $X$ è compatto.
\end{lm}

\begin{proof}
  Sia $\mathcal{U}=\{U_i\}_{i \in I}$ un qualsiasi ricoprimento aperto di $X$. Per definizione di ricoprimento si ha che per ogni $x \in X$ esiste $i(x) \in I$ t.c. $x \in U_{i(x)}$ e per definizione di base esiste $B_x \in \mathcal{B}$ t.c. $x \in B_x \subseteq U_{i(x)}$.
  $\{B_x\}_{x \in X}$ è un ricoprimento di $X$ con aperti di $B$ (si dice che $\{B_x\}_{x \in X}$ è un \textit{raffinamento} di $\mathcal{U}$).
  Per ipotesi esistono allora $x_1, \dots, x_n$ t.c. $X=B_{x_1} \cup \dots \cup B_{x_n} \subseteq U_{i(x_1)} \cup \dots \cup U_{i(x_n)}$, da cui la tesi.
\end{proof}

\begin{thm} \label{prod_comp}
  $X, Y$ compatti $\implies$ $X \times Y$ compatto.
\end{thm}

\begin{proof}
  Per il lemma \ref{comp_base}, possiamo partire da un ricoprimento $\mathcal{U}=\{U_i \times V_i\}_{i \in I}$ dove $U_i$ è aperto in $X$, $V_i$ è aperto in $Y$ per ogni $i \in I$.
  Per ogni $x \in X$ il sottoinsieme $\{x\} \times Y \subseteq X \times Y$ è compatto in quanto omeomorfo a $Y$, per cui esiste $J_x \subseteq I$ finito t.c. $\displaystyle \{x\} \times Y \subseteq \bigcup_{i \in J_x} (U_i \times V_i)$. Poniamo $\displaystyle U_x= \bigcap_{i \in J_x} U_i$, è aperto in quanto intersezione finita di aperti e per costruzione
  $\displaystyle U_x \times Y \subseteq \bigcup_{i \in J_x} (U_i \times V_i)$. Per compattezza di $X$ e poiché $\{U_x\}_{x \in X}$ è un ricoprimento aperto, abbiamo che $X=U_{x_1} \cup \dots \cup U_{x_n}$ per qualche $x_1, \dots, x_n$.
  Allora $\displaystyle X \times Y=\bigcup_{k=1}^n (U_{x_k} \times Y) \subseteq \bigcup_{k=1}^n \bigcup_{i \in J_{x_k}} (U_i \times V_i)$, da cui la tesi.
\end{proof}

\begin{oss}
  Se $A \subseteq X, B \subseteq Y$, la topologia prodotto di $A \times B$, ciascuno dotato della topologia di sottospazio, coincide con la topologia di sottospazio di $X \times Y$. Dunque, per il teorema \ref{prod_comp}, se $A, B$ sono sottospazi compatti, $A \times B$ è un sottospazio compatto di $X \times Y$.
\end{oss}


\subsection{Compattificazione di Alexandroff ed esaustioni}
\begin{ex}
    Sia $p=(1,0,\dots,0)$ un punto di $S^n$. Voglio far vedere che $S^n\setminus\{p\}$ \`e isomorfo a $\mathbb{R}^n$. In effetti la proiezione stereografica mi fornisce la mappa
    \[
        f\colon(x_0, \dots, x_n)\mapsto\left(\frac{x_1}{1-x_0}, \dots, \frac{x_n}{1-x_0}\right)
    \]
    che \`e un isomorfismo.

    Si ha anche che aggiungendo $p$ allo spazio di partenza si ha un compatto che contiene isomorficamente $\mathbb{R}^n$. Lo scopo della sezione \`e quello di generalizzare questa idea. Cio\`e ottenere un compatto che contiene uno spazio dato aggiungendo un punto.
\end{ex}

\begin{defn}
    Sia $(X,\tau)$ uno spazio topologico. Sia anche $\hat{X} = X \cup \{\infty\}$. Definisco
    \[
        \hat{\tau} = \tau \cup \{\hat{X}\setminus K\; |\ K \subseteq X\  \text{chiuso e compatto}\}
    \]
    Allora $(\hat{X}, \hat{\tau})$ \`e detta \textsc{compattificazione di Alexandroff}.
\end{defn}

\begin{oss}
    $(\hat{X}, \hat{\tau})$ \`e uno spazio topologico compatto e vi \`e una naturale immersione aperta da $X$ a $\hat{X}$.
\end{oss}

\begin{defn}
    Uno spazio topologico si dice \textsc{localmente compatto} se ogni suo punto ammette un intorno compatto.
\end{defn}

\begin{prop}
    La compattificazione di uno spazio $X$ \`e T2 se e solo se $X$ \`e T2 e localmente compatto.
\end{prop}
\begin{proof}
    Hint: la dimostrazione, come molte di quelle che riguardano la compattificazione, si fa distinguendo il caso in cui si prende un punto di $X$ e quello in cui si prende il punto $\infty$.
\end{proof}

\begin{prop}
    Sia $f\colon X \longrightarrow Y$ un'immersione aperta tra spazi T2. Si definisca $g\colon Y \longrightarrow \hat{X}$ con
    \[
        g(y) =
        \begin{cases}
            x\quad \text{se}\ f^{-1}(y) = \{x\}\\
            \infty \quad\text{se}\ f^{-1}(y) = \emptyset
        \end{cases}
    \]
    Allora $g$ \`e continua.
\end{prop}
\begin{proof}
    Hint: si prende un aperto nello spazio compattificato e si cerca di dire che la sua controimmagine \`e aperta. Per farlo si distingue il caso in cui l'infinito \`e nell'aperto da quello in cui non \`e nell'aperto.
\end{proof}

\begin{cor}
    Dato uno spazio $X$ T2 compatto, esso \`e isomorfo alla compattificazione di $X$ privato di un punto.
\end{cor}

\begin{prop}
    Sia $f\colon X \longrightarrow Y$ continua con $Y$ T2. Si prenda ${\hat{f}\colon \hat{X}\longrightarrow\hat{Y}}$ che manda $x$ in $f(x)$ e l'infinito nell'infinito. Allora $\hat{f}$ \`e continua se e solo se $f$ \`e propria.
\end{prop}

\begin{defn}
    Sia $X$ uno spazio topologico. Un'\textsc{esaustione in compatti} di $X$ \`e una famiglia ${\{K_n\}_{n\in\mathbb{N}}}$ di compatti di $X$ tali che ${K_n\subseteq\ \open{K_{n+1}}}$ e $\bigcup_nK_n = X$.
\end{defn}

\begin{prop}
    Sia $K_n$ un'esaustione in compatti e $H$ un compatto. Allora $H$ \`e contenuto in $K_i$ per qualche $i$.
\end{prop}
\begin{proof}
    Poich\'e $H\subseteq \bigcup_n\open{K_n}$, $H$ \`e contenuto in un'unione finita di $K_i$, allora \`e contenuto nel pi\`u grande di questi.
\end{proof}

\begin{ex}
    $\mathbb{R}\times(0,1)$ non \`e isomorfo a $\mathbb{R}\times[0,1]$. %non mi convince tanto la dimostrazione...
\end{ex}
\begin{proof}
    Si ha che $K_n = [-n,n]\times[\frac{1}{n}, 1-\frac{1}{n}]$ \`e un esaustione in compatti di $\mathbb{R}\times(0,1)$, mentre $H=\{0\}\times[0,1]$ \`e un compatto dell'altro spazio. Per\`o l'immagine isomorfa di $H$ non \`e contenuta in alcun $K$. Quindi si ha un assurdo.
\end{proof}

\begin{defn}
    Dato $X$ spazio topologico, $\pi_0(X)$ \`e l'insieme delle componenti connesse per archi di $X$.
\end{defn}

\begin{ex}
    $\mathbb{R}^n$ privato di $s$ punti non \`e isomorfo a $\mathbb{R}^n$
    privato di $t$ punti per $s\ne t$.
\end{ex}
\begin{proof}
    L'idea di base \`e quella di prendere esaustioni in compatti dei due spazi.
    Allora i compatti delle esaustioni (almeno definitivamente) avranno un buco per ogni punto tolto. Per ottenere un assurdo, basta contare le componenti connesse delle esaustioni.
\end{proof}


\subsection{Compattezza per successioni}
Sia $\{a_n\}$ una successione a valori in $X$ spazio topologico, allora una sottosuccessione di $\{a_n\}$ è una sottofamiglia $\{a_{n_i}\}$, dove $n_i, i=1, 2, \dots$ è una successione strettamente crescente in $\mathbb{N}$ (si tratta di una successione indicizzata da $i$).

\begin{defn}
  Uno spazio topologico $X$ si dice \textsc{compatto per successioni} se ogni successione in $X$ ammette una sottosuccessione convergente.
\end{defn}

\begin{prop} \label{N1->(c->cs)}
  Sia $X$ spazio topologico primo numerabile. Se $X$ è compatto, allora è anche compatto per successioni.
\end{prop}

\begin{proof}
  Sia $\{x_n\} \subseteq X$ una successione. Per ogni $m \in \mathbb{N}$ sia $C_m=\overline{\{x_n, n \ge m\}}$. Per ogni $m$, $C_m$ è chiuso e $C_{m+1} \subseteq C_m$. Per il teorema \ref{pif}, $\displaystyle \bigcap_{m \in \mathbb{N}} C_m \not=\emptyset$.
  Sia $\displaystyle \bar{x} \in \bigcap_{m \in \mathbb{N}} C_m \not$, cerchiamo una sottosuccessione che tende a $\bar{x}$. Sia $\{U_i\}_{i \in \mathbb{N}}$ un sistema fondamentale numerabile di intorni di $\bar{x}$ e supponiamo senza perdita di generalità $U_{i+1} \subseteq U_i$ per ogni $i \in \mathbb{N}$. Costruiamo la sottosuccessione per induzione:
  $\bar{x} \in C_0=\overline{\{x_n\}} \implies$ esiste $n_0$ t.c. $x_{n_0} \in U_0$. Per ogni $i \ge 0$, $\bar{x} \in C_{n_i+1}=\overline{\{x_n, n \ge n_i+1\}} \implies$ esiste $n_{i+1}>n_i$ t.c.
  $x_{n_{i+1}} \in U_{i+1}$. È un semplice esercizio verificare che $\{x_{n_i}\}$ funziona.
\end{proof}

\begin{thm}
  Sia $X$ a base numerabile. Allora $X$ è compatto $\Leftrightarrow$ è compatto per successioni.
\end{thm}

\begin{proof}
  ($\implies$) Per la proposizione \ref{N1->(c->cs)} e il teorema \ref{N2power} abbiamo immediatamente quest'implicazione.

  ($\leftarrow$) Mostriamo la contronominale. Sia $X$ non compatto e sia $\mathcal{U}$ un ricoprimento aperto di $X$ che non ammette sottoricoprimenti finiti. $X$ a base numerabile $\implies$ $\mathcal{U}$ ammette un raffinamente numerabile, dunque un sottoricoprimento numerabile. In alternativa, per il lemma \label{comp_base} potevamo dire che esiste un ricoprimento con aperti di base che non ammette sottoricoprimenti finiti, ed essendo $X$ a base numerabile potevamo scegliere quello come $\mathcal{U}$.
  Dunque senza perdita di generalità $\mathcal{U}=\{U_i\}_{i \in \mathcal{N}}$. Sia $x_i \in  X \setminus (U_0 \cup U_1 \cup \dots \cup U_i)$, che è ben definita in quanto $\mathcal{U}$ non ammette sottoricoprimenti finiti. Per assurdo, sia $\bar{x}$ il limite di una sottosuccessione di $\{x_i\}$.
  Poiché $\displaystyle X=\bigcup_{i \in \mathbb{N}} U_i$, $\bar{x} \in U_{i_0}$ per qualche $i_0 \in \mathbb{N}$. $U_{i_0}$ aperto $\implies$ è intorno di
  $\bar{x}$ $\implies$ $|\{i \in \mathbb{N} | x_i \in U_{i_0} \}|=|\mathbb{N}|$, dove l'ultima implicazione segue dal fatto che $\bar{x}$ è il limite di una sottosuccessione, ma $x_i \not\in U_{i_0}$ per ogi $i \ge i_0$ per definizione, assurdo.
\end{proof}

\begin{ex}
  \begin{ftt}
    $Fun([0, 1], [0, 1])$ con la topologia della convergenza puntuale è compatto ma non compatto per successioni.
  \end{ftt}
  \begin{proof}
    È compatto per il teorema di Tychonoff \ref{tychonoff}.

    Siano $f_n:[0, 1] \rightarrow [0, 1]$ definite da $f_n(x)=10^nx-\lfloor 10^nx \rfloor$. Allora $\{f_n\}$ non ha sottosuccessioni puntualmente convergenti.
  \end{proof}
\end{ex}


\subsection{Completezza e caratterizzazione dei metrici compatti}
All'interno di tutto questo paragrafo, $(X, d)$ indica una spazio metrico.

\begin{defn}
  $\{x_n\} \subseteq X$ è \textsc{di Cauchy} se per ogni $\epsilon>0$ esiste $n_0$ t.c. $d(x_n, x_m) \le \epsilon$ per ogni $n, m \ge n_0$.
\end{defn}

\begin{lm} \label{conv->cauchy}
  Se $\{x_n\}$ è convergente, allora è di Cauchy.
\end{lm}

\begin{proof}
  Sia $x \in X$ il limite a cui converge $\{x_n\}$ e sia $\epsilon>0$, allora esiste $n_0$ t.c. per ogni $n \ge n_0$ $d(x_n, x) < \epsilon/2$. Allora per la disuguaglianza triangolare si ha che per ogni $m, n \ge n_0$ vale che $d(x_n, x_m) \le d(x_n, x)+d(x, x_m)<\epsilon/2+\epsilon/2=\epsilon$, da cui la tesi.
\end{proof}

\begin{defn}
  $(X, d)$ è \textsc{completo} se ogni successione di Cauchy in $X$ converge.
\end{defn}

\begin{ex}
  $\mathbb{R}^n$ è completo.
\end{ex}

\begin{ex}
  $\mathbb{Q}$ non è completo.
\end{ex}

\begin{lm}
  Sia $\{x_n\}$ di Cauchy, se ha una sottosuccessione convergente allora è convergente.
\end{lm}

\begin{proof}
  Sia $\{x_{n_i}\}$ la sottosuccessione che converge al limite $x \in X$ e sia $\epsilon>0$. Allora esiste $i_0$ t.c. per ogni $i \ge i_0$ $d(x_{n_i}, x) < \epsilon/2$. Inoltre, esiste $\bar{n}$ t.c. per ogni $n, m \ge \bar{n}$ $d(x_n, x_m) \le \epsilon/2$.
  Poniamo $n^*=\max{\{n_{\bar{n}}, n_{i_0}\}}$ dove abbiamo usato $n_{\bar{n}} \ge \bar{n}$ al posto di $\bar{n}$ stesso per assicurarci che $n^*$ faccia parte degli indici della sottosuccessione convergente.
  Si ha allora che per ogni $n \ge n^*$ $d(x_n, x) \le d(x_n, x_{n^*})+d(x_{n^*} , x)<\epsilon/2+\epsilon/2=\epsilon$, da cui la tesi.
\end{proof}

\begin{cor} \label{cs->compl}
  $(X, d)$ compatto per successioni $\implies$ $(X, d)$ completo.
\end{cor}

\begin{lm}
  Siano $(X, d)$ spazio metrico completo, $Y \subseteq X$ con la metrica indotta, allora $Y$ è completo $\Leftrightarrow$ $Y$ è chiuso in $X$.
\end{lm}

\begin{proof}
  Per il fatto \ref{metr-N1} e la proposizione \ref{chiusoxsucc}, possiamo fare la dimostrazione supponendo o dimostrando (a seconda della freccia) $Y$ chiuso per successioni.

  ($\implies$) Sia $\{y_n\}$ una successione in $Y$ convergente a un elemento di $X$. Per il lemma \ref{conv->cauchy} sappiamo che $\{y_n\}$ è di Cauchy, ma stiamo supponendo $Y$ completo, dunque $\{y_n\}$ converge in $Y$, dunque $Y$ è chiuso per successioni. Notiamo che per quest'implicazione non era necessaria l'ipotesi $(X, d)$ completo.

  ($\Leftarrow$) Sia $\{y_n\}$ una successione di Cauchy in $Y$. Dato che $X$ è completo, questa converge a un elemento di $X$, ma $Y$ è chiuso per successioni, dunque l'elemento appartiene anche a $Y$ e la successione converge in $Y$.
\end{proof}

\begin{defn}
  $(X, d)$ è \textsc{totalmente limitato} se per ogni $\epsilon>0$ esiste un ricoprimento finito di $X$ fatto con palle di raggio $\epsilon$.
\end{defn}

\begin{lm}
  $(X, d)$ totalmente limitato implica $(X, d)$ limitato.
\end{lm}

\begin{proof}
  Prendiamo $\epsilon=1$, fissiamo il centro $x_0$ di una delle palle e sia $D$ la massima distanza tra $x_0$ e uno qualsiasi degli altri centri, che esiste perché le palle, dunque anche i loro centri, sono in numero finito. Allora è una banale applicazione della disuguaglianza triangolare verificare che la palla di centro $x_0$ e raggio $D+1$ ricopre tutto $X$.
\end{proof}

\begin{ex}
  Consideriamo su $\mathbb{R}$ la distanza $d(x, y)=\min{\{|x-y|, 1\}}$, allora $(\mathbb{R}, d)$ è limitato ma non totalmente limitato.
\end{ex}

\begin{prop} \label{tl->bn}
  $(X, d)$ totalmente limitato implica $(X, d)$ a base numerabile.
\end{prop}

\begin{proof}
  Per la proposizione \ref{metr-num} basta vedere che $X$ è separabile. Per la totale limitatezza, per ogni $n \ge 1$ esiste $F_n \subseteq X$ finito con $\displaystyle X=\bigcup_{p \in F_n} B(p, 1/n)$. Allora $\displaystyle \bigcup_{n \ge 1} F_n$ è numerabile, mostriamo dunque che è denso. Sia dunque $U \subseteq X$ aperto non vuoto, allora esiste $n_0 \ge 1, z \in U$ con $B(z, 1/n_0) \subseteq U$.
  Abbiamo che $z \in B(p, 1/n_0)$ per qualche $p \in F_{n_0} \implies d(p, z)<1/n_0 \implies p \in B(z, 1/n_0) \implies p \in U$, da cui la tesi.
\end{proof}

\begin{thm}
  Sia $(X, d)$ uno spazio metrico. Allora le seguenti sono equivalenti:
  \begin{nlist}
    \item $X$ è compatto;
    \item $X$ è compatto per successioni;
    \item $X$ è totalmente limitato e completo.
  \end{nlist}
\end{thm}

\begin{proof}
  ((i) $\implies$ (ii)) $(X, d)$ è metrico, dunque per il fatto \ref{metr-N1} è primo numerabile, allora per la proposizione \ref{N1->(c->cs)} otteniamo che (i) $\implies$ (ii).

  ((ii) $\implies$ (iii)) Per il corollario \ref{cs->compl} abbiamo che $X$ è completo. Supponiamo per assurdo che non sia totalmente limitato. Allora esiste $\epsilon>0$ tale che non è possibile ricoprire $X$ con un numero finito di palle di raggio $\epsilon$. Fissiamo $x_0 \in X$ e definiamo $\{x_n\}$ induttivamente fissando $x_{n+1} \in X \setminus (B(x_0, \epsilon) \cup B(x_1, \epsilon) \cup \dots \cup B(x_n, \epsilon))$, che esiste per l'ipotesi assurda.
  Allora dati due naturali qualsiasi $m \not= n$ abbiamo che $d(x_n, x_m) \ge \epsilon$, quindi non si possono estrarre da $\{x_n\}$ sottosuccessioni di Cauchy, in particolare non si possono estrarre sottosuccessoni convergenti, assurdo perché siamo sotto l'ipotesi (ii).

  ((iii) $\implies$ (i)) Per la proposizione \ref{tl->bn} e per il teorema \ref{bn->(c_sse_cs)}, sotto l'ipotesi (iii) abbiamo che (i) $\Leftrightarrow$ (ii). Dimostriamo allora (ii). Sia $\{x_n\}$ una successione in $X$. $X$ è completo per ipotesi, quindi basta trovare una sottosuccessione di Cauchy.
  Per totale limitatezza eiste un ricoprimento con un numero finito di palle di raggio $2^{-0}$. Allora esiste una di queste palle che contiene infiniti $\{x_n\}$. Fissiamo $n_0$ come il minimo indice tale che $x_{n_0}$ sta in questa palla e procediamo induttivamente.
  Abbiamo $x_{n_i}$ che appartiene a una palla di raggio $2^{-i}$ contenente infiniti $x_n$ (in particolare, infiniti di indice maggiore di $n_i$). Ricopriamo $X$ con un numero finito di palle di raggio $2^{-(i+1)}$. Allora ce ne dev'essere una che contiene infiniti di quegli $x_n$ che stavano nella palla di raggio $2^{-i}$ contenete $x_{n_i}$. Prendiamo dunque $n_{i+1}$ come il minimo indice maggiore di $n_i$ tale che $x_{n_{i+1}}$ sta nella palla di raggio $2^{-(i+1)}$ trovata. È una semplice verifica dimostrare che la sottosuccessione $\{x_{n_i}\}$ è di Cauchy.
\end{proof}


\subsection{Spazi di Baire}
\begin{defn}
    Un sottoinsieme di uno spazio topologico \`e raro se l'apertura della chiusura \`e vuota. Un sottoinsieme si dice magro se \`e unione numerabile di rari. Uno spazio si dice di Baire se le aperture dei magri sono vuote.
\end{defn}
\begin{prop}
    Sono equivalenti:
    \begin{nlist}
        \item X \`e di Baire
        \item Unione numerabile di chiusi rari ha parte interna vuota
        \item Intersezione numerabile di aperti densi \`e densa
    \end{nlist}
\end{prop}
\begin{proof}
    Le ultime due sono equivalenti per passaggio al complementare. La seconda \`e la definizione dell'essere di Baire. Per fare vedere che la seconda implica la prima, sia $Y\subseteq X$ magro, $Y=\bigcup_{n\in\mathbb{N}}Y_n$ con gli $Y_n$ rari. Allora
    \[
    \open{Y}=\open{(\bigcup Y_n)}\subseteq\open{\bigcup \bar{Y_n}}=\emptyset
    \]
\end{proof}
\begin{thm}(Versione debole del teorema di Baire)
    Ogni $X$ metrico completo \`e di Baire.
\end{thm}
\begin{proof}
    Siano $A_n, n \in \mathbb{N}$ aperti densi. Dati $x_0 \in X$ e $r_0>0$, è sufficiente mostrare che $\displaystyle \left(\bigcap_{n \in \mathbb{N}} A_n \right)\cap B(x_0,r_0)\not=\emptyset$. Per farlo, costruiamo la seguente successione di palle: \\
    $B_0=B(x_0,r_0)$; \\
    $B_{n+1}$: dato che $A_{n+1}$ è denso, $A_{n+1}\cap B_n \not=\emptyset$ ed è un aperto, allora possiamo trovare una palla $B_{n+1}=B(x_{n+1},r_{n+1})$ t.c. $\overline{B_{n+1}} \subseteq B_n \cap A_{n+1}$ e $r_{n+1} \le 1/(n+1)$. \\
    In questo modo, i centri delle palle definiscono una successione $x_n$ che è di Cauchy, quindi da $X$ completo abbiamo $x_n \longrightarrow x$. Abbiamo anche che la successione sta definitivamente in ogni $B_n$, dunque $x \in \overline{B_n} \implies x \in A_n$ per ogni $n$, ma $x \in \overline{B_1} \subset\ B_0=B(x_0,r_0)$, da cui la tesi.
\end{proof}

\begin{defn}
    Un sottoinsieme di uno spazio si dice perfetto se \`e chiuso e privo di punti isolati.
\end{defn}
\begin{prop} Un sottoinsieme $Y\subseteq\mathbb{R}$ perfetto \`e pi\`u che numerabile
\end{prop}
\begin{proof}
    $Y$ \`e completo, quindi di Baire. Allora scrivendo per assurdo $Y=\bigcup_{n\in\mathbb{N}}\{y_n\}$ e $\open{Y}=\emptyset$, non posso avere $\open{\bar{\{y_n\}}}=\open{\{y_n\}}=\emptyset$
\end{proof}


\subsection{Azioni proprie}
Sia $G$ gruppo che agisce tramite omeomorfismi su $X$ spazio topologico. Sia quindi $\phi$ definita come:
\begin{align*}
\phi:G \times X &\longrightarrow X \times X\\
(g,x) &\longmapsto (x, g\cdot x)
\end{align*}
Dotiamo allora $G$ della topologia discreta, e osserviamo che $\phi$ è continua se e solo se, date $\phi _1$ e $\phi _2$ come:
\begin{align*}
\phi _1:G \times X & \longrightarrow X & \phi _2:G \times X &\longrightarrow X \\
(g,x) &\longmapsto x & (g,x) &\longmapsto g \cdot x
\end{align*}
sono entrambe continue. In particolare:
\begin{itemize}
\item $\phi _1$ è continua perché proiezione
\item $\phi _2 ^{-1}(U)=\{(g,x) \mid g \cdot x \in U\}=\displaystyle \bigcup _{g \in G} \{g\} \times g^{-1}U$ aperto se $U \subset X$ aperto
\end{itemize}

Indichiamo che $G$ agisce su $X$ con $G \curvearrowright X$.

\begin{defn}
L'azione $G \curvearrowright X$ è \textsc{propria} se $\phi$ è un'applicazione propria (cioè per cui $\phi ^{-1} (K)$ è compatto se $K \subset X \times X$ è compatto).
\end{defn}

\begin{defn}
Sia $G \curvearrowright X$. Dati $Z,Y \subset X$, il \textsc{trasportatore} di $Y$ in $Z$ è:
$$(Y|Z)_G=\{g \in G \mid gY \cap Z \neq \emptyset\}$$
\end{defn}

\begin{thm}
Sia $G \curvearrowright X$ tramite omeomorfismi con $X$ spazio di Hausdorff localmente compatto. Allora sono equivalenti:
\begin{nlist}
\item L'azione è propria
\item $(K|K)_G$ è finito $\forall K \subset X$ compatto
\item $\forall x,y \in X, \exists U \in I(x), V \in I(y)$ tali che $(U|V)_G$ è finito
\end{nlist}
\end{thm}
\begin{proof}
(i) $\Longrightarrow$ (ii): Sia $K \subset X$ compatto. Allora:
$$K \times K \subset X \times X \text{ compatto } \Longrightarrow \phi ^{-1} (K \times K) \subseteq G \times X \text{ compatto}$$
Dove $\phi ^{-1}(K \times K)=\{(g,x) \mid x \in K, gx \in K\}$. Allora $p_1(\phi ^{-1}(K \times K))=(K|K)_G$ (ricordando che $p_1$ è la proiezione $G\times X \longrightarrow G$) è compatto in quanto immagine di compatto. $G$ discreto $\Longrightarrow$ $(K|K)_G$ compatto e discreto $\Longrightarrow$ $(K|K)_G$ finito.\\
(ii) $\Longrightarrow$ (iii): Siano $x,y \in X$ e $U \in I(x), V \in I(y)$ intorni compatti $\Longrightarrow$ $U \cup V$ compatto $\Longrightarrow$ $(U \cup V | U \cup V)_G \supset (U|V)_G$ è finito.\\
(iii) $\Longrightarrow$ (i): Manca
\end{proof}

\begin{ex}
$\mathbb{Z}^2 \curvearrowright \mathbb{R}^2$ per traslazione con:
$$(a,b) \cdot (x,y)=(x+a,y+b)$$
L'azione è propria. Vediamo ora (ii). Sia $D_{a,b}=[a,a+1] \times [b,b+1]$. Allora $(a,b) \cdot D_{0,0}$, e inoltre
$$\mathbb{R}^2=\bigcup _{a,b \in \mathbb{Z}} D_{a,b}$$
In generale, $\exists ! g \in \mathbb{Z}^2 \mid gD_{a,b}=D_{n,m}$ dati $a,b,m,n \in \mathbb{Z}$. Sia $K \subset \mathbb{R}^2$ compatto, allora esistono finiti $D_{a,b}$ tali che $K \cap D_{a,b} \neq \emptyset$. Siano $D_{a_1,b_1}, \dots ,D_{a_n,b_n}$, e sia $gK \cap K \neq \emptyset$ tale che $g \in G$ permuta i $D_{a_i,b_i}$. Allora $\exists i$ tale che $gD_{a_i,b_i}=D_{a_j,b_j}$, ma questo determina unicamente $g$, e quindi $(K|K)_G$ è finito.
\end{ex}

\begin{exc}
  Seguendo la stessa strada dell'esempio, si mostri che $\mathbb{Z}^n \curvearrowright \mathbb{R}^n$ per traslazione è un'azione propria.
\end{exc}

\begin{defn}
L'azione $G \curvearrowright X$ (con $X$ Hausdorff e localmente compatto) è detta \textsc{vagante} se $\forall x \in X, \exists U \in I(x)$ tale che $(U|U)_G$ è finito.
\end{defn}

\begin{defn}
L'azione $G \curvearrowright X$, nelle solite ipotesi, si dice \textsc{propriamente discontinua} se $\forall x \in X, \exists U \in I(x)$ tale che $(U|U)_G=\{1\}$.
\end{defn}

\begin{defn}
L'azione $G \curvearrowright X$, nelle solite ipotesi, si dice \textsc{libera} se $\forall g 	\in G,g \neq 1, \forall x \in X$ vale che $gx \neq x$.
\end{defn}

\begin{prop}
Sempre nelle stesse ipotesi:
\begin{nlist}
\item Le azioni proprie sono vaganti
\item Un'azione è propriamente discontinua se e solo se è vagante e libera
\end{nlist}
\end{prop}

\begin{proof}
(i): Per quanto visto precedentemente, $\forall K \subseteq X$ compatto, $(K|K)_G$ è finito, e quindi abbiamo finito poiché $X$ è localmente compatto.\\
(ii): Propriamente discontinua $\Longrightarrow$ vagante e libera.\\
Supponiamo $gx=x$. Allora $\forall U \in I(x), g \in (U|U)_G \Longrightarrow g=1$.\\
($\Longleftarrow$) Per esercizio.
\end{proof}

\begin{oss}
Abbiamo dunque mostrato che propria $\Longrightarrow$ vagante, propriamente discontinua $\Longrightarrow$ vagante.
\end{oss}

\begin{ex}
$\mathbb{Z} \curvearrowright \mathbb{R}^2$, con:
$$n \cdot (x,y)=(2^nx,2^{-n}y)$$
Per $\mathbb{R}^2 \smallsetminus \{0\}$, l'azione è libera. Vediamo che $\forall x,y \in \mathbb{R}^2 \smallsetminus \{0\}, \exists U \in I(x)$ tale che $(U|U)_G=\{1\}$. Supponiamo $U \in I(x,y)$ con $x \neq 0$, e consideriamo il disco di centro $(x,y)$ e raggio $\frac{|x|}{4}$. Allora $nU \cap U \neq [=] \emptyset \quad \forall n \neq 0 [\forall n=0]$.\\
Vediamo che l'azione non è propria: consideriamo $p=(0,1)$ e $q=(1,0)$. Siano $U \in I(p), V \in I(q)$. Allora $\exists n_0$ tale che $\forall n \ge n_0$ vale $(2^{-n},1) \in U, (1,2^{-n}) \in V$, ma $n(2^{-n},1)=(1,2^{-n})$, ovvero l'azione di $n$ li scambia. Allora $n \in (U|V)_G, \forall n \ge n_0$.
\end{ex}

\begin{defn}
$Z \subset X$ è \textsc{$G$-stabile} se $\forall x \in Z, \forall g \in G, gx \in Z$.
\end{defn}

\begin{thm}
Sia $G \curvearrowright X$ un'azione vagante. Allora $G \curvearrowright X$ è propria se e solo se $\faktor {X}{G}$ è di Hausdorff.
\end{thm}
\begin{proof}
($\Longrightarrow$) $\faktor{X}{G}$ è di Hausdorff se e solo se $\forall x,y \in X$ con $Gx \neq Gy,\ \exists U_1 \in I(x),\ V_1 \in I(y)$ $G$-stabili con $U_1 \cap V_1 =\emptyset$. Per ipotesi esistono $U \in I(x),\ V \in I(y)$ tali che $(U|V)_G$ è finito, fissati $x,y \in X$. Poniamo allora:
$$(U|V)_G=\{g_1,\dots,g_n\}$$
e consideriamo $g_ix$ e $y$. Poiché $X$ è di Hausdorff, quindi possiamo separarli $\forall i=1,\dots,n$. Siano allora $U_i \in I(g_ix)$ e $V_i \in I(y)$ tali che $U_i \cap V_i = \emptyset$. Definiamo inoltre:
$$U'=U \cap \bigcap _{i=1}^n g_i ^{-1} (U_i) \qquad \qquad V'=V \cap \bigcap _{i=1}^n V_i$$
Allora $U' \in I(x),\ V' \in I(y)$ e vediamo che $(U'|V')_G=\emptyset$. Infatti, supponiamo:
$$gU' \cap V' \neq \emptyset \Longrightarrow g \in (U|V)_G \Longrightarrow g \in \{g_1,\dots,g_n\}$$
Ma allora, $g_iU' \cap V' \subset U_i \cap V_i =\emptyset \Longrightarrow (U'|V')_G=\emptyset$. \\
Poniamo adesso $U''=A=\displaystyle \bigcap _{g \in G}gU'$ e $B=\displaystyle \bigcap _{g \in G}gV'$, dunque $A$ e $B$ sono aperti e $G$-stabili. Vediamo che sono disgiunti: abbiamo che $A \cap B= \displaystyle \bigcap _{g,h \in G} gU' \cap hV'$, e d'altra parte $gU' \cap hV' \neq \emptyset$ se e solo se $h^{-1}gU' \cap V' \neq \emptyset$. Ma $h^{-1}g \notin (U'|V')_G=\emptyset$. Dunque $gU' \cap hV'=\emptyset,\ \forall g,h \in G$. Dunque $\faktor {X}{G}$ è di Hausdorff.
\end{proof}

\begin{ex}
$\mathbb{Z}^n \curvearrowright \mathbb{R}^n$ per traslazione $\Longrightarrow \faktor {\mathbb{R}^n}{\mathbb{Z}^n} \simeq (S^1)^n$\\
Osserviamo che il prodotto di identificazioni aperte è un'identificazione aperta, e quindi:
$$\faktor {\mathbb{R}^n}{\mathbb{Z}^n} \simeq \left(\faktor {\mathbb{R}}{\mathbb{Z}}\right) ^n \simeq (S^1)^n$$
Consideriamo, per semplicità, il caso $n=2$:\\
Sia $D=[0,1] \times [0,1]$, allora $D$ interseca tutte le orbite di $\mathbb{Z}^2 \curvearrowright \mathbb{R}^2$. Posso vedere $\faktor {\mathbb{R}^2}{\mathbb{Z}^2}$ come quoziente del quadrato?\\
Definiamo una relazione indotta su $D$: $x \sim y \Longleftrightarrow \exists g \in G : gx=y$. Otteniamo il diagramma:\\
\begin{center}
\begin{tikzcd}
D \arrow[r, "i"] \arrow[d, "\pi _1"]
& \mathbb{R}^2 \arrow[d, "\pi _2"] \\
\faktor {D}{\sim} \arrow[r, "f"]
& \faktor {\mathbb{R}^2}{\mathbb{Z}^2}
\end{tikzcd}
\end{center}
Osserviamo che $f$ è continua, vediamo che è un omeomorfismo. \\
$\faktor {D}{\sim} \simeq S^1 \times S^1$, $f$ è continua e biunivoca, ma $\faktor {D}{\sim}$ è compatto perché immagine di un compatto. Allora $\faktor {\mathbb{R}^2}{\mathbb{Z}^2}$ Hausdorff implica che $f$ è un omeomorfismo.
\end{ex}

\begin{defn}
Sia $G \curvearrowright X$ tramite omeomorfismo. Un \textsc{dominio fondamentale} per l'azione è un chiuso $D \subset X$ tale che:
\begin{nlist}
\item $D=\overline{\mathop D\limits ^\circ}$
\item $g \neq \id$, allora $g\mathop D\limits ^\circ \cap \mathop D\limits ^\circ =\emptyset$ (cioè $(\mathop D\limits ^\circ|\mathop D\limits ^\circ)_G=\{\id\}$)
\item $X=\displaystyle \bigcup _{g \in G} gD$ è un ricoprimento localmente finito
\end{nlist}
\end{defn}

\begin{ex}
$\mathbb{Z}^n \curvearrowright \mathbb{R}^n$ per traslazione. Allora $[0,1]^n$ è un dominio fondamentale.
\end{ex}

\begin{ex}
$\faktor {\mathbb{Z}}{n\mathbb{Z}} \curvearrowright \mathbb{R}^2$ per rotazione di angolo $\frac{2\pi}{n}$. Allora il cono di angolo $\frac{2\pi}{n}$ è un dominio fondamentale.
\end{ex}

\begin{thm}
Sia $G \curvearrowright X$ tramite omeomorfismo, $X$ Hausdorff e localmente compatto, e $D \subset X$ un dominio fondamentale. Allora $\faktor {D}{\sim} \simeq \faktor {X}{G}$. Inoltre l'azione di $G$ su $X$ è propria.
\end{thm}
\begin{proof}
Consideriamo il seguente diagramma:
\begin{center}
\begin{tikzcd}
D \arrow[r, "i", hook] \arrow[d, "\pi _2"]
& X \arrow[d, "\pi _1"] \\
\faktor {D}{\sim} \arrow[r, "f"] & \faktor {X}{G}
\end{tikzcd}
\end{center}
dove $\sim$ è una relazione indotta dall'azione di $G$ su $D$. Come nell'esempio precedente, $f$ è continua e biunivoca, vediamo allora che è anche aperta (e che dunque è un omeomorfismo). Sia $A \subseteq X$ aperto, allora $f(A) \subset \faktor {X}{G}$ è aperto se e solo se $\pi _1 ^{-1} (f(A)) \subseteq X$ è aperto e $\pi _2 ^{-1} (A) \subseteq D,\ \exists B \subseteq D : \pi _2 ^{-1} (A)=B \cap D$ aperto. Allora:
$$\pi _1 ^{-1} (f(A))=\bigcup _{g \in G} g(B \cap D)=V$$
Mostriamo che $V$ è aperto. Per fare ciò, è sufficiente vedere che $\forall x \in B \cap D,\ \exists U \subset X$ aperto con $U \subset V$ (cioé $x \in U$). Allora, poiché $\{gD\}$ è un ricoprimento localmente finito di $X$, esiste $U' \in I(x)$ aperto tale che $U' \cap gD \neq \emptyset$ per un numero finito di $g$.\\
Assumendo allora che $U' \cap gD \neq \emptyset \Longrightarrow x \in gD$, basta sostituire $U'$ con $\displaystyle U' \smallsetminus \bigcup _{x \notin gD} gD$, ed ottengo ancora un intorno aperto di $x$ con le stesse proprietà. Siano adesso $g_1,\dots,g_n \in G$ gli elementi per cui $x \in g_iD$, allora $U' \subset g_1D \cup \cdots \cup g_nD$. Per ogni $i$ abbiamo che:
$$g_i^{-1}x \in D \Longrightarrow g_i^{-1}x \in \pi _2 ^{-1}(x) \Longrightarrow g_i^{-1}(x) \in B \subset \pi _2^{-1}(A) \Longrightarrow x \in g_iB$$
Poniamo:
$$U=U' \cap \bigcap _{i=1}^n g_iB \Longrightarrow  x \in U$$
intorno aperto di $x$. Vediamo che $U \subseteq V$. Infatti:
$$y \in U \Longrightarrow y \in U' \Longrightarrow y \in g_iD, \exists g_i :x \in g_iD \Longrightarrow y \in g_i(D \cap B) \subset V$$
\end{proof}


\section{Spazi proiettivi e varietà}

\subsection{Spazi proiettivi}
\begin{defn}
Sia $\mathbb{K}$ un campo, e sia $V$ un $\mathbb{K}$-spazio vettoriale. Definiamo su $V$ la relazione di equivalenza:
$$v \sim w \Longleftrightarrow \exists \lambda \in \mathbb{K}^*=\mathbb{K}\smallsetminus \{0\} : v=\lambda w$$
Allora chiamiamo \textsc{spazio proiettivo} associato a $V$ lo spazio:
$$\mathbb{P}(V)=\faktor {V \smallsetminus \{0\}}{\sim}=\faktor{V \smallsetminus \{0\}}{G} \qquad \text{ con }G=\{\lambda \id \mid \lambda \neq 0\}$$
$\mathbb{P}(V)$ è lo spazio delle rette di $V$.
\end{defn}

\begin{defn}
$W \subseteq \mathbb{P}(V)$ è un \textsc{sottospazio} di dimensione $k$ se $W=\pi (H\smallsetminus \{0\})$, con $H$ sottospazio di $V$ di dimensione $k+1$ e $\pi :V\smallsetminus \{0\} \longrightarrow \mathbb{P}(V)$. In particolare, $\dim \mathbb{P}(V)=\dim V -1$. Se $V=\mathbb{K}^n$, $\mathbb{P}(\mathbb{K}^n)$ si indica con $\mathbb{P}^{n-1}(\mathbb{K})$.
\end{defn}

Considereremo adesso in particolare i casi $\mathbb{K}=\mathbb{R}$ oppure $\mathbb{K}=\mathbb{C}$. In questi casi assumiamo $\mathbb{P}^n(\mathbb{K})$ dotato della topologia quoziente di $\mathbb{R}^{n+1}\smallsetminus \{0\}$.

\begin{oss}
$\mathbb{C}^{n+1} \smallsetminus \{0\} \simeq \mathbb{R}^{2(n+1)} \smallsetminus \{0\}$
\end{oss}

\begin{oss}
Se $S^n \subseteq \mathbb{R}^{n+1}$ è la sfera unitaria, $S^n$ interseca ogni retta in due punti della forma $\pm v$, per cui, come insieme:
$$\mathbb{P}^n(\mathbb{R})=\faktor{S^n}{\pm \id}$$
Vediamo che $\mathbb{P}^n(\mathbb{R})$ è omeomorfo a $\faktor{S^n}{\pm \id}$. Infatti abbiamo il diagramma:
\begin{center}
\begin{tikzcd}
S^n \arrow[r, "i", hook] \arrow[d] & \mathbb{R}^{n+1} \smallsetminus \{0\} \arrow[d, "\pi"]\\
\faktor{S^n}{\pm \id} \arrow[r, "\bar{i}"] & \mathbb{P}^n(\mathbb{R})
\end{tikzcd}
\end{center}
Poiché $i \circ \pi$ è continua, per la proprietà universale della topologia quoziente, $\bar{i}$ è continua. Consideriamo allora il diagramma:
\begin{center}
\begin{tikzcd}
\mathbb{R}^{n+1} \smallsetminus \{0\} \arrow[r, "r"] \arrow[d, "\pi"] & S^n \arrow[d]\\
\mathbb{P}^n(\mathbb{R}) \arrow[r, "\bar{r}"] & \faktor{S^n}{\pm \id}
\end{tikzcd}
\end{center}
con $r(x)=\frac{x}{||x||}$. Analogamente, $\bar{r}$ è continua e $\bar{r}$ e $\bar{i}$ sono una l'inversa dell'altra, dunque sono omeomorfismi.
\end{oss}

\begin{oss}
$\mathbb{P}^n(\mathbb{R})$ è compatto T2. Infatti è quoziente di $S^n$ (che è compatto T2) per l'azione di un gruppo finito, azione che perciò è sicuramente propria (in realtà è propriamente discontinua).
\end{oss}

\begin{prop}
Sia $D^n=\{x \in \mathbb{R}^n \mid ||x|| \le 1 \}$, e sia $\sim _D$ definita da:
$$x \sim _D y \Longleftrightarrow x=y \text{ o } (||x||=||y||=1 \text{ e } x=-y)$$
Allora $\mathbb{P}^n(\mathbb{R}) \simeq \faktor{D^n}{\sim _D}$.
\end{prop}
\begin{proof}
Sia $H=\{(x_0,\dots,x_n) \in S^n \mid x_0 \ge 0 \}$ e sia $\sim _H$ data da:
$$v \sim _H w \Longleftrightarrow v=w \text{ o } (v=-w \text{ e } x_0(v)=x_0(w)=0)$$
La composizione $H \longrightarrow S^n \longrightarrow \mathbb{P}^n(\mathbb{R})$ induce una bigezione continua $\faktor{H}{\sim _H} \longrightarrow \mathbb{P}^n(\mathbb{R})$, continua per la proprietà universale, mentre la bigezione deriva da ragioni insiemistiche.\\
$H$ compatto $\Longrightarrow \faktor{H}{\sim _H}$ compatto. Inoltre $\mathbb{P}^n(\mathbb{R})$ è T2, dunque $\mathbb{P}^n(\mathbb{R}) \simeq \faktor{H}{\sim _H}$. Infine, l'omeomorfismo
\begin{align*}
H &\longrightarrow D^n \\
(x_0,\dots,x_n) &\longmapsto (x_1,\dots,x_n)
\end{align*}
con inversa $(x_1,\dots,x_n) \longmapsto (\sqrt{1-x_1^2-\cdots-x_n^2},x_1,\dots,x_n)$ induce un omeomorfismo $\faktor{H}{\sim _H} \simeq \faktor{D}{\sim _D}$.
\end{proof}

\begin{oss}
$\mathbb{P}^1(\mathbb{R}) \simeq S^1$ in quanto:
$$ \mathbb{P}^1(\mathbb{R})=\faktor{D^2}{\sim}=\faktor{[-1,1]}{\{-1,1\}}=S^1 $$
\end{oss}

\begin{oss}
$\mathbb{P}^2(\mathbb{R})=\faktor{D^2}{\sim}$ dove $x\sim y \Longleftrightarrow x=y \text{ o } ||x||=||y||=1$ è il nastro di Moebius.
\end{oss}

In generale, l'inclusione $\mathbb{R}^n \smallsetminus \{0\} \longrightarrow \mathbb{R}^{n+1} \smallsetminus \{0\}$ induce un'inclusione $\mathbb{P}^{n-1}(\mathbb{R}) \hookrightarrow \mathbb{P}^n(\mathbb{R})$, tale che $\mathbb{P}^n(\mathbb{R})\smallsetminus \mathbb{P}^{n-1}(\mathbb{R}) \simeq (\mathop D\limits ^\circ)^n$.

\begin{defn}
Un punto di $\mathbb{P}^n(\mathbb{K})$ è descritto da una $(n+1)$-upla a meno di multipli. La classe di $(x_0,\dots,x_n)$ si denota con $[x_0: \dots:x_n]$ (\textsc{coordinate omogenee}).
\end{defn}

\begin{oss}
Se $p \in \mathbb{K}[x_0,\dots,x_n]$ è un polinomio, non ha senso chiedersi quanto vale $p([x_0: \dots:x_n])$. Inoltre, se $p$ è omogeneo di grado $d,\ \forall \lambda \in \mathbb{K}^*$, vale:
$$p(\lambda x_0,\dots,\lambda x_n)=\lambda ^d p(x_0,\dots,x_n)$$
dunque, perlomeno, è ben definito il fatto che $p$ si annulli su $[x_0: \dots:x_n]$, cosa che avviene per definizione.
\end{oss}

Torniamo adesso ad esaminare il caso generale. Prendiamo $\mathbb{P}^n(\mathbb{K})$, e siano $x_0,\dots,x_n$ le coordinate di $\mathbb{K}^{n+1}$. Poniamo allora
$$U_i=\{x_i \neq 0\}=\{[x_0: \dots : x_n] \in \mathbb{P}^n(\mathbb{K}) \mid x_i \neq 0\}$$

\begin{prop}
Esiste una bigezione naturale tra $U_i$ e $\mathbb{K}^n$ che, nel caso $\mathbb{K}=\mathbb{R}$, è un omeomorfismo.
\end{prop}
\begin{proof}
Siano $\varphi :U_i \rightarrow \mathbb{K}^n,\ \psi :\mathbb{K}^n \rightarrow U_i$ definite da:
\begin{align*}
\varphi ([x_0: \dots :x_n]) &= \left( \dfrac{x_0}{x_i},\cdots,\dfrac{x_{i-1}}{x_i},\dfrac{x_{i+1}}{x_i},\cdots,\dfrac{x_n}{x_i}\right) \\
\psi ((x_1,\dots,x_n)) &=[x_1: \dots :x_{i-1}:1:x_{i+1}: \dots :x_n]
\end{align*}
Allora
\begin{align*}
&\varphi (\psi ((x_1,\dots,x_n)))=(x_1,\dots,x_n) \\
&\psi (\varphi ([x_0: \dots :x_n])=\psi \left( \left( \dfrac{x_0}{x_i},\cdots,\dfrac{x_n}{x_i}\right) \right) \\&=\left[\dfrac{x_0}{x_i}: \dots :\dfrac{x_{i-1}}{x_i}:1:\dfrac{x_{i+1}}{x_i}: \dots :\dfrac{x_n}{x_i}\right]=[x_0: \dots :x_n]
\end{align*}
Dunque $\varphi$ e $\psi$ sono una l'inversa dell'altra, e dunque sono bigezioni. Rimane da vedere che, se $\mathbb{K}=\mathbb{R}$, sono continue. $\psi$ è ovviamente continua, in quanto composizione delle mappe continue $\mathbb{R}^n \hookrightarrow \mathbb{R}^{n+1} \smallsetminus \{0\} \rightarrow \mathbb{P}^n(\mathbb{R})$. Inoltre $\varphi$ si ottiene per passaggio al quoziente da $\tilde{\varphi} : \pi ^{-1} (U_i) \rightarrow \mathbb{R}^n$, dove:
$$\tilde{\varphi}(x_0,\dots,x_n)=\left( \dfrac{x_0}{x_i},\cdots,\dfrac{x_{i-1}}{x_i},\dfrac{x_{i+1}}{x_i},\cdots,\dfrac{x_n}{x_i}\right)$$
con $\pi ^{-1}(U_i)=\{x \in \mathbb{R}^{n+1} \mid x_i \neq 0\}$. Allora $\tilde{\varphi}$ è chiaramente continua, e dunque lo è anche $\varphi$ (infatti la topologia quoziente di $\faktor {\pi ^{-1} (U_i)}{\sim}$ coincide con quella di $U_i \subseteq \mathbb{P}^n (\mathbb{R})$ come sottospazio, in quanto $\pi ^{-1} (U_i)$ è saturo. Dunque $\forall p \in \mathbb{P}^n (\mathbb{R}),\ \exists U$ aperto in $\mathbb{P}^n (\mathbb{R})$ con $p \in U$ e $U$ omeomorfo a $\mathbb{R}^n$.
\end{proof}

\begin{defn}
Sia $X$ spazio topologico. Allora $X$ si dice \textsc{varietà} $n$-dimensionale se:
\begin{nlist}
\item $X$ è T2
\item $\forall p \in X,\ \exists$ aperto $U$ con $p \in U$ con $U$ omeomorfo a un aperto $V$ di $\mathbb{R}^n$
\item $X$ è a base numerabile
\end{nlist}
\end{defn}

\begin{oss}
Poiché le palle aperte sono una base della topologia di $\mathbb{R}^n$ e una palla aperta è omeomorfa a $\mathbb{R}^n$, (ii) è equivalente a richiedere che esista $U$ aperto con $p \in U$ e $U$ omeomorfo ad una palla (o a $\mathbb{R}^n$).
\end{oss}

\begin{oss}
Poiché $\mathbb{P}^n(\mathbb{R})$ è a base numerabile, $\mathbb{P}^n(\mathbb{R})$ è una $n$-varietà.
\end{oss}

\begin{oss}
(i), (ii), (iii), sono indipendenti. Ad esempio:
$$\faktor{\mathbb{R} \times \{-1,1\}}{\sim} \quad \text{con} \quad (x,t) \sim (y,s) \Leftrightarrow (x,t)=(y,s) \text{ o } (x=y \text{ e }x,y \neq 0)$$
Verifica (ii) poiché è localmente euclidea, ma non è T2.
\end{oss}


\subsection{Proiettivi complessi}
Con dimostrazioni analoghe, si mostra che $\mathbb{P}^n(\mathbb{C})$ è compatto, T2, ed è ricoperto da aperti $U_i=\{z_i \neq 0\}$ omeomorfi a $\mathbb{C}^n \simeq \mathbb{R}^{2n}$. Essendo a base numerabile, $\mathbb{P}^n(\mathbb{C})$ è una varietà di dimensione $2n$.

\begin{prop}
$\mathbb{P}^1(\mathbb{C}) \simeq S^2$
\end{prop}
\begin{proof}
Infatti:
$$\mathbb{P}^1(\mathbb{C})=\{z_0=0\} \cup \{z_0 \neq 0\}=\{[0:1]\} \cup U_0$$
Poiché $\mathbb{P}^1(\mathbb{C}$ è compatto T2, per l'unicità della compattificazione di Alexandroff, $\mathbb{P}^1(\mathbb{C})=\hat{U}_0=\hat{\mathbb{R}}^2=S^2$.
\end{proof}

In generale:
$$\mathbb{P}^n(\mathbb{K})=\{x_0=0\} \cup \{x_0 \neq 0\} \simeq \mathbb{P}^{n-1}(\mathbb{K}) \cup U_0 \simeq \mathbb{P}^{n-1}(\mathbb{K}) \cup \mathbb{K}^n$$

\begin{ex}
La restrizione di $\pi :\mathbb{C}^{n+1} \smallsetminus \{0\} \rightarrow \mathbb{P}^n(\mathbb{C})$ a:
$$S^{n+1}=\{v \in \mathbb{C}^{n+1}=\mathbb{R}^{2n+2} \mid ||v||=1\}$$
è surgettiva, perché $\pi(v)=\pi(\frac{v}{||v||})$ e ciò implica $\mathbb{P}^n(\mathbb{C})$ compatto. Tuttavia, $\forall \theta \in \mathbb{R},\ \pi(e^{i\theta}v)=\pi(v) \ \forall v \in \mathbb{C}^{n+1} \smallsetminus \{0\}$. Consideriamo la mappa $f: S^{2n+1} \rightarrow \mathbb{P}^n(\mathbb{C})$ ottenuta restringendo $\pi$. Allora, $\forall p \in \mathbb{P}^n(\mathbb{C}),\ f^{-1}(p) \subseteq S^{2n+1}$ è omeomorfo a $S^1$. Infatti, se $v \in f^{-1}(p)$ scelto a caso, allora:
$$f^{-1}(p)=\{\lambda v \mid \lambda \in \mathbb{C},\ |\lambda|=1\}$$
per cui la mappa 
\begin{align*}
i:S^1=\{\lambda \in \mathbb{C} \mid |\lambda|=1\} &\longrightarrow f^{-1}(p) \\
\lambda &\longmapsto \lambda v
\end{align*}
è bigettiva e omeomorfismo (infatti va da un compatto a un T2).\\
Perciò, se $n=1,\ \exists f:S^3 \rightarrow S^2$ surgettiva tale che $f^{-1}(p) \simeq S^1$ per ogni $p$ in $S^2$. Si ottiene che $S^3 \simeq S^2 \times S^1$. Questa $f$ si chiama \textsc{fibrato di Lefschetz}, ed è un esempio di fibrato non banale.
\end{ex}

\begin{ex}(\emph{Uno spazio compatto per successioni non compatto})\\
Sia $X=[0,1]^{[0,1]}$ con la topologia prodotto, cioè la topologia della convergenza puntuale. Sia inoltre, $\forall f \in X$:
$$\text{supp}(f)=\{x \in [0,1] \mid f(x) \neq 0\}$$
il supporto di $f$. Prendiamo allora:
$$Y=\{f \in X \mid \#\text{supp}(f) \le \#\mathbb{N}\}$$
Allora:
\begin{nlist}
\item $Y$ è denso in $X$
\item $X$ è T2
\item $Y$ non è compatto (compatto in T2 $\Rightarrow$ chiuso)
\item $Y$ è compatto per successioni
\end{nlist}
Sia $f_n$ una successione in $Y$, ne devo trovare una estratta puntualmente convergente a una $f \in Y$. Sia:
$$A=\bigcup _{n \in N} \text{supp}(f)$$
Allora $\#A \le \#\mathbb{N}$, quindi posso scrivere anche $A=\{a_0,\dots,a_n,\dots\}$. La successione $\{f_n(a_0)\}_{n \in \mathbb{N}}$ è contenuta in [0,1] che è compatto, dunque esiste una scelta crescente di indice $k_0(n)$ tale che $f_{k_0(n)} (a_0) \longrightarrow \ell _0$ per $n \longrightarrow +\infty$.\\
Analogamente $\{f_{k_0(n)}(a_1)\}_{n \in \mathbb{N}} \subseteq [0,1]$, dunque esiste una sottosuccessione $k_1(n)$ estratta da $k_0(n)$ con $f_{k_1(n)}(a_1) \longrightarrow \ell _1 \in [0,1]$. Procedendo così, $\forall m$ costruisco una sottosuccessione $k_m(n)$ con $k_{m+1}(n)$ estratta da $k_m(n)$ e tali che
$$\lim _{n \rightarrow +\infty} f_{k_m(n)}(a_m)=\ell _m \in [0,1]$$
Segue che, $\forall m \in \mathbb{N},\ \displaystyle \lim _{i \rightarrow +\infty} f_{k_i(i)} (a_m)=\ell _m$.\\
Infatti $f_{k_i(i)}$ è definitivamente una sottosuccessione di $f_{k_i(n)}$, con $n \in \mathbb{N}$. Poniamo allora:
$$f(x)=\begin{cases} \ell _i & \mbox{se }x=a_i \in A \\ 0 & \mbox{se }x \notin A \end{cases}$$
Allora $f \in Y$ ($\text{supp}(f)$ è numerabile) e $f_{k_i(i)} \longrightarrow f$ puntualmente.
\end{ex}

\section{Omotopia}

\subsection{Omotopie, equivalenze omotopiche e retrazioni}
\begin{defn}
  Siano $f, g:X \rightarrow Y$ continue. Una \textsc{omotopia} tra $f$ e $g$ è una mappa continua $H:X \times [0, 1] \rightarrow Y$ t.c. $H(x, 0)=f(x), H(x, 1)=g(y)$ per ogni $x \in X$.
\end{defn}

\begin{ftt}
  Per ogni $t \in [0, 1]$, $H_t(x)=H(x, t), H_t:X \rightarrow Y$ è continua, per cui $H$ descrive un'interpolazione tra $f$ e $g$: deformo $f$ in $g$.
\end{ftt}

Nel seguito, $I=[0, 1]$.

\begin{defn}
  $f$ si dice \textsc{omotopa} a $g$ (e si scrive $f \sim g$) se esiste un'omotopia tra $f$ e $g$.
\end{defn}

\begin{ftt}
  \begin{nlist}
    \item $f \sim f$;
    \item $f \sim g \implies g \sim f$;
    \item $f \sim g$ e $g \sim h \implies f \sim h$.
  \end{nlist}
  Dunque l'omotopia è una relazione di equivalenza su $C(X, Y)$ (le funzioni continue da $X$ in $Y$). L'insieme delle classi di omotopia si denota con $[X, Y]$.
\end{ftt}

\begin{ex}
  Se $Y$ è un convesso di $\mathbb{R}^n$, $\left|[X, Y]\right|=1$, cioè tutte le mappe $f:X \rightarrow Y$ sono omotope fra loro. Infatti, date $f$ e $g$, basta prendere $H(x, t)=tf(x)+(1-t))g(x)$. In realtà, dato che l'omotopia è una relazione di equivalenza, otteniamo di più: ci basta che $Y$ sia "stellato" rispetto a un punto, cioè che esista $p \in Y$ t.c. per ogni $y \in Y, t \in I$, $ty+(1-t)p \in Y$. Allora si dimostra che tutte le funzioni sono omotope alla funzione che vale costantemente $p$.
\end{ex}

\begin{defn}
  Sia $X$ spazio topologico, $\pi_0(X)$ è l'insieme delle componente connesse per archi di $X$.
\end{defn}

\begin{lm}
  $f:X \rightarrow Y$ continua induce una funzione ben definita $f_{\star}:\pi_0(X) \rightarrow \pi_0(Y)$ definita dalla richiesta che $f(C) \subseteq f_{\star}(C)$ per ogni $C \in \pi_0(X)$. Allora il lemma dice questo: se $f \sim g$, $f_{\star}=g_{\star}$.
\end{lm}

\begin{proof}
  Sia $C \in \pi_0(X)$ e sia $x_0 \in C$. Se $H$ è un'omotopia tra $f$ e $g$, la mappa $\gamma:[0, 1] \rightarrow Y, \gamma(t)=H(x_0, t)$ è un arco continuo in $Y$ che congiunge $f(x_0)$ con $g(x_0)$.
\end{proof}

\begin{ftt}
  Se $X \subseteq \mathbb{R}^n$ è stellato rispetto a $p$, allora c'è una bigezione tra $[X, Y]$ e $\pi_0(Y)$.
\end{ftt}

\begin{proof}
  $X$ stellato $\implies$ $X$ connesso per archi $\implies$ $|\pi_0(X)|=1$. Dunque possiamo definire $\psi:C(X, Y) \rightarrow \pi_0(Y)$ data da $\psi(f)=f_{\star}(X)$. Per il lemma, $\psi$ induce al quoziente una ben definita $\varphi:[X, Y] \rightarrow \pi_0(Y)$. $\varphi$ è ovviamente suriettiva (basta considerare le funzioni costanti). Per l'iniettività: $f \in C(X, Y)$ è omotopa alla costante $f(p)$ tramite l'omotopia $H(x, t)=f(tx+(1-t)p)$.
  Se prendiamo $f$ e $g$ t.c. $\varphi([f])=\varphi([g])$, allora dev'essere che $f(p)$ e $g(p)$ stanno nella stessa componente connessa per archi di $Y$, dunque le costanti $f(p)$ e $g(p)$ sono omotope tramite $K(x, t)=\gamma(t)$ dove $\gamma$ è il cammino che congiunge i due punti, da cui $[f]=[g]$ e quindi $\varphi$ è iniettiva.
\end{proof}

\begin{defn}
  $f:X \rightarrow Y$ continua è un'\textsc{equivalenza omotopica} se ammetta un'\textsc{inversa omotopica}, cioè $g:Y \rightarrow X$ continua t.c. $f \circ g \sim \id_Y, g \circ f \sim \id_X$. $X$ e $Y$ si dicono \textsc{omotopicamente equivalenti} (o \textsc{omotopi}) se esiste un'equivalenza omotopica tra loro.
\end{defn}

\begin{exc}
  Essere omotopicamente equivalenti è una relazione di equivalenza. Per la transitività si usa il seguente lemma.
\end{exc}

\begin{lm}
  $f_0, f_1:X \rightarrow Y, g_0, g_1:Y \rightarrow Z$ continue, $f_0 \sim f_1, g_0 \sim g_1 \implies g_0 \circ f_0 \sim g_1 \circ f_1$.
\end{lm}

\begin{proof}
  $(x, y) \mapsto K(H(x, t), t)$, dove $H$ è l'omotopia tra $f_0$ e $f_1$ e $K$ quella tra $g_0$ e $g_1$, è l'omotopia cercata.
\end{proof}

\begin{defn}
  $X$ è \textsc{contraibile} (o \textsc{contrattile}) se è omotopicamente equivalente a un punto.
\end{defn}

\begin{prop}
  $X \subseteq \mathbb{R}^n$ stellato $\implies$ $X$ contrattile.
\end{prop}

\begin{proof}
  Sia $Y=\{q\}$ un punto. Siano $fX: \rightarrow Y$ la costante e $g:Y \rightarrow X$ t.c. $g(q)=x_0 \in X$ (a caso, non necessariamente il punto rispetto a cui $X$ è stellato). $f$ e $g$ sono continue e $f \circ g$ è ovviamente l'identità. $g \circ f:X \rightarrow Y$ è omotopa a $\id_X$ perché $X$ stellato $\implies$ $\left|[X, X]\right|=1$.
\end{proof}

\begin{oss}
  $X$ contrattile $\implies$ $X$ connesso per archi. Infatti, se $f:X \rightarrow Y$ è un'equivalenza omotopica, $f_{\star}: \pi_0(X) \rightarrow \pi_0(Y)$ è bigettiva. Lo si dimostra considerando l'inversa $g$ e facendo vedere che $(f \circ g)_{\star}=f_{\star} \circ g_{\star}$.
\end{oss}

\begin{defn}
  $X$ spazio topologico, $C \subseteq X$ si dice \textsc{retratto} se esiste una mappa $r$ (detta \textit{retrazione}), $r:X \rightarrow C$ continua t.c. $r(x)=x$ per ogni $x \in C$.

  $C$ si dice \textsc{retratto di deformazione} se esiste $r:X \rightarrow C$ come sopra e t.c. sei $i:C \rightarrow X$ è l'inclusione, esiste un omotopia $H$ tra $\id_X$ e $i \circ r$ t.c. $H(x, t)=x$ per ogni $x \in C, t \in I$.
\end{defn}

Segue che se $Y$ è un retratto di deformazione $X$ e $Y$ sono omotopicamente equivalenti ($i$ e $r$ sono equivalenze omotopiche).

\begin{ex}
  $p \in X \implies \{p\}$ è un retratto di $X$.
\end{ex}

\begin{ex}
  $S^n$ è un retratto di deformazione di $X=\mathbb{R}^{n+1} \setminus \{0\}$. Basta prendere $r: X \rightarrow S^n$ definita da $r(x)=x/\left\|x\right\|$, l'omotopia tra $\id_X$ e $i \circ r$ è $H(x, t)=(1-t)x+tx/\left\|x\right\|$.
\end{ex}

\begin{ex}
  $GL^+(n, \mathbb{R})$ (le matrici con determinante positivo) si ritrae per deformazione a $SL(n, \mathbb{R})=\{ A \in GL(n, \mathbb{R}) | \det{A}=1\}$. Infatti, se $A \in GL^+(n, \mathbb{R})$, $\det{A}>0$, per cui
  $tA+\frac{1-t}{\sqrt[n]{\det{A}}}A=\left(t+\frac{1-t}{\sqrt[n]{\det{A}}}\right)A \in GL^+(n, \mathbb{R})$ per ogni $t \in I$ dato che $t+\frac{1-t}{\sqrt[n]{\det{A}}}>0$ per ogni $t \in I$.
  Allora è un semplice esercizio verificare che la funzione $H:GL^+(n, \mathbb{R}) \times I \rightarrow GL^+(n, \mathbb{R})$ data da $H(A, t)=\left(t+\frac{1-t}{\sqrt[n]{\det{A}}}\right)A$ è l'omotopia che cerchiamo.
\end{ex}

Si può dimostrare analogamente che $GL^+(n, \mathbb{R})$ si retrae per deformazione su $SO(n, \mathbb{R})$ (le matrici ortonormali con determinante $1$), ma la dimostrazione è difficile (più che altro lunga, secondo Frigerio) ed è lasciata al corso di geometria e topologia differenziale.

\begin{exc}
  $X=(\mathbb{R} \times \{0\}) \cup (\mathbb{Q} \times \mathbb{R}) \subseteq \mathbb{R}^2$. allora
  \begin{nlist}
    \item $\{(0, 0)\}$ è un retratto di deformazione di $X$, che perciò è contrattile;
    \item $\{(0, 1)\}$ non è un retratto di deformazione di $X$.
  \end{nlist}
\end{exc}

\begin{sol}
  \begin{nlist}
    \item Hint: prima schiacciare il "pettine" sulla retta, poi la retta sul punto.
    \item Per assurdo sia $H:X \times I \rightarrow X$ tra $\id$ e la costante $(0, 1)$ t.c. $H((0,1), t)=(0, 1)$ per ogni $t \in I$. Per ogni $x \not=0$, i punti $(0, 1)$ e $(x, 1)$ giacciono in componenti connesse distinte di $X \setminus \{y=0\}$, ma l'arco $\gamma_x:[0, 1] \rightarrow X$ dato da
    $\gamma_x(t)=H((x, 1), t)$ è continuo e congiunge $(x, 1)$ a $(0, 1)$, dunque deve passare per $\{y=0\}$. Allora per ogni $x \not=0$ esiste $t(x) \in I$ t.c. $H((x, 1), t(x))$ ha ordinata nulla. Consideriamo la successione $(1/n, 1)$. Chiamando $t_n=t(1/n)$ e sfruttando la compattezza di $I$, a meno di estrarre una sottosuccessione possiamo supporre senza perdita di generalità $t_n \rightarrow \bar{t}$.
    Per continuità, se $\pi_y: \mathbb{R}^2 \rightarrow \mathbb{R}$ è la proiezione alla seconda coordinata, per continuità di $H$ e $\pi_y$ otteniamo che $\displaystyle 0=\lim_{n \rightarrow +\infty} \pi_y(H((1/n, 1), t_n))=\pi_y(H((0, 1), \bar{t}))=\pi_y((0, 1))=1$, assurdo.
  \end{nlist}
\end{sol}


\subsection{Gruppo fondamentale}
Siano $X$ spazio topologico e $a, b \in X$. $\Omega(a, b):=\{\gamma:[0, 1] \rightarrow X \text{ continuo t.c. } \gamma(0)=a, \gamma(1)=b\}$.

\begin{defn}
  $\alpha, \beta \in \Omega(a, b)$ sono \textsc{omotopi} (\textsc{come cammini} o \textsc{a estremi fissi}) se esiste $H:[0, 1] \times I \rightarrow X$ omotopia tra $\alpha$ e $\beta$ e t.c. $H(0, t)=a, H(1, t)=b$ per ogni $t \in I$. Essere omotopi è una relazione di equivalenza $\sim$ tra cammini.
\end{defn}

\begin{defn}
  Come insieme, il \textsc{gruppo fondamentale di $X$ con punto base $a$} è $\pi_1(X, a)=\faktor{\Omega(a, a)}{\sim}$.
\end{defn}

Il nostro prossimo obiettivo sarà quello di mettere su $\pi_1(X, a)$ una struttura di gruppo.

Se $\alpha, \beta \in \Omega(a, a)$, è ben definita $\alpha * \beta \in \Omega(a, a)$ data da $\alpha * \beta (t)=\begin{cases} \alpha(2t) \text{ se } 0 \le t \le 1/2 \\
\beta(2t-1) \text{ se } 1/2 \le t \le 1 \end{cases}$.

Notazione: $1_a \in \Omega(a, a)$ è il cammino costante in $a$. Se $\alpha \in \Omega(a, b)$, $\bar{\alpha} \in \Omega(b, a)$ è $\bar{\alpha}(t)=\alpha(1-t)$.

È vero che se $\alpha, \beta, \gamma \in \Omega(a, a)$, allora $(\alpha * \beta) * \gamma=\alpha * (\beta * \gamma)$? No! Però sono omotopi. \marginpar{\warningsign}

\begin{lm}
  Sia $\varphi:[0, 1] \rightarrow [0, 1]$ continua e t.c. $\varphi(0)=0, \varphi(1)=1$ e sia $\gamma \in \Omega(a, b)$, allora $\gamma \sim \gamma \circ \varphi$.
\end{lm}

\begin{proof}
  L'omotopia cercata è data da $H(t, s)=\gamma(st+(1-s)\varphi(t))$.
\end{proof}

\begin{cor}
  $\alpha \in \Omega(a, b), \beta \in \Omega(b, c), \gamma \in \Omega(c, d)$, allora $(\alpha * \beta) * \gamma \sim \alpha * (\beta * \gamma)$ in quanto sono uno riparametrizzazione dell'altro.
\end{cor}

\begin{lm}
  $\alpha, \alpha' \in \Omega(a, b), \beta, \beta' \in \Omega(b, c), \alpha \sim \alpha', \beta \sim \beta' \implies \alpha * \beta \sim \alpha' * \beta'$ in $\Omega(a, c)$.
\end{lm}

\begin{proof}
  Sia $H$ l'omotopia tra $\alpha$ e $\alpha'$ e $K$ quella tra $\beta$ e $\beta'$, allora la mappa
  $$(t, s) \mapsto \begin{cases} H(2t, s) \text{ se } 0 \le t \le 1/2 \\ K(2t-1, s) \text{ se } 1/2 \le t \le 1 \end{cases}$$ è l'omotopia cercata.
\end{proof}

\begin{cor}
  L'operazione $\pi_1(X, a) \times \pi_1(X, a) \rightarrow \pi_1(X, a)$ t.c. $([\alpha], [\beta]) \mapsto [\alpha * \beta]$ è ben definita e associativa.
\end{cor}

Notazione: con un abuso, d'ora in poi indicheremo con $\alpha * \beta * \gamma$ il cammino $(\alpha * \beta) * \gamma$ o $\alpha * (\beta * \gamma)$, che non dà problemi a meno di riparametrizzazione/omotopia. Stessa cosa per $\alpha_1 * \alpha_2 * \dots * \alpha_n$.

Elemento neutro: $1_a$ è il cammino costante. Allora $1_a * \alpha$ e $\alpha * 1_a$ sono riparametrizzazioni di $\alpha$ per ogni $\alpha \in \Omega(a, a)$, per cui $[1_a]\cdot[\alpha]=[1_a * \alpha]=[\alpha]=[\alpha * 1_a]=[\alpha]\cdot[1_a]$.

Inverso: vogliamo dire che $[\bar{\alpha}]\cdot[\alpha]=[\alpha]\cdot[\bar{\alpha}]=[1_a]=1$ per ogni $\alpha \in \pi_1(X, a)$. Mostriamo che $\alpha * \bar{\alpha} \sim 1_a$, che lo stesso vale per $\bar{\alpha} * \alpha$ è analogo.
$$H(t, s)=\begin{cases} \alpha(2t) \text{ se } t \le s/2 \\ \alpha(s) \text{ se } s/2 \le t \le 1-s/2 \\ \bar{\alpha(2t-1)}=\alpha(2-2t) \text{ se } t \ge 1-s/2 \end{cases}$$ è l'omotopia cercata.

Abbiamo così dimostrato che $\pi_1(X, a)$ è un gruppo con l'operazione $[\alpha]\dots[\beta]=[\alpha * \beta]$. Vogliamo adesso studiare la dipendenza di  $\pi_1(X, a)$ da $a$. D'ora in poi, se non detto diversamenenvte, assumeremo $X$ connesso per archi (se $Y$ è la componente connessa per archi di $a$ in $X$, $\pi_1(X,a ) \cong \pi_1(Y, a)$).

Siano $a, b \in X$ e $\gamma \in \Omega(a, b)$. Poniamo $\gamma_{\sharp}: \pi_1(X, a) \rightarrow \pi_1(X, b)$ t.c. $\gamma_{\sharp}([\alpha])=[\bar{\gamma} * \alpha * \gamma]$. Osserviamo che è ben definita.

\begin{thm}
  $\gamma_{\sharp}: \pi_1(X, a) \rightarrow \pi_1(X, b)$ è un isomorfismo di gruppi.
  \begin{proof}
    $\gamma_{\sharp}$ è un omomorfismo in quanto $\gamma_{\sharp}([\alpha][\beta])=\gamma_{\sharp}([\alpha * \beta])=[\bar{\gamma} * \alpha * \beta * \gamma]=[\bar{\gamma} * \alpha * \gamma * \bar{\gamma} * \beta * \gamma]=$
    $[\bar{\gamma} * \alpha * \gamma][\bar{\gamma} * \beta * \gamma]=\gamma_{\sharp}([\alpha]) \cdot \gamma_{\sharp}([\beta])$.

    È un isomorfismo con inversa $\bar{\gamma}_{\sharp}$. Infatti,
    $\bar{\gamma}_{\sharp}(\gamma_{\sharp}([\alpha]))=\bar{\gamma}_{\sharp}([\bar{\gamma} * \alpha * \gamma])=[\bar{\bar{\gamma}} * \bar{\gamma} * \alpha * \gamma * \bar{\gamma}]=[\alpha]$.
    Analogamente, $\gamma_{\sharp}(\bar{\gamma}_{\sharp}([\beta]))=[\beta]$.
  \end{proof}
\end{thm}

\begin{cor}
  Il tipo di isomorfismo di $\pi_1(X, a)$ non dipende da $a$, per cui a volte si parla del gruppo fondamentale di $X$, $\pi_1(X)$.
\end{cor}

\begin{defn}
  Sia $\Omega(S^1, a)=\{ \gamma: Sì1 \rightarrow X \text{ con } \gamma(1)=a\}$ ($S^1$ lo vediamo in $\mathbb{C}$, cioè $1 \in S^1$ è $(1, 0)$). C'è una bigezione canonica tra $\Omega(a, a)$ e $\Omega(S^1, a)$ data da: se $\alpha \in \Omega(a, a)$, poiché
  $\alpha(0)=\alpha(1)$, $\alpha$ definisce una $\hat{\alpha}: \faktor{[0, 1]}{\{0, 1\}} \rightarrow X$ continua. Identifichiamo d'ora in poi $\faktor{[0, 1]}{\{0, 1\}}$ tramite l'identificazione $\pi: [0, 1] \rightarrow S^1, \pi(t)=e^{2 \pi i t}$, per cui
  $\hat{\alpha}:S^1 \rightarrow X$ e $\hat{\alpha}(1)=\gamma(1)=a$. L'inverso della mappa $\alpha \mapsto \hat{\alpha}$ è $\alpha(t)=\hat{\alpha}(\pi(t))$.
\end{defn}

\begin{lm} \label{Q/C=D^2}
  $Q=[0, 1] \times [0, 1], C \subseteq Q, C=\{s=1\} \cup \{t=0\} \cup \{t=1\}$. Allora $\faktor{Q}{C} \cong D^2$ tramite un omeomorfismo che manda $[t, 0]$ in $e^{2 \pi i t}$ per ogni $t \in [0, 1]$.
\end{lm}

\begin{proof}
  Non è stata fatta esplicitamente perché i conti sono orrendamente schifosi, per cui vi chiediamo di fare uno sforzo di immaginazione cercando di vedere la dimostrazione con "l'occhio della mente" (aiutatevi con un disegno e attenzione a non sbagliare con i bordi).
\end{proof}

\begin{prop}
  $\alpha \in \Omega(a, a)$, allora $[\alpha]=1$ in $\pi_1(X, a)$ $\Leftrightarrow$ $\hat{\alpha}: S^1 \rightarrow X$ si estende in maniera continua a $D^2$.
\end{prop}

\begin{proof}
  ($\implies$) Se $\alpha \sim 1_a$, esiste $H:Q \rightarrow X$ t.c. $H(t, 0)=\alpha(t), H(C)=\{a\}$ data dall'omotopia. $H$ definisce per passaggio al quoziente una funzione continua. $\hat{H}: \faktor{Q}{C} \rightarrow X$, cioè tramite l'identificazione del lemma \ref{Q/C=D^2} $\hat{H}: D^2 \rightarrow X$. Per costruzione,
  $\hat{H} \restrict(S^1)=\hat{\alpha}$, che dunque si estende come voluto.

  ($\Leftarrow$) Viceversa, se $\hat{\alpha}$ si estende a $f:D^2 \rightarrow X$, la mappa $H: Q \rightarrow X$ data da $H=f \circ \pi$, dove $\pi: \faktor{Q}{C} \rightarrow D^2$ è l'identificazione, dà un'omotopia tra $\alpha$ e $1_a$.
\end{proof}

\begin{cor}
  Sia $P \subseteq \mathbb{R}^2$ un poligono convesso con lati $e_1, \dots, e_n$ parametrizzati da $\phi_i: [0, 1] \rightarrow e_i$ e siano $\gamma: \partial P=e_1 \cup \dots \cup e_n \rightarrow X, \alpha_1=\gamma \circ \phi_i$ per ogni $i$. Allora
  $\alpha_1 * \alpha_2 * \dots * \alpha_n \sim 1_{\alpha_1(0)}$ $\Leftrightarrow$ $\gamma$ si estende a $P$ in maniera continua.
\end{cor}

\begin{proof}
  Esiste un omeomorfismo $f: P \rightarrow D^2$ con $f(\partial P)=S^1$, per cui la tesi segue da quanto già visto.
\end{proof}


\subsection{Il gruppo fondamentale come funtore}
Sia $f:X \rightarrow Y$ continua. Se $\alpha: [0, 1] \rightarrow X$ è continua, lo è anche $f \circ \alpha$, per cui $f$ induce $f_{\star}: \Omega(a, a) \rightarrow \Omega(f(a), f(a))$. Se $\alpha \sim \beta$ in $\Omega(a, a)$ e $H: I^2 \rightarrow X$ è un'omotopia di cammini tra $\alpha$ e $\beta$,
$f \circ H: I^2 \rightarrow Y$ è un'omotopia di cammini tra $f_{\star}(\alpha)$ e $f_{\star}(\beta)$. Dunque $f_{\star}$ induce $f_{\star}: \pi_1(X, a) \rightarrow \pi_1(Y, f(a))$.
\begin{nlist}
  \item $f_{\star}$ è un omomorfismo di gruppi: $f_{\star}([\alpha][\beta])=f_{\star}([\alpha * \beta])=[f \circ (\alpha \star \beta)]=[(f \circ \alpha) * (f \circ \beta)]=$
  $[f \circ \alpha][f \circ \beta]=f_{\star}([\alpha]) \cdot f_{\star}([\beta])$.
  \item Se $g: Y \rightarrow Z$, $g_{\star} \circ f_{\star}=(g \circ f)_{\star}: \pi_1(X, a) \rightarrow \pi_1(Z, g(f(a)))$.
  Infatti, $(g \circ f)_{\star}([\alpha])=[g \circ f \circ \alpha]=g_{\star}([f \circ \alpha])=g_{\star}(f_{\star}([\alpha]))=(g_{\star} \circ f_{\star})([\alpha])$.
  \item Data $\id: X \rightarrow X$, $\id_{\star}: \pi_1(X, a) \rightarrow \pi_1(X, a)$ è, ovviamente, l'identità.
\end{nlist}
Le proprietà appena elencate ci dicono che abbiamo trovato un funtore tra la categoria degli spazi topologici e quella dei gruppi.

\begin{cor}
  $f:X \rightarrow Y$ omeomorfismo $\implies$ $f_{\star}: \pi_1(X, a) \rightarrow \pi_1(Y, f(a))$ isomorfismo di gruppi.
\end{cor}

\begin{proof}
  Sia $g: X \rightarrow Y$ l'inversa continua di $f$, $g_{\star}: \pi_1(Y, f(a)) \rightarrow \pi_1(X, a)$. $\id_{\pi_1(X, a)}=(\id_X)_{\star}=(g \circ f)_{\star}=g_{\star} \circ f_{\star}$.
  $\id_{\pi_1(Y, f(a))}=(\id_Y)_{\star}=(f \circ g)_{\star}=f_{\star} \circ g_{\star}$. Questo implica che $f_{\star}$ e $g_{\star}$ sono isomorfismi e sono uno l'inverso dell'altro.
\end{proof}

\begin{oss}
  $X$ omeomorfo a $Y$ implica $X$ omotopicamente equivalente a $Y$. Domanda: $X$ omotopicamente equivalente a $Y$ implica $\pi_1(X) \cong \pi_1(Y)$?
\end{oss}

\begin{prop} \label{id->gamma}
  Sia $f:X \rightarrow X$ omotopa all'identità tramite $H: X \times I \rightarrow X$ e sia $\gamma: [0, 1] \rightarrow X$ t.c. $\gamma(s)=H(a, s)$. Allora le mappe $f_{\star}: \pi_1(X, a) \rightarrow \pi_1(X, f(a))$ e $\gamma_{\sharp}: \pi_1(X, a) \rightarrow \pi_1(X, f(a))$ coincidono.
\end{prop}

\begin{proof}
  Per ogni $\alpha \in \Omega(a, a)$ definisco $K: I^2 \rightarrow X, K(t, s)=H(\alpha(t), s)$. Dunque la giunzione $\alpha * \gamma * \bar{f_{\star}(\alpha)} * \bar{\gamma}$ è omotopa all'identità (esercizio; hint: "disegnate" $K$, poi al tempo $t$ fermatevi prima lungo $\gamma$), dunque per il corollario \ref{pol_est} si estende da $\partial I^2$ a $I^2$. Lo stesso vale per
  $f_{\star}(\alpha) * \bar{\gamma} * \bar{\alpha} * \gamma$ che è perciò omotopo come cammino a $1_{f(a)}$ e ciò implica che
  $1_{\pi_1(X, f(a))}=[f_{\star}(\alpha) * (\bar{\gamma} * \bar{\alpha} * \gamma)]=[f_{\star}(\alpha)][\bar{\gamma} * \bar{\alpha} * \gamma)]$, cioè
  $f_{\star}([\alpha])=[\bar{\gamma} * \bar{\alpha} * \gamma)]^{-1}=[\bar{\gamma} * \alpha * \gamma]=\gamma_{\sharp}([\alpha])$.
\end{proof}

\begin{thm} \label{eq_omo->iso}
  $f: X \rightarrow Y$ equivalenza omotopica $\implies$ $f_{\star}: \pi_1(X, a) \rightarrow \pi_1(Y, f(a))$ è un isomorfismo.
\end{thm}

\begin{proof}
  Sia $g: Y \rightarrow X$ un'inversa omotopica di $f$. Poiché $g \circ f \sim \id_X$, per la proposizione \ref{id->gamma} $(g \circ f)_{\star}=g_{\star} \circ f_{\star}: \pi_1(X, a) \rightarrow \pi_1(X, g(f(a)))$ coincide con $\gamma_{\sharp}$ per qualche $\gamma \in \Omega(a, g(f(a)))$, dunque è un isomorfismo.
  Analogamente, $f_{\star} \circ g_{\star}$ è un isomorfismo. Dunque $f_{\star}$ è iniettiva e suriettiva, da cui la tesi.
\end{proof}

\begin{defn}
  $X$ è \textsc{semplicemente connesso} se $X$ è connesso per archi e $\pi_1(X, a)={1}$ per ogni $a \in X$.
\end{defn}

\begin{cor}
  Se $X$ è contrattile, $X$ è semplicemente connesso.
\end{cor}

\begin{prop}
  $A \subseteq X$ retratto, con inclusione $i: A \rightarrow X$ e retrazione $r: X \rightarrow A$. Allora
  \begin{nlist}
    \item $i_{\star}$ è iniettiva e $r_{\star}$ è suriettiva (per qualsiasi scelta del punto base in $A$);
    \item se la retrazione è per deformazione, $i_{\star}$ e $r_{\star}$ sono isomorfismi.
  \end{nlist}
\end{prop}

\begin{proof}
  \begin{nlist}
    \item Dato $a \in A$, $r \circ i= \id_A \implies r_{\star} \circ i_{\star}=\id_{\pi_1(A, a)}$, da cui la tesi.
    \item Segue dal fatto che $i$ e $r$ sono equivalenze omotopiche e dal teorema \ref{eq_omo->iso}.
  \end{nlist}
\end{proof}

\begin{oss}
  Se $A \subseteq X$, $i_{\star}: \pi_1(A, a) \rightarrow \pi_1(X, a)$ può non essere iniettiva. Vedremo che è così per $X=D^2$ e $A=S^1$.
\end{oss}

\begin{exc}
  Abbiamo studiato la mappa $\Omega(a, a) \rightarrow \Omega(S^1, a)$ t.c. $\alpha \mapsto \hat{\alpha}$ e abbiamo osservato che se $[\alpha]=[\beta]$ in $\pi_1(X, a)$ allora $\hat{\alpha} \sim \hat{\beta}$ come mappe $S^1 \rightarrow X$. Perciò è ben definita la mappa $\psi: \pi_1(X, a) \rightarrow [S^1, X]$ t.c.
  $\psi([\alpha])=[\hat{\alpha}]$. Dimostrare che, se $X$ è connesso per archi, $\psi$ è suriettiva e $\psi([\alpha])=\psi([\beta])$ $\Leftrightarrow$ $[\alpha]$ è coniugato a $[\beta]$ in $\pi_1(X, a)$ (dunque $[S^1, X]$ è in bigezione con le classi di coniugio di $\pi_1(X, a)$).
\end{exc}


\subsection{Rivestimenti}
\begin{defn}
  $f:X \longrightarrow Y$ è un \textit{omeomorfismo locale} se per ogni $p \in X$ esistono $U$ aperto con $p \in U$ e $V$ aperto di $Y$ con $f(p) \in V$ t.c. $f(U)=V$ e $f \restrict{U}:U \longrightarrow V$ è un omeomorfismo.
\end{defn}

\begin{ftt}
  \begin{nlist}
    \item Un omeomorfismo locale è una mappa aperta;
    \item $f$ omeomorfismo locale $\implies$ $f^{-1}(y)$ è discreto per ogni $y \in Y$.
  \end{nlist}
\end{ftt}

\begin{defn}
  Una mappa continua $p:E \longrightarrow X$ è un \textsc{rivestimento} se $X$ è connesso per archi e per ogni $x \in X$ esiste $U$ aperto di $X$, $x \in U$ t.c. $\displaystyle p^{-1}(U)=\bigsqcup_{i \in I} V_i$, $I \not=\emptyset$ e $V_i$ aperto in $E$ e $p \restrict{V_i}:V_i \longrightarrow U$ è un omeomorfismo per ogni $i \in I$.
\end{defn}

\begin{ftt}
  \begin{nlist}
    \item Un rivestimento è anche un omeomorfismo locale;
    \item un rivestimento è suriettivo;
    \item un intorno $U$ come quello dato dalla definizione di rivestimento si dice \textit{ben rivestito}.
  \end{nlist}
\end{ftt}

\begin{ex}
  \begin{nlist}
    \item $p: \mathbb{R} \longrightarrow S^1$, $p(t)=(\cos{2\pi t}, \sin{2\pi t})$ è un rivestimento;
    \item $p: (-1, 1) \longrightarrow S^1$, $p(t)=(\cos{2\pi t}, \sin{2\pi t})$ è un omeomorfismo locale suriettivo, ma non un rivestimento;
    \item $\pi: \faktor{ \mathbb{R} \times \{ -1, 1 \} }{\sim} \longrightarrow \mathbb{R}$, $(x, \epsilon) \sim (y, \epsilon') \Leftrightarrow (x, \epsilon)=(y, \epsilon') \lor x=y \not=0, \pi([x, \epsilon])=x$ è un omeomorfismo locale suriettivo, ma non un rivestimento;
    \item $\pi: S^n \longrightarrow \mathbb{P}^n(\mathbb{R})$, $\pi(x)=[x]$ è un rivestimento.\
  \end{nlist}
\end{ex}

\begin{defn}
  Siano $p: E \longrightarrow X$ un rivestimento e $f: Y \longrightarrow X$ una funzione continua. Una funzione continua $\tilde{f}: Y \longrightarrow E$ si dice \textsc{sollevamento di $f$ rispetto a $p$} se $p \circ \tilde{f}=f$.
\end{defn}

\begin{prop} \label{unic_soll}
  (Unicità del sollevamento)
  Sia $p: E \longrightarrow X$ un rivestimento, $Y$ connesso, $f: Y \longrightarrow X$ continua. Siano $\tilde{f}, \tilde{g}$ sollevamenti di $f$ rispetto a $p$. Se esiste $y_0 \in Y$ con $\tilde{f}(y_0)=\tilde{g}(y_0)$, allora $\tilde{f}=\tilde{g}$.
\end{prop}

\begin{proof}
  Basta vedere che $\Omega=\{ y \in Y | \tilde{f}(y)=\tilde{g}(y)\} \subseteq Y$ è sia aperto che chiuso. Per ipotesi $\Omega \not= \emptyset$ e $Y$ è connesso, per cui avremmo necessariamente $\Omega=Y$, da cui la tesi.

  $\Omega$ è aperto: sia $y \in \Omega$. Allora $\tilde{f}(y)=\tilde{g}(y)=\tilde{x}_0$. Siano $x_0=p(\tilde{x}_0)=f(y)$ e $U$ un intorno di $x_0$ ben rivestito.
  $\displaystyle p^{-1}(U)=\bigsqcup_{i \in I} V_i$. Sia $i_0 \in I$ t.c. $\tilde{x}_0 \in V_{i_0}$.
  $\tilde{f}, \tilde{g}$ continue $\implies$ esiste un intorno $W$ di $y$ t.c. $\tilde{f}(W) \subseteq V_{i_0}, \tilde{g}(W) \subseteq V_{i_0}$.
  $p \restrict{V_{i_0}}$ è iniettiva, dunque da $p \circ \tilde{f}=p \circ \tilde{g}$ ne deduciamo che $\tilde{f} \restrict{W}=\tilde{g} \restrict{W} \implies W \subseteq \Omega$.

  $\Omega$ chiuso: $\Omega^C=Y \setminus \Omega$ è aperto. Sia $y \in \Omega^C \implies \tilde{f}(y) \not= \tilde{g}(y)$. Tuttavia $(p \circ \tilde{f})(y)=(p \circ \tilde{g})(y)=x_0$.
  Se $U$ è un intorno ben rivestito di $x_0$, $\displaystyle p^{-1}(U)=\bigsqcup_{i \in I} V_i$ e da $\tilde{f}(y) \not= \tilde{g}(y)$ ne deduciamo che $\tilde{f}(y) \in V_{i_0}, \tilde{g}(y)=V_{i_1}, i_0 \not= i_1$. Per continuità di $\tilde{f}$ e $\tilde{g}$ si conclude.
\end{proof}

\begin{thm}
  Sia $p:E \longrightarrow X$ un rivestimento, $\gamma:[0, 1] \longrightarrow X$ un cammino continuo. Siano $x_0=\gamma(0)$ e $\tilde{x}_0 \in p^{-1}(x_0)$.
  Allora esiste un unico $\tilde{\gamma}:[0, 1] \longrightarrow E$ cammino continuo con $\tilde{\gamma}(0)=\tilde{x}_0$ e $p \circ \tilde{\gamma}=\gamma$.
\end{thm}

\begin{proof}
  L'unicità segue dalla proposizione \ref{unic_soll}. Ricopriamo adesso $X$ con aperti ben rivestiti $\{U_i\}_{i \in I}$ e sia $\epsilon>0$ numero di Lebesgue per il ricoprimento $\{\gamma^{-1}(U_i), i \in I\}$ di $[0, 1]$.
  Perciò, se $1/n < \epsilon$, per ogni $k=0, 1, \dots, n-1$, $\gamma([k/n, (k+1)/n]) \subseteq U_{i_k}$ ben rivestito.
  Definiamo induttivamente $\tilde{\gamma}$ su $[k/n, (k+1)/n]$ come segue: per definizione di rivestimento esiste $V_{i_0}$ aperto di $E$ t.c. $\tilde{x}_0 \in V_{i_0}$ e $p_0=p \restrict{V_{i_0}} V_{i_0} \longrightarrow U_{i_0}$ è un omeomorfismo;
  per ogni $t \in [0, 1/n]$, $\tilde{\gamma}(t)=p_0^{-1}(\gamma(t))$. Una volta definito $\tilde{\gamma}$ continuo su $[0, k/n]$, troviamo $V_{i_k} \subseteq E$ t.c. $\tilde{\gamma}(k/n) \in V_{i_k}$ e $p_k=p \restrict{V_{i_k}}: V_{i_k} \longrightarrow U_{i_k}$ è un omeomorfismo.
  Poniamo dunque $\tilde{\gamma}(t)=p_k^{-1}(\gamma(t))$ per ogni $t \in [k/n, (k+1)/n]$. È ora facile verificare che questa definizione soddisfa le condizioni richieste.
\end{proof}

\begin{prop}
  Sia $p:E \longrightarrow X$ un rivestimento, $x_0, x_1 \in X$. Allora esiste una bigezione tra $p^{-1}(\{x_0\})$ e $p^{-1}(\{x_1\})$. La cardinalità di $p^{-1}(\{x_0\})$ si dice \textit{grado} del rivestimento e si indica con $\deg{p}$.
\end{prop}

\begin{proof}
  $X$ connesso per archi $\implies$ esiste $\gamma \in \Omega(x_0, x_1)$ e, dato $\tilde{x}_0 \in p^{-1}(\{x_0\})$, esiste un sollevamento $\tilde{\gamma}_{\tilde{x}_0}: [0, 1] \longrightarrow E$ con punto iniziale $\tilde{x}_0$. È ben definita
  \begin{align*}
  \psi:p^{-1}(\{x_0\}) &\longrightarrow p^{-1}(\{x_1\}) \\
  \tilde{x}_0 &\longmapsto \tilde{\gamma}_{\tilde{x}_0}(1)
  \end{align*}
  che ammette inversa ottenuta da $\bar{\gamma} \in \Omega(x_1, x_0)$.
\end{proof}

\begin{defn}
  (Sezione) Dato $U \subseteq X$, una \textsc{sezione} $s:U \longrightarrow E$ di un rivestimento $p:E \longrightarrow X$ è una mappa continua t.c. $p(s(x))=x$ per ogni $x \in U$. Si dice \textit{locale} se $U$ è un aperto di $X$, \textit{globale} se $U=X$.
\end{defn}

\begin{oss}
  Per definizione, se $p$ è un rivestimento, per ogni $x \in X$ esiste $U \subseteq X$ aperto con sezione locale definita su $U$.
\end{oss}

\begin{thm} \label{soll_omo}
  (Sollevamento delle omotopie) Siano $p:E \longrightarrow X$ un rivestimento, $f:Y \longrightarrow X$ continua con $Y$ localmente connesso per archi. Sia $F:Y \times[0, 1] \longrightarrow X$ t.c. $F(y, 0)=f(y)$ per ogni $y \in Y$. Sia $\tilde{f}:Y \longrightarrow E$ un sollevamento di $f$, cioè $p \circ \tilde{f}=f$.
  Allora esiste $\tilde{F}:Y \times [0, 1] \longrightarrow E$ che solleva $F$, cioè $p \circ \tilde{F}=F$, ed estende $\tilde{f}$, cioè $\tilde{F}(y, 0)=\tilde{f}(y)$ per ogni $y \in Y$.
\end{thm}

\begin{proof}
  Poniamo, per ogni $y_0 \in Y$, $\gamma_{y_0}:[0, 1] \longrightarrow X$ il cammino $\gamma_{y_0}(t)=F(y_0, t)$. Sia $\tilde{\gamma}_{y_0}$ il sollevamento di $\gamma_{y_0}$ a partire da $\tilde{f}(y_0)$. Poniamo infine $\tilde{F}(y_0, t)=\tilde{\gamma}{y_0}(t)$.
  È chiaro che $p \circ \tilde{F}=F$. Serve $\tilde{F}$ continua, basta vedere che per ogni $(y_0, t) \in Y \times [0, 1]$ esiste un intorno $U$ in $Y \times [0, 1]$ t.c. $\tilde{F}\restrict{U}=s \circ F\restrict{U}$ dove $s$ è una sezione locale (continua) del rivestimento.
  Poiché $[0, 1]$ è compatto, analizzando il ricoprimento $\{F^{-1}(W) | W \text{ aperto ben rivestito di } X\}$ otteniamo un numero finito di aperti $Z_1, \dots, Z_n$ di $Y \times [0, 1]$ che ricoprono $y_0 \times [0, 1] \cong [0, 1]$ e sono t.c. $F(Z_i)$ è contenuto in un intorno ben rivestito di $X$ per ogni $i$.
  Possiamo supporre $Z_i=A_i \times B_i$, $A_i$ aperto di $Y$ e $B_i$ aperto di $[0, 1]$, $y_0 \in A_i$ per ogni $i$. Sia $A= \bigcap_{i=1}^n A_i$ aperto di $Y$ con $y_0 \in A$ e osserviamo che $\bigcup_{i=1}^n B_i=[0, 1]$.
  Sia $1/k<$ numero di Lebesgue di $\{B_1, \dots, B_n\}$. Allora per ogni $j=0, 1, \dots, k-1$ $A \times [j/k, (j+1)/k] \subseteq A_i \times B_i$ per qualche $i$, perciò $F(A \times [j/k, (j+1)/k]) \subseteq U_j$ con $U_j$ aperto ben rivestito.
  Ora, per definizione di sollevamenti di cammini, $\tilde{F} \restrict{A \times [0, 1/k]}=s_0 \circ F$ dove $s_0: U_0 \longrightarrow E$ è una sezione locale (si suppone $A$ connesso per archi). Induttivamente $\tilde{F} \restrict{A \times [j, (j+1)/k]}=s_j \circ F$ con $s_j: U_j \longrightarrow E$ sezione locale per ogni $j$, da cui la tesi.
\end{proof}

\begin{cor}
  Siano $\gamma_1, \gamma_2:[0, 1] \longrightarrow X$ cammini omotopi (a estremi fissi), $\gamma_1(0)=\gamma_2(0)=x_0$.
  Allora, se $\tilde{x}_0 \in p^{-1}(x_0)$, $(\tilde{\gamma}_1)_{\tilde{x}_0}$ è omotopo a $(\tilde{\gamma}_2)_{\tilde{x}_0}$ (a estremi fissi).
  In particolare, $(\tilde{\gamma}_1)_{\tilde{x}_0}(1)=(\tilde{\gamma}_2)_{\tilde{x}_0}(1)$.
\end{cor}

\begin{proof}
  Si prenda come $F$ l'omotopia tra i due cammini. Bisogna verificare che $\tilde{F}$ data dal teorema \ref{soll_omo} è un'omotopia tra i sollevamenti dei due cammini, cioè che è a estremi fissi e solleva le cose giuste. Tutto ciò segue dall'unicità del sollevamento data dalla proposizione \ref{unic_soll}.
\end{proof}

\begin{cor}
  Sia $p:E \longrightarrow X$ un rivestimento. Allora $p_{\star}:\pi_1(E, \tilde{x}_0) \longrightarrow \pi_1(X, x_0)$ è iniettiva.
\end{cor}

\begin{proof}
  $\alpha=[\gamma] \in \ker{p_{\star}} \implies p \circ \gamma \sim c_{x_0}$ (cammino costante)
  $\implies \gamma=\widetilde{(p \circ \gamma)}_{\tilde{x}_0} \sim \widetilde{(c_{x_0})}_{\tilde{x}_0}=c_{\tilde{x}_0} \implies [\gamma]=1 \in \pi_1(E, \tilde{x}_0)$.
\end{proof}

\begin{defn}
  (Azione di monodromia) Siano $p:E \longrightarrow X$ un rivestimento, $x_0 \in X$, $F=p^{-1}(x_0)$. Allora esiste un'azione destra di $\pi_1(X, x_0)$ su $F$ definita così:
  \begin{align*}
    F \times \pi_1(X, x_0) &\longrightarrow F \\
    (\tilde{x}, [\gamma]) &\longmapsto \tilde{x} \cdot [\gamma]=\tilde{\gamma}_{\tilde{x}}(1),
  \end{align*}
  detta \textsc{azione di monodromia}.
\end{defn}

\begin{oss}
  Sia $p:E \longrightarrow X$ un rivestimento, $F=p^{-1}(x_0) \subseteq E, x_0 \in X$. La monodromia $F \times \pi_1(X, x_0)$ è transitiva se e solo se $E$ è connesso per archi.
\end{oss}

\begin{defn}
  Un rivestimento $p:E \longrightarrow X$ si dice \textsc{universale} se $E$ è semplicemente connesso, in particolare connesso per archi.
\end{defn}

\begin{prop}
  Sia $p:E \longrightarrow X$ un rivestimento universale, $x_0 \in X$, $\tilde{x}_0 \in F=p^{-1}(x_0)$. La mappa $\psi:\pi_1(X, x_0) \longrightarrow F, \psi([\gamma])=\tilde{x}_0 \cdot [\gamma]$ è bigettiva, per cui $|\pi_1(X, x_0)|=|F|$.
\end{prop}

\begin{proof}
  $E$ connesso per archi $\implies$ la suriettività viene dal fatto che l'azione è transitiva.
  Se $\psi([\gamma_1])=\psi([\gamma_2])$, $\tilde{x}_0 \cdot [\gamma_1]=\tilde{x}_0 \cdot [\gamma_2] \implies (\tilde{\gamma}_1)_{\tilde{x}_0}(1)=(\tilde{\gamma}_2)_{\tilde{x}_0}(1)$.
  Poiché $E$ è semplicemente connesso, $(\tilde{\gamma}_1)_{\tilde{x}_0}=(\tilde{\gamma}_2)_{\tilde{x}_0} \implies \gamma_1=p \circ \tilde{\gamma}_1 \sim p \circ \tilde{\gamma}_2=\gamma_2 \implies [\gamma_1]=[\gamma_2]$, da cui l'iniettività.
\end{proof}


\subsection{Alcuni gruppi fondamentali}
Abbiamo adesso gli strumenti per poter determinare il gruppo fondamentale di alcuni spazi topologici. Iniziamo da quello di $S^1$.

\begin{thm}
  $\pi_1(S^1) \simeq \mathbb{Z}$.
\end{thm}

\begin{proof}
  Analizziamo il rivestimento
  \begin{align*}
    p: \mathbb{R} &\longrightarrow S^1 \\
    t &\longmapsto (\cos{2\pi t}, \sin{2\pi t}),
  \end{align*}
  $F=p^{-1}(1, 0)=\mathbb{Z}$. Poniamo $\psi:\pi_1(S^1, (1, 0)) \longrightarrow \mathbb{Z}$, $\psi([\gamma])=0 \cdot [\gamma]$. $\mathbb{R}$ è semplicemente connesso, quindi basta vedere che $\psi$ è un omeomorfismo.
  Dati $[\alpha], [\beta] \in \pi_1(S^1, x_0), x_0=(1, 0)$,
  abbiamo che $\psi([\alpha] \cdot [\beta])=\psi([\alpha * \beta])=\widetilde{(\alpha * \beta)}_0(1)=\tilde{\alpha}_0*\tilde{\beta}_{\tilde{\alpha}_0(1)}(1)=\tilde{\beta}_{\tilde{\alpha}_0(1)}(1)$.
  Ora, $\tilde{\beta}_{\tilde{\alpha}_0(1)}$ e $\tilde{\alpha}_0(1)+\tilde{\beta}_0$ sono entrambi sollevamenti di $\beta$ a partire dallo stesso punto iniziale.
  Dunque $\psi([\alpha] \cdot [\beta])=(\tilde{\alpha}_0(1)+\tilde{\beta}_0)(1)=\tilde{\alpha}_0(1)+\tilde{\beta}_0(1)=\psi([\alpha])+\psi([\beta])$, da cui la tesi.
\end{proof}

\begin{prop}
  $R \subseteq X$ retratto, allora per ogni $x_0 \in R$ $i_{\star}: \pi_1(R, x_0) \longrightarrow \pi_1(X, x_0)$ è iniettiva e $r_{\star}: \pi_1(X, x_0) \longrightarrow \pi_1(R, x_0)$ è suriettiva.
\end{prop}

\begin{proof}
  Poiché $r \circ i$ è l'identità di $R$, $r_{\star} \circ i_{\star}=\id_{\pi_1(R, x_0)}$, da cui segue la tesi.
\end{proof}

\begin{cor}
  $S^1=\partial D^2$ non è un retratto di $D^2$.
\end{cor}

\begin{thm}
  (Teorema del punto fisso di Brouwer) $f:D^2 \longrightarrow D^2$ continua $\implies$ $f$ ha almeno un punto fisso.
\end{thm}

\begin{proof}
  Se $f(x)\not=x$ per ogni $x \in D^2$, costruiamo una retrazione \\
  $r:D^2 \longrightarrow S^1$ come segue:
  \begin{center}
  \begin{tikzpicture}[line cap=round,line join=round,>=triangle 45,x=1.0cm,y=1.0cm]
    \clip(2.99,-4.05) rectangle (10.59,2.55);
    \draw(6.82,-0.66) circle (2.8cm);
    \draw [domain=2.99:10.59] plot(\x,{(-4.53--0.68*\x)/1.6});
    \begin{scriptsize}
      \fill [color=black] (9.06,1.02) circle (1.5pt);
      \draw[color=black] (9.14,1.35) node {$r(x)$};
      \fill [color=black] (7.46,0.34) circle (1.5pt);
      \draw[color=black] (7.54,0.65) node {$x$};
      \fill [color=black] (5.68,-0.42) circle (1.5pt);
      \draw[color=black] (5.7,-0.09) node {$f(x)$};
    \end{scriptsize}
  \end{tikzpicture}
\end{center}
cerchiamo $t>0$ t.c. $\|f(x)+t(x-f(x))\|^2=1$, poniamo poi $r(x)=f(x)+t(x-f(x))$ per cui basta vedere che $t$ dipende in modo continuo da $x$. Risolviamo $1=\|f(x)\|^2+2t\left \langle f(x), x-f(x) \right \rangle+t^2\|(x-f(x))\|^2$ e prendiamo $t>0$.
\end{proof}

\begin{oss}
  Tramite $\pi_1(S^1) \simeq \mathbb{Z}$, l'elemento $n \in \mathbb{Z}$ è identificato da $\gamma(t)=(\cos{2\pi t}, \sin{2\pi t})$ in quanto $\tilde{\gamma}_0(t)=nt$ e $\tilde{\gamma}_0(1)=n$.
\end{oss}

\begin{exc}
  \begin{align*}
    f:\mathbb{C} \setminus \{0\} &\longrightarrow \mathbb{C} \setminus \{0\} \\
    z &\longmapsto z^n
  \end{align*}
  è un rivestimento di grado $n$.
\end{exc}

\begin{thm}
  Dati $X, Y$ e $x_0 \in X, y_0 \in Y$, \\ $\pi_1(X \times Y, (x_0, y_0))=\pi_1(X, x_0) \times \pi_1(Y, y_0)$.
\end{thm}

\begin{proof}
  È lasciata come esercizio per il lettore. Traccia: $\pi_X:X \times Y \longrightarrow X, \pi_Y:X \times Y \longrightarrow Y$ le proiezioni,
  \begin{align*}
    i:X &\longrightarrow X \times Y \\
    x &\longmapsto (x, y_0), \\
    j:Y &\longrightarrow X \times Y \\
    y &\longmapsto (x_0, y).
  \end{align*}
  Poniamo $\psi:\pi_1(X \times Y, (x_0, y_0)) \longrightarrow \pi_1(X, x_0) \times \pi_1(Y, y_0)$, $\psi(\alpha)=((\pi_X)_{\star}(\alpha), (\pi_Y)_{\star}(\alpha))$. Si verifica che funziona.
\end{proof}

\begin{cor}
  $\pi_1(S^1 \times S^1)=\mathbb{Z} \oplus \mathbb{Z}$.
\end{cor}

\begin{exc}
  Siano $\alpha$ che genera $\mathbb{Z} \times \{0\}$ e $\beta$ che genera $\{0\} \times \mathbb{Z}$ sul toro. Visualizzare l'omotopia di cammini tra $\alpha * \beta$ e $\beta * \alpha$.
\end{exc}


\subsection{Gruppi liberi}
Si fa un po' di algebra perch\'e s\`i.

\begin{defn}
    Siano $\{G_i\}$ con $i\in I$ dei gruppi. Il loro prodotto libero \`e una coppia $(G, \{\phi_i\}_{i\in I})$ con $G$ gruppo e i $\phi_i\colon G_i\longrightarrow G$ omomorfismi tali che per ogni gruppo $H$ e famiglia di omomorfismi $\psi_i\colon G_i \longrightarrow H$ esista unico un omomrfismo $\psi\colon G\longrightarrow H$ con la propriet\`a che $\psi\phi_i = \psi_i$ per ogni $i\in I$.
    Spesso si indicher\`a il prodotto libero con il simbolo $\Conv_i G_i$.
\end{defn}

\begin{prop}
    Il prodotto libero \`e unico a meno di isomorfisimi.
\end{prop}
\begin{proof}
    Siano $(G, \phi_i)$ e $(G', \phi'_i)$ prodotti liberi dei $G_i$. La dimostrazione segue lo schema che si \`e sempre fatto con le propriet\`a universali. Inizio con lo scrivere la definizione del prodotto libero $G$, utilizzando $H=G'$ e $\psi_i=\phi'_i$. Dunque ho garantita un'unica funzione $\phi'\colon G \longrightarrow G'$ tale che $\phi'\phi_i=\phi'_i$.
    Analogamente trovo una funzione $\phi\colon G' \longrightarrow G$.
    Allora $\phi_i = \phi\phi'\phi_i$ e $\phi'_i=\phi'\phi\phi'_i$, da cui si ricava che $\phi\phi'$ e $\phi'\phi$ sono identit\`a, e dunque $\phi$ e $\phi'$ sono gli isomorfisimi tra $G$ e $G'$.
\end{proof}


\begin{prop}
    Il prodotto libero dei $G_i$ esiste.
\end{prop}
\begin{proof}
    Sia $W = \cup_{i\in I}(G_i\setminus \{e_i\})$, dove gli $e_i$ sono le unit\`a nei rispettivi gruppi. Sia $W^*$ l'insieme delle parole finite (eventualmente vuote) sull'alfabeto $W$. Definisco anche
        \[
        G = \{g \in W^* \quad | \quad \forall k \in \mathbb{N},\quad g_k\in G_i \Rightarrow g_{k+1} \notin G_i\}
        \]
        Su $G$ voglio definire un'operazione che lo renda un gruppo. Lo faccio con una concatenazione pi\`u un'eventuale riduzione cio\`e:
        \[
        (g_1, \dots, g_i)(h_1,\dots, h_j) =
        \begin{cases}
            (g_1, \dots, g_i, h_1, \dots, h_j) & \text{se $G_i \neq H_1$}\\
            (g_1, \dots, g_ih_1, \dots, h_j) & \text{se $G_i = H_1$ e $g_ih_1 \neq e$}\\
            (g_1, \dots, g_{i-1})(h_2, \dots, h_j) & \text{altrimenti}

        \end{cases}
        \]

        Questo \`e un gruppo con elemento neutro la parola vuota $()$. Bisogna trovare degli omomorfismi e poi far vedere che rispettano la propriet\`a universale per essere il prodotto libero cercato.

        Possiamo equipaggiare ogni gruppo con una funzione $\phi_i\colon G_i\longrightarrow G$ che associa $\phi_i(x)=(x)$ la parola di una lettera. Si prenda un gruppo $H$ e degli omomorfismi $\psi\colon G_i\longrightarrow H$. Pongo allora $\psi\colon G\longrightarrow H$ come
        \[
            \psi((g_1, \dots, g_n)) = \psi_{i_1}(g_1)\cdots\psi_{i_n}(g_n)
        \]
        Si verifica che tale $\psi$ \`e effettivamente l'omomorfismo cercato, unico per costruzione.
\end{proof}

\begin{defn}
    Dato $S=\{x_i\}$ l'insieme $F(S)=\Conv_i G_i$ \`e il gruppo libero generato da $S$, dove $G_i=\{x_i^m \ |\ m\in\mathbb{Z}\}\cong \mathbb{Z}$.
\end{defn}

\begin{oss}
    Si ha canonicamente un'immersione $i\colon S\hookrightarrow F(S)$, tale per cui, preso un gruppo $H$ e una funzione $\psi\colon S \rightarrow H$, esiste unico un omomorfismo $\phi\colon F(S)\rightarrow H$ tale che $\phi \circ i = \psi$.
\end{oss}

\begin{defn}
    Dato un gruppo $G$ e un sottoinsieme $S\subseteq G$, la chiusura normale di $S$, detta $N(S)$, \`e il pi\`u piccolo sottgruppo normale contenente $S$.
\end{defn}

\begin{oss}
    Ogni chiusura normale si pu\`o esprimere come Span dei coniugati.
\end{oss}

\begin{defn}
    Dato un insieme $S$ e un sottoinsieme $R\subseteq F(S)$, si definisce la presentazione come $\langle S \ |\ R\rangle = F(S)/N(R)$.
\end{defn}

\begin{ex}
    \begin{nlist}
        \item $\mathbb{Z} \cong \langle 1 \rangle$;
        \item $\mathbb{Z}*\mathbb{Z}\cong F(\{a,b\})$;
        \item $\mathbb{Z}\times\mathbb{Z}\cong\langle a,b\ |\ aba^{-1}b^{-1}\rangle$.
    \end{nlist}
\end{ex}

\begin{prop}
    Sia $R\subseteq F(S)$, $\psi\colon F(S)\rightarrow H$ un omomrfismo. Se $\psi(R)=\{e\}$, allora esiste un omomorfismo $\bar{\psi}\colon \langle S\ |\ R \rightarrow H$ tale che $\bar{\psi}([s])=\psi(s)$ per ogni $s\in F(S)$.
\end{prop}

\begin{prop}
    Sia $G_i = \langle S_i\ |\ R_i\rangle$ $i=1,2,3$, $\phi_1\colon G_0 \rightarrow G_1$, $\phi_2\colon G_0\rightarrow G_2$. Sia $G = G_1 * G_2 /N$ con $N = N(\{\phi_1(g)\phi_2(g)^{-1}\ |\ g\in G_0\})$.

    Allora $$G\cong \langle S_1 \cup S_2\ |\ R_1\cup R_2\cup R\rangle$$ con $R =\{\tilde{\phi_1}(s)\tilde{\phi_2}(s)^{-1}\ |\ s\in S_0\}$, $\tilde{\phi_i}\colon F(S_0)\rightarrow F(S_i)$ le estensioni naturali di $\phi_i$.
\end{prop}


\subsection{Il teorema di Van Kampen}
\begin{thm}(di Van Kampen)
    Sia $X$ uno spazio topologico con $X=A\cup B$ dove $A$, $B$ e $A\cap B$ sono aperti connessi per archi. Si considerino le inclusioni:
    \begin{align*}
        \alpha\colon& A\cap B \longrightarrow A\\
        \beta\colon& A\cap B \longrightarrow B\\
        f\colon& A \longrightarrow X\\
        g\colon& B \longrightarrow X
    \end{align*}
    Sia inoltre $G$ un gruppo e si considerino ancora gli omomorfismi
    \begin{align*}
        h\colon& \pi_1(A)\longrightarrow G\\
        k\colon& \pi_1(B)\longrightarrow G
    \end{align*}
    tali per cui $h\ \alpha_* = k\ \beta_*$.

    Allora esiste unico un omomorfismo $\phi\colon\pi_1(X)\longrightarrow G$ tale che $\phi\ f_* = h$ e $\phi\ g_*=k$.
\end{thm}

\begin{proof}
    La dimostrazione \`e una cosa bestiale in termini di grafici e disegni.

     Arriver\`a.
\end{proof}

\begin{cor}
    Nelle notazioni del teorema precedente, si ha che
    \[
        \pi_1(X)\cong \frac{\pi_1(A) * \pi_1(B)}{N}
    \]
    con $N = N(\{i\alpha_*(g)\ j \beta_*(G)^{-1}\ |\ g\in\pi_1(A\cap B)\})$, dove $i$ e $j$ sono le immersioni dei $\pi_1$.
\end{cor}
\begin{proof}
    Applico la definizione del prodotto libero con gli omomorfismi $f_*$ e $g_*$. Questo mi d\`a un omomorfismo $\psi$ che fa commutare il diagramma.
    \begin{center}\begin{tikzcd}
    \pi_1(A) \arrow[rd, "i"'] \arrow[rrrd, "f_*"] &                                      &  &          \\
                                                  & \pi_1(A)*\pi_1(B) \arrow[rr, "\psi"] &  & \pi_1(X) \\
    \pi_1(B) \arrow[ru, "j"] \arrow[rrru, "g_*"]  &                                      &  &
\end{tikzcd}\end{center}
    Con un po' di conti si mostra che $\psi(N)=\{e\}$, il che mi permette di applicare una proposizione sui prodotti liberi che mi garantisce l'esistenza di un omomorfismo
    \[
        \bar{\psi}\colon\pi_1(X)\longrightarrow \frac{\pi_1(A) * \pi_1(B)}{N}
    \]
    Per esercizio si verifichi che la funzione data da Van Kampen (utilizzando come $G$ il quoziente di gruppi cercato, $h=i$ e $k=j$) \`e un'inversa di $\psi$.
\end{proof}

\begin{ex}
    $\pi_1(S^n)=\{e\}$ per $n\geq 2$.
\end{ex}
\begin{proof}
    Prendo $A=S^n\setminus\{(1,0,\dots,0)\}$ e $A=S^n\setminus\{(1,0,\dots,0)\}$. La proiezione stereografica mostra che sono entrambi semplicemente connessi, e dunque anche $S^n$ lo \`e.
\end{proof}

\begin{ex}
    $\pi_1(\mathbb{P}^n(\mathbb{C}))$ \`e banale per ogni $n$.
\end{ex}
\begin{proof}
    Per induzione, il caso $n=0$ funziona. Prendiamo $H = \{[x_0:\dots:x_n]\ | \ x_0=0\}$, $A = \mathbb{P}^n\setminus H$ e $B= \mathbb{P}^n\setminus \{[1:0:\dots:0]\}$. Si ha che $A\cong \mathbb{C}^n$ con
    \[
        [x_0:\dots:x_n]\mapsto(\frac{x_1}{x_0},\dots, \frac{x_n}{x_0})
    \]
    Quindi $A$ \`e semplicemente connesso.
    $A\cap B$ \`e semplicemente connesso perch\'e \`e isomorfo a tutto $\mathbb{C}^n$ tolta l'origine.

    Sia $r\subseteq\mathbb{C}^{n+1}$ la retta generata da $(1,0,\dots, 0)$.
    Sia $h\colon\mathbb{C}^{n+1}\setminus \times [0,1] r\longrightarrow \mathbb{C}^{n+1}$ con $h((x_0, \dots, x_n), t)=(tx_0, \dots, x_n)$.
    Si ha che $h$ passa al quoziente che definisce lo spazio proiettivo, e ho ottenuto che $B$ \`e omotopo a $\mathbb{P}^{n-1}$. Quindi per ipotesi induttiva si conclude.
\end{proof}

\begin{defn}
    $\Sigma_g$ \`e la superficie con $g$ buchi. Per esempio $\Sigma_0$ \`e la sfera e $\Sigma_1$ \`e il toro.
\end{defn}

\begin{ex}
    Vogliamo calcolare il gruppo fondamentale di $\Sigma_g$. Sia $P$ un $4g$-agono, con i lati indicati con $a_1$, $b_1$, $a_1$, $b_1$, $a_2$, \dots
    Allora $\Sigma_g$ \`e $P$ quozientato l'identificazione dei lati con lo stesso nome, in modo simile alla costruzione del toro. Non abbiamo formalizzato questo passaggio, basta soltanto essere in grado di vederlo.

    Sia $p$ un punto interno a $P$. Chiamo $A = \pi(P\setminus\{p\})$ e $B=\pi(P\setminus\partial P)$, dove $\pi$ \`e la proiezione al quoziente.  Si ha innanzi tutto che $B$ ha gruppo fondamentale banale.

    Inoltre $A$ si retrae (radialmente) su $\pi(\partial P)$. Ora un altro passaggio di cui si richiede solo l'intuito geometrico: $\pi(\partial P)$ \`e il wedge di $2g$ cerchi. Dunque $\pi_1(A)$ \`e il prodotto libero di $2g$ copie di $\mathbb{Z}$. Infine $A\cap B$ \`e omotopo a una circonferenza. Un rappresentante $\alpha$ del generatore del gruppo fondamentale \`e la proiezione di un laccio che fa un giro intorno a $p$.
    Allora dentro $\pi_1(A)$ l'immagine di $\alpha$ \`e $\prod_{i=1}^g a_ib_ia_i^{-1}b_i^{-1}$.
    Allora
    \[
        \pi_1(\Sigma_g)=\langle a_1, \dots, b_g\ |\ \prod_{i=1}^ga_ib_ia_i^{-1}b_i^{-1}
    \]
\end{ex}
\begin{prop}
    Detto $\Gamma_g = \pi_1(\Sigma_g)$, si ha che $\Gamma_g /[\Gamma_g, \Gamma_g] \cong \mathbb{Z}^{2g}$.
\end{prop}
\begin{proof}
    Costruisco una funzione $\phi\colon \Gamma_g\longrightarrow \mathbb{Z}^{2g}$ che manda $a_1$ e $b_i$ distinti in generatori distinti. Si verifica che $\psi$ \`e un omomorfismo surgettivo. Si mostra con rapido conto che $[\Gamma_g, \Gamma_g]\subseteq Ker\ \psi$. Infine si costruisce a mano l'inversa di $\bar{\psi}\colon \Gamma_g /[\Gamma_g, \Gamma_g]\rightarrow \mathbb{Z}^{2g}$.
\end{proof}

\begin{cor}
    Si ha che $\Sigma_m \sim \Sigma_n$ se e solo se $\Sigma_m \cong \Sigma_n$ se e solo se $n=m$.
\end{cor}


\end{document}
