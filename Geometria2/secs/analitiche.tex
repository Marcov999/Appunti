\begin{defn}
  (Funzione analitica) Sia $U \subset \mathbb{C}$ un aperto. Una funzione $f:U \longrightarrow \mathbb{C}$ si dice \textsc{analitica in $z_0 \in U$} se
  \begin{nlist}
    \item esiste una serie di potenze $\displaystyle \sum_{n \ge 0} a_n(z-z_0)^n$ che converge assolutamente per $|z-z_0|<r$ per un qualche $r>0$ e
    \item $\displaystyle f(z)=\sum_{n \ge 0} a_n(z-z_0)^n$ per $|z-z_0|<r$.
  \end{nlist}
  Diremo che $f$ è \textsc{analitica in $U$} se $f$ è analitica in $z_0$ per ogni $z_0 \in U$.
\end{defn}

\begin{oss}
  $\displaystyle \sum_{n \ge 0} a_n(z-z_0)^n=a_0+\sum_{n \ge 1} a_n(z-z_0)^n$.
\end{oss}

\begin{ex}
  Sono funzioni analitiche:
  \begin{nlist}
    \item i polinomi;
    \item $e^z, \cos{z}, \sin{z}, \dots$
  \end{nlist}
\end{ex}

\begin{prop}
  Sia $U \subset \mathbb{C}$ un aperto e siano $f, g:U \longrightarrow \mathbb{C}$ funzioni analitiche in $U$. Allora
  \begin{nlist}
    \item $f+g$ è analitca in $U$;
    \item $fg$ è analitica in $U$;
    \item $\dfrac{f}{g}$ è definita e analitica in qualunque aperto contenuto in $\{z \in U \mid g(z) \not=0\}$.
  \end{nlist}
\end{prop}

\begin{prop}
  Siano $U$ e $V$ aperti di $\mathbb{C}$. Siano $f:U \longrightarrow V$ e $g:V \longrightarrow \mathbb{C}$ funzioni analitiche. Allora $g \circ f$ è analitica in $U$.
\end{prop}

\begin{prop}
  Sia $U \subset \mathbb{C}$ aperto e $f:U \longrightarrow \mathbb{C}$ una funzione. Se $f$ è analitica in $U$, allora $f$ è continua in $U$.
\end{prop}

\begin{proof}
  Sia $z_0 \in U$. Assumiamo che $\displaystyle f(z)=\sum_{n \ge 0}a_n(z-z_0)^n$ per $|z-z_0|<r$ per qualche $r>0$. Senza perdita di generalità $z_0=0$ e $f(z_0)=f(0)=0$ $\implies$ $a_0=f(z_0)=f(0)=0$.
  $\displaystyle f(z)=\sum_{n \ge 0} a_n(z-z_0)^n=\sum_{n \ge 0} a_nz^n=\sum_{n \ge 1} a_nz^n=z\sum_{n \ge 1} a_nz^{n-1}$. Se $0<\rho<r$ e $|z|<\rho$, $\displaystyle |f(z)| \le \sum_{n \ge 1} |a_n||z^n| \le |z| \sum_{n \ge 1} |a_n||z|^{n-1} \le |z| \sum_{n \ge 1} |a_n| \rho^{n-1}$.
  Adesso notiamo che $\displaystyle \sum_{n \ge 1} |a_n| \rho^{n-1}$ non dipende da $|z|$, quindi $|f(z)| \longrightarrow 0$ per $|z| \longrightarrow 0$.
\end{proof}

\begin{prop}
  Sia $z_0 \in \mathbb{C}$. Sia $\displaystyle \sum_{n \ge 0} a_n(z-z_0)^n$ una serie di potenze assolutamente convergente nel disco aperto $D_r(z_0)=\{z \mid |z-z_0|<r\}$ per qualche $r>0$. Allora la funzione
  \begin{align*}
    f:D_r(z_0) &\longrightarrow \mathbb{C}\\
    z &\longmapsto \sum_{n \ge 0} a_n(z-z_0)^n
  \end{align*}
  è analitica in $D_r(z_0)$.
\end{prop}

\begin{proof}
  Senza perdita di generalità $z_0=0$ $\implies$ $\displaystyle f(z)=\sum_{n \ge 0} a_nz^n$. Sia $a \in D_r(0)$ e sia $s \in \mathbb{R}, s>0$ t.c. $|a|+s<r$. Notiamo che $\displaystyle z^n=((z-a)+a)^n=\displaystyle \sum_{0 \le k \le n} \binom{n}{k}(z-a)^ka^{n-k}$.
  Quindi $\displaystyle f(z)=\sum_{n \ge 0} a_nz^n=\sum_{n \ge 0} a_n\left(\sum_{0 \le k \le n} \binom{n}{k}(z-a)^na^{n-k}\right)$. Se $|z-a|<s$, $|a|+|z-a|<r$, quindi la serie $\displaystyle \sum_{n \ge 0} |a_n|(|a|+|z-a|)^n$ converge.
  Scambiando l'ordine delle sommatorie, possiamo scrivere $\displaystyle f(z)=\sum_{k \ge 0} \left(\sum_{n \ge k} a_n\binom{n}{k}a^{n-k}\right)\cdot (z-a)^k=\sum_{n \ge 0}b_k(z-a)^k$ per $|z-a|<s$ se $\displaystyle \sum_{n \ge k} a_n\binom{n}{k}a^{n-k}$ converge per ogni $k$.
  Per mostrarlo, osserviamo che $\displaystyle \sum_{l \ge 0} c_lz^l$ assolutamente convergente implica, con una semplice induzione, $\displaystyle \sum_{l \ge s} l(l-1)\dots(-s+1)c_lz^{l-s}$ assolutamente convergente, con lo stesso raggio di convergenza, per ogni $s$.
  Adesso $\displaystyle \sum_{l \ge s} l(l-1)\dots(-s+1)c_lz^{l-s}=\sum_{l \ge s} \dfrac{l!}{(l-s)!}c_lz^{l-s}=s! \sum_{l \ge s} \binom{l}{s}c_lz^{l-s}$, per cui $\displaystyle \sum_{l \ge s} \binom{l}{s}c_lz^{l-s}$ è assolutamente convergente con lo stesso raggio di convergenza di $\displaystyle \sum_{l \ge 0} c_lz^l$.
  Visto che $\displaystyle \sum_{n \ge 0} a_nz^n$ è assolutamente convergente per $|z|<r$, abbiamo che $\displaystyle \sum_{n \ge k} a_n\binom{n}{k}a^{n-k}$ converge dato che $|a|<r$.
\end{proof}

\begin{thm}
  Sia $U \subset \mathbb{C}$ aperto. Se $f:U \longrightarrow \mathbb{C}$ è analitica in $U$, allora $f$ è olomorfa in $U$ e la sua derivata $f'$ è una funzione analitica in $U$.
\end{thm}

\begin{proof}
  Sia $z_0 \in U$, senza perdita di generalità $z_0=0$. Per ipotesi esiste una serie $\displaystyle \sum_{n \ge 0} a_n(z-z_0)^n=\sum_{n \ge 0} a_nz^n$ che converge assolutamente per $|z-z_0|=|z|<r$ per qualche $r>0$. Sia $\delta>0$ t.c. $|z|+\delta<r$.
  Per ogni $h \in \mathbb{C}, h\not=0, |h|<\delta$, abbiamo $\displaystyle f(z+h)=\sum_{n \ge 0} a_n(z+h)^n=\sum_{n \ge 0} a_n(z^n+nhz^{n-1}+h^2P_n(z, h))$ dove $P_n(z, h)$ è un polinomio in $z$ e $h$ a coefficienti interi positivi.
  $\displaystyle P_n(z, h)=\sum_{k=2}^n \binom{n}{k}h^{k-2}z^{n-k} \implies |P_n(z, h)| \le \sum_{k=2}^n \binom{n}{k} \delta^{k-2}|z|^{n-k}=P_n(|z|, \delta)$, che non dipende da $h$.
  Allora $\displaystyle f(z+h)=f(z)+\sum_{n \ge 1} a_nnhz^{n-1}+h^2\sum_{n \ge 2}a_nP_n(z, h) \implies f(z+h)-f(z)-\sum_{n \ge 1} a_nnhz^{n-1}=h^2\sum_{n \ge 2} a_nP_n(z, h)$. Per ipotesi la somma al membro di sinistra è assolutamente convergente, dunque lo è anche quella al membro di destra.
  Dividiamo per $h$: $\displaystyle \frac{f(z+h)-f(z)}{h}-\sum_{n \ge 1}a_nnz^{n-1}=h\sum_{n \ge 2} a_nP_n(z, h)$. $\displaystyle \sum_{n \ge 1}a_nnz^{n-1}$ è la cosiddetta serie derivata di $\displaystyle \sum_{n \ge 0}a_nz^n$.
  Per $|h|<\delta$, $\displaystyle \left|\sum_{n \ge 2} a_nP_n(z, h)\right| \le \sum_{n \ge 2} |a_n|P_n(|z|, \delta)$ che non dipende da $h$ e converge.
  Dunque per $h \longrightarrow 0$ abbiamo che $\displaystyle h\sum_{n \ge 2} a_nP_n(z, h) \longrightarrow 0$ e $\displaystyle \sum_{n \ge 1} a_nnz^{n-1} \longrightarrow 0$, quindi $\displaystyle \frac{f(z+h)-f(z)}{h} \longrightarrow f'(z)=\sum_{n \ge 1} a_nnz^{n-1}$ $\implies$ $f'$ è analitica.
\end{proof}

Concludiamo il paragrafo con la dimostrazione della proposizione \ref{serie_log}.

\begin{proof}
  Sia $D=B(0, 1)$ e siano $f:D \longrightarrow \mathbb{C}, f(z)=\log(1+z)$, $\displaystyle g:D \longrightarrow \mathbb{C}, g(z)=\sum_{n=1}^{+\infty} (-1)^{n+1}\frac{z^n}{n}$.
  Poiché il raggio di convergenza della serie data è $1$, $g$ è ben definita, analitica, dunque olomorfa e $\displaystyle g'(z)=\sum_{n=1}^{+\infty} (-1)^{n+1}n\frac{z^{n-1}}{n}=\sum_{n=1}^{+\infty} (-1)^{n+1}z^{n-1}=\sum_{n=0}^{+\infty} (-1)^nz^n=\sum_{n=0}^{+\infty} (-z)^n=\frac{1}{1-(-z)}=\frac{1}{1+z}$.
  Perciò, se $h=f-g$, $h'(z)=f'(z)-g'(z)=\dfrac{1}{1+z}-\dfrac{1}{1+z}=0$ per ogni $z \in D$, dunque $h' \equiv 0$ su $D$ e $h$ è costante su $D$ (una funzione con differenziale nullo ovunque è costante su un connesso). Dunque $h(z)=c$ per ogni $z \in D$.
  Per trovare $c$ osserviamo che $\displaystyle h(0)=f(0)-g(0)=\log{1}-\sum_{n=1}^{+\infty} 0=0$. Dunque $h \equiv 0$, cioè $f \equiv g$.
\end{proof}
