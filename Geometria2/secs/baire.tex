\begin{defn}
    Un sottoinsieme di uno spazio topologico \`e raro se l'apertura della chiusura \`e vuota. Un sottoinsieme si dice magro se \`e unione numerabile di rari. Uno spazio si dice di Baire se le aperture dei magri sono vuote.
\end{defn}
\begin{prop}
    Sono equivalenti:
    \begin{nlist}
        \item X \`e di Baire
        \item Unione numerabile di chiusi rari ha parte interna vuota
        \item Intersezione numerabile di aperti densi \`e densa
    \end{nlist}
\end{prop}
\begin{proof}
    Le ultime due sono equivalenti per passaggio al complementare. La seconda \`e la definizione dell'essere di Baire. Per fare vedere che la seconda implica la prima, sia $Y\subseteq X$ magro, $Y=\bigcup_{n\in\mathbb{N}}Y_n$ con gli $Y_n$ rari. Allora
    \[
    \open{Y}=\open{(\bigcup Y_n)}\subseteq\open{\bigcup \bar{Y_n}}=\emptyset
    \]
\end{proof}
\begin{thm}(Versione debole del teorema di Baire)
    Ogni $X$ metrico completo \`e di Baire.
\end{thm}
\begin{proof}
    La dimostrazione \`e uguale uguale a quella fatta da Manetti a pagina 115.
\end{proof}

\begin{defn}
    Un sottoinsieme di uno spazio si dice perfetto se \`e chiuso e privo di punti isolati.
\end{defn}
\begin{prop} Un sottoinsieme $Y\subseteq\mathbb{R}$ perfetto \`e pi\`u che numerabile
\end{prop}
\begin{proof}
    $Y$ \`e completo, quindi di Baire. Allora scrivendo per assurdo $Y=\bigcup_{n\in\mathbb{N}}\{y_n\}$ e $\open{Y}=\emptyset$, non posso avere $\open{\bar{\{y_n\}}}=\open{\{y_n\}}=\emptyset$
\end{proof}
