\begin{defn}
    Un sottoinsieme di uno spazio topologico \`e raro se l'apertura della chiusura \`e vuota. Un sottoinsieme si dice magro se \`e unione numerabile di rari. Uno spazio si dice di Baire se le aperture dei magri sono vuote.
\end{defn}
\begin{prop}
    Sono equivalenti:
    \begin{nlist}
        \item X \`e di Baire
        \item Unione numerabile di chiusi rari ha parte interna vuota
        \item Intersezione numerabile di aperti densi \`e densa
    \end{nlist}
\end{prop}
\begin{proof}
    Le ultime due sono equivalenti per passaggio al complementare. La seconda \`e la definizione dell'essere di Baire. Per fare vedere che la seconda implica la prima, sia $Y\subseteq X$ magro, $Y=\bigcup_{n\in\mathbb{N}}Y_n$ con gli $Y_n$ rari. Allora
    \[
    \open{Y}=\open{(\bigcup Y_n)}\subseteq\open{\bigcup \bar{Y_n}}=\emptyset
    \]
\end{proof}
\begin{thm}(Versione debole del teorema di Baire)
    Ogni $X$ metrico completo \`e di Baire.
\end{thm}
\begin{proof}
    Siano $A_n, n \in \mathbb{N}$ aperti densi. Dati $x_0 \in X$ e $r_0>0$, è sufficiente mostrare che $\displaystyle \left(\bigcap_{n \in \mathbb{N}} A_n \right)\cap B(x_0,r_0)\not=\emptyset$. Per farlo, costruiamo la seguente successione di palle: \\
    $B_0=B(x_0,r_0)$; \\
    $B_{n+1}$: dato che $A_{n+1}$ è denso, $A_{n+1}\cap B_n \not=\emptyset$ ed è un aperto, allora possiamo trovare una palla $B_{n+1}=B(x_{n+1},r_{n+1})$ t.c. $\overline{B_{n+1}} \subseteq B_n \cap A_{n+1}$ e $r_{n+1} \le 1/(n+1)$. \\
    In questo modo, i centri delle palle definiscono una successione $x_n$ che è di Cauchy, quindi da $X$ completo abbiamo $x_n \longrightarrow x$. Abbiamo anche che la successione sta definitivamente in ogni $B_n$, dunque $x \in \overline{B_n} \implies x \in A_n$ per ogni $n$, ma $x \in \overline{B_1} \subset\ B_0=B(x_0,r_0)$, da cui la tesi.
\end{proof}

\begin{defn}
    Un sottoinsieme di uno spazio si dice perfetto se \`e chiuso e privo di punti isolati.
\end{defn}
\begin{prop} Un sottoinsieme $Y\subseteq\mathbb{R}$ perfetto \`e pi\`u che numerabile
\end{prop}
\begin{proof}
    $Y$ \`e completo, quindi di Baire. Allora scrivendo per assurdo $Y=\bigcup_{n\in\mathbb{N}}\{y_n\}$ e $\open{Y}=\emptyset$, non posso avere $\open{\bar{\{y_n\}}}=\open{\{y_n\}}=\emptyset$
\end{proof}
