\begin{defn}
    Uno \textsc{spazio topologico} è una coppia $(X, \tau),\; {\tau \subseteq
    \mathcal{P}(X)}$, t.c.
    \begin{nlist}
        \item $\emptyset,\ X \in \tau$;
        \item $A_1,\ A_2 \in \tau \Rightarrow A_1 \cap A_2 \in \tau$;
        \item se $I$ è un insieme e $A_i \in \tau \, \forall i \in I$, allora
        $\displaystyle \bigcup_{i \in I} A_i \in \tau$.
    \end{nlist}
    Allora $\tau$ si dice \textsc{topologia} di $X$ e gli elementi della
    topologia sono detti \textsc{aperti} di $\tau$.
\end{defn}

\begin{prop}
    Se $(X, d)$ è uno spazio metrico, gli aperti rispetto a $d$ definiscono una
    topologia.
\end{prop}

\begin{defn}
    Uno spazio topologico $(X, \tau)$ è detto \textsc{metrizzabile} se $\tau$ è
    indotta da una distanza su $X$.
\end{defn}

\begin{defn}
    Sia $(X, \tau)$ uno spazio topologico, $C \subseteq X$ è \textsc{chiuso} se
    $X \setminus C$ è aperto.
\end{defn}

\begin{oss}
\begin{nlist}
\item possono esistere insiemi né aperti né chiusi;
\item $\emptyset$ e $X$ sono chiusi;
\item unione finita di chiusi è chiusa;
\item intersezione arbitraria di chiusi è chiusa.
\end{nlist}
\end{oss}

\begin{ex}
    Ecco alcuni esempi di spazi topologici:
    \begin{nlist}
        \item tutte le topologie indotte da una metrica;
        \item la topologia discreta, cioè $\tau=\mathcal{P}(X)$, indotta dalla
        distanza discreta;
        \item la topologia indiscreta, cioè $\tau=\{ \emptyset, X \}$;
        \item la topologia cofinita, dove gli aperti sono l'insieme vuoto più
        tutti e soli gli insiemi il cui complementare è un insieme finito;
        \item la topologia della semicontinuità (inferiore) su $\mathbb{R}$, $\tau=\{\emptyset, \mathbb{R}\} \cup \{\mathcal{O}_y | y \in \mathbb{R}\}$ dove $\mathcal{O}_y=\{x \in \mathbb{R} | x>y\}$.
    \end{nlist}
\end{ex}

\begin{defn}
	Dato uno spazio topologico $X$ e un insieme $B \subseteq X$, si chiama
	\textit{parte interna} di $B$, e la si indica con $B^{\circ}$, il più grande
	aperto contenuto in $B$. Analogamente, la \textit{chiusura} di $B$, indicata
	con $\overline{B}$, è il più piccolo chiuso che contiene $B$. Definiamo
	infine la \textit{frontiera} (o \textit{bordo}) di un insieme come l'insieme
	$\partial B= \overline{B} \setminus B^{\circ}$.
\end{defn}

Notiamo che parte interna e chiusura sono ben definite: per la prima basta
prendere l'unione di tutti gli aperti contenuti in $B$ (alla peggio c'è solo il
vuoto, che è contenuto in ogni insieme), per la seconda si prende l'intersezione
di tutti i chiusi che lo contengono (alla peggio c'è solo $X$). Per la stabilità
degli aperti per unione arbitraria e dei chiusi per intersezione arbitraria, gli
insiemi così ottenuti sono ancora un aperto e un chiuso e sono rispettivamente
il più grande aperto contenuto e il più piccolo chiuso che contiene per come
sono stati costruiti. Infine, è banale mostrare che la parte interna di un
aperto è l'aperto stesso e la chiusura di un chiuso è il chiuso stesso.

\begin{ftt}
	$X= B^{\circ} \sqcup\ \partial B\ \sqcup (X
	\setminus B)^{\circ}$ dove con $\sqcup$ si indica l'unione disgiunta.
\end{ftt}

\begin{proof}
	Per definizione $B^{\circ} \cup\ \partial B=\overline{B}$ e i due insiemi
	sono disgiunti. Inoltre, essendo $\overline B$ il più piccolo chiuso che
	contiene $B$, il suo complementare dev'essere il più grande aperto disgiunto
	da $B$, cioè il più grande aperto contenuto in $X \setminus B$, che è la
	parte interna di quest'ultimo. La tesi segue facilmente.
\end{proof}

\begin{exc}
  Sono lasciati come semplice esercizio alcuni fatti su parte interna e chiusura:
  \begin{nlist}
    \item $\overline{A \cup B}=\overline{A} \cup \overline{B}$, ma $\displaystyle \bigcup_{q \in \mathbb{Q}} \overline{\{q\}}=\mathbb{Q}, \overline{\bigcup_{q \in \mathbb{Q}}\{q\}}=\mathbb{R}$, quindi l'uguaglianza non vale in generale per unioni infinite;
    \item $\overline{A \cap B} \subseteq \overline{A} \cap \overline{B}$, ma prendendo $A=\mathbb{Q}, B=\mathbb{R} \setminus \mathbb{Q}$ otteniamo che può non valere l'uguaglianza;
    \item gli analoghi per la parte interna (attenzione: unione e intersezione si scambiano);
    \item $\left(\overline{(\overline{A})^{\circ}}\right)^{\circ}=(\overline{A})^{\circ}$ e $(\overline{\overline{A^{\circ}})^{\circ}}=\overline{A^{\circ}}$;
    \item trovare un esempio di $X$ spazio topologico e $A \subseteq X$ t.c. \\ $A, \overline{A}, (\overline{A})^{\circ}, \overline{(\overline{A})^{\circ}}, A^{\circ}, \overline{A^{\circ}}, (\overline{A^{\circ}})^{\circ}$ sono tutti diversi.
  \end{nlist}
\end{exc}

Vediamo adesso la caratterizzazione della chiusura in spazi metrici. Premettiamo una definizione.

\begin{defn}
  Siano $(X, d)$ spazio metrico e $E \subseteq X$. Diremo che $x_0 \in X$ è \textit{aderente} a $E$ se per ogni $r>0$ vale che $E \cap B(x_0, r) \not= \emptyset$.
\end{defn}

Vale allora la seguente affermazione.

\begin{prop}
  Dato $E$ in uno spazio metrico, $\overline{E}$ è l'insieme dei punti aderenti a $E$.
\end{prop}

\begin{proof}
  Sicuramente l'insieme dei punti aderenti contiene $E$, inoltre è un semplice esercizio verificare che è un insieme chiuso. Dobbiamo adesso dimostrare che è il più piccolo chiuso che contiene $E$. Sia dunque $C$ un chiuso che contiene $E$, è ancora una volta un semplice esercizio verificare che il suo complementare è disgiunto dall'insieme dei punti aderenti, che è dunque contenuto in $C$ e per minimalità della chiusura deve essere proprio $\overline{E}$.
\end{proof}
