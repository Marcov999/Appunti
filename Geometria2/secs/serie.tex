Vediamo ora, senza dimostrazione, alcuni risultati sulle serie di numeri complessi che ci torneranno utili.

\begin{defn}
  Sia $(c_n)_{n \in \mathbb{N}}$ una successione di numeri complessi. Diciamo che $\displaystyle \sum_{n \ge 0} c_n$ è \textit{assolutamente convergente} se la serie $\displaystyle \sum_{n \ge 0} |c_n|$ è convergente.
\end{defn}

\begin{exc}
  $\displaystyle \sum_{n \ge 0} \frac{i^n}{n!}$ è convergente?
\end{exc}

\begin{prop} \label{sum&prod}
  Siano $\displaystyle \sum_{n \ge 0} a_n$ e $\displaystyle \sum_{n \ge 0} b_n$ due serie assolutamente convergenti.
  \begin{nlist}
    \item $\displaystyle \sum_{n \ge 0} a_n+b_n$ è assolutamente convergente ed è uguale a $\displaystyle \sum_{n \ge 0} a_n + \sum_{n \ge 0} b_n$ (la serie delle somme è la somma delle serie);
    \item se $\displaystyle c_n:=\sum_{p=0}^n a_pb_{n-p}$, allora $\displaystyle \sum_{n \ge 0} c_n$ è assolutamente convergente e la sua somma è uguale al prodotto delle somme delle due serie date.
  \end{nlist}
\end{prop}

\begin{defn}
  Sia $\displaystyle \sum_{n \ge 0} a_nz^n$ una serie di potenze. Chiamiamo \textit{raggio di convergenza} la quantità $\displaystyle \rho:=\sup\{r \in \mathbb{R}, r>0 \mid \sum_{n \ge 0} |a_n|r^n \text{ è convergente}\}$.
\end{defn}

\begin{oss}
  $\rho$ può essere finito e in questo caso $\rho \ge 0$, oppure $\rho=+\infty$.
\end{oss}

\begin{defn}
  Chiamiamo \textit{disco di convergenza} l'insieme $\{z \in \mathbb{C} \mid |z|<\rho\}$.
\end{defn}

\begin{oss}
  Il disco di convergenza è aperto e $\rho=0$ $\implies$ disco$=\emptyset$.
\end{oss}

\begin{prop}
  Data una serie di potenze $\displaystyle \sum_{n \ge 0} a_nz^n$, esiste $0 \le \rho \le +\infty$ t.c.:
  \begin{enumerate}
    \item caso $\rho=0$ la serie converge solo per $z=0$;
    \item caso $\rho=+\infty$: la serie converge assolutamente per ogni $z \in \mathbb{C}$;
    \item caso $0<\rho<+\infty$: se $|z|<\rho$ la serie converge assolutamente, se $|z|>\rho$ la serie non converge.
  \end{enumerate}
  Inoltre si ha la formula di Hadamard:
  $$\frac{1}{\rho}=\limsup_{n \longrightarrow +\infty} |a_n|^{1/n}$$
  con la convenzione che $\rho=+\infty$ se il limite è $0$ e $\rho=0$ se il limite è $+\infty$.
\end{prop}

\begin{exc}
  Calcolare il raggio di convergenza di \\
  $\displaystyle \sum_{n \ge 0} n!z^n, \sum_{n \ge 0} \frac{z^n}{n!}, \sum_{n \ge 0} (-1)^n \frac{z^{2n}}{(2n)!}, \sum_{n \ge 0} (-1)^n \frac{z^{2n+1}}{(2n+1)!}$.
\end{exc}

\begin{ex}
  Proviamo che $\displaystyle \sum_{n \ge 0} z^n=\frac{1}{1-z}$ per $|z|<1$. La somma della serie è $\displaystyle \lim_{n \longrightarrow +\infty}S_n, S_n=z^0+z^1+\dots+z^n=\frac{1-z^{n+1}}{1-z}$. Lo studio del limite è lasciato per esercizio.
\end{ex}

\begin{ftt}
  \begin{nlist}
    \item Siano $\displaystyle \sum_{n \ge 0} a_nz^n$ e $\displaystyle \sum_{n \ge 0} b_nz^n$ due serie di potenze con raggio di convergenza $\ge R$ per qualche $R$.
    Allora, definendo $\displaystyle S(z):=\sum_{n \ge 0} a_nz^n+\sum_{n \ge 0}b_nz^n, P(z):=\left(\sum_{n \ge 0}a_nz^n\right)\left(\sum_{n \ge 0}b_nz^n\right)$, abbiamo che $S(z)$ e $P(z)$ hanno raggio di convergenza $\ge R$.
    Inoltre, per ogni $r \in \mathbb{C}$ con $|r|<R$, $\displaystyle S(r)=\sum_{n \ge 0}a_nr^n+\sum_{n \ge 0} b_nr^n$ e $\displaystyle P(r)=\left(\sum_{n \ge 0}a_nr^n\right)\left(\sum_{n \ge 0}b_nr^n\right)$.
    \item Sia $\displaystyle f(z)=\sum_{n \ge 0} a_nz^n$. Chiamiamo \textit{serie derivata} la serie di potenze $\displaystyle \sum_{n \ge 1} na_nz^{n-1}$ e la denotiamo con $f'(z)$. Allora $f$ e $f'$ hanno lo stesso raggio di convergenza.
  \end{nlist}
\end{ftt}

Passiamo ora a definire alcune funzioni complesse tramite serie di potenze. Cominciamo da esponenziale, seno e coseno.

\begin{defn}
  Fissato $z \in \mathbb{C}$, chiamiamo \textit{esponenziale del numero complesso $z$} la quantità $\displaystyle e^z:=\sum_{n \ge 0} \frac{1}{n!}z^n$.
  Chiamiamo \textit{coseno di $z$} $\displaystyle \cos{z}:=\sum_{n \ge 0}(-1)^n\frac{z^{2n}}{(2n)!}$ e \textit{seno di $z$} $\displaystyle \sum_{n \ge 0}(-1)^n \frac{z^{2n+1}}{(2n+1)!}$.
\end{defn}

\begin{oss}
  Le definizioni sono ben poste perché le serie hanno raggio di convergenza infinito.
\end{oss}

\begin{exc}
  Provare che le definizioni coincidono con quelle già date.
\end{exc}

\begin{ex}
  Dati $z, z' \in \mathbb{C}$, $e^{z+z'}=e^ze^{z'}$. Siano $a_n=\dfrac{1}{n!}z^n, b_n=\dfrac{1}{n!}(z')^n$.
  Sia $\displaystyle c_n=\sum_{p=0}^n a_pb_{n-p}=\sum_{p=0}^n \left(\frac{1}{p!}z^p\right)\left(\frac{1}{(n-p)!}(z')^{n-p}\right)=\frac{1}{n!}\sum_{p=0}^n\binom{n}{p}z^p(z')^{n-p}=\frac{1}{n!}(z+z')^n$,
  allora per il punto (ii) della proposizione \ref{sum&prod} abbiamo proprio $e^{z+z'}=e^ze^{z'}$.
\end{ex}

\begin{oss}
  $z, z' \in \mathbb{C}$ t.c. $e^z=e^{z'}$. Questo equivale a $e^{z'-z}=1$ $\iff$ $|e^{z'-z}|=e^{\mathfrak{Re}(w)}=1, \cos{\mathfrak{Im}(w)}+i\sin{\mathfrak{Im}(w)}=1$ ($w=z'-z$) $\iff$ $\mathfrak{Re}(w)=0$ e $\mathfrak{Im}(w)=2\pi k$ per $k \in \mathbb{Z}$.
  Da questo segue che $e^z=e^{z'}$ $\iff$ $z'=z+i(2\pi k)$ per un certo $k \in \mathbb{Z}$.
\end{oss}

Adesso vogliamo definire il logaritmo.

\begin{defn}
  Sia $z \in \mathbb{C}, z \not=0$. Chiamiamo \textit{logaritmo del numero complesso $z$} la quantità $\log{z}:=\log{|z|}+i\arg{z}$.
\end{defn}

\begin{oss}
  $\log{z} \in \mathbb{C}$ è definito modulo $2\pi i \mathbb{Z}$ perché $\arg{z} \in \faktor{\mathbb{R}}{2\pi \mathbb{Z}}$.
\end{oss}

\begin{prop}
  Per ogni $z \in \mathbb{C}, z \not=0$, $e^{\log{z}}=z$.
\end{prop}

\begin{ftt}
  $\log{(zz')}=\log{z}+\log{z'} \mod{2\pi i \mathbb{Z}}$.
\end{ftt}

Domanda: come definiamo $z \longmapsto \log{z}$? Abbiamo bisogno della definizione di branca.

\begin{defn}
  Sia $D$ aperto connesso di $\mathbb{C}$ t.c. $0 \not\in D$ e $f:D \longrightarrow \mathbb{C}$ continua. Diciamo che $f$ è una \textit{branca} di $\log{z}$ se $e^{f(z)}=z$ (cioè se $f(z)$ è uno dei possibili valori di $\log{z}$).
\end{defn}

\begin{prop}
  Assumiamo che esista una branca $f(z)$ di $\log{z}$ in $D$. Allora ogni altra branca di $\log{z}$ in $D$ è della forma $f(z)+k(2\pi i)$ per qualche intero $k$. Viceversa, $f(z)+k(2 \pi i)$ è una branca di $\log{z}$ per ogni intero $k$, a partire dalla branca fissata $f(z)$.
\end{prop}

\begin{proof}
  Siano $f(z)$ e $g(z)$ due branche di $\log{z}$ in $D$ ($e^{f(z)}=z, e^{g(z)}=z$ per ogni $z \in D$). $h(z):=\dfrac{1}{2\pi i}(g(z)-f(z)):D \longrightarrow \mathbb{C}$. $h$ è continua e t.c. $\Ima{h} \subseteq \mathbb{Z}$.
  Visto che $D$ è connesso, $h$ è costantemente uguale a un certo intero $k$ $\implies$ $g(z)=f(z)+k(2\pi i)$. Il viceversa è ovvio.
\end{proof}

\begin{oss}
  \begin{nlist}
    \item Possiamo dare una definizione di branca anche per $\arg{z}$;
    \item ogni branca di $\arg{z}$ definisce una branca di $\log{z}$ (e viceversa) (se $f(z)$ è una branca di $\arg{z}$, $\log{|z|}+if(z)$ è una branca di $\log{z}$).
  \end{nlist}
\end{oss}

\begin{oss}
  Sia $D=\{z \in \mathbb{C} \mid \mathfrak{Re}(z)>0\}$. Per ogni $z \in D$ esiste un unico $-\pi/2<\phi<\pi/2$ t.c. $\phi$ è un argomento di $z$; poniamo allora $\text{Arg}(z):=\phi$.
\end{oss}

\begin{prop}
  La funzione
  \begin{align*}
    \text{Arg}(z):D &\longrightarrow \mathbb{C}\\
    z &\longmapsto \phi
  \end{align*}
  è continua.
\end{prop}

\begin{proof}
  Per ogni $z \in \mathbb{C}, z \not=0$, $\arg{z}=\arg{\left(\dfrac{z}{|z|}\right)}$.
  \begin{align*}
    f:D &\longrightarrow U=\{u \in \mathbb{C} \mid |u|=1, \mathfrak{Re}(u)>0\}\\
    z &\longmapsto \frac{z}{|z|}
  \end{align*}
  \begin{center}
    \begin{tikzcd}
      D \arrow[rr, "f"] \arrow[rd, "\text{Arg}"] & & U \arrow[ld, "\text{Arg}"]\\
      & \mathbb{C}
    \end{tikzcd}
  \end{center}
  $f$ è continua. Ci rimane da provare che
  \begin{align*}
    \text{Arg}:U &\longrightarrow \mathbb{C}\\
    u &\longmapsto \text{Arg}(u)=y
  \end{align*}
  è continua.
  Per costruzione, è l'inversa di
  \begin{align*}
    g:(-\pi/2, \pi/2) &\longrightarrow U\\
    y &\longmapsto e^{iy}
  \end{align*}
  $g$ è continua e biettiva. $g$ si estende a $\tilde{g}$ continua e biettiva da $[-\pi/2, \pi/2]$ (compatto) a $\{u \in \mathbb{C} \mid \mathfrak{Re}(u) \ge 0\}$ (Hausdorff) $\implies$ $\tilde{g}$ è un omeomorfismo, quindi la sua inversa è continua, quindi l'inversa di $g$ è continua, quindi $\text{Arg}$ è continua.
\end{proof}

\begin{defn}
  Chiamiamo branca \textit{principale} di $\log{z}$ la funzione continua $\log{z}+i\text{Arg}(z)$ per $z \in D=\{z \in \mathbb{C} \mid \mathfrak{Re}(z)>0\}$.
\end{defn}

\begin{lm} \label{der_inv}
  Se $f:D \longrightarrow D'$ è olomorfa e biettiva e $g:D \longrightarrow D'$ è l'inversa di $f$, se $f'(z_0)\not=0$, allora $g$ è olomorfa in $f(z_0)$ e vale l'usuale formula $g'(f(z_0))=\dfrac{1}{f'(z_0)}$.
\end{lm}

\begin{proof}
  Da $g \circ f=\id$ e dal teorema della funzione inversa (Analisi II), poiché $f'(z_0)=a+ib\not=0$, $\diff f_{z_0}=\begin{pmatrix}
  a & -b\\
  b & a
\end{pmatrix}$ è invertibile, per cui $g$ è differenziabile in $f(z_0)$ e $\diff g_{f(z_0)}=(\diff f_{z_0})^{-1}$. Infine, si verifica che se $\diff f_{z_0}=A=\begin{pmatrix}
a & -b\\
b & a
\end{pmatrix}$, $a^2+b^2\not=0$ (cioè $a+ib\not=0$), $\diff g_{f(z_0)}=A^{-1}=\begin{pmatrix}
  c & -d\\
  d & c
\end{pmatrix}$ con $c+id=\dfrac{1}{a+ib}$ su $\mathbb{C}$.
\end{proof}

\begin{prop}
  Se $f(z)$ è una branca di $\log{z}$ in un insieme aperto connesso $D$, la funzione $f$ è olomorfa e $\dfrac{\diff}{\diff z}f(z)=\dfrac{1}{z}$ (ovviamente $0 \not\in D$).
\end{prop}

\begin{proof}
  $f(\exp(z))=z+c$, $c$ costante. Per il lemma \ref{der_inv}, $f'(\exp(z))=\dfrac{1}{\exp(z)}$, cioè $f'(y)=\dfrac{1}{y}$.
  Detto bene, fissato $z_0 \in D, z_0\not=0$, $y_0=f(z_0) \in f(D)$ è t.c. $z_0=\exp(y_0)$. Per continuità di $\exp$, dato un intorno $V$ di $z_0$ in $D$ esiste un intorno $W$ di $y_0$ in $\mathbb{C}$ t.c. $\exp(W) \subseteq V$ e le uguaglianze sopra sono verificate (in particolare, $f \circ \exp=\id$, senza l'aggiunta di una costante).
\end{proof}

\begin{prop} \label{serie_log}
  La serie di potenze $\displaystyle \sum_{n \ge 1} (-1)^{n+1}\frac{z^n}{n}$ converge per $|z|<1$ ed è uguale alla branca principale di $\log{(z+1)}$.
\end{prop}

\begin{proof}
  Si veda più avanti, alla fine del paragrafo sulle funzioni analitiche.
\end{proof}

\begin{defn}
  Per ogni $z, \alpha \in \mathbb{C}, z \not=0$ poniamo $z^{\alpha}=e^{\alpha\log{z}}$.
\end{defn}
