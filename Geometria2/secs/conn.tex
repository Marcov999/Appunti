\begin{defn}
Uno spazio topologico $X$ si dice \textsc{sconnesso} se vale una delle seguenti condizioni (equivalenti):
\begin{nlist}
\item $X=A \sqcup B$ (cioè $X=A \cup B$ con $A \cap B = \emptyset$) con $A,B$ aperti non vuoti. Allora in particolare $A$ e $B$ sono anche chiusi, in quanto sono uno il complementare dell'altro.
\item $X=A \sqcup B$ con $A,B$ chiusi non vuoti
\item $\exists A \subseteq X, A \neq \emptyset, A \neq X, A$ è sia aperto che chiuso
\end{nlist}
\end{defn}

\begin{defn}
$X$ si dice \textsc{connesso} se non è sconnesso. In altre parole, se: \\ $\forall A \subseteq X, A \neq \emptyset, A$ aperto e chiuso, allora $A=X$.
\end{defn}

\begin{ex}
$\mathbb{R} \smallsetminus \{0\}$ è sconnesso in quanto unione degli aperti $(-\infty,0)$ e $(0,+\infty)$.
\end{ex}

\begin{thm}
$[0,1] \subset \mathbb{R}$ è connesso.
\end{thm}
\begin{proof}
Siano $A,B$ aperti non vuoti di [0,1], con $[0,1]=A \sqcup B$. Posso supporre $0 \in A$. Dunque, se $t_0=\inf B$ (esiste poiché $B$ è non vuoto e limitato inferiormente), allora $t_0 \ge \varepsilon >0$, perché altrimenti avrei $B \cap [0,\varepsilon) \neq \emptyset$, che contraddice le ipotesi. $B$ è chiuso ($A$ e $B$ sono entrambi aperti e chiusi), quindi $t_0 \in B$. Ma $B$ è anche aperto e $t_0 >0$, per cui $\exists \delta >0$ con $(t_0-\delta, t_0] \subseteq B$, e ciò contraddice il fatto che $t_0=\inf B$.
\end{proof}

\begin{defn}
$X$ si dice \textsc{connesso per archi} se $\forall x_0,x_1 \in X$, \\
$\exists \alpha:[0,1] \longrightarrow X$ continua tale che $\alpha (0)=x_0, \alpha (1)=x_1$.
\end{defn}

\begin{prop}
$X$ connesso per archi $\Longrightarrow$ $X$ connesso.
\end{prop}
\begin{proof}
Se $X$ fosse sconnesso, allora $X=A \sqcup B$ aperti non vuoti. Prendo allora $X_0 \in A, x_1 \in B$. Se $\alpha:[0,1] \longrightarrow X$ con $\alpha (0)=x_0, \alpha (1)=x_1$ fosse continua, avrei una partizione $[0,1]=\alpha ^{-1}(A) \sqcup \alpha ^{-1}(B)$ di [0,1] in aperti non vuoti, il che è assurdo.
\end{proof}

\begin{prop}
Sia $f:X\longrightarrow Y$ continua. Allora:
\begin{nlist}
\item Se $X$ è connesso, $f(X)$ è connesso
\item Se $X$ è connesso per archi, $f(X)$ è connesso per archi
\end{nlist}
\end{prop}
\begin{proof}
Se $f(X)=A \sqcup B$, allora $X=f^{-1}(A) \sqcup f^{-1}(B)$, quindi $X$ è sconnesso (abbiamo dimostrato la contronominale della (i)). \\
Dati $y_0,y_1 \in f(X), \exists x_0, x_1 \in X$ con $f(x_0)=y_0, f(x_1)=y_1$, esiste $\alpha:[0,1]\longrightarrow X$ con $\alpha (0)=x_0, \alpha (1)=x_1$. Allora $f \circ \alpha:[0,1] \longrightarrow Y$ è un arco continuo che connette $y_0$ a $y_1$.
\end{proof}

\begin{lm}
Siano $X$ spazio topologico, $Y \subseteq X$ connesso e $Z \subseteq X$ tale che $Y \subseteq Z \subseteq \overline{Y}$. Allora $Z$ è connesso.
\end{lm}
\begin{proof}
Osserviamo inanzitutto che $Y$ è denso in $Z$, in quanto la chiusura di $Y$ in $Z$ è $\overline{Y} \cap Z=Z$ poiché $Z \subseteq \overline{Y}$. Sia $A \neq \emptyset$ un aperto e chiuso di $Z$. Allora $A \cap Y$ è aperto e chiuso in $Y$ (per definizione di topologia di sottospazi), e $A \cap Y \neq \emptyset$ in quanto $Y$ è denso in $Z$ e $A$ è aperto in $Z$. Allora, per connessione di $Y$, si ha che $A \cap Y=Y$, cioè $Y \subseteq A$. Ma $Y$ è denso in $Z$, per cui anche $A$ lo è. Ma $A$ è anche chiuso in $Z$, per cui $A=Z$.
\end{proof}

\begin{lm}
Siano $\{Y_i\}_{i \in I}$ sottospazi connessi di $X$, e sia $x_0 \in X$ tale che $x_0 \in Y_i$, $\forall i \in I$ (cioè $x_0 \in \bigcup _{i \in I} Y_i \neq \emptyset$). Allora $\bigcup _{i \in I} Y_i$ è connesso.
\end{lm}
\begin{proof}
Sia $A \neq \emptyset$ aperto e chiuso di $Y=\bigcup Y_i$. A meno di sostituire $A$ con il suo complementare $Y \smallsetminus A$, posso supporre $x_0 \in A$. $\forall i \in I, A \cap Y_i$ è aperto e chiuso in $Y_i$, e non è vuoto perché contiene $x_0$. Per cui, $Y_i$ connesso implica che $A \cap Y_i=Y_i$, cioè $Y_i \subseteq A$. Ma allora $Y \subseteq A$, cioè $Y=A$, da cui la tesi.
\end{proof}

\begin{defn}
Sia $x_0 \in X$. Allora esiste il più grande sottospazio connesso di $X$ che contiene $x_0$. Lo indicheremo con $C(x_0)$ e lo chiameremo \textsc{componente connessa} di $x_0$.
\end{defn}
\begin{proof}(\emph{Esistenza di $C(x_0)$})\\
Pongo $C(x_0)=\bigcup \{Y \mid Y \subseteq X, Y \text{ connesso},x_0 \in Y\}$. Per il lemma appena visto, $C(x_0)$ è connesso, e contiene per costruzione qualsiasi connesso che contiene $x_0$, perciò è anche il più grande sottospazio connesso contenente $x_0$.
\end{proof}

\begin{prop}
Le componenti connesse realizzano una partizione dello spazio in chiusi.
\end{prop}
\begin{proof}
$\forall x_0,x_1 \in X, x_0 \in C(x_0)$ poiché $\{x_0\}$ è connesso. Se $C(x_0) \cap C(x_1) \neq \emptyset$, per il lemma precedente $C(x_0) \cup C(x_1)$ è connesso $\Longrightarrow C(x_0) \cup C(x_1) \subseteq C(x_0)$ e $C(x_0) \cup C(x_1) \subseteq C(x_1) \Longrightarrow C(x_0)=C(x_1)$.\\
Infine, $\overline{C(x_0)}$ è connesso per un lemma precedente e contiene $x_0$, dunque $\overline{C(x_0)} \subseteq C(x_0)$, per cui $C(x_0)=\overline{C(x_0)}$, ovvero $C(x_0)$ è chiuso.
\end{proof}

\begin{oss}
$A \subseteq X$ aperto, chiuso e connesso $\Longrightarrow A$ è una componente connessa. Infatti, se $B \supseteq A$ è connesso, $A$ è aperto e chiuso in $B$ e non vuoto, allora $A=B$, per cui $A$ è un connesso massimale, ovvero è una componente connessa.
\end{oss}

\begin{oss}
Le componenti connesse di $\mathbb{Q}$ sono i punti, cioè $C(x_0)=\{x_0\}$, $\forall x_0 \in \mathbb{Q}$. Si dice allora che $\mathbb{Q}$ è \textsc{totalmente sconnesso}. Infatti, supponiamo per assurdo che esista una componente connessa $C$ che contiene due punti $x_0,x_1 \in \mathbb{Q}$, con $x_0<x_1$, e sia $y \in \mathbb{R} \smallsetminus \mathbb{Q}$ con $x_0<y<x_1$. Allora $(C \cap (-\infty,y))\sqcup (C \cap (y,+\infty))$ è una partizione non banale di $C$ in aperti. Ma allora $C$ è sconnesso, il che contraddice l'ipotesi iniziale. \\
Dunque, in generale, le componenti connesse non sono aperte.
\end{oss}

Studiamo adesso i connessi di $\mathbb{R}$:\\
$A \subset \mathbb{R}$ è convesso se $\forall x,y \in A$ vale:
$$tx+(1-t)y \in A \qquad \forall t \in [0,1]$$

\begin{oss}
\begin{align*}
A \subset \mathbb{R} \text{ convesso } &\Longrightarrow A \text{ connesso per archi}\\
&\Longrightarrow A \text{ connesso}
\end{align*}
\end{oss}

\begin{prop}
Se $I \subset \mathbb{R}$, sono equivalenti:
\begin{nlist}
\item $I$ è convesso ($I$ è un intervallo)
\item $I$ è connesso per archi
\item $I$ è connesso
\end{nlist}
\end{prop}
\begin{proof}
(i) $\Longrightarrow$ (ii) $\Longrightarrow$ (iii) lo sappiamo già per l'osservazione precedente.\\
Vediamo allora (iii) $\Longrightarrow$ (i). Supponiamo $I$ non convesso. Allora $\exists a<b<c$ tali che $a,c \in I, b \notin I$. Allora:
$$I=((-\infty,b) \cap I) \cup ((b,+\infty) \cap I)$$
che è assurdo per ipotesi.
\end{proof}

\begin{ex}
(0,1) non è omeomorfo a [0,1). Sia infatti $f:[0,1) \longrightarrow (0,1)$ omeomorfismo. Allora:\\
$[0,1) \smallsetminus \{0\}$ è connesso\\
$(0,1) \smallsetminus \{f(0)\}$ non è connesso.
\end{ex}

\begin{ex}(\emph{Connesso $\not\Rightarrow$ connesso per archi})\\
Sia $Y=\{(x, \sin \frac{1}{x}) \mid x>0\} \subset \mathbb{R}^2$. Poiché $Y=F(0,+\infty)$ con:
\begin{align*}
F:(0,+\infty) &\longrightarrow \mathbb{R}^2 \\
x &\longmapsto \left(x,\sin \frac{1}{x}\right)
\end{align*}
allora $Y$ è connesso per archi. Sia:
$$X=\overline{Y}=Y \cup \{(0,t) \mid |x| \le 1\}$$
che è connesso in quanto chiusura di un connesso. Vediamo allora che non è connesso per archi. Supponiamo che esista il cammino $\alpha$ definito come:
$$\alpha:[0,1] \longrightarrow X \qquad \text{con} \qquad \alpha(0)=(0,0) \text{  e  } \alpha(1) \in Y$$
Consideriamo allora:
\begin{align*}
\alpha(t)&=(x(t),y(t))\\
x&:[0,1] \longrightarrow [0,+\infty]\\
y&:[0,1] \longrightarrow [-1,1]
\end{align*}
Sia $\Omega=\{t \in [0,1] \mid x(t)=0\}$ chiuso, e sia $t_0 =\max \Omega, t_0<1$ per ipotesi. Supponiamo $y(t_0) \ge 0$ (l'altro caso è analogo). Per continuità:
$$\exists \delta >0 : y(t) \ge -\dfrac{1}{2} \qquad \forall t \in [t_0,t_0+\delta]$$
Vediamo che $x([t_0,t_0+\delta]) \subset [0,+\infty)$ è un connesso contenente $0=x(t_0)$ Allora $\exists \varepsilon >0$ tale che $[0,\varepsilon]\subset x([t_0,t_0+\delta])$. Cerchiamo allora $\lambda \in [0,\varepsilon]$ con $\sin \dfrac{1}{\lambda}=-1$:
$$\lambda=\dfrac{1}{-\frac{\pi}{2}+2k\pi} \quad \text{ con } k\gg 0$$
Allora $\sin \dfrac{1}{\lambda}=-1$, vale a dire che se $x(t)=\lambda$, allora $y(t)=\sin \dfrac{1}{\lambda}=-1$.
\end{ex}

Torniamo adesso ad un discorso generale sulle connessioni.

\begin{defn}
Dato $x \in X$, la \textsc{componente connessa per archi} di $x$ è il più grande connesso per archi che contiene $x$.\\
\underline{Equivalentemente}: Dato $x \in X$, la sua componente connessa per archi $C_a(x)$ è:
$$C_a(x)=\{y \in X \mid \exists \alpha:[0,1] \longrightarrow X \text{ con } \alpha(0)=x, \alpha(1)=y\}$$
\end{defn}

Definiamo una relazione $\sim$ su $X$:
$$x \sim y \text{ se } \exists \alpha:[0,1] \longrightarrow X \text{ con } \alpha(0)=x, \alpha(1)=y$$
Vediamo che $\sim$ è una relazione di equivalenza, dunque $C_a(x)=[x]$:
\begin{nlist}
\item $\sim$ è riflessiva. Sia:
\begin{align*}
\alpha:[0,1] &\longrightarrow X \\
t &\longmapsto x
\end{align*}
Allora $x \sim x$.
\item $\sim$ è simmetrica. Sia $\alpha:[0,1] \longrightarrow X$ continua con $\alpha(0)=x,\alpha(1)=y$. Definiamo allora $\beta:[0,1]\longrightarrow X$ tale che $\beta (t)=\alpha (1-t)$, e osserviamo che è continua in quanto composizione di funzioni continue. Allora, poiché $\beta (0)=y, \beta(1)=x$, si ha che $y \sim x$.
\item $\sim$ è transitiva. Siano $\alpha,\beta :[0,1]\longrightarrow X$ continue con $\alpha(0)=x, \alpha(1)=\beta(0)=y, \beta(1)=z$. Definiamo allora la \textsc{giunzione} di $\alpha$ e $\beta$ nel seguente modo:
$$
(\alpha * \beta)(t)=\begin{cases} \alpha (2t) & \mbox{se }t \in [0, 1/2] \\ \beta (2t-1) & \mbox{se }t \in [1/2,1]
\end{cases}
$$
Allora abbiamo che:
\begin{align*}
&(\alpha * \beta)(0) = \alpha (0) =x \\
&(\alpha * \beta)\left(\dfrac{1}{2}\right) =\alpha (1) = \beta (0)=y\\
&(\alpha * \beta)(1) = \beta (1) =z\\
\end{align*}
Allora $[0,1]=[0,1/2] \cup [1/2,1]$ è un ricoprimento chiuso localmente finito, di conseguenza è fondamentale, e quindi $\alpha * \beta$ è continua in quanto le sue restrizioni sono continue.
\end{nlist}

\begin{ex}
$X=X_1 \cup X_2$ dove:
$$ X_1=\{(0,t) \mid |t| \le 1\} \qquad X_2=\left\{ \left(x,\sin \dfrac{1}{x}\right) \mid x>0 \right\}$$
$X$ ha due componenti connesse per archi: $X_1$ e $X_2$. Infatti:
\begin{align*}
X_2&=X \cap \{x>0\} \qquad \text{aperta, non chiusa}\\
X_1&=X \cap \{x=0\} \qquad \text{chiusa, non aperta}
\end{align*}
Dunque in generale le componenti connesse per archi non sono né aperte né chiuse.
\end{ex}

\begin{prop}
Supponiamo che ogni punto abbia un intorno connesso. Allora le componenti connesse sono aperte.
\end{prop}
\begin{proof}
Sia $x \in X$, con $C(x)$ la sua componente connessa. Sia $y \in C(x)$, e sia $U \in I(y)$ intorno connesso. Allora $C(x) \cup U$ è unione di due connessi che si intersecano in $y$, e quindi $U \subset C(x)$.
\end{proof}

\begin{prop}
Supponiamo che ogni punto abbia un intorno connesso per archi. Allora le componenti connesse per archi sono aperte, e coincidono con le componenti connesse.
\end{prop}
\begin{proof}
Sia $x \in X$, con $C_a(x)$ la sua componente connessa per archi. Sia $y \in C_a(x)$, e sia $U \in I(y)$ intorno componente connessa per archi. Allora $C_a(x) \cup U$ è connesso per archi, e quindi $U \subset C_a(x)$. Vediamo adesso che $C_a(x)=C(x)$. Chiaramente, $C_a(x) \subset C(x)$ aperto, infatti:
$$C(x)=\bigsqcup _{y \in C(x)} C_a(y)$$
Allora $C(x) \smallsetminus C_a(x)$ è aperto e $C(x)$ è connesso, quindi $C_a(x)=C(x)$.
\end{proof}

\begin{defn}
$X$ è detto \textsc{localmente connesso} se ogni punto ammette un sistema fondamentale di intorni connessi.
\end{defn}

\begin{defn}
$X$ è detto \textsc{localmente connesso per archi} se ogni punto ammette un sistema fondamentale di intorni connessi per archi.
\end{defn}

\begin{ex}
$(\mathbb{R}\times \{0\}) \cup (\mathbb{Q}\times \mathbb{R})$ è uno spazio connesso per archi non localmente connesso.
\end{ex}
