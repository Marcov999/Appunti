\begin{defn}
  Uno spazio topologico $X$ è \textsc{compatto} se ogni suo ricoprimento aperto ammette un sottoricoprimento finito, cioè se dato $\mathcal{U}=\{U_i\}_{i \in I}$ t.c. $\displaystyle X=\bigcup_{i \in I} U_i$ esistono $i_0, \dots, i_n$ t.c. $X=U_{i_0} \cup \dots \cup U_{i_n}$. Un sottospazio $Y \subseteq X$ è compatto se lo è con la topologia di sottospazio.
\end{defn}

\begin{defn}
  Uno spazio metrico $X$ è \textsc{limitato} se esistono $X_0 \in X$ e $R>0$ t.c. $X=B(x_0, R)$ o equivalentemente se $diam(X)=\sup{\{d(x, y), x, y \in X\}}<+\infty$.
\end{defn}

\begin{lm} \label{mc->l}
  Se $X$ è metrico compatto, allora $X$ è limitato.
\end{lm}

\begin{proof}
  Scelto $x_0 \in X$, $U_n=B(x_0, n), n \in \mathbb{N}$ è un ricoprimento aperto di $X$. Esistono allora $i_0, \dots, i_n$ t.c. $X=U_{i_0} \cup \dots \cup U_{i_n}$. Poniamo allora $R=\max\{i_0, \dots, i_n\}$ e otteniamo $X=B(x_0, R)$.
\end{proof}

\begin{cor}
  $\mathbb{R}$ non è compatto.
\end{cor}

\begin{oss}
  La compattezza è invariante per omeomorfismo, dunque $(-1, 1)$ non è compatto in quanto omeomorfo a $\mathbb{R}$ tramite $x \longmapsto \tan{\frac{\pi}{2}x}$, con inversa $y \longmapsto \frac{2}{\pi}\tan^{-1}{y}$. Segue, per omeomorfismo affine con $(-1, 1)$, che nessun intervallo aperto $(a, b)$ è compatto.
\end{oss}

\begin{cor}
  Metrico e limitato non implica compatto.
\end{cor}

\begin{thm}
  $[0, 1]$ è compatto.
\end{thm}

\begin{proof}
  Sia $\mathcal{U}=\{U_i\}_{i \in I}$ un ricoprimento aperto di $[0, 1]$ e consideriamo $t_0=\sup{A}$ dove $A=\{t \in [0, 1] \text{ t.c. c'è un sottoricoprimento finito che ricopre } [0, t]\}$.
  Chiaramente l'insieme è non vuoto e contiene elementi maggiori di $0$, infatti c'è un aperto $U_{i_0}$ che contiene $0$ e dunque contiene un intervallo della forma $[0, \epsilon), \epsilon>0$: qualunque numero maggiore di $0$ e minore di $\epsilon$ sta dunque in $A$. Il $\sup$ di $A$ è anche un massimo: infatti esiste un aperto $U_{i_{t_0}}$ che contiene $t_0$, dunque contiene un intorno della forma $(t_0-\delta, t_0]$.
  Prendendo allora un sottoricoprimento finito che ricopre $[0, t_0-\delta/2]$ è aggiungendoci $U_{i_{t_0}}$ otteniamo un sottoricoprimento finito che ricopre $[0, t_0]$, da cui $t_0 \in A$ e dunque è un massimo.
  Supponiamo per assurdo che $t_0 \not=1$, cioè $0<t_0<1$, allora sempre in $U_{i_{t_0}}$ c'è un intorno della forma $[t_0, t_1)$ con $t_0<t_1<1$, quindi tutti i punti in $(t_0, t_1)$ stanno in $A$, contro la massimalità di $t_0$, assurdo.
\end{proof}

\begin{thm} \label{cont_comp}
  $f:X \rightarrow Y$ continua, $X$ compatto $\Rightarrow$ $f(X)$ compatto.
\end{thm}

\begin{proof}
  Sia $\{U_i\}$ un ricoprimento aperto di $f(X)$ con gli $U_i$ aperti di $f(X)$, allora esistono $\{V_i\}$ aperti di $Y$ t.c. $U_i=f(X) \cap V_i$.
  Allora $\{f^{-1}(U_i)\}$ è un ricoprimento di $X$ e $f^{-1}(U_i)=f^{-1}(f(X) \cap V_i)=f^{-1}(f(X)) \cap f^{-1}(V_i)=X \cap f^{-1}(V_i)$ (il penultimo uguale è vero perché $f^{-1}(V_i) \subseteq f^{-1}(f(X))$), dunque per la continuità di $f$ è un'intersezione di un aperto con tutto, cioè è aperto, dunque $\{f^{-1}(U_i)\}$ è un ricoprimento aperto di $X$, ne estraiamo un sottoricoprimento, prendiamo le immagini degli aperti così trovati e questi sono un sottoricoprimento finito di $f(X)$ estratto da $\{U_i\}$.
\end{proof}

\begin{ftt}
  I seguenti sono fatti banali:
  \begin{nlist}
    \item ogni spazio finito è compatto;
    \item unione finita di spazi compatti è compatta.
  \end{nlist}
\end{ftt}

\begin{thm} \label{ccc}
  $X$ compatto, $Y \subseteq X$ chiuso $\implies$ $Y$ compatto.
\end{thm}

\begin{proof}
  Sia $\{U_i\}$ un ricoprimento di $Y$ con gli $U_i$ aperti di $Y$, allora per ogni $i$ esiste un $V_i$ aperto di $X$ t.c. $U_i=Y \cap V_i$ e $\{V_i\}$ è ancora un ricoprimento di $Y$. Notiamo che anche $W=X \setminus Y$ è un aperto e dunque $\{V_i\} \cup \{W\}$ è un ricoprimento aperto di $X$.
  Allora, essendo $X$ compatto, esiste un sottoricoprimento finito $W, V_{i_0}, \dots V_{i_n}$ (a priori, $W$ potrebbe non essere necessario, ma aggiungerlo non cambia niente). Essendo $W \cap Y=\emptyset$ dev'essere che $V_{i_0}, \dots, V_{i_n}$ ricoprono $Y$, dunque i rispettivi $U_{i_0}, \dots, U_{i_n}$ sono il sottoricoprimento finito cercato.
\end{proof}

\begin{oss}
  Abbiamo visto (e useremo ancora) che $Y \subseteq X$ è compatto se e solo se per ogni famiglia $\{U_i\}_{i \in I}$ di aperti di $X$ t.c. $\displaystyle Y \subseteq \bigcup_{i \in I} U_i$ esiste $\mathcal{I} \subseteq I$ finito con $\displaystyle Y \subseteq \bigcup_{i \in \mathcal{I}} U_i$.
\end{oss}

\begin{ex}
  Sia $X$ un insieme con la topologia cofinita, allora ogni sottoinsieme di $X$ è compatto. Infatti, sia $Y \subseteq X$ e supponiamo $\displaystyle Y \subseteq \bigcup_{i \in I} U_i$, $U_i$ aperti di $X$. Se $Y=\emptyset$ allora è compatto, altrimenti esiste $y_0 \in Y$ e $y_0 \in U_{i_0}$ per qualche $i_0 \in I$.
  Poiché $X \setminus U_{i_0}$ è finito, anche $Y \setminus U_{i_0}$ è finito, diciamo $Y \setminus U_{i_0}=\{y_1, \dots, y_n\}$, allora per ogni $j=1, \dots, n$ esiste $i_j$ con $y_j \in U_{i_j}$ e dunque $Y \subseteq U_{i_0} \cup U_{i_1} \dots \cup U_{i_n}$.

  Ad esempio, se $X=\mathbb{Z}$ con la topologia cofinita, $\mathbb{N} \subseteq \mathbb{Z}$ è compatto ma non chiuso.
\end{ex}

\begin{thm} \label{chc}
  Sia $X$ spazio topologico T2, $Y \subseteq X$, $Y$ compatto. Allora $Y$ è chiuso.
\end{thm}

\begin{proof}
  Dimostriamo che $X \setminus Y$ è aperto. Sia $x \in X \setminus Y$. Dato che $X$ è T2, per ogni $y \in Y$ esistono due aperti disgiunti $U_y, V_y$ t.c. $x \in U_y, y \in V_y$.
  $\{V_y\}_{y \in Y}$ è un ricoprimento aperto di $Y$, che è compatto, dunque esiste un sottoricoprimento finito $V_{y_1}, \dots V_{y_n}$. Notiamo allora che $V=V_{y_1} \cup \dots \cup V_{y_n}, U=U_{y_1} \cap \dots \cap U_{y_n}$ sono aperti disgiunti t.c. $Y \subseteq V, x \in U$.
  In particolare, $U$ è un intorno aperto di $x$ disgiunto da $Y$ e, dato che lo possiamo trovare per ogni $x \in X \setminus Y$, possiamo concludere che $X \setminus Y$ è aperto e dunque $Y$ è chiuso.
\end{proof}

\begin{lm} \label{ch->r}
  $X$ compatto T2 $\implies$ $X$ regolare.
\end{lm}

\begin{proof}
  T2 $\implies$ T1, dobbiamo dimostrare T3. Siano $Y \subseteq X$ chiuso, $x_0 \in X \setminus Y$. Dal teorema \ref{ccc} $Y$ è compatto, prendendo gli aperti $U, V$ della dimostrazione del teorema \ref{chc} abbiamo esattamente quello che ci serve per soddisfare T3.
\end{proof}

\begin{thm} \label{ch->n}
  $X$ compatto T2 $\implies$ $X$ normale.
\end{thm}

\begin{proof}
  T2 $\implies$ T1, dobbiamo dimostrare T4. Siano $C$ e $D$ due chiusi disgiunti in $X$. Dal lemma \ref{ch->r} sappiamo che $X$ è regolare, dunque per ogni $x \in C$ esistono $U_x, V_x$ aperti disgiunti $x \in U_x, D \subseteq V_x$. $\{U_x\}_{x \in C}$ è un ricoprimento aperto di $C$, che per il teorema \ref{ccc} è compatto, dunque esiste un sottoricoprimento finito $U_{x_1}, \dots, U_{x_n}$.
  Allora prendendo $U=U_{x_1} \cup \dots \cup U_{x_n}, V=V_{x_1} \cap \dots \cap V_{x_n}$ abbiamo che sono due aperti disgiunti t.c. $C \subseteq U, D \subseteq V$, che è esattamente quello che ci serve per dimostrare T3.
\end{proof}

\begin{prop}
  $X$ T2, $C, D \subseteq X$ compatti e disgiunti, allora esistono $U, V$ aperti disgiunti con $C \subseteq U, D \subseteq V$.
\end{prop}

\begin{proof}
  Dato che sfrutta le stesse idee delle dimostrazioni precedenti, è lasciata come esercizio al lettore.
\end{proof}

\begin{thm}
  $X$ spazio topologico, $Y_i, i \in I$ famiglia di sottoinsiemi chiusi con $Y_{i_0}$ compatto per qualche $i_0 \in I$. Se per ogni $J \subseteq I$ finito vale che $\displaystyle \bigcap_{i \in J} Y_i \not = \emptyset$ (\textsc{proprietà dell'intersezione finita}), allora $\displaystyle \bigcap_{i \in I} Y_i \not=\emptyset$.
\end{thm}

\begin{proof}
  Per ogni $i \in I, i \not=i_0$, sia $Z_i=Y_{i_0} \cap Y_i$. $Z_i$ è un chiuso di $Y_{i_0}$, dunque $W_i=Y_{i_0} \setminus Z_i=Y_{i_0} \setminus Y_i$ è un aperto di $Y_{i_0}$. Quindi, se per assurdo $\displaystyle \bigcap_{i \in I} Y_i=\emptyset$, avremmo che $\{W_i, i \in I\}$ sarebbe un ricoprimento aperto di $Y_{i_0}$.
  Allora, per compattezza di $Y_{i_0}$, esiste un sottoricoprimento finito $W_{i_1}, \dots, W_{i_n}$,
  cioè $Y_{i_0}=W_{i_1} \cup \dots \cup W_{i_n}=(Y_{i_0} \setminus Y_{i_1}) \cup \dots \cup (Y_{i_0} \setminus Y_{i_n})=Y_{i_0} \setminus (Y_{i_1} \cap \dots \cap Y_{i_n}) \implies$
  $\implies Y_{i_0} \cap Y_{i_1} \cap \dots \cap Y_{i_n}=\emptyset$, assurdo.
\end{proof}

\begin{ex}
  Serve un $Y_i$ compatto: $X=\mathbb{R}, Y_n=[n, +\infty)$ è un controesempio. \marginpar\warningsign
\end{ex}

\begin{cor}
  Se $Y_n, n \in \mathbb{N}$ è una famiglia di sottoinsiemi non vuoti di $X$ con $Y_0$ compatto, $Y_i$ chiuso per ogni $i \in \mathbb{N}$, $Y_{n+1} \subseteq Y_n$ per ogni $n \in \mathbb{N}$, allora $\displaystyle \bigcap_{n \in \mathbb{N}} Y_n \not= \emptyset$.
\end{cor}

\begin{lm} \label{comp_base}
  Sia $X$ spazio topologico e $\mathcal{B}$ una base di $X$. Se ogni ricoprimento di $X$ con aperti di $\mathcal{B}$ ammette un sottoricoprimento finito, allora $X$ è compatto.
\end{lm}

\begin{proof}
  Sia $\mathcal{U}=\{U_i\}_{i \in I}$ un qualsiasi ricoprimento aperto di $X$. Per definizione di ricoprimento si ha che per ogni $x \in X$ esiste $i(x) \in I$ t.c. $x \in U_{i(x)}$ e per definizione di base esiste $B_x \in \mathcal{B}$ t.c. $x \in B_x \subseteq U_{i(x)}$.
  $\{B_x\}_{x \in X}$ è un ricoprimento di $X$ con aperti di $B$ (si dice che $\{B_x\}_{x \in X}$ è un \textit{raffinamento} di $\mathcal{U}$).
  Per ipotesi esistono allora $x_1, \dots, x_n$ t.c. $X=B_{x_1} \cup \dots \cup B_{x_n} \subseteq U_{i(x_1)} \cup \dots \cup U_{i(x_n)}$, da cui la tesi.
\end{proof}

\begin{thm} \label{prod_comp}
  $X, Y$ compatti $\implies$ $X \times Y$ compatto.
\end{thm}

\begin{proof}
  Per il lemma \ref{comp_base}, possiamo partire da un ricoprimento $\mathcal{U}=\{U_i \times V_i\}_{i \in I}$ dove $U_i$ è aperto in $X$, $V_i$ è aperto in $Y$ per ogni $i \in I$.
  Per ogni $x \in X$ il sottoinsieme $\{x\} \times Y \subseteq X \times Y$ è compatto in quanto omeomorfo a $Y$, per cui esiste $J_x \subseteq I$ finito t.c. $\displaystyle \{x\} \times Y \subseteq \bigcup_{i \in J_x} (U_i \times V_i)$. Poniamo $\displaystyle U_x= \bigcap_{i \in J_x} U_i$, è aperto in quanto intersezione finita di aperti e per costruzione
  $\displaystyle U_x \times Y \subseteq \bigcup_{i \in J_x} (U_i \times V_i)$. Per compattezza di $X$ e poiché $\{U_x\}_{x \in X}$ è un ricoprimento aperto, abbiamo che $X=U_{x_1} \cup \dots \cup U_{x_n}$ per qualche $x_1, \dots, x_n$.
  Allora $\displaystyle X \times Y=\bigcup_{k=1}^n (U_{x_k} \times Y) \subseteq \bigcup_{k=1}^n \bigcup_{i \in J_{x_k}} (U_i \times V_i)$, da cui la tesi.
\end{proof}

\begin{oss}
  Se $A \subseteq X, B \subseteq Y$, la topologia prodotto di $A \times B$, ciascuno dotato della topologia di sottospazio, coincide con la topologia di sottospazio di $X \times Y$. Dunque, per il teorema \ref{prod_comp}, se $A, B$ sono sottospazi compatti, $A \times B$ è un sottospazio compatto di $X \times Y$.
\end{oss}

\begin{prop}
  $C \subseteq \mathbb{R}^n$ è compatto $\Leftrightarrow$ è chiuso e limitato.
\end{prop}

\begin{proof}
  ($\implies$) Segue dal fatto che $\mathbb{R}^n$ è un metrico T2, dal lemma \ref{mc->l} e dal teorema \ref{chc}.

  ($\Leftarrow$) Se $C$ è limitato, esiste $R>0$ t.c. $C \subseteq [-R, R]^n \subseteq \mathbb{R}^n$, e $[-R, R]^n$ è compatto in quanto prodotto finito di copie del compatto $[-R, R]$ (che è compatto perché isomorfo a $[0, 1]$), quindi $C$, essendo chiuso in un compatto, è compatto per il teorema \ref{ccc}.
\end{proof}

\begin{thm}
  $f:X \rightarrow Y$ continua, $X$ compatto, $Y$ T2, allora $f$ è chiusa.
\end{thm}

\begin{proof}
  Se $C \subseteq X$ è chiuso, allora è chiuso in un compatto, dunque per il teorema \ref{ccc} è compatto, quindi per il teorema \ref{cont_comp} $f(C)$ è compatto in $Y$, allora è compatto in un T2, dunque per il teorema \ref{chc} è chiuso.
\end{proof}

\begin{defn}
  $X$ spazio topologico si dice \textsc{compattamente generato} se i compatti di $X$ formano un ricoprimento fondamentale.
\end{defn}

\begin{lm}
  Se ogni $x \in X$ ha un intorno compatto, $X$ è compattamente generato.
\end{lm}

\begin{proof}
  Basta far vedere che se $A \subseteq X$ è t.c. $A \cap K$ è aperto in $K$ per ogni compatto $K \subseteq X$, allora $A$ è aperto in $X$. Sia $p \in A$, basta mostrare che $p \in A^{\circ}$. Per ipotesi esiste un intorno compatto $K$ di $p$, cioè $p \in K^{\circ} \subseteq K$.
  Ora, $A \cap K^{\circ}$ è aperto in $K^{\circ}$, ed è perciò aperto in $X$ in quanto aperti di aperti sono aperti, dunque $p \in A \cap K^{\circ} \subseteq A$, ma $A \cap K^{\circ}$ è un aperto, da cui $p \in A^{\circ}$ come voluto.
\end{proof}

\begin{defn}
  $f:X \rightarrow Y$ continua è \textsc{propria} se $f^{-1}(K)$ è un compatto di $X$ per ogni $K \subseteq Y$ compatto.
\end{defn}

\begin{ex}
  $f:\mathbb{R}^n \rightarrow \mathbb{R}^n$ continua è propria se e solo se per ogni successione $\{x_n\}_{n \in \mathbb{N}}$ con $\displaystyle \lim_{n \rightarrow +\infty} ||x_n||=+\infty$ si ha che $\displaystyle \lim_{n \rightarrow +\infty} ||f(x_n)||=+\infty$. La dimostrazione è lasciata per esercizio al lettore.
\end{ex}

\begin{thm}
  $f:X \rightarrow Y$ continua e propria. Se $Y$ è T2 e compattamente generato, allora $f$ è chiusa.
\end{thm}

\begin{proof}
  Sia $C \subseteq X$ chiuso. Poiché $Y$ è compattamente generato, basta vedere che $f(C) \cap K$ è chiuso in $K$ per ogni $K \subseteq Y$ compatto. Ora $f(C) \cap K=f(C \cap f^{-1}(K))$. Poiché $f$ è propria, $f^{-1}(K)$ è compatto, per cui $C \cap f^{-1}(K)$ è chiuso in compatto, perciò è compatto per il lemma \ref{ccc}, dunque $f(C \cap f^{-1}(K))$ è compatto in $Y$ che è T2, che è chiuso per il lemma \ref{chc}, da cui la tesi.
\end{proof}

\begin{ex}
  Ogni punto di $\mathbb{R}^n$ ha un intorno compatto $\implies$ $\mathbb{R}^n$ è compattamente generato, dunque se $f:\mathbb{R}^n \rightarrow \mathbb{R}^m$ è propria, è chiusa.
  Se invece consideriamo $\pi: \mathbb{R}^2 \rightarrow \mathbb{R}$ t.c. $\pi(x, y)=x$, non è propria poiché $\pi^{-1}(\{0\})=\{0\} \times \mathbb{R}$, che non è compatto, e infatti non è neanche chiusa (si consideri ad esempio l'iperbole).
\end{ex}

\begin{ex}
  Nessun $p \in \mathbb{Q}$ ammette un intorno compatto (in $\mathbb{Q}$). Infatti se $U \subseteq \mathbb{Q}$ è un intorno di $p$ in $\mathbb{Q}$, esiste $V \subseteq \mathbb{Q}$ aperto di $\mathbb{Q}$ con $p \in V \subseteq U$,
  perciò esiste $\epsilon>0$ t.c. $p \in ((p-\epsilon, p+\epsilon) \cap \mathbb{Q}) \subseteq U$. Se $U$ fosse compatto sarebbe compatto anche in $\mathbb{R}$ (ad esempio perché immagine della sua immersione, che è una funzione continua), dunque sarebbe chiuso in $\mathbb{R}$,
  per cui avremmo $\overline{(p-\epsilon, p+\epsilon) \cap \mathbb{Q}}=[p-\epsilon, p+\epsilon] \subseteq U \subseteq \mathbb{Q}$ (dove la chiusura è in $\mathbb{R}$), assurdo.
\end{ex}
