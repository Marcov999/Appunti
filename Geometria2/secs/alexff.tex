\begin{ex}
    Sia $p=(1,0,\dots,0)$ un punto di $S^n$. Voglio far vedere che $S^n\setminus\{p\}$ \`e isomorfo a $\mathbb{R}^n$. In effetti la proiezione stereografica mi fornisce la mappa
    \[
        f\colon(x_0, \dots, x_n)\mapsto(\frac{x_1}{1-x_0}, \dots, \frac{x_n}{1-x_0})
    \]
    che \`e un isomorfismo.

    Si ha anche che aggiungendo $p$ allo spazio di partenza si ha un compatto che contiene isomorficamente $\mathbb{R}^n$. Lo scopo della sezione \`e quello di generalizzare questa idea. Cio\`e ottenere un compatto che contiene uno spazio dato aggiungendo un punto.
\end{ex}

\begin{defn}
    Sia $(X,\tau)$ uno spazio topologico. Sia anche $\hat{X} = X \cup \{\infty\}$. Definisco
    \[
        \hat{\tau} = \tau \cup \{\hat{X}\setminus K\; |\ K \subseteq \hat{X}\  \text{chiuso e compatto}\}
    \]
    Allora $(\hat{X}, \hat{\tau})$ \`e detta compattificazione di Alexandroff.
\end{defn}

\begin{oss}
    $(\hat{X}, \hat{\tau})$ \`e uno spazio topologico compatto e vi \`e una naturale immersione aperta da $X$ a $\hat{X}$.
\end{oss}

\begin{defn}
    Uno spazio topologico si dice compatto se ogni suo punto ammette un intorno compatto.
\end{defn}

\begin{prop}
    La compattificazione di uno spazio $X$ \`e T2 se e solo se $X$ \`e T2 e localmente compatto.
\end{prop}
\begin{proof}
    Hint: la dimostrazione, come molte di quelle che riguardano la compattificazione, si fa distinguendo il caso in cui si prende un punto di $X$ e quello in cui si prende il punto $\infty$.
\end{proof}
