\begin{ex}
    Sia $p=(1,0,\dots,0)$ un punto di $S^n$. Voglio far vedere che $S^n\setminus\{p\}$ \`e isomorfo a $\mathbb{R}^n$. In effetti la proiezione stereografica mi fornisce la mappa
    \[
        f\colon(x_0, \dots, x_n)\mapsto(\frac{x_1}{1-x_0}, \dots, \frac{x_n}{1-x_0})
    \]
    che \`e un isomorfismo.

    Si ha anche che aggiungendo $p$ allo spazio di partenza si ha un compatto che contiene isomorficamente $\mathbb{R}^n$. Lo scopo della sezione \`e quello di generalizzare questa idea. Cio\`e ottenere un compatto che contiene uno spazio dato aggiungendo un punto.
\end{ex}

\begin{defn}
    Sia $(X,\tau)$ uno spazio topologico. Sia anche $\hat{X} = X \cup \{\infty\}$. Definisco
    \[
        \hat{\tau} = \tau \cup \{\hat{X}\setminus K\; |\ K \subseteq \hat{X}\  \text{chiuso e compatto}\}
    \]
    Allora $(\hat{X}, \hat{\tau})$ \`e detta compattificazione di Alexandroff.
\end{defn}

\begin{oss}
    $(\hat{X}, \hat{\tau})$ \`e uno spazio topologico compatto e vi \`e una naturale immersione aperta da $X$ a $\hat{X}$.
\end{oss}

\begin{defn}
    Uno spazio topologico si dice compatto se ogni suo punto ammette un intorno compatto.
\end{defn}

\begin{prop}
    La compattificazione di uno spazio $X$ \`e T2 se e solo se $X$ \`e T2 e localmente compatto.
\end{prop}
\begin{proof}
    Hint: la dimostrazione, come molte di quelle che riguardano la compattificazione, si fa distinguendo il caso in cui si prende un punto di $X$ e quello in cui si prende il punto $\infty$.
\end{proof}

\begin{prop}
    Sia $f\colon X \longrightarrow Y$ un'immersione aperta tra spazi T2. Si definisca $g\colon Y \longrightarrow \hat{X}$ con
    \[
        g(y) =
        \begin{cases}
            x\quad \text{se}\ f^{-1}(y) = \{x\}\\
            \infty \quad\text{se}\ f^{-1}(y) = \emptyset
        \end{cases}
    \]
    Allora $g$ \`e continua.
\end{prop}
\begin{proof}
    Hint: si prende un aperto nello spazio compattificato e si cerca di dire che la sua controimmagine \`e aperta. Per farlo si distingue il caso in cui l'infinito \`e nell'aperto da quello in cui non \`e nell'aperto.
\end{proof}

\begin{cor}
    Dato uno spazio $X$ T2 compatto, esso \`e isomorfo alla compattificazione di $X$ privato di un punto.
\end{cor}

\begin{prop}
    Sia $f\colon X \longrightarrow Y$ continua con $Y$ T2. Si prenda ${\hat{f}\colon \hat{X}\longrightarrow\hat{Y}}$ che manda $x$ in $f(x)$ e l'infinito nell'infinito. Allora $\hat{f}$ \`e continua se e solo se $f$ \`e propria.
\end{prop}

\begin{defn}
    Sia $X$ uno spazio topologico. Un'esaustione in compatti di $X$ \`e una famiglia ${\{K_n\}_{n\in\mathbb{N}}}$ di compatti di $X$ tali che ${K_n\subseteq\ \open{K_{n+1}}}$ e $\bigcup_nK_n = X$.
\end{defn}

\begin{prop}
    Sia $K_n$ un'esaustione in compatti e $H$ un compatto. Allora $H$ \`e contenuto in $K_i$ per qualche $i$.
\end{prop}
\begin{proof}
    Poich\'e $H\subseteq \bigcup_n\open{K_n}$, $H$ \`e contenuto in un'unione finita di $K_i$, allora \`e contenuto nel pi\`u grande di questi.
\end{proof}

\begin{ex}
    $\mathbb{R}\times(0,1)$ non\`e isomorfo a $\mathbb{R}\times[0,1]$.
\end{ex}
\begin{proof}
    Si ha che $K_n = [-n,n]\times[\frac{1}{n}, 1-\frac{1}{n}]$ \`e un esaustione in compatti di $\mathbb{R}\times(0,1)$, mentre $H=\{0\}\times[0,1]$ \`e un compatto dell'altro spazio. Per\`o l'immagine isomorfa di $H$ non \`e contenuta in alcun $K$. Quindi si ha un assurdo.
\end{proof}

\begin{defn}
    Dato $X$ spazio topologico, $\pi_0(X)$ \`e l'insieme delle componenti connesse per archi di $X$.
\end{defn}

\begin{ex}
    $\mathbb{R}^n$ privato di $s$ punti non \`e isomorfo a $\mathbb{R}^n$
    privato di $t$ punti per $s\ne t$.
\end{ex}
\begin{proof}
    L'idea di base \`e quella di prendere esaustioni in compatti dei due spazi.
    Allora i compatti delle esaustioni (almeno definitivamente) avranno un buco per ogni punto tolto. Per ottenere un assurdo, basta contare le componenti connesse delle esaustioni.
\end{proof}
