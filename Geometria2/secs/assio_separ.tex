In un generico spazio topologico, vorremmo usare gli aperti per dire in qualche modo che punti o insiemi (che prenderemo chiusi) disgiunti sono separati anche a livello della topologia. Per questo richiediamo che gli spazi soddisfino alcune condizioni, i cosiddetti \textsc{assiomi di separazione}. Negli enunciati degli assiomi $X$ è uno spazio topologico con una topologia data. Ecco i primi due.

(T1) $\forall \, x, y \in X, x \not=y$ esistono aperti $U, V \subseteq X$ con $x \in U \setminus V, y \in V \setminus U$;

(T2) $\forall \, x, y \in X, x \not=y$ esistono aperti $U, V \subseteq X$ con $U \cap V= \emptyset, x \in U, y \in V$.

\begin{defn} Se $X$ gode di T2 è detto \textsc{spazio di Hausdorff}, o spazio topologico \textit{separato}.
\end{defn}

Chiaramente T2 $\implies$ T1.

\begin{prop}
  Uno spazio topologico $X$ è T1 $\Leftrightarrow$ i punti sono chiusi.
\end{prop}

\begin{proof}
  ($\implies$) Supponiamo $X$ sia T1 e sia $x \in X$. Per ogni $y \in X, y \not=x$ esistono $U_y, V_y$ aperti che soddisfano le condizioni imposte da T1. Ma allora $x \not \in V_y, y \in V_y$, da cui $\displaystyle X \setminus \{x\}=\bigcup_{y \in X, y \not=x} V_x$, essendo unione di aperti è aperto, perciò il suo complementare, $\{x\}$, è chiuso, come voluto.

  ($\Leftarrow$) Supponiamo adesso che i punti di $X$ siano chiusi. Allora, dati $x, y \in X, \\ x \not= y$, scegliendo gli aperti $U=X \setminus \{y\}, V=X \setminus \{x\}$ si ha che questi soddisfano le condizioni di T1.
\end{proof}

\begin{oss}
  Segue facilmente dalla proposizione sopra che in uno spazio T1 tutti gli insiemi cofiniti sono aperti. D'altra parte, se tutti i cofiniti sono aperti i punti sono chiusi. Dunque, se abbiamo uno spazio topologico generico $(X, \tau)$, possiamo affermare che esso è T1 $\Leftrightarrow$ $\tau$ è più fine della topologia cofinita.
\end{oss}

\begin{prop}
  $X$ è di Hausdorff $\Leftrightarrow$ $\Delta_X \subseteq X \times X$ è chiuso, dove $\Delta_x=\{ (x, x) | x \in X \}$ è la \textit{diagonale di $X$}.
\end{prop}

\begin{proof}
  ($\implies$) Sia $(x, y) \in X \times X \setminus \Delta_X$. Allora $x \not= y$, dunque esistono due aperti disgiunti $U_x, V_y$ t.c. $x \in U_x, y \in U_y$, da cui $(x, y) \in U_x \times V_y$, che è un aperto della topologia prodotto. Dunque $\displaystyle X \times X \setminus \Delta_X=\bigcup_{(x, y) \in X \times X \setminus \Delta_X} U_x \times V_y$ è unione di aperti, dunque è aperto e il complementare, $\Delta_X$, è chiuso.

  ($\Leftarrow$) Supponiamo adesso $\Delta_X$ chiuso. Dati due punti $x, y \in X, x \not= y$ la coppia $(x, y)$ sta nel complementare di $\Delta_X$, che è aperto. Esiste allora un elemento della base della topologia prodotto su $X \times X$ tutto contenuto in $X \times X \setminus \Delta_X$ che contiene $(x, y)$. Ma questo è della forma $U \times V$ per qualche $U$ e $V$ aperti con $x \in U, y \in V$ e dato che non interseca $\Delta_X$, significa che non possiamo trovare lo stesso elemento in entrambi gli insiemi, cioè $U \cap V= \emptyset$. Segue allora che $X$ è di Hausdorff.
\end{proof}

\begin{ex} \label{metrizz->T2}
  Un esempio di spazio di Hausdorff è un qualsiasi spazio metrizzabile. Lasciamo come semplice esercizio la dimostrazione che metrizzabile implica Hausdorff.

  Un esempio, invece, di uno spazio T1 ma non T2, è un qualsiasi spazio infinito con la topologia cofinita. Lasciamo anche in questo caso la verifica di quanto appena detto come semplice esercizio.
\end{ex}

\begin{prop}
  Sottospazi e prodotti arbitrari di spazi topologici T1 (rispettivamente T2) sono ancora T1 (rispettivamente T2).
\end{prop}

\begin{proof}
  Per quanto riguarda i sottospazi, basta prendere l'intersezione degli aperti in questione con il sottospazio stesso ed è semplice verificare che rispettano ancora le condizioni, sia per T1 che per T2.

  Per quanto riguarda i prodotti, ricordando la loro definizione come insiemi di funzioni, se due funzioni differiscono per almeno un elemento allora basta prendere gli aperti dati dall'assioma del caso (T1 o T2) per lo spazio topologico relativo a quell'elemento e tutto lo spazio per tutti gli altri elementi.
\end{proof}

\begin{prop}Supponiamo per assurdo $f \not=g$. Allora esiste $x \in X$ t.c. $f(x) \not= g(x)$.
  Questa proposizione segue dal fatto che se abbiamo più aperti allora di sicuro possiamo trovarne che rispettano le condizioni degli assiomi, anzi, magari ne troviamo anche più di prima. Sia dunque $X$ un insieme e $\tau_1< \tau_2$ topologie su $X$, se $\tau_1$ è T1 (rispettivamente T2), allora anche $\tau_2$ è T1 (rispettivamente T2).
\end{prop}

Vediamo ora alcune applicazioni di questi primi due assiomi di separazione.

\begin{nlist}
  \item $f: X \rightarrow Y$ continua, $Y$ T2, $\Gamma_f=\{ (x, f(x)) | x \in X\} \subseteq X \times Y$ ($\Gamma_f$ è il grafico di $f$), allora $\Gamma_f$ è chiuso in $X \times Y$;
  \item $f, g: X \rightarrow Y$ continue, $Y$ T2, allora $\{ x \in X | f(x)=g(x) \} \subseteq X$ è chiuso in $X$;
  \item $f:X \rightarrow X$ continua, $X$ T2, allora $Fix(f) = \{ x \in X | f(x)=x \} \subseteq X$ è chiuso in $X$;
  \item $f, g:X \rightarrow Y$ continue, $Y$ T2, $Z \subseteq X$ denso t.c. $f(z)=g(z) \, \forall z \in Z$, allora $f=g$;
  \item $X$ T2, $\{x_n\}$ successione convergente, allora il limite è unico.
\end{nlist}

\begin{proof}
  \begin{nlist}
    \item Sia $(x, y) \in X \times Y \setminus \Gamma_f$. Allora $y \not=f(x)$ e poiché $Y$ è T2 si ha che esistono due aperti disgiunti $U_{f(x)}, V_y$ t.c. $f(x) \in U_{f(x)}, y \in V_y$. Per la continuità di $f$, abbiamo che $U_x=f^{-1}(U_{f(x)})$ è un aperto contenente $x$ t.c. $U_x \times V_y \subseteq X \times Y$ è un aperto della topologia prodotto contenete $(x, y)$ che per quanto detto prima non interseca mai $\Gamma_f$. Allora
    $\displaystyle X \times Y \setminus \Gamma_f=\bigcup_{(x, y) \in X \times Y \setminus \Gamma_f} U_x \times V_y$ è unione di aperti, cioè apertp, e il suo complementare, $\Gamma_f$, è chiuso.
    \item Sia $X_{|f=g}$ l'insieme che vogliamo mostrare essere chiuso e $x \in X \setminus X_{|f=g}$. Allora $f(x) \not= g(x)$ e dato che $Y$ è T2 esistono due aperti disgiunti $U_x', V_x'$ t.c. $f(x) \in U_x', g(x) \in V_x'$.Supponiamo per assurdo $f \not=g$. Allora esiste $x \in X$ t.c. $f(x) \not= g(x)$.
    Per la continuità di $f$ e $g$ ho che $U_x=f^{-1}(U_x'), V_x=g^{-1}(V_x')$ sono due aperti di $X$ con intersezione non vuota (c'è almeno $x$), sia dunque $A_x$ la loro intersezione. Questo è un intorno aperto di $x$ t.c. per ogni $y \in A_x$ si ha $f(y) \not=g(y)$, per come è stato definito l'aperto. Allora $\displaystyle X \setminus X_{|f=g} = \bigcup_{x \in X \setminus X_{|f=g}} A_x$ è unione di aperti, cioè è aperto, dunque il suo complementare, $X_{|f=g}$, è chiuso.
    \item Segue direttamente dalla precedente ponendo $Y=X$ e $g=id$.
    \item Segue ancora una volta dalla seconda affermazione: poiché $Z \subseteq X_{|f=g}$, che è un chiuso, allora $X \subseteq \overline{Z} \subseteq X_{|f=g} \subseteq X \implies X_{|f=g} =X$.
    \item Supponiamo per assurdo che il limite non sia unico e siano $x \not= y$ due elementi di $X$ a cui converge la successione. Poiché $X$ è T2, esistono due aperti disgiunti $U, V$ con $x \in U, y \in V$. Ma questi sono rispettivamente intorni di $x$ e $y$, dunque per definizione di limite la successione deve essere definitvamente contenuta in entrambi gli insiemi, assurdo poiché essi sono disgiunti.
  \end{nlist}
\end{proof}

Vediamo adesso gli altri due assiomi di separazione.

(T3) Per ogni chiuso $C \subseteq X, x \in X \setminus C$, esistono aperti $U, V \subseteq X$ disgiunti con $x \in U, C \subseteq V$;

(T4) per ogni coppia di chiusi disgiunti $C, D \subseteq X$ esistono aperti $U, V \subseteq X$ disgiunti con $C \subseteq U, D \subseteq V$.

\begin{oss}
  T4+T1 $\implies$ T3+T1 $\implies$ T2 $\implies$ T1. La dimostrazione è lasciata come utile esercizio al lettore.
\end{oss}

\begin{defn}
  $X$ è detto \textsc{normale} se soddisfa T1+T4. $X$ è detto \textsc{regolare} se soddisfa T1+T3.
\end{defn}

\begin{oss}
  Possiamo riformulare la condizione T4 come segue: per ogni coppia di chiusi $C, D \subseteq X$ esiste $U \subseteq X$ aperto t.c. $C \subseteq U \subseteq \overline{U} \subseteq X \setminus D$.
\end{oss}

\begin{proof}
  Chiaramente se esiste un siffatto $U$ possiamo prendere lui e l'aperto $X \setminus \overline{U}$ per soddisfare T4. Viceversa, siano $U, V$ gli aperti dati dalla condizione T4. Poiché $X \setminus V$ è chiuso e contenuto in $X \setminus D$ abbiamo dunque che $C \subseteq U \subseteq \overline{U} \subseteq X \setminus V \subseteq X \setminus D$.
\end{proof}

\begin{prop}
  $X$ s.t. metrizzabile $\implies$ $X$ normale.
\end{prop}

\begin{proof}
  Dall'esempio \ref{metrizz->T2} sappiamo già che $X$ è T2, dunque anche T1. Resta da dimostrare che $X$ è T4.

  Siano dunque $C, D \subseteq X$ due chiusi disgiunti e sia $d$ la distanza su $X$. Definiamo $d_C(x)=\inf_{y \in C} d(x, y)$. Mostriamo che $d_C(x)=0 \Leftrightarrow x \in C$. Una freccia è banale. Per l'altra, se esistesse $x \in X \setminus C$ t.c. $d_C(x)=0$ basta prendere le palle di raggio $1/n$ per estrarre una successione in $C$ che tende a $ x \not\in C$, asssurdo per la proposizione \ref{chiusoxsucc}. Analogamente definiamo $d_D(x)$ e
  dimostriamo che $d_D(x)=0 \Leftrightarrow x \in D$. Queste due funzioni così definite sono chiaramente continue. Definiamo adesso $\displaystyle f(x)=\frac{3d_C(x)}{d_C(x)+d_D(x)}$, che è chiaramente continua perché ottenuta da funzioni continue con operazioni che conservano la continuità. Inoltre è ben definita, poiché essendo $D$ e $C$ disgiunti non può mai essere che $d_C(x)=d_D(x)=0$, perciò il denominatore è sempre positivo. Osserviamo ora che $f:X \rightarrow [0, 3], C=f^{-1}(0), D=f^{-1}(3)$. Inoltre, $[0, 3]$ ha la topologia di sottospazio, perciò $[0, 1), (2, 3]$ sono due aperti disgiunti e per la continuità di $f$ anche le loro controimmagini sono aperti disgiunti che contengono rispettivamente $C$ e $D$, che è quello che serve per soddisfare T4.
\end{proof}

\begin{prop}
  Sottospazi e prodotti arbitrari di T3 sono T3; sottospazi chiusi di T4 sono T4; in generale, sottospazi arbitrari di T4 o prodotti arbitrari di T4 non sono T4.
\end{prop}

\begin{proof}
  Sia $X$ T3 e $Y \subseteq X$ un sottoinsieme con la topologia di sottospazio. Siano adesso $C \subseteq Y$ un chiuso e $x \in Y \setminus C$. Per definizione della topologia di sottospazio, $C=Y \cap D$ dove $D$ è un chiuso di $X$. Ovviamente $x \not\in D$, altrimenti, dato che $x \in Y$, avremmo $x \in Y \cap D=C$. Allora esistono due aperti disgiunti $U, V$ di $X$ t.c. $x \in U, D \subseteq V$. Allora $U'=U \cap Y, V'=V \cap Y$ sono due aperti
  disgiunti di $Y$ che soddisfano le condizioni per avere T3. Per quanto riguarda T4, questa dimostrazione non funziona, dato che due chiusi disgiunti in $Y$ potrebbero non essderlo in $X$. Se però $Y$ fosse chiuso in $X$, allora i due chiusi di $Y$ sarebbero definiti a loro volta come intersezione di $Y$, che è chiuso in $X$, e un altro chiuso di $X$, quindi sarebbero a loro volta chiusi in $X$ e potremmo ripetere la dimostrazione come sopra. Serve un esempio per dire che in generale non si può fare: pensateci.

  Passiamo adesso ai prodotti. Sia $\displaystyle X=\prod_{\alpha \in A} X_{\alpha}$ dove $X_{\alpha}$ è T3 per ogni $\alpha \in A$ e siano $C \subseteq X$ un chiuso, $x \in X \setminus C$. Indichiamo con $x_{\alpha}$ la proiezione di $x$ su $X_{\alpha}$. Allora esiste un elemento $B$ della base che contiene $x$ e tutto contenuto in $X \setminus C$, perciò il suo complementare è un chiuso che contiene $C$. Sia $S \subseteq A$ il
  sottoinsieme finito per cui, se scriviamo $\displaystyle B=\prod_{\alpha \in A} U_{\alpha}$, abbiamo che $U_{\alpha}=X_{\alpha} \Leftrightarrow \alpha \not\in S$. Allora grazie alla proprietà T3 di ogni $X_{\alpha}$ troviamo, per ogni $\alpha \in S$, due aperti disgiunti $V_{\alpha}, V_{\alpha}'$  t.c.
  $x_{\alpha} \in V_{\alpha}, X_{\alpha} \setminus U_{\alpha} \subseteq V_{\alpha }'$. Ponendo dunque $\displaystyle U=\prod_{\alpha \in A} V_{\alpha}, V=\prod_{\alpha \in A} V_{\alpha}'$ (dove se $\alpha \not\in S$
  allora $V_{\alpha}=V_{\alpha}'=X_{\alpha}$) è facile verificare che questi sono due aperti disgiunti che separano $x$ e $C$, dunque $X$ è T3. Un esempio di prodotto di T4 non T4 sarà visto tra poco.
\end{proof}

Riassumendo, metrizzabile $\stackrel{\text{(i)}}{\implies}$ normale $\stackrel{\text{(ii)}}{\implies}$ regolare $\stackrel{\text{(iii)}}{\implies}$ T2. Vediamo dei controesempi che dimostrano che non valgono le implicazioni inverse.

\begin{ex}
  \begin{nlist}
    \item retta di Sorgenfrey;
    \item piano di Sorgenfrey;
    \item la topologia su $\mathbb{R}$ generata dagli intervalli $(a, b)$ e dagli insiemi $(a, b) \setminus K$ dove $K= \{ 1/n | n \in \mathbb{N}, n \not=0 \}$, detta topologia $\mathbb{R}_K$.
\end{nlist}
\end{ex}

\begin{proof}
  Per quanto visto nell'esempio \ref{Sorgenfrey-retta} sappiamo già che la retta di Sorgenfrey non è metrizzabile. Mostriamo adesso che è normale. Essendo più fine della topologia euclidea, sicuramente è T1. Per dire che è T4, siano $C, D \subseteq \mathbb{R}$ due chiusi disgiunti. Ovviamente $C \subseteq \mathbb{R} \setminus D, D \subseteq \mathbb{R} \setminus C$ e questi ultimi sono aperti. Dunque per ogni $c \in C$ esiste un  $r_c>0$ t.c. $[c, c+r_c) \subseteq \mathbb{R} \setminus D$ e analogamente per ogni
  $d \in D$ esiste un $r_d>0$ t.c. $[d, d+r_d) \subseteq \mathbb{R} \setminus C$. Siano dunque $\displaystyle U =\bigcup_{c \in C} [c, c+r_c), V=\bigcup_{d \in D} [d, d+r_d)$. Questi sono ovviamente aperti in quanto unioni di aperti e $C \subseteq U, D \subseteq V$. Se per assurdo non fossero disgiunti, allora esisterebbero $c \in C, d \in D$ t.c. $[c, c+r_c) \cap [d, d+r_d) \not= \emptyset$. Senza perdita di generalità $c<d$, allora dev'essere
  $d \in [c, c+r_c) \subseteq \mathbb{R} \setminus D$, assurdo.
  \item Ancora una volta la topologia è T1 in quanto più fine della topologia euclidea. Inoltre è T3: siano infatti $C \subseteq \mathbb{R^2}$ un chiuso e $x \not \in C$. Allora esiste un intorno aperto di $x$ della forma $[a, b) \times [c, d)$ tutto contenuto in $\mathbb{R} \setminus C$. È un semplice esercizio verificare che tale intorno è anche chiuso, quindi ponendo $U=[a, b) \times [c, d)$ e $V$ il suo complementare troviamo due aperti disgiunti t.c. $x \in U, C \subseteq V$.
  Per mostrare che non è T4, dimostriamo prima un lemma.
  \begin{lm}
    Siano $X$ T4 e separabile e $D \subseteq X$ chiuso e discreto (cioè t.c. la topologia di sottospazio è quella discreta). Allora $|D| \subseteq |\mathbb{R}|$.
  \end{lm}
  \begin{proof}
    Sia $Q \subseteq X$ un denso numerabile e sia $S \subseteq D$. $S$ è chiuso in $D$, ma ciò significa che è intersezione di un chiuso in $X$ e di $D$, che è chiuso in $X$, quindi $S$ è chiuso in $X$, e analogamente $D \setminus S$. Esistono allora due aperti disgiunti $U_S, V_S$ t.c. $S \subseteq U_S, D \setminus S \subseteq V_S$. Definiamo $f: \mathcal{P}(D) \rightarrow \mathcal{P}(Q)$ di modo che $f(S)=Q \cap A_S$. $f$ è iniettiva:
    siano infatti $S, T \subseteq D$ disgiunti e supponiamo senza perdita di generalità $S \setminus T \not= \emptyset$. Notiamo che $S \setminus T \subseteq D \setminus T \subseteq V_T$, perciò $S \setminus T \subseteq U_S \cap V_T$, quindi $U_S \cap V_T \not= \emptyset$, ma essendo intersezione di
    due aperti è un aperto, perciò esiste $x \in Q$ t.c. $x \in U_S \cap V_T \Rightarrow x \in f(S)$, ma allora $x \not\in U_T \Rightarrow x \not\in Q \cap U_T =f(T)$, il che vuol dire che $f$ è iniettiva. Perciò $|D| < |\mathcal{P}(D)| \le |\mathcal{P}(Q)|=|\mathbb{R}|$ poiché $Q$ è numerabile.
  \end{proof}
  Adesso basta ricordare dall'esempio \ref{Sorgenfrey-piano} che la retta $y=-x$, che ha la cardinalità del continuo, ha come topologia di sottospazio quella discreta. È facile mostrare che il piano di Sorgenfrey è separabile, quindi segue facilmente dal lemma che non è T4.
  \begin{oss}
    Il piano di Sorgenfrey è un esempio di prodotto di T4 che non è T4.
  \end{oss}
  \item Essendo la topologia considerata più fine della topologia euclidea, sappiamo già che è T2. Mostriamo che non è T3. Si verifica facilmente che $K$ è chiuso e $0 \not\in K$. Se esistessero due aperti disgiunti che li separano, quello di $0$ sarebbe necessariamente unione di intorni della forma $(a, b) \setminus K$. Sia quindi, senza perdita di generalità, $(-\epsilon, \epsilon) \setminus K, \epsilon>0$ uno di questi intorni. Consideriamo allora $n \in \mathbb{N} \setminus \{0\}$ t.c. $1/n < \epsilon$. Allora qualunque intorno contenente $1/n$ interseca banalmente $(-\epsilon, \epsilon)$, dunque due aperti disgiunti che separino $0$ e $K$ non possono esistere.
\end{proof}

\begin{thm}
  Se $X$ è regolare a base numerabile, allora $X$ è normale.
\end{thm}

\begin{proof}
  Essendo $X$ regolare, ovviamente è T1. Dobbiamo verificare che sia T4. Siano quindi $C, D \subseteq X$ due chiusi disgiunti e $\mathcal{B}$ una base numerabile. Per ogni $c \in C$ esiste un aperto $U_c$ t.c. $c \in U_c \subseteq  \overline{U_c} \subseteq X \setminus D$. Analogamente possiamo
  scegliere un $V_d$ per ogni $d \in D$ che li separi allo stesso modo da $C$. Notiamo che possiamo prendere $U_c, V_d \in \mathcal{B}$ per ogni $c \in C, d \in D$. Definiamo adesso $\mathcal{C} \subseteq \mathcal{B}$ come $\{U_c \in \mathcal{B} | \overline{U_c} \cap D=\emptyset\}$ e analogamente $\mathcal{D}$.
  Per quello che abbiamo detto prima, $\mathcal{C}$ ricopre $C$ e $\mathcal{D}$ ricopre $D$. Essendo $\mathcal{B}$ numerabile, possiamo scrivere $\mathcal{C}=\{U_n\}, \mathcal{D}=\{V_n\}$. Siano adesso $\displaystyle A_n= U_n \setminus \bigcup_{j=1}^n \overline{V_j}, B_n =V_n \setminus \bigcup_{j=1}^n \overline{U_j}$. Si può notare facilmente che sono tutti
  aperti, dunque $\displaystyle A= \bigcup_{n \in \mathbb{N}} A_n$ e $\displaystyle B= \bigcup_{n \in \mathbb{N}} B_n$ sono aperti, inoltre, dato che gli $U_n$ ricoprono $C$ mentre $C \cap \overline{V_n}$ per ogni $n$, abbiamo $C \subseteq A$ e analogamente $D \subseteq B$. È un semplice
  esercizio verificare che $A_n \cap B_n = \emptyset$ per ogni $n, m \in \mathbb{N}$, dunque $A$ e $B$ sono due aperti disgiunti contenenti rispettivamente $C$ e $D$, da cui segue che $X$ è T4.
\end{proof}

\begin{thm}
  Teorema di Urysohn: se $X$ è regolare a base numerabile, allora $X$ è metrizzabile.
\end{thm}

\begin{proof}
  Non verrà dimostrato in questo corso.
\end{proof}
