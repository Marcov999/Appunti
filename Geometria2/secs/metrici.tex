\begin{defn}
	Uno \textsc{spazio metrico} è una coppia $(X, d)$, $X$ insieme,
	${d: X \times X \rightarrow \mathbb{R}}$ t.c. per ogni ${x, y, z \in X}$
    \begin{nlist}
    	\item $d(x, y) \ge 0$ e $d(x, y)=0 \Leftrightarrow x=y$;
    	\item $d(x, y)=d(y, x)$;
    	\item $d(x, z) \le d(x, y)+d(y,z)$ (disuguaglianza triangolare).
    \end{nlist}
	In tal caso $d$ si dice \textsc{distanza} o \textsc{metrica}.
\end{defn}

\begin{ex}
    Ecco alcuni esempi di distanze:
    \begin{itemize}
        \item La distanza $d_1$ su $\mathbb{R}^n$, che alla coppia di elementi
        ${x=(x_1, \dots, x_n)}, {y=(y_1, \dots, y_n)}$ associa il numero
        $${\displaystyle d_1(x, y)=\sum_{i=1}^n |x_i-y_i|};$$
        \item la distanza $d_2$ (o $d_E$, distanza euclidea) su $\mathbb{R}^n$,
        $$\displaystyle d_2(x, y)=\sqrt{\sum_{i=1}^n (x_i-y_i)^2};$$
        \item la distanza $d_{\infty}$ su $\mathbb{R}^n$, ${\displaystyle
        d_{\infty}(x, y)=\sup_{i=1, \dots, n} \{ |x_i-y_i| \}}$;
        \item
        la distanza discreta su un generico insieme $x$, $d(x, y)=1$ se $x \not=
        y$ e $0$ altrimenti;
        \item le distanze $d_1$, $d_2$, $d_{\infty}$ sullo spazio delle funzioni
        continue da $[0, 1]$ in $\mathbb{R}$, definite rispettivamente
        $${\displaystyle d_1(f, g)=\int_0^1 |f(t)-g(t)| dt},$$
        $${\displaystyle d_2(f, g)=\sqrt{\int_0^1 (f(t)-g(t))^2 dt}},$$
        $${\displaystyle d_{\infty}(f, g)=\sup_{t \in [0, 1]} |f(t)-g(t)|}.$$
    \end{itemize}
\end{ex}

\begin{defn}
	$f:(X, d) \rightarrow (Y, d')$ viene detto \textsc{embedding isometrico} se
	$d'(f(x_1), f(x_2))=d(x_1, x_2)$ per ogni $x_1, x_2 \in X$.
\end{defn}

\begin{oss}
    Valgono i seguenti fatti:
    \begin{itemize}
        \item l'identità è un embedding isometrico;
        \item composizione di embedding isometrici è un embedding isometrico;
        \item se un embedding isometrico $f$ è biettivo, anche $f^{-1}$ è un
        embedding isometrico e $f$ si dice \textsc{isometria};
        \item un embedding isometrico è sempre iniettivo, dunque è un'isometria
        se e solo se è suriettivo;
        \item se $(X, d)$ è fissato, l'insieme delle isometrie da $X$ in sé è un
        gruppo con la composizione, chiamato $\Isom (X,d)$.
    \end{itemize}
\end{oss}

\begin{defn}
    Sia $V$ uno spazio vettoriale. Una norma su $V$ \`e una funzione ${\norm{\cdot}\colon V \longrightarrow [0, \infty)}$ che rispetta le seguenti propriet\`a, per ogni $x, y \in V$, $\lambda \in \mathbb{R}$:
    \begin{nlist}
        \item $\norm{x} = 0$ se e solo se $x=0$;
        \item $\norm{\lambda x} = |\lambda|\norm{x}$;
        \item $\norm{x+y} \leq \norm{x}+\norm{y}$.
    \end{nlist}
    Ogni norma induce una distanza $d$ sullo spazio definita come ${d(x,y)=\norm{x-y}}$.
\end{defn}

\begin{thm}
Le norme su uno spazio vettoriale reale di dimensione finita inducono distanze topologicamente equivalenti.
\end{thm}
\begin{proof}
    Sia $n$ la dimensione dello spazio. Poich\`e lo ogni spazio reale di tale dimensione \`e isomorfo a $\mathbb{R}^n$, assumo senza perdita di generalit\`a che $V=\mathbb{R}^n$. Sia allora $\{e_1, \dots, e_n\}$ la base canonica. Sia inoltre $M = \max_{i = 1, \dots, n}\norm{e_i}$. Si ha dunque che
    \[
        \norm{x}-\norm{y} \leq \norm{x-y} \leq M \sum_i |x_i-y_i| = M d_1(x,y)
    \]
    Ho ottenuto allora che $\norm{\cdot}$ \`e Lipschitziana nella distanza $d_1$, e dunque \`e anche continua. Poich\`e $d_1$ e $d_2$ sono equivalenti, la norma \`e continua anche con $d_2$. Allora ammette massimo e minimo sui compatti, e in particolare sulla sfera unitaria. Siano essi rispettivamente $M$ e $m$. Per ogni vettore $v$ dello spazio, vale allora
    \[
        \norm{v} = \norm{\norm{v}_2 \frac{v}{\norm{v}_2}} = \norm{v}_2 \norm{\frac{v}{\norm{v}_2}}
    \]
    che porta a
    \[
        m\norm{v}_2 \leq \norm{v} \leq M\norm{v}_2
    \]
    Passando alle distanze indotte si ottiene l'equivalenza desiderata.
\end{proof}
