Abbiamo visto che analitica $\implies$ olomorfa. Quanto segue è di interesse indipendente e servirà per dimostrare l'implicazione opposta, cioè
\begin{center}
  olomorfa $\implies$ analitica
\end{center}
(in particolare, l'esistenza della derivata prima complessa implica l'esistenza delle derivate di qualsiasi ordine).

\begin{thm}
  Sia $f:D \longrightarrow \mathbb{C}$ olomorfa. Allora $f(z)\diff z$ è una $1$-forma chiusa su $D$.
\end{thm}

\begin{proof}
  Assumendo $f$ $C^1$, $\omega=f(z)\diff z=f(z)\diff x+if(z)\diff y$ è $C^1$ e $\diff\omega=\left(\dfrac{\partial if}{\partial x}-\dfrac{\partial f}{\partial y}\right)\diff x\diff y=\left(i\dfrac{\partial f}{\partial x}-\dfrac{\partial f}{\partial y}\right)\diff x\diff y$.
  Poiché $f$ è olomorfa, $\dfrac{\partial f}{\partial x}=\diff f(1)=f'$, mentre $\dfrac{\partial f}{\partial y}=\diff f(i)=i\diff f(1)=if'$ $\implies$ $\diff\omega=(if'-if')\diff x\diff y=0$. Adesso facciamo il caso generale.

  Sia $f$ olomorfa, per assurdo $f(z)\diff z$ non è chiusa. Allora esiste un rettangolo $R \subseteq D$ t.c. $\displaystyle \int_{\partial R} \omega=\alpha(R)\not=0$. D'ora in poi, se $\overline{R}$ è un qualsiasi rettangolo, poniamo $\alpha(\overline{R})=\int_{\partial \overline{R}} \omega$. Dividiamo $R$ in quattro rettangoli uguali $A_1, A_2, A_3, A_4$ con i lati paralleli a quelli di $R$ e aventi un vertice in comune nel centro di $R$.
  $\displaystyle \alpha(R)=\int_{\partial R} \omega=\int_{\partial A_1} \omega+\int_{\partial A_2} \omega+\int_{\partial A_3} \omega+\int_{\partial A_4} \omega=\alpha(A_1)+\alpha(A_2)+\alpha(A_3)+\alpha(A_4)$ ed esiste perciò un $A_i$ t.c. $|\alpha(A_i)|\ge \dfrac{|\alpha(R)|}{4}$.
  Poniamo $R_1$ uguale a questo $A_i$. Iterando la costruzione di $R_1$, otteniamo per induzione una successione di rettangoli inscatolati $R \supseteq R_1 \supseteq R_2 \supseteq R_3 \supseteq \dots \supseteq R_n \supseteq \dots$ con $|\alpha(R_n)|\ge \dfrac{|\alpha(R)|}{4^n}$. Ora prendiamo $\displaystyle z_0 \in \bigcap_{n \in \mathbb{N}} R_n \not=\emptyset$ in quanto intersezione decrescente di compatti chiusi non vuoti.
  Poiché $f$ è olomorfa, abbiamo $f(z)=f(z_0)+f'(z_0)(z-z_0)+\epsilon(z)|z-z_0|$, dove $\displaystyle \lim_{z \longrightarrow z_0} |\epsilon(z)|=0$, per cui $\displaystyle \alpha(R_n)=\int_{\partial R_n} (f(z_0)+f'(z_0)(z-z_0)+\epsilon(z)|z-z_0|)\diff z=\int_{\partial R_n} f(z_0)\diff z+\int_{\partial R_n} f'(z_0)(z-z_0)\diff z+\int_{\partial R_n} \epsilon(z)|z-z_0|\diff z$.
  $\diff z$ e $(z-z_0)\diff z$ ammettono primitive $z$ e $(z-z_0)^2/2$, dunque sono esatte, in particolare chiuse, quindi i primi due integrali fanno $0$. Perciò $\displaystyle \alpha(R_n)=\int_{\partial R_n} \epsilon(z)|z-z_0|\diff z$.
  Ora, poiché $\displaystyle \lim_{z \longrightarrow z_0} \epsilon(z)=0$, possamo scegliere $\delta>0$ t.c. $|z-z_0|<\delta \implies |\epsilon(z)|\le \dfrac{|\alpha(R)|}{4(a+b)^2}$, dove $a, b$ sono le lunghezze dei lati di $R$.
  Poiché $z_0 \in R_n$ e $diam(R_n) \le \dfrac{a+b}{2^n}$, basta scegliere $n$ opportuna e avremo che $|\epsilon(z)|\le \dfrac{|\alpha(R)|}{4(a+b)^2}$ per ogni $z \in R_n$.
  Per tale $n$, se $z \in \partial R_n$, $|z-z_0| \le diam(R_n) \le \dfrac{a+b}{2^n}$, per cui $\displaystyle |\alpha(R_n)|=\left|\int_{\partial R_n} \epsilon(z)|z-z_0|\diff z\right|\le\int_0^{\frac{2(a+b)}{2^n}} |\epsilon(\gamma(t))|\cdot|\gamma(t)-z_0|\diff t \le \dfrac{2(a+b)}{2^n}\cdot\dfrac{|\alpha(R)|}{4(a+b)^2}\cdot\dfrac{(a+b)}{2^n}=\dfrac{1}{2}\dfrac{|\alpha(R)|}{4^n}$, il che contraddice $|\alpha(R_n)| \ge \dfrac{|\alpha(R)|}{4^n}$.
\end{proof}

\begin{cor}
  $f:D \longrightarrow \mathbb{C}$ olomorfa $\implies$ per ogni $p \in D$ esistono un aperto $U \subseteq D$, $p \in U$ e $F:U \longrightarrow \mathbb{C}$ olomorfa con $F'=f\restrict{U}$.
\end{cor}

\begin{proof}
  $f(z)\diff z$ chiusa $\implies$ esistono $U$ come nell'enunciato, $F:U \longrightarrow \mathbb{C}$ con  $\diff F=f(z)\diff z$ su $U$, cioè $F$ è olomorfa e $F'=f$ su $U$.
\end{proof}

\begin{cor}
  $f:D \longrightarrow \mathbb{C}$ olomorfa $\implies$ $\displaystyle \int_{\gamma} f(z)\diff z=0$ per ogni $\gamma$ laccio in $D$ omotopicamente banale.
\end{cor}

\begin{prop}
  Sia $f:D \longrightarrow \mathbb{C}$ continua, $D$ aperto, e $f$ olomorfa su $D \setminus r$, $r$ una retta orizzontale. Allora $f(z)\diff z$ è chiusa.
\end{prop}

\begin{proof}
  Sia $R \subseteq D$ rettangolo, vogliamo $\displaystyle \int_{\partial R} \omega=0$. Se $R \cap r=\emptyset$, segue da quanto già visto. Se $R \cap r\not=\emptyset$, supponiamo senza perdita di generalità che un lato orizzontale di $r$ giaccia su $r$ (se $r$ taglia $R$ da qualche parte a metà, si separa $R$ in due rettangoli con lati giacenti su $R$: è ovvio che l'integrale lungo il bordo di $R$ è la somma degli integrali lungo i bordi degli integrali ottenuti). Allora possiamo costruire una successione di rettangoli $R_n$ in questo modo: un lato orizzontale coincide con il lato orizzontale di $R$ che non giace su $R$, i lati verticali sono contenuti nei lati verticali di $R$ e il rimanente lato orizzontale si avvicina al crescere di $n$ al lato di $R$ che giace su $r$.
  $R_n$ "tende" a $R$. Allora, usando che $f$ è continua, si vede facilmente che $\displaystyle \int_{\partial R} f(z)\diff z=\lim_{n \longrightarrow +\infty} \int_{\partial R_n} f(z)\diff z=0$ in quanto $\displaystyle \int_{\partial R_n} f(z)\diff z=0$ per ogni $n$ (sempre perché $R_n \subseteq D \setminus r$).
\end{proof}
