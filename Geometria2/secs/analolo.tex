Abbiamo visto che analitica $\implies$ olomorfa. Quanto segue è di interesse indipendente e servirà per dimostrare l'implicazione opposta, cioè
\begin{center}
  olomorfa $\implies$ analitica
\end{center}
(in particolare, l'esistenza della derivata prima complessa implica l'esistenza delle derivate di qualsiasi ordine).

\begin{thm}
  Sia $f:D \longrightarrow \mathbb{C}$ olomorfa. Allora $f(z)\diff z$ è una $1$-forma chiusa su $D$.
\end{thm}

\begin{proof}
  Assumendo $f$ $C^1$, $\omega=f(z)\diff z=f(z)\diff x+if(z)\diff y$ è $C^1$ e $\diff\omega=\left(\dfrac{\partial if}{\partial x}-\dfrac{\partial f}{\partial y}\right)\diff x\diff y=\left(i\dfrac{\partial f}{\partial x}-\dfrac{\partial f}{\partial y}\right)\diff x\diff y$.
  Poiché $f$ è olomorfa, $\dfrac{\partial f}{\partial x}=\diff f(1)=f'$, mentre $\dfrac{\partial f}{\partial y}=\diff f(i)=i\diff f(1)=if'$ $\implies$ $\diff\omega=(if'-if')\diff x\diff y=0$. Adesso facciamo il caso generale.

  Sia $f$ olomorfa, per assurdo $f(z)\diff z$ non è chiusa. Allora esiste un rettangolo $R \subseteq D$ t.c. $\displaystyle \int_{\partial R} \omega=\alpha(R)\not=0$. D'ora in poi, se $\overline{R}$ è un qualsiasi rettangolo, poniamo $\alpha(\overline{R})=\int_{\partial \overline{R}} \omega$. Dividiamo $R$ in quattro rettangoli uguali $A_1, A_2, A_3, A_4$ con i lati paralleli a quelli di $R$ e aventi un vertice in comune nel centro di $R$.
  $\displaystyle \alpha(R)=\int_{\partial R} \omega=\int_{\partial A_1} \omega+\int_{\partial A_2} \omega+\int_{\partial A_3} \omega+\int_{\partial A_4} \omega=\alpha(A_1)+\alpha(A_2)+\alpha(A_3)+\alpha(A_4)$ ed esiste perciò un $A_i$ t.c. $|\alpha(A_i)|\ge \dfrac{|\alpha(R)|}{4}$.
  Poniamo $R_1$ uguale a questo $A_i$. Iterando la costruzione di $R_1$, otteniamo per induzione una successione di rettangoli inscatolati $R \supseteq R_1 \supseteq R_2 \supseteq R_3 \supseteq \dots \supseteq R_n \supseteq \dots$ con $|\alpha(R_n)|\ge \dfrac{|\alpha(R)|}{4^n}$. Ora prendiamo $\displaystyle z_0 \in \bigcap_{n \in \mathbb{N}} R_n \not=\emptyset$ in quanto intersezione decrescente di compatti chiusi non vuoti.
  Poiché $f$ è olomorfa, abbiamo $f(z)=f(z_0)+f'(z_0)(z-z_0)+\epsilon(z)|z-z_0|$, dove $\displaystyle \lim_{z \longrightarrow z_0} |\epsilon(z)|=0$, per cui $\displaystyle \alpha(R_n)=\int_{\partial R_n} (f(z_0)+f'(z_0)(z-z_0)+\epsilon(z)|z-z_0|)\diff z=\int_{\partial R_n} f(z_0)\diff z+\int_{\partial R_n} f'(z_0)(z-z_0)\diff z+\int_{\partial R_n} \epsilon(z)|z-z_0|\diff z$.
  $\diff z$ e $(z-z_0)\diff z$ ammettono primitive $z$ e $(z-z_0)^2/2$, dunque sono esatte, in particolare chiuse, quindi i primi due integrali fanno $0$. Perciò $\displaystyle \alpha(R_n)=\int_{\partial R_n} \epsilon(z)|z-z_0|\diff z$.
  Ora, poiché $\displaystyle \lim_{z \longrightarrow z_0} \epsilon(z)=0$, possamo scegliere $\delta>0$ t.c. $|z-z_0|<\delta \implies |\epsilon(z)|\le \dfrac{|\alpha(R)|}{4(a+b)^2}$, dove $a, b$ sono le lunghezze dei lati di $R$.
  Poiché $z_0 \in R_n$ e $diam(R_n) \le \dfrac{a+b}{2^n}$, basta scegliere $n$ opportuna e avremo che $|\epsilon(z)|\le \dfrac{|\alpha(R)|}{4(a+b)^2}$ per ogni $z \in R_n$.
  Per tale $n$, se $z \in \partial R_n$, $|z-z_0| \le diam(R_n) \le \dfrac{a+b}{2^n}$, per cui $\displaystyle |\alpha(R_n)|=\left|\int_{\partial R_n} \epsilon(z)|z-z_0|\diff z\right|\le\int_0^{\frac{2(a+b)}{2^n}} |\epsilon(\gamma(t))|\cdot|\gamma(t)-z_0|\diff t \le \dfrac{2(a+b)}{2^n}\cdot\dfrac{|\alpha(R)|}{4(a+b)^2}\cdot\dfrac{(a+b)}{2^n}=\dfrac{1}{2}\dfrac{|\alpha(R)|}{4^n}$, il che contraddice $|\alpha(R_n)| \ge \dfrac{|\alpha(R)|}{4^n}$.
\end{proof}

\begin{cor}
  $f:D \longrightarrow \mathbb{C}$ olomorfa $\implies$ per ogni $p \in D$ esistono un aperto $U \subseteq D$, $p \in U$ e $F:U \longrightarrow \mathbb{C}$ olomorfa con $F'=f\restrict{U}$.
\end{cor}

\begin{proof}
  $f(z)\diff z$ chiusa $\implies$ esistono $U$ come nell'enunciato, $F:U \longrightarrow \mathbb{C}$ con  $\diff F=f(z)\diff z$ su $U$, cioè $F$ è olomorfa e $F'=f$ su $U$.
\end{proof}

\begin{cor}
  $f:D \longrightarrow \mathbb{C}$ olomorfa $\implies$ $\displaystyle \int_{\gamma} f(z)\diff z=0$ per ogni $\gamma$ laccio in $D$ omotopicamente banale.
\end{cor}

\begin{prop} \label{retta_orizzontale}
  Sia $f:D \longrightarrow \mathbb{C}$ continua, $D$ aperto, e $f$ olomorfa su $D \setminus r$, $r$ una retta orizzontale. Allora $f(z)\diff z$ è chiusa.
\end{prop}

\begin{proof}
  Sia $R \subseteq D$ rettangolo, vogliamo $\displaystyle \int_{\partial R} \omega=0$. Se $R \cap r=\emptyset$, segue da quanto già visto. Se $R \cap r\not=\emptyset$, supponiamo senza perdita di generalità che un lato orizzontale di $r$ giaccia su $r$ (se $r$ taglia $R$ da qualche parte a metà, si separa $R$ in due rettangoli con lati giacenti su $R$: è ovvio che l'integrale lungo il bordo di $R$ è la somma degli integrali lungo i bordi degli integrali ottenuti). Allora possiamo costruire una successione di rettangoli $R_n$ in questo modo: un lato orizzontale coincide con il lato orizzontale di $R$ che non giace su $R$, i lati verticali sono contenuti nei lati verticali di $R$ e il rimanente lato orizzontale si avvicina al crescere di $n$ al lato di $R$ che giace su $r$.
  $R_n$ "tende" a $R$. Allora, usando che $f$ è continua, si vede facilmente che $\displaystyle \int_{\partial R} f(z)\diff z=\lim_{n \longrightarrow +\infty} \int_{\partial R_n} f(z)\diff z=0$ in quanto $\displaystyle \int_{\partial R_n} f(z)\diff z=0$ per ogni $n$ (sempre perché $R_n \subseteq D \setminus r$).
\end{proof}

Per proseguire nella dimostrazione che le funzioni olomorfe sono analitiche, avremo bisogno di una definizione e di alcune sue proprietà. Sia $a \in \mathbb{C}, x_0=1+a$. Ricordiamo che $\pi_1(\mathbb{C}\setminus\{a\}, x_0) \simeq \mathbb{Z}$. Fissiamo un isomorfismo canonico $f:\pi_1(\mathbb{C}\setminus\{a\}, x_0)\longrightarrow \mathbb{Z}$ che manda $[\gamma]$ in $1$, dove $\gamma$ è una parametrizzazione di $\partial B(a, 1)$ che percorre la circonferenza in senso antiorario.
Ricordiamo anche che $[S^1, \mathbb{C}\setminus\{a\}]$ indica le classi di omotopia (libera) di mappe continue da $S^1$ in $\mathbb{C}\setminus\{a\}$. Abbiamo visto che esiste una bigezione naturale
\begin{align*}
  \Omega(x_0, x_0) &\longrightarrow \Omega(S^1, x_0)\\
  \gamma &\longmapsto \hat{\gamma}
\end{align*}
\begin{center}
  \begin{tikzcd}
    \left[0,1\right] \arrow[r, "\gamma"] \arrow[d, "\pi"] & \mathbb{C}\setminus\{a\}\\
    S^1\simeq\faktor{[0,1]}{\{0,1\}} \arrow[ru, "\hat{\gamma}"]
  \end{tikzcd}
\end{center}
Questa bigezione induce un omomorfismo
\begin{align*}
  \partial:\pi_1(\mathbb{C}\setminus\{a\}, x_0) &\longrightarrow [S^1, \mathbb{C}\setminus\{a\}]\\
  [\gamma] &\longmapsto [\hat{\gamma}]
\end{align*}
Inoltre $\partial$ è suriettiva e $\partial([\gamma])=\partial([\gamma'])$ $\iff$ $[\gamma]$ e $[\gamma']$ sono coniugati in $\pi_1(\mathbb{C}\setminus\{a\}, x_0)$, che però è abeliano, dunque $\partial$ è iniettiva.
Definiamo $\psi:[S^1, \mathbb{C}\setminus\{a\}] \longrightarrow \mathbb{Z}$ come $\psi=\partial^{-1}\circ f$.

\begin{defn}
  Sia $\gamma:[0,1] \longrightarrow \mathbb{C}\setminus\{a\}$ un cammino chiuso in $x_0$. Chiamiamo \textsc{indice di $\gamma$ rispetto ad $a$} l'intero $\psi([\hat{\gamma}])=:I(\gamma, a)$ dove $\hat{\gamma}:S^1 \longrightarrow \mathbb{C}\setminus\{a\}$ è la mappa indotta da $\gamma$.
\end{defn}

\begin{thm}
  Sia $\gamma:[0,1] \longrightarrow \mathbb{C}\setminus\{a\}$ un cammino chiuso. Allora $\displaystyle I(\gamma, a)=\frac{1}{2\pi i}\int_{\gamma} \frac{1}{z-a}\diff z$.
\end{thm}

\begin{proof}
  Abbiamo già visto che $\dfrac{1}{z-a}$ è olomorfa in $\mathbb{C}\setminus\{a\}$ $\implies$ $\omega=\dfrac{1}{z-a}\diff z$ è chiusa $\implies$ possiamo integrare $\omega$ lungo curve $\gamma$ continue. Inoltre, abbiamo già visto che, se $\gamma'$ è liberamente omotopo a $\gamma$, $\displaystyle \int_{\gamma}\omega=\int_{\gamma'}\omega$. Quindi la seguente mappa è ben definita:
  \begin{align*}
    \varphi:[S^1, \mathbb{C}\setminus\{a\}] &\longrightarrow \mathbb{C}\\
    [\hat{\gamma}] &\longmapsto \frac{1}{2\pi i}\int_{\gamma}\omega
  \end{align*}
  Ci resta da dimostrare $\varphi=\psi$, cioè $\varphi([\hat{\gamma}])=\psi([\hat{\gamma}])$ per ogni $[\hat{\gamma}] \in [S^1, \mathbb{C}\setminus\{a\}]$.
  Notiamo che $[S^1, \mathbb{C}\setminus\{a\}]=\{[\hat{\gamma}_n] \mid \gamma_n:[0,1]\longrightarrow \mathbb{C}\setminus\{a\}, \gamma_n(t)=a+e^{2\pi i t}\} \implies \psi([\hat{\gamma}_n])=n$.
  Abbiamo già visto d'altro canto che $\displaystyle \int_{\gamma_n} \frac{1}{z-a}\diff z=2\pi i n$, da cui la tesi.
\end{proof}

\begin{oss}
  Per definizione, $\omega=\dfrac{1}{z-a}\diff z$ è chiusa, cioè localmente esatta. Le primitive locali di $\omega$ sono date dalle branche di $\log(z-a)$. Una primitiva di $\omega$ lungo $\gamma$ è una funzione continua $f:[0,1]\longrightarrow \mathbb{C}$ t.c. $e^{f(t)}=\gamma(t)-a \implies I(\gamma, a)=\dfrac{1}{2\pi i}(f(1)-f(0))$.
\end{oss}

Elenchiamo ora alcune proprietà dell'indice.

\begin{oss}
  Siano $\gamma, \gamma'$ due curve $C^0$ in $\mathbb{C}\setminus\{a\}$, $\gamma$ liberamente omotopo a $\gamma'$: Allora $I(\gamma, a)=I(\gamma',a)$.
\end{oss}

\begin{prop}
  Sia $\gamma:[0,1]\longrightarrow \mathbb{C}$ una curva chiusa $C^0$. La funzione
  \begin{align*}
    \mathbb{C}\setminus\Ima{\gamma} &\longrightarrow \mathbb{Z}\\
    z &\longmapsto I(\gamma,z)
  \end{align*}
  è continua e localmente costante (in particolare, è costante su ogni componente connessa di $\mathbb{C}\setminus\Ima{\gamma}$).
\end{prop}

\begin{proof}
  Fissiamo $a \in \mathbb{C}\setminus\Ima{\gamma}$. Mostreremo che per ogni $h \in \mathbb{C}$, $|h|$ sufficientemente piccolo, $I(\gamma, a+h)=I(\gamma, a)$. Sia $0<\delta_0<\min\{|\gamma(t)-a| \mid t \in [0,1]\}\not=0$ perché $a \not \in \mathbb{C}\setminus\Ima{\gamma}$.
  Per ogni $h \in \mathbb{C}$ t.c. $|h|<\delta_0$ abbiamo: $\displaystyle I(\gamma, a+h)=\frac{1{2\pi i}}\int_{\gamma} \frac{1}{z-(a+h)}\diff z=\frac{1}{2\pi i} \int_{\gamma} \frac{\diff z}{(z-h)-a}$. Facciamo un cambio di variabile: $z'=z-h$.
  Otteniamo che $\displaystyle I(\gamma, a+h)=\frac{1}{2\pi i}\int_{\gamma'} \frac{1}{z'-a}\diff z'$ dove $\gamma'(t)=\gamma(t)-h$ per $t \in [0,1]$.
  Per concludere, mostreremo che è uguale a $\displaystyle \frac{1}{2\pi i}\int_{\gamma} \frac{1}{z-a}\diff z$ perché $\gamma$ e $\gamma'$ sono liberamente omotopi. $\gamma \sim \gamma'$ tramite l'omotopia $F(t,s)=\gamma(t)-sh$ per ogni $t,s \in [0,1]$ $\implies$ $I(\gamma, a+h)=I(\gamma', a)=I(\gamma, a)$.
\end{proof}

\begin{prop}
  Sia $a \in \mathbb{C}$. Sia $\gamma$ una curva chiusa $C^0$ t.c. $\Ima{\gamma} \subset D \subset D \subset \mathbb{C}\setminus\{a\}$, $D$ è un aperto semplicemente connesso. Allora $I(\gamma, a)=0$.
\end{prop}

\begin{proof}
  $\omega=\dfrac{1}{z-a}\diff z$ è chiusa in $D$ $\implies$ $\omega$ è esatta $\implies$ $\displaystyle \int_{\gamma} \omega=0$ $\implies$ $I(\gamma, a)=0$.
\end{proof}

\begin{exc}
  Sia $\gamma$ una curva chiusa $C^0$, $a \in \mathbb{C}\setminus\Ima{\gamma}$. Allora $I(\gamma, a)=0$ per tutti gli $a$ in una componente connessa illimitata di $\mathbb{C}\setminus\Ima{\gamma}$.
\end{exc}

\begin{ex}
  Sia $\gamma:t \longmapsto Re^{it}$ con $R>0$ e $t \in [0, 2\pi]$. Per $|z|<R$, $I(\gamma, z)=1$. Per $|z|>R$, $I(\gamma, z)=0$.
\end{ex}

\begin{prop}
  Sia $f:\{z \in \mathbb{C} \mid |z| \le R\} \longrightarrow \mathbb{C}$ una mappa continua, sia $\gamma(t)=f(Re^{2\pi it})$ per $t \in [0,1]$. Se $a \not\in \Ima(\gamma)$ e $I(\gamma, a)\not=0$ allora esiste $z \in \mathbb{C}, |z|<R$ t.c. $f(z)=a$.
\end{prop}

\begin{proof}
  Per assurdom assumiamo che $f(z)\not=a$ per ogni $|z|<R$. Allora $f(z)\not=a$ per ogni $|z| \le R$ visto che $a \not\in \Ima{\gamma}$. Definiamo $F(t,s)=f(t, Re^{2\pi is})$ per ogni $t,s \in [0,1]$. $F$ ci dà un'omotopia tra $\gamma$ e un cammino costante in $f(0)$.
  Infatti $F$ è continua, $F(1,s)=\gamma(s)$, $F(0,s)=f(0)$, $F(t,0)=F(t,1)$ e $F(t, s) \in \mathbb{C}\setminus\{a\}$ per ogni $t, s$. $\gamma$ è dunque omotopo al cammino costante in $f(0)$ $\implies$ $\displaystyle \int_{\gamma} \frac{1}{z-a}\diff z=0$, ma $I(\gamma, a)\not=0$, assurdo.
\end{proof}

\begin{defn}
  Siano $\gamma_1, \gamma_2$ due curve $C^0$. Allora Poniamo
  \begin{align*}
    \gamma_1\gamma_2:t &\longmapsto \gamma_1(t)\gamma_2(t)\\
    \gamma_1+\gamma_2:t &\longmapsto \gamma_1(t)+\gamma_2(t)
  \end{align*}
\end{defn}

\begin{thm}
  Siano $\gamma_1, \gamma_2$ due curve $C^0$ chiuse t.c. $0 \not\in \Ima{\gamma_1}\cup\Ima{\gamma_2}$. Allora $I(\gamma_1\gamma_2, 0)=I(\gamma_1, 0)+I(\gamma_2, 0)$.
\end{thm}

\begin{proof}
  Sia $\omega=\dfrac{\diff z}{z}$ e $\gamma_i:[0,1] \longrightarrow \mathbb{C}\setminus\{0\}$, $i=1,2$. Sia $f_i:[0,1] \longrightarrow \mathbb{C}$ continua t.c. $e^{f_i(t)}=\gamma_i(t)$ per ogni $t \in [0,1], i=1,2$.
  Allora $\gamma_1(t)\gamma_2(t)=e^{f_1(t)}e^{f_2(t)}=e^{f_1(t)+f_2(t)}$ $\implies$ $f=f_1+f_2$ è una primitiva di $\omega$ lungo $\gamma_1\gamma_2$. $I(\gamma_1, \gamma_2)=\dfrac{f_1(1)+f_2(1)-f_1(0)-f_2(0)}{2\pi i}=\dfrac{f_1(1)-f_1(0)}{2\pi i}+\dfrac{f_2(1)-f_2(0)}{2\pi i}=I(\gamma_1, 0)+I(\gamma_2, 0)$.
\end{proof}

\begin{thm}
  Siano $\gamma_1, \gamma$ curve chiuse $C^0$ t.c. $0 \not\in \Ima{\gamma_1}\cup\Ima{\gamma}$. Assumiamo che $0<|\gamma_1(t)|<|\gamma(t)|$ per ogni $t \in [0,1]$. Allora $I(\gamma_1+\gamma_2, 0)=I(\gamma, 0)$.
\end{thm}

\begin{proof}
  $\gamma(t)+\gamma_1(t)=\gamma(t)\left(1+\dfrac{\gamma_1(t)}{\gamma(t)}\right)=\gamma(t)\beta(t)$. $I(\gamma_1+\gamma, 0)=I(\gamma\beta, 0)=I(\gamma, 0)+I(\beta, 0)$.
  Notiamo che $I(\beta, 0)=0$, infatti $|\beta(t)-1|=\left|1+\dfrac{\gamma_1(t)}{\gamma(t)}-1 \right|=\left|\dfrac{\gamma_1(t)}{\gamma(t)} \right|<1$ $\implies$ $\Ima{\beta} \subset D(1,1)$, un aperto semplicemente connesso che non contiene $0$, dunque $I(\beta, 0)=0$.
\end{proof}

\begin{thm}
  (Formula integrale di Cauchy) Sia $D \subset \mathbb{C}$ un aperto, $a \in D$, sia $\gamma:[0,1] \longrightarrow D$ un cammino chiuso omotopicamente banale, con $a \not\in \Ima{\gamma}$. Sia $f:D \longrightarrow \mathbb{C}$ olomorfa. Allora $\displaystyle \frac{1}{2\pi i}\int_{\gamma} \frac{f(z)}{z-a}\diff z=I(\gamma, a)f(a)$.
\end{thm}

\begin{proof}
  Per ogni $z \in D$ sia $g(z)=\begin{cases}
    \dfrac{f(z)-f(a)}{z-a} & \mbox{se }z\not=a\\
    f'(a) & \mbox{se }z=a
\end{cases}$. $g$ è continua in $D$ ed è olomorfa in $D \setminus r$ dove $r$ è la retta orizzontale che passa per $a$.
Per la proposizione \ref{retta_orizzontale}, $g(z)\diff z$ è chiusa in $D$. Visto che $\gamma$ è omotopicamente banale in $D$ e $g(z)\diff z$ è chiusa in $D$, abbiamo che $\displaystyle \int_{\gamma}g(z)\diff z=0$.
$\displaystyle 0=\int_{\gamma} g(z)\diff z=\int_{\gamma} \frac{f(z)-f(a)}{z-a}\diff z=\int_{\gamma} \frac{f(z)}{z-a}\diff z-f(a)\int_{\gamma} \frac{1}{z-a}\diff z$. Dividendo per $2\pi i$ otteniamo la tesi.
\end{proof}

\begin{ex}
  Sia $R>0$ t.c. $B(a,R)=\{z\in \mathbb{C}\mid |z-a|\le R\} \subset D$ e sia $\gamma$ una parametrizzazione di $\partial B(a,R)$ in senso antiorario. Applicando la formula integrale di Cauchy otteniamo che $\displaystyle \frac{1}{2\pi i}\int_{\gamma} \frac{f(z)}{z-a}\diff z=f(a)$ ($I(\gamma,a)=1$) per ogni $f:D \longrightarrow \mathbb{C}$ olomorfa.
\end{ex}

\begin{oss}
  La formula integrale di Cauchy implica il teorema di Cauchy. Infatti, se $f$ è olomorfa su un aperto $D\subset \mathbb{C}$, vogliamo $\omega=f(z)\diff z$ chiusa in $D$. Sia $\gamma$ omotopicamente banale in $D$ e $a \not\in \Ima{\gamma}$. Sia $F(z)=(z-a)f(z)$.
  Applichiamo la formula integrale di Cauchy a $F(z)$: $\displaystyle 0=F(a)I(\gamma, a)=\frac{1}{2\pi i}\int_{\gamma} \frac{F(z)}{z-a}\diff z=\frac{1}{2\pi i}\int_{\gamma}f(z)\diff z$.
  Abbiamo $\displaystyle\int_{\gamma}\omega=0$ per ogni $\gamma$ omotopicamente banale con $\Ima{\gamma} \subset D$, dunque per il teorema \ref{chiusa^int=0} $\omega$ è chiusa.
\end{oss}

\begin{thm}
  (Analitica $\implies$ olomorfa) Sia $f$ olomorfa nel disco aperto $\{z \in \mathbb{C} \mid |z|<R\}$ per $0<R\le+\infty$. Allora $f$ è analitica in $\{z \in \mathbb{C} \mid |z|<R\}$. In particolare, esiste una serie di potenze $\displaystyle \sum_{n \ge 0} a_nz^n$ con raggio di convergenza $\rho \ge R$ t.c. $\displaystyle f(z)=\sum_{n \ge 0} a_nz^n$ per $|z|<R$.
\end{thm}

\begin{proof}
  Sia $0<r_0<R$. Definiamo $\gamma:t \longmapsto r_0e^{2\pi i t}$ per ogni $t \in [0,1]$. Sia $z \in \mathbb{C}$ t.c. $|z|<r_0$ (per notazione, $|z|=r$). Allora $I(\gamma,z)=1$ e la formula integrale di Cauchy ci dice che $\displaystyle f(z)=\dfrac{1}{2\pi i}\int_{\gamma}\frac{f(w)}{w-z}\diff w$.
  Notiamo che $\displaystyle \frac{1}{w-z}-\frac{1}{w}\frac{\left(\frac{z}{w}\right)^{n+1}}{1-\frac{z}{w}}=\frac{1}{w}\frac{1-\left(\frac{z}{w}\right)^{n+1}}{1-\frac{z}{w}}=\frac{1}{w}\sum_{k=0}^n \left(\frac{z}{w}\right)^k$.
  $\displaystyle \frac{1}{w-z}=\frac{1}{w}\sum_{k=0}^n \left(\frac{z}{w}\right)^k+\frac{1}{w}\frac{\left(\frac{z}{w}\right)^{n+1}}{1-\frac{z}{w}}$.
  Sostituiamolo dentro la formula con l'integrale per $f(z)$ per ottenere $\displaystyle f(z)=\frac{1}{2\pi i}\left[\int_{\gamma}\frac{f(w)}{w}\sum_{k=0}^n \left(\frac{z}{w}\right)^k\diff w+\int_{\gamma}\frac{f(w)}{w}\frac{\left(\frac{z}{w}\right)^{n+1}}{1-\frac{z}{w}}\diff w\right]=\frac{1}{2\pi i}\left[\sum_{k=0}^n z^k\left(\int_{\gamma} \frac{f(w)}{w^{k+1}}\diff w\right)+\int_{\gamma}\frac{f(w)}{w}\frac{\left(\frac{z}{w}\right)^{n+1}}{1-\frac{z}{w}}\diff w\right]$.
  Definiamo $\displaystyle a_k=\frac{1}{2\pi i}\int_{\gamma} \frac{f(w)}{w^{k+1}}\diff w$, $\displaystyle R_n=\frac{1}{2\pi i}\int_{\gamma}\frac{f(w)}{w}\frac{\left(\frac{z}{w}\right)^{n+1}}{1-\frac{z}{w}}\diff w$.
  Abbiamo ottenuto $\displaystyle f(z)=\sum_{k=0}^n a_kz^k+R_n$. Definiamo $M(r_0)=\sup\{|f(\gamma(t))| \mid t \in [0,1]\}$.
  $\displaystyle R_n=\frac{1}{2\pi i}\int_{\gamma}\frac{f(w)}{w}\frac{\left(\frac{z}{w}\right)^{n+1}}{1-\frac{z}{w}}\diff w=\frac{1}{2\pi i}\int_0^1 \frac{f(\gamma(t))}{\gamma(t)}\frac{\left(\frac{z}{\gamma(t)}\right)^{n+1}}{1-\frac{z}{\gamma(t)}}r_02\pi ie^{2\pi it}\diff t$.
  $\displaystyle |R_n| \le \frac{1}{2\pi}\int_0^1 \frac{|f(\gamma(t))|}{|\gamma(t)|}\frac{\left|\frac{z}{\gamma(t)}\right|^{n+1}}{\left|1-\frac{z}{\gamma(t)}\right|}r_02\pi \diff t$.
  $|z|=r<r_0=|\gamma(t)| \implies \left|1-\dfrac{z}{\gamma(t)}\right| \ge 1-\dfrac{r}{r_0}$.
  $\displaystyle |R_n| \le \frac{1}{2\pi} \int_0^1 \frac{|f(\gamma(t))|}{r_0}\frac{\left(\frac{r}{r_0}\right)^{n+1}}{1-\frac{r}{r_0}}r_02\pi\diff t \le M(r_0) \frac{\left(\frac{r}{r_0}\right)^{n+1}}{1-\frac{r}{r_0}}$.
  Quindi, poiché $r/r_0<1$, $|R_n| \longrightarrow 0$ per $n \longrightarrow +\infty$ $\implies$ $\displaystyle f(z)=\sum_{k \ge 0}a_kz^k$ per $|z|<R$.
  Osserviamo che $\displaystyle a_k=\frac{1}{2\pi i}\int_{\gamma} \frac{f(w)}{w^{k+1}}\diff w$ è indipendente da $r_0$ per ogni $0<r_0<R$. Usando stime analoghe alle precedenti otteniamo $\displaystyle |a_k| \le \frac{1}{2\pi} \int_0^1 \frac{|f(\gamma(t))|}{|\gamma(t)|^{k+1}}r_02\pi \diff t \le \frac{M(r_0)}{r_0^k}$.
  $$ |a_k| \le \frac{M(r_0)}{r_0^k} \, (\star)$$
  Per la proposizione \ref{convergenza_serie} $\displaystyle \sum_{k \ge 0} a_kz^k$ converge uniformemente per $|z| \le r$ per ogni $r<r_0$ e converge assolutamente per $|z|<r_0$ per ogni $0<r_0<R$ $\implies$ il raggio di convergenza della serie è $\rho \ge R$.
\end{proof}

\begin{oss}
  $\displaystyle f(z)=\sum_{k \ge 0} a_kz^k \implies 2\pi if(z)=\sum_{k \ge 0}\left(\int_{\gamma} \frac{f(w)}{w^{k+1}}\diff w\right)z^k=\int_{\gamma}f(w)\left(\sum_{k \ge 0}\frac{z^k}{w^{k+1}}\right)\diff w$.
\end{oss}

\begin{cor}
  Sia $D \subset \mathbb{C}$ un aperto. Una funzione $f$ è olomorfa in $D$ $\iff$ $f$ è analitica in $D$. In particolare, la serie di Taylor di $f$ in $a \in D$ ha raggio di convergenza $\rho \ge \sup\{r>0 \mid \{z \in \mathbb{C} \mid |z-a|<r\}\subseteq D\}$.
\end{cor}

\begin{proof}
  Segue da quanto visto nella dimostrazione del teorema precedente.
\end{proof}

\begin{cor} \label{derivate_olo}
  Se $f$ è olomorfa su $D$ aperto, allora $f$ è $C^{\infty}$ in $D$ e per ogni $n \ge 0$ $f^{(n)}$ è olomorfa in $D$.
\end{cor}

\begin{proof}
  $f$ olomorfa $\implies$ $f$ analitica $\implies$ $f \in C^{\infty}$ e $f^{(n)}$ analitica $\implies$ $f^{(n)}$ olomorfa.
\end{proof}

\begin{cor}
  Sia $f$ olomorfa in $D$ aperto. Sia $R>0$ t.c. $B(a, R) \subset D$. Sia $\gamma$ la curva che parametrizza $\partial B(a, R)$ in senso antiorario. Allora $\displaystyle f^{(n)}(a)=\frac{n!}{2\pi i}\int_{\gamma}\frac{f(z)}{(z-a)^{n+1}}\diff z$.
\end{cor}

\begin{proof}
  $\displaystyle \frac{f^{(n)}(a)}{n!}=a_n=\frac{1}{2\pi i}\int_{\gamma}\frac{f(z)}{(z-a)^{n+1}}\diff z$.
\end{proof}

\begin{thm}
  (Teorema di Morera) Sia $f$ una funzione continua in un aperto $D$. Se $\omega=f(z)\diff z$ è chiusa in $D$, allora $f$ è olomorfa in $D$.
\end{thm}

\begin{proof}
  $\omega$ è chiusa in $D$. Per definizione, per ogni $a \in D$ esiste un intorno aperto $U$ di $a$ t.c. $\omega=\diff F$ in $U$, $F$ di classe $C^1$ in $U$.
  $f(z)\diff z=\omega=\diff F=\dfrac{\partial F}{\partial z}\diff z+\dfrac{\partial F}{\partial\bar{z}}\diff\bar{z} \iff f(z)=\dfrac{\partial F}{\partial z}, \dfrac{\partial F}{\partial\bar{z}}=0$.
  $\dfrac{\partial F}{\partial\bar{z}}=0$ $\iff$ $F$ soddisfa le condizioni di Cauchy-Riemann $\implies$ $F$ olomorfa $\implies$ $F'$ olomorfa per il corollario \ref{derivate_olo}, ma $F'=f$.
\end{proof}

\begin{cor}
  Sia $f:D \longrightarrow \mathbb{C}$ continua, $D$ aperto. De $f$ è olomorfa in $D \setminus r$, $r$ retta orizzontale, allora $f$ è olomorfa in $D$.
\end{cor}

\begin{proof}
  Segue dal teorema di Morera e dalla proposizione \ref{retta_orizzontale}.
\end{proof}

\begin{thm}
  Sia $f:D \longrightarrow \mathbb{C}$ una funzione, $D \subset \mathbb{C}$ un aperto. Le seguenti sono equivalenti:
  \begin{nlist}
    \item $f$ è olomorfa in $D$;
    \item $f$ è analitica in $D$;
    \item $f \in C^1$ e soddisfa le condizioni di Cauchy-Riemann in $D$;
    \item $f$ è continua in $D$ e olomorfa in $D\setminus r$, $r$ retta orizzontale;
    \item $f$ è continua e $\omega=f(z)\diff z$ è chiusa in $D$.
  \end{nlist}
\end{thm}

\begin{proof}
  Segue da quanto visto finora.
\end{proof}

\begin{cor}
  (Disuguaglianze di Cauchy) Sia $a \in \mathbb{C}$, $\displaystyle f(z)=\sum_{n \ge 0} a_n(z-a)^n$ con raggio di convergena $\rho>0$. Allora per ogni $0 <r<\rho$ $a_n=\dfrac{f^{(n)}(a)}{n!}$ e $|a_n| \le M(r)$ dove $M(r)=\sup\{|f(z)| \mid |z-a|=r\}$.
\end{cor}

\begin{proof}
  Vista nelle dimostrazioni sopra.
\end{proof}
