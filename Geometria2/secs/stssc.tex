\begin{defn}
  Uno spazio topologico $X$ localmente connesso per archi si dice \textsc{semilocalmente semplicemente connesso} se per ogni $x \in X$ esiste $U \subseteq X$ aperto con $x \in U$ t.c. l'inclusione $i:U \rightarrow X$ induce il morfismo banale $i_{\star}:\pi_1(U, x) \rightarrow \pi_1(X, x)$. Ovvero, per ogni $x \in X$ esiste un intorno $U$ di $x$ t.c. tutti i lacci basati in $x$ e contenuti in $U$ sono banali in $X$.
\end{defn}

Questa proprietà è verificata ad esempio se ogni punto ha un intorno semplicemente connesso (per esempio le varietà).

\begin{oss}
  Se $X$ ammette un rivestimento universale, allora è semilocalmente semplicemente connesso. Infatti, dato $x \in X$, ne possiamo prendere un intorno connesso per archi e ben rivestito $U$. Se $p:E \rightarrow X$ è il rivestimento universale e $V \subseteq p^{-1}(U)$ è un aperto con $p\restrict{V}:V \rightarrow U$ omeomorfismo,
  \begin{center}
    \begin{tikzcd}
      V \arrow[r, hook, "j"] \arrow[d, shift left, swap, "p\restrict{V}" right] & E \arrow[d, "p"]\\
      U \arrow[r, hook, "i"] \arrow[u, shift left, "s"] & X
    \end{tikzcd}
  \end{center}
  $i=p \circ j \circ s$ $\implies$ $i_{\star}=p_{\star} \circ j_{\star} \circ s_{\star}$, ma $j_{\star}$ è banale perché $\pi_1(E) \cong \{1\}$, dunque $i_{\star}:\pi_1(U, x) \rightarrow \pi_1(X, x)$ è banale.
\end{oss}

\begin{lm}
  Quando esiste, il rivestimento universale è unico a meno di isomorfismo.
\end{lm}
