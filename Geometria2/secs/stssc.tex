\begin{defn}
  Uno spazio topologico $X$ localmente connesso per archi si dice \textsc{semilocalmente semplicemente connesso} se per ogni $x \in X$ esiste $U \subseteq X$ aperto con $x \in U$ t.c. l'inclusione $i:U \rightarrow X$ induce il morfismo banale $i_{\star}:\pi_1(U, x) \rightarrow \pi_1(X, x)$. Ovvero, per ogni $x \in X$ esiste un intorno $U$ di $x$ t.c. tutti i lacci basati in $x$ e contenuti in $U$ sono banali in $X$.
\end{defn}

Questa proprietà è verificata ad esempio se ogni punto ha un intorno semplicemente connesso (per esempio le varietà).

\begin{oss}
  Se $X$ ammette un rivestimento universale, allora è semilocalmente semplicemente connesso.
\end{oss}
