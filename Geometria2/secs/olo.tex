Adesso studiamo le funzioni complesse.

Sia $U \subset \mathbb{C}$ aperto, $f:U \longrightarrow \mathbb{C}$. Possiamo scrivere $f(z)=u(z)+iv(z)$ per ogni $z \in U$. Chiamiamo $u(z)$ la parte reale di $f$, $u:U \longrightarrow \mathbb{R}$, $v(z)$ la parte immaginaria di $f$, $v:U \longrightarrow \mathbb{R}$.

\begin{ftt}
  $f:U \longrightarrow \mathbb{C}$ è continua in $z_0 \in U$ (rispettivamente in tutto $U$) se e solo se $u$ e $v$ sono continue in $z_0 \in U$ (rispettivamente in tutto $U$).
\end{ftt}

Importante: dal punto di vista della continuità le funzioni di $U$ a valori complessi possono essere semplicemente viste come funzioni da $U \subseteq \mathbb{R}^2$ a valori reali. Domanda: vale la stessa cosa quando parliamo di derivabilità?

\begin{defn}
  (Differenziabilità) La funzione $f:U \longrightarrow \mathbb{C}$ è \textit{differenziabile in $z_0 \in U$} se esiste $A:\mathbb{C} \longrightarrow \mathbb{C}$ applicazione $\mathbb{R}$-lineare t.c.
  \begin{nlist}
    \item $f(z)=f(z_0)+A(z-z_0)+r(z)$;
    \item $\dfrac{|r(z)|}{|z-z_0|} \xrightarrow[]{z \longrightarrow z_0} 0$.
  \end{nlist}
  In questo caso, chiamiamo l'applicazione $A$ \textit{jacobiano di $f$}, $A=: Jf_{z_0}$.
\end{defn}

\begin{oss}
  $f$ differenziabile in $z_0$ $\implies$ $f$ continua in $z_0$.
\end{oss}

Fissiamo la base $\{1, i\}$ di $\mathbb{C}$ come $\mathbb{R}$-spazio vettoriale.
\begin{align*}
  f:U &\longrightarrow \mathbb{C}\\
  z=x+iy &\longmapsto u(x, y)+iv(x, y)
\end{align*}

\begin{ftt}
  \begin{nlist}
    \item Se $f$ è differenziabile in $x_0+iy_0=z_0 \in U$, esistono le derivate parziali di $f$ in $z_0$ e $Jf_{z_0}=\begin{pmatrix}
      \dfrac{\partial u}{\partial x}(x_0, y_0) & \dfrac{\partial u}{\partial y}(x_0, y_0)\\
      \dfrac{\partial v}{\partial x}(x_0, y_0) & \dfrac{\partial v}{\partial y}(x_0, y_0)
  \end{pmatrix}$;
  \item (teorema del differenziale totale): se esistono $\dfrac{\partial u}{\partial x}, \dfrac{\partial u}{\partial y}, \dfrac{\partial v}{\partial x}, \dfrac{\partial v}{\partial y}$ in un intorno di $z_0$ e sono continue in $z_0$, allora $f$ è differenziabile in $z_0$.
  \end{nlist}
\end{ftt}

\begin{exc}
  \begin{align*}
    f:\mathbb{R}^2 &\longrightarrow \mathbb{R}\\
    (x, y) &\longmapsto \begin{cases} x & \mbox{se }y\not=x^2 \\ 0 & \mbox{se }y=x^2 \end{cases}
  \end{align*}
  è differenziabile in $(0, 0)$?
\end{exc}

\begin{defn}
  (Funzione olomorfa) Siano $U \subseteq \mathbb{C}$ aperto, $f:U \longrightarrow \mathbb{C}$ continua. Diciamo che $f$ è \textsc{olomorfa in $z_0 \in \mathbb{C}$} se esiste $\displaystyle \lim_{\substack{h \longrightarrow 0 \\ h\not=0}} \frac{f(z_0+h)-f(z_0)}{h}$. Se tale limite esiste, lo chiamiamo derivata di $f$ in $z_0$ e lo denotiamo con $f'(z_0)$.
  Diciamo che $f$ è \textsc{olomorfa in $U$} se è olomorfa per ogni $z_0 \in U$. Diciamo che $f$ è \textsc{intera} se è definita su tutto $\mathbb{C}$ ed è olomorfa in $\mathbb{C}$.
\end{defn}

\begin{prop}
  Sia $U \subset \mathbb{C}$ aperto, $f, g:U \longrightarrow \mathbb{C}$ olomorfe in $z_0 \in U$.
  \begin{nlist}
    \item $f+g$ è olomorfa in $z_0$ e $(f+g)'(z_0)=f'(z_0)+g'(z_0)$;
    \item $f \cdot g$ è olomorfa in $z_0$ e $(f \cdot g)'(z_0)=f'(z_0)g(z_0)+f(z_0)g'(z_0)$;
    \item se $g(z_0) \not=0$, $\dfrac{f}{g}$ è olomorfa in $z_0$ e $\left(\dfrac{f}{g}\right)9(z_0)=\dfrac{f'(z_0)g(z_0)-f(z_0)g'(z_0)}{g(z_0)^2}$.
  \end{nlist}
\end{prop}

\begin{prop}
  Siano $U, V$ aperti di $\mathbb{C}$, $f:U \longrightarrow V$, $g:V \longrightarrow \mathbb{C}$, $f$ olomorfa in $z_0 \in U$ è $g$ olomorfa in $f(z_0)$. Allora $g \circ f$ è olomorfa in $z_0$ e $(g \circ f)'(z_0)=g'(f(z_0))\cdot f'(z_0)$.
\end{prop}

\begin{ex}
  \begin{nlist}
    \item $f(z)=z$ è una funzione intera. Infatti, $\dfrac{f(z+h)-f(z)}{h}=\dfrac{z+h-z}{h}=1$ $\implies$ $f$ è olomorfa in $z \in \mathbb{C}$ e $f'(z)=1$;
    \item $f(z)=z^n, n \ge 1$ è intera.
    Infatti, $\displaystyle \frac{f(z+h)-f(z)}{h}=\frac{(z+h)^n-z^n}{h}=\frac{\sum_{k=0}^n \binom{n}{k}z^kh^{n-k}-z^n}{h}=\sum_{k=0}^{n-1} \binom{n}{k}z^kh^{n-k-1} \xrightarrow[\substack{h \longrightarrow 0 \\ h\not=0}]{} \binom{n}{n-1} z^{n-1}=$\\
    $=nz^{n-1}=:f'(z)$.
  \end{nlist}
\end{ex}

\begin{exc}
  Provare che $f(z)=z^n, n<0$ è olomorfa in $\mathbb{C}\setminus\{0\}$.
\end{exc}

\begin{oss}
  I polinomi sono funzioni intere. Le funzioni razionali $\dfrac{P(z)}{Q(z)}$ con $P$ e $Q$ polinomi sono olomorfe in $\mathbb{C} \setminus \{z \in \mathbb{C} \mid Q(z)=0\}$.
\end{oss}

\begin{ex}
  (Una funzione non olomorfa) $f(z)=\bar{z}$ non è olomorfa. $\dfrac{f(z+h)-f(z)}{h}=\dfrac{\bar{z}+\bar{h}-\bar{z}}{h}=\dfrac{\bar{h}}{h}$ per $h \in \mathbb{C}, h \not=0$. Se esiste $f'(z)$, allora il limite dovrebbe esistere per ogni possibile direzione per cui $h \longrightarrow 0, h\not=0$. Se esiste $f'(z)$, dobbiamo per esempio avere che
  \begin{align*}
    1=\lim_{\substack{h \longrightarrow 0 \\ h\not=0 \\ \mathfrak{Im}(h)=0}} \frac{\bar{h}}{h}=\lim_{\substack{h \longrightarrow 0 \\ h\not=0 \\ \mathfrak{Im}(h)=0}} \frac{f(z+h)-f(z)}{h}=\\
    =\lim_{\substack{h \longrightarrow 0 \\ h\not=0 \\ \mathfrak{Re}(h)=0}} \frac{f(z+h)-f(z)}{h}=\lim_{\substack{h \longrightarrow 0 \\ h\not=0 \\ \mathfrak{Re}(h)=0}} \frac{\bar{h}}{h}=-1,
  \end{align*}
  assurdo $\implies$ non può esistere $f'(z)$.
\end{ex}

Domanda: qual è il legame tra l'olomorficità di $f:U \longrightarrow \mathbb{C}$ e la differenziabilità di $f$ vista come funzione da $U \subset \mathbb{R}^2$ in $\mathbb{R}^2$?

\begin{oss}
  Come nel caso della differenziabilità in ambito reale, $f$ olomorfa in $z_0$ $\implies$ $f$ continua in $z_0$. Infatti, $f(z)-f(z_0)=\dfrac{f(z)-f(z_0)}{z-z_0}(z-z_0)$; al limite per $z \longrightarrow z_0$, abbiamo che tende  $\displaystyle f'(z_0) \cdot \lim_{z \longrightarrow z_0} (z-z_0)=f'(z_0) \cdot 0=0$.
\end{oss}

\begin{ex}
  $g(z)=\bar{z}$ non è olomorfa ma nella base $\{1, i\}$ di $\mathbb{C}$
  \begin{align*}
    g: \mathbb{R}^2 &\longrightarrow \mathbb{R}^2\\
    (x, y) &\longmapsto (x, -y)
  \end{align*}
   è differenziabile.
\end{ex}

\begin{thm} \label{car_olo}
  Sia $U \subset \mathbb{C}$ un aperto e $f:U \longrightarrow \mathbb{C}$ continua. $f$ è olomorfa in $z_0 \in U$ $\iff$ le seguenti due condizioni sono soddisfatte:
  \begin{nlist}
    \item $f$ è differenziabile in $z_0$;
    \item $Jf_{z_0}:\mathbb{C} \longrightarrow \mathbb{C}$ corrisponde alla moltiplicazione per $a \in \mathbb{C}$.
  \end{nlist}
\end{thm}

\begin{proof}
  ($\implies$) \\ L'olomorficità implica che $f(z_0+h)=f(z_0)+f'(z_0)h+r(h) \, (\star)$ con \\ $\dfrac{|r(h)|}{|h|} \xrightarrow[h \longrightarrow 0, h \not=0]{} 0$.
  Nella base $\{1, i\}$ di $\mathbb{C}$, possiamo riscrivere l'equazione $(\star)$ come ($z_0=x_0+iy_0, h=\alpha+i\beta$) $f(x_0+\alpha, y_0+\beta)=f(x_0, y_0)+f'(z_0)(\alpha+i\beta)+r(\alpha, \beta)$ con $\dfrac{|r(\alpha, \beta)|}{\sqrt{\alpha^2+\beta^2}} \longrightarrow 0$.
  La mappa
  \begin{align*}
    \mathbb{C} &\longrightarrow \mathbb{C}\\
    z &\longmapsto az,
  \end{align*}
  $a:=f'(z_0)$ è $\mathbb{R}$-lineare $\implies$ $f$ differenziabile in $z_0$ e $Jf_{z_0}=\text{moltiplicazione per }a=f'(z_0)$.

  ($\Leftarrow$) La dimostrazione è analoga:
  \begin{enumerate}
    \item si parte dalla definizione di differenziabilità nelle variabili $x, y$;
    \item si riscrive in termini di $z$;
    \item si usa che $Jf_{z_0}=\text{moltiplicazione per }a \in \mathbb{C} \implies f'(z_0)=a$.
  \end{enumerate}
\end{proof}

Obiettivo: sostituire la condizione (ii) con le cosiddette condizioni di Cauchy-Riemann.

\begin{lm} \label{c-lineare}
  Sia $A: \mathbb{C} \longrightarrow \mathbb{C}$ $\mathbb{R}$-lineare. Le seguenti affermazioni sono equivalenti:
  \begin{nlist}
    \item $A$ è indotta dalla moltiplicazione per un numero complesso, cioè $A(z)=a \cdot z$ per un certo $a \in \mathbb{C}$;
    \item $A$ è $\mathbb{C}$-lineare;
    \item $A(i)=iA(1)$;
    \item $A=\text{moltiplicazione per la matrice }\begin{pmatrix}
      \alpha & -\beta\\
      \beta & \alpha
  \end{pmatrix}$ dove $\alpha, \beta \in \mathbb{R}$ (nella base $\{1, i\}$ di $\mathbb{C}$).
  \end{nlist}
\end{lm}

\begin{proof}
  Per esercizio.
\end{proof}

\begin{oss}
  $a=\alpha+i\beta$.
\end{oss}

\begin{thm}
  (Cauchy-Riemann) Sia $U \subset \mathbb{C}$ aperto, $f:U \longrightarrow \mathbb{C}$ continua. $f(x, y)=u(x, y)+iv(x, y)$ nella base $\{1, i\}$ di $\mathbb{C}$. $f$ è olomorfa in $z_0 \in U$ $\iff$ valgono le seguenti:
  \begin{nlist}
    \item $f$ è differenziabile in $z_0$;
    \item valgono le \textsc{condizioni di Cauchy-Riemann}: $\dfrac{\partial u}{\partial x}(x_0, y_0)=\dfrac{\partial v}{\partial y}(x_0, y_0)$ e $\dfrac{\partial u}{\partial y}(x_0, y_0)=-\dfrac{\partial v}{\partial x}(x_0, y_0)$.
  \end{nlist}
  Nel caso $f$ sia olomorfa, $f'(z_0)=\dfrac{\partial u}{\partial x}(z_0)+i\dfrac{\partial v}{\partial x}(z_0)=$espressioni equivalenti usando le condizioni di Cauchy-Riemann.
\end{thm}

\begin{proof}
  Dobbiamo semplicemente provare che (ii) del teorema \ref{car_olo} equivale a (ii) di questo teorema. $Jf_{z_0}=\begin{pmatrix}
    \frac{\partial u}{\partial x}(z_0) & \frac{\partial u}{\partial y}(z_0)\\
    \frac{\partial v}{\partial x}(z_0) & \frac{\partial v}{\partial y}(z_0)
\end{pmatrix}$ che, per il punto (ii) del teorema \ref{car_olo}, equivale alla moltiplicazione per $a \in \mathbb{C}$, che per il lemma \ref{c-lineare} corrisponde all'applicazione lineare data dalla matrice $\begin{pmatrix}
  \alpha & -\beta\\
  \beta & \alpha
\end{pmatrix}$. Imponendo l'uguaglianza troviamo le condizioni di Cauchy-Riemann.
\end{proof}
