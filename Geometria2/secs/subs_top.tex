\begin{defn}
    Sia $(X, \tau)$ uno spazio topologico, $Y \subseteq X$ un sottoinsieme,
    cio\`e esiste una mappa iniettiva $i: Y \hooklongrightarrow X$ di
    inclusione. La topologia ristretta su $Y$ da $\tau$ \`e la topologia meno
    fine che rende $i$ continua.
\end{defn}

\begin{oss}
    Per avere $i$ continua, serve che per ogni $U$ aperto di $\tau$, si abbia
    $i^{-1}(U)$ aperto nella topologia ristretta $\tau \restrict{Y}$. Poiché
    $i^{-1}(U) = U \cap Y$, si ha in effetti che per ogni aperto $U$ di $\tau$,
    $U \cap Y \in \tau \restrict{Y}$.

    Siccome $\sigma = \{U \cap Y \;|\; U \in \tau\}$ \`e una topologia su $Y$
    che rende continua $i$ e $\sigma \subseteq \tau\restrict{Y}$, deve valere
    l'uguaglianza in quanto $\tau\restrict{Y}$ \`e la meno fine con tali
    proprietà.

    Questo ci permette anche di caratterizzare gli aperti (rispettivamente chiusi) di $Y$ come l'intersezione degli aperti (rispettivamente chiusi) di $X$ con $Y$.
\end{oss}

\begin{exc}
  Siano $A \subseteq Y \subseteq X$, allora la chiusura di $A$ in $Y$ con la topologia di sottospazio è $\overline{A} \cap Y$, dove con $\overline{A}$ s'intende la chiusura di $A$ in $X$.
\end{exc}

\begin{oss}
    Vale anche che se $\mathcal{B}$ \`e base per $\tau$, allora
    \[
    \mathcal{B'} = \{B \cap Y \;|\; B \in \mathcal{B}\}
    \]
    \`e una base per $\tau\restrict{Y}$. La dimostrazione \`e analoga a quella
    fatta per le topologie.
\end{oss}

\begin{defn}
    Sia $(X, d)$ uno spazio metrico e $Y \subseteq X$ un sottoinsieme generico.
    La distanza $d$ pu\`o essere ristretta a $Y \times Y$. In tal caso
    $d\restrict{Y \times Y}$, che verr\`a indicata anche come $d\restrict{Y}$
    \`e una distanza su $Y$.
\end{defn}

\begin{prop}
    Nel setting della definizione precedente, siano $\tau$ la topologia indotta
    da $d$ su $X$, $\tau\restrict{Y}$ la restrizione di $\tau$ a $Y$, e $\sigma$
    la topologia indotta da $d\restrict{Y}$ su $Y$. Allora $\tau\restrict{Y} =
    \sigma$.
\end{prop}

\begin{proof}
    Siano $\mathcal{B}$ e $\mathcal{D}$ rispettivamente basi per
    $\tau\restrict{y}$ e $\sigma$. Cio\`e:
    \begin{align*}
        \mathcal{B}\ =&\ \{Y\ \cap\ \{x \in X \tc d(x, x_0) < r,\quad x_0 \in
        X,\ r \in \mathbb{R}^+\}\}\\
        =&\ \{y \in Y \tc d(y, x_0) < r,\quad x_0 \in X,\ r \in \mathbb{R}^+\}
    \end{align*}
    \[
        \mathcal{D}\ =\ \{y \in Y \tc d(y, y_0) < r,\quad y_0 \in Y,\ r \in
        \mathbb{R}^+\}
    \]
    Chiaramente $\mathcal{D} \subseteq \mathcal{B}$, quindi anche $\sigma
    \subseteq \tau\restrict{Y}$.

    Inoltre, poiché le palle sono aperte, preso un $B \in \mathcal{B}$, per ogni
    $y \in B \cap Y$ esiste un $r_y>0$ tale che $D_y = B(y,r_y)$ sia contenuto
    in $B$. per cui si ha che
    \[
        B = \bigcup_{y \in B \cap Y} D_y.
    \]
    Ma ogni $D_y$ \`e un elemento della base $\mathcal{D}$, quindi ogni aperto
    di $\tau\restrict{Y}$ \`e un aperto di $\sigma$, concludendo l'ultima
    inclusione.
\end{proof}

\begin{defn}
    Siano $(X, \tau)$ spazio topologico, $Y \subseteq X$. Allora $Y$ si dice
    discreto in $X$ se $\tau\restrict{Y} = \mathcal{P}(Y)$, cioè la topologia
    ristretta a $Y$ \`e quella discreta.
\end{defn}

\begin{oss}
    Se $Y$ \`e discreto in $X$, allora per ogni $y \in Y$ esiste un aperto $U$
    di $X$ tale che $U \cap Y = \{y\}$.
\end{oss}

\begin{thm}
    \emph{Proprietà universale delle immersioni.} Sia $X$ spazio topologico e $Y
    \subseteq X$ con $i: Y \hooklongrightarrow X$ mappa di immersione. Allora
    per ogni spazio topologico $Z$ e per ogni funzione $f: Z \longrightarrow Y$,
    si ha che $f$ \`e continua se e solo se $i \circ f$ \`e continua.
\end{thm}

\begin{proof}
    Poiché $i$ \`e continua per la definizione della topologia su $Y$, e poiché
    la composizione di continue \`e continua, si ha che se $f$ \`e continua,
    anche $i \circ f$ lo \`e.

    Per l'altra implicazione, supponiamo $i \circ f$ continua e sia $A$ un
    aperto di $Y$. Allora esiste $U$ aperto di $X$ tale che ${U\cap Y = A}$, e
    vale ${i^{-1}(U) = A}$. Allora $(i\circ f)^{-1}(U) = f^{-1}(i^{-1}(U)) =
    f^{-1}(A)$ \`e aperto in $Z$.
\end{proof}

\begin{thm}
    \emph{Universalit\`a della proprietà universale.} La proprietà universale
    delle immersioni caratterizza in modo unico la topologia ristretta. Cio\`e
    dato $(X, \tau)$ spazio topologico con $Y \subseteq X$ e immersione ${i: Y
    \hooklongrightarrow X}$, $\tau\restrict{Y}$ \`e l'unica topologia su $Y$ che
    rispetta la proprietà universale.
\end{thm}

\begin{proof}
    Sia $\sigma$ una topologia su $Y$ che rispetta la propriet\`a universale.
    \begin{nlist}
        \item Prendo come $Z$ lo spazio $(Y, \tau\restrict{Y})$ e come $f$
        l'identità su $Y$. Il diagramma della proprietà universale \`e allora il
        seguente:

        \begin{center}\begin{tikzcd}
            (Y, \tau\restrict{Y}) \arrow[r, "\id"] \arrow[rd, "g=i"]
            & (Y, \sigma) \arrow[d, "i", hookrightarrow]\\
            & (X,\tau)
        \end{tikzcd}\end{center}

        Per definizione $g$ \`e continua, $i$ \`e continua, quindi $\id$ \`e
        continua. Allora $\tau\restrict{Y} \subseteq \sigma$.

        \item Questa volata prendo come $Z$ lo spazio $(Y, \sigma)$ mantenendo
        come $f$ l'identità. Il diagramma risulta essere:

        \begin{center}\begin{tikzcd}
            (Y, \sigma) \arrow[r, "\id"] \arrow[rd, "g=i"]
            & (Y, \sigma) \arrow[d, "i", hookrightarrow]\\
            & (X,\tau)
        \end{tikzcd}\end{center}

        Per definizione $\id$ \`e continua, quindi lo \`e anche $g=i$ per la
        propriet\`a universale. Allora $\sigma$ rende continua $i$, e dunque
        vale $\sigma \subseteq \tau\restrict{Y}$ per la definizione della
        restrizione di $\tau$.
    \end{nlist}
\end{proof}
\begin{defn}
    Sia $f:X\longrightarrow Y$ una funzione. Essa si dice \textsc{aperta} se
    manda aperti in aperti e \textsc{chiusa} se manda chiusi in chiusi.
\end{defn}
% TODO: dimostrazione?

\begin{oss}
    Sia $f$ come sopra continua e bigettiva. Allora essa \`e un omeomorfismo se
    e solo se \`e aperta, e se solo se \`e chiusa.
\end{oss}

\begin{ex}
    Si consideri l'insieme $\mathbb{Z}$ con la topologia $\tau =
    \mathcal{P}(\mathbb{N})\cup\mathbb{Z}$. Allora $f: n \longmapsto n-1$ \`e
    bigettiva, continua, non aperta e non omeomerfismo.
\end{ex}

\begin{defn}
    Sia $f:X \hooklongrightarrow Y$ una funzione iniettiva continua. Allora $f$
    si dice \textsc{immersione topologica} se per ogni $A \subseteq X$ aperto
    esiste un aperto $U \subseteq Y$ tale che $A = f^{-1}(U)$.
\end{defn}

\begin{oss}
    Sia $f: X \longrightarrow Y$ continua. Allora $f$ \`e chiusa e iniettiva se
    e solo se \`e un'immersione chiusa, se e solo se f \`e immersione con $f(X)$
    chiuso. Idem con gli aperti.
\end{oss}

\begin{oss}
    Sia $Y \subseteq X$ un sottospazio. Allora:
    \begin{nlist}
        \item se $X$ \`e N1 allora $Y$ \`e N1;

        \item idem con N2;

        \item se $X$ e separabile e metrizzabile allora $Y$ \`e separabile.
    \end{nlist}
\end{oss}

\begin{proof}
    I primi punti sono ovvi. Per quanto riguarda l'ultimo, se $X$ \`e
    metrizzabile e separabile, \`e N1. Dunque anche $Y$ \`e metrizzabile e N1, e
    quindi anche separabile.
\end{proof}

\begin{ex} \label{Sorgenfrey-piano}
    \emph{Il piano di Sorgenfrey}. Si prenda lo spazio $\mathbb{R}^2$ dotato
    della topologia generata dai rettangoli del tipo ${[a,b) \times [c.d)}$.
    Esso \`e chiamto il piano di Sorgenfrey. Si consideri il sottospazio della
    retta $y=-x$. La topologia indotta \`e la discreta, infatti per ogni punto
    $(k, -k)$, il rettangolo ${[k, k+1)\times[-k, 1-k)}$ interseca la retta solo
    nel punto considerato. Dunque si \`e trovato uno spazio separabile
    (prendendo per esempio $\mathbb{Q}^2$ come denso numerabile) che ha un
    sottospazio non separabile.
\end{ex}
