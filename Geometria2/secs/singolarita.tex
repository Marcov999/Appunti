\begin{thm}
    Sia $D\subseteq\C$ aperto, $f:D\setminus\{z_0\}\rightarrow\C$ olomorfa, con
    $z_0\in\C$. Allora $f$ si estende a una funzione olomorfa $g:D\rightarrow\C$
    se e solo se $f$ \`e limitata in un intorno di $z_0$.
\end{thm}
\begin{proof}
    il $(\Rightarrow)$ \`e facile, infatti se $g$ \`e olomorfa, \`e continua,
    dunque localmente limitata. Per l'altra freccia, sia $\epsilon$ tale che
    $B = B(z_0,\epsilon)\subseteq D$ e $|f(z)|\leq M$ per $z\in B\setminus
    \{z_0\}$. Sappiamo che $f$ ha uno sviluppo in serie di Laurent su
    $B\setminus \{z_0\}$:
    \[
        f(z) = \sum_{-\infty}^\infty a_n (z-z_0)^n
    \]
    con $|a_n|\leq\frac{M}{R*n}$, $R<\epsilon$, $M=sup\{|f(z)|\ t.c.\
    |z-z_0|=R\}$.

    Se $n<0$, $\lim_{R\rightarrow 0^+} M/R^n = 0$, per cui $a_n = 0$. Allora la
    serie di Laurent ha solo termini con esponente non negativo, quindi $f$ si
    estende a tale funzione olomorfa.
\end{proof}

\begin{defn}
    Nel caso descritto nel teorema precedente, $z_0$ s chiama singolarit\`a
    eliminabile o rimuovibile di $f$. In caso contrario, si chiama singolarit\`a
    isolata.
\end{defn}

\begin{defn}
    Sia $z_0$ singolarit\`a isolata di $f:D\setminus\{z_0\}\rightarrow\C$
    olomorfa con sviluppo di Laurent $\sum a_n (z-z_0)^n$ in un piccolo disco
    puntato centrato in $z_0$. Allora $z_0$ si dice:
    \begin{nlist}
        \item
            Polo di ordine $n_0$ se $X=\{a_n, n<0, a_n\neq 0\}$ \`e finito e
            $n_0 = -min(X)$.
        \item
            Singolarit\`a essenziale se $X$ \`e infinito.
    \end{nlist}
\end{defn}

\begin{oss}
    Il fatto che se $a_n\neq 0 $ per qualche $n<0$ allora la singolarit\`a non
    \`e eliminabile pu\`o essere vista nel seguente modo: sia $g(z) =
    f(z)(z-z_0)^{|n_0|-1}$, $a_{n_0}\neq 0$, $n_0<0$. Se $f$ fosse estendibile,
    lo sarebbe anche $g$. Se $b_n$ \`e il coefficiente $n$-esimo dello sviluppo
    di laurent di $g$, avremmo $b_{-1} = a_{n_0} \neq 0$.
    Ma
    \[
        b_{-1} = \frac{1}{2\pi i} \int_\gamma \frac{g(w)}{(w-z_0)^{-1+1}} = 0
    \]
    poich\'e $g$ olomorfa anche in $z_0$.
\end{oss}

\begin{prop}
    Sia $z_0$ una singolarit\`a isolata di $f:D\setminus\{z_0\}\rightarrow\C$,
    con $D$ aperto, $z_0\in D$, $f$ olomorfa. Sono fatti equivalenti:
    \begin{nlist}
    \item
        $z_0$ \`e un polo di ordine $n_0$ di $f$;
    \item
        $f(z) = g(z)/(z-z_0)^{n_0}$, con $g$ olomorfa su $D$ e $g(z_0)\neq 0$.
    \item
        $1/f$ si estende a una funzione olomorfa $k:U\rightarrow\C$ in un
        intorno di $z_0$, con $z_0$ zero di ordine $n_0$ per $k$.
    \end{nlist}
\end{prop}
\begin{proof}
    ($i\Rightarrow ii)$. Si ha che $f_z = \sum_{n\geq n_0} a_n(z-z_0)^n$ e
    $a_{-n_0} \neq 0$. Alora semplicemente si raccolglie il $(z-z_0)^{-n_0}$ e
    quello che rimane \`e una serie di potenze, cio\`e la nostra funzione
    olomorfa $g$.

    ($ii\Rightarrow iii)$. In un intorno di $z_0$, $g$ non si annulla, quindi
    \`e ben definita e olomorfa $k=1/f = (z-z_0)^{n_0}/g(z)$, che ha in $z_0$
    uno zero di ordine $n_0$.

    ($iii\Rightarrow i)$. Se $k(z) = h(z)(z-z_0)^{n_0}$, $h(z_0)\neq 0$ in una
    palla bucata centrata in $z_0$, si ha
    \[
        f(z) = \frac{(z-z_0)^{-n_0}}{h(z)} = \sum_{n\geq -n_0}
        a_{n+n_0}(z-z_0)^n.
    \]
    dove $1/h(z)$ \`e analitica.
\end{proof}

\begin{defn}
    Sia $D\subseteq\C$ aperto. Una funzione meromorfa su $D$ \`e una funzione
    olomorfa $f:D\setminus S\rightarrow\C$, dove $S$ \`e un insieme discreto (e
    chiuso in $D$) e per ogni $z_0\in S$, $f$ ha un polo in $z_0$.
\end{defn}

\begin{exc}
    Rapporti di funzioni olomorfe sono meromorfi.
\end{exc}

\begin{thm}
    (di Weierstrass). Sia $z_0$ una singolarit\`a essenziale per
    $f:D\setminus\{z_0\}\rightarrow\C$. Allora per ogni intorno $U\subseteq D$
    di $z_0$, l'insieme $f(U\setminus\{z_0\})$ \`e denso in $\C$.
\end{thm}
\begin{proof}
    Per assurdo, se $f(U\setminus\{z_0\})$ non \`e denso, esiste un complesso
    $a$ con $B(a,R)\cap f(U\setminus\{z_0\})=\emptyset$, $R>0$.

    Sia $g:U\setminus\{z_0\}\rightarrow\C$, $g(z) = \frac{1}{f(z)-a}$. Si ha
    dunque $|g(z)| \leq 1/R$, per $z\in U$. Allora $z_0$ \`e una singolarit\`a
    eliminabile per $g$, che si estende dunque ad una funzione olomorfa (che
    denotiamo comunque con $g$) su $U$. Ma allora $f$ avrebbe un polo in $z_0$,
    che \`e contro le ipotesi.
\end{proof}

\begin{oss}
    Se una funzione ha una singolarit\`a essenziale in $z_0$, allora non ammette
    limite per $z\rightarrow z_0$.
\end{oss}

\subsubsection{Teorema dei Residui}

\begin{defn}
    Sia $D\subseteq\C$ chiuso con parte interna non vuota, e sia
    $\Gamma=\partial D$. Diciamo che $D$ ha bordo $C^1$ a tratti se per ogni
    componente connessa $\Gamma_i$ di $\Gamma$, esiste una curva in $\Gamma_i$
    che sia $C^1$ a tratti, surgettiva e iniettiva eccetto al pi\`u che negli
    estremi.

    Le componenti di bordo di un tale $D$ si possono orientare canonicamente
    come segue: diciamo che $\gamma:J\rightarrow \Gamma_i$ con $J$ intervallo
    reale \`e positiva se per tutti i $t_0$ per cui $\gamma'(t_0)$ \`e definita,
    il vettore $-i\gamma'(t_0)$ punta all'esterno di $D$ in $\gamma(t_0)$.

    Il significato di "punta all'esterno" a livello intuitivo \`e chiaro, a
    livello formale un vettore $v$ punta all'esterno di $D$ in $z_0\in\partial D$ se
    esiste un$\epsilon>0$ tale che $\{t\in(-\epsilon,\epsilon) \ :\ z_0+tv\in D
    \} = (\epsilon,0]$.

    In questi casi, se $R\subseteq D$ \`e un dominio con bordo $C^1$, $\Gamma =
    \partial R  \Gamma_1 \cup \dots \cup \Gamma_k$, con parametrizzazioni
    positive $\gamma_i:J_i\rightarrow \Gamma_i$, e $\omega$ \`e una 1-forma su
    $D$, poniamo
    \[
        \int_{\partial R}\omega = \sum_{i=1}^k \int_{\gamma_i} \omega
    \]
\end{defn}

\begin{thm}
    Sia $D\subseteq \C$ aperto, $f:D\setminus S\rightarrow \C$ olomorfa con $S$
    chiuso e discreto in $D$. Sia $R\subseteq D$ compatto e con bordo $C^1$
    atratti, per cui $R\cap S = \{z_1,\dots,z_k\}$ \`e finito, $\partial R \cap
    S \neq \emptyset$. Allora
    \[
        \int_{\partial R} f(z)dz = 2 \pi i \sum_{i=1}^k \Res(f,z_i)
    \]
\end{thm}

\begin{proof}
  Limitiamoci al caso in cui $R$ è omeomorfo a un disco, il caso generale è simile ma non verrà mai usato nelle applicazioni, per cui non viene dimostrato. Abbiamo dunque $f$ olomorfa in $D\setminus S$ con $R \cap S=\{z_1, \dots, z_k\}$ singolarità isolate.
  Sia $\gamma=\partial R$ (percorso in senso antiorario) e $\beta=\gamma*l_k*\bar{\alpha}_k*\bar{l}_k*l_{k-1}*\bar{\alpha}_{k-1}*\bar{l}_{k-1}*\dots*l_1*\bar{\alpha}_1*\bar{l}_1$, dove $\alpha_i$ è una piccola circonferenza intorno a $z_i$ percorsa in senso antiorario e gli $l_i$ sono tratti che uniscono un punto di $\gamma$ a gli $\alpha_i$.
  È possibile scegliere gli $\alpha_i$ e $l_i$ di modo che $\beta$ sia omotopicamente banale in $R\setminus \{z_1,\dots, z_k\}$.
  Poiché $f$ è olomorfa su $R \setminus \{z_1, \dots, z_k\}$, $f(z)\diff z$ è chiusa su $R \setminus \{z_1, \dots, z_k\}$ e poiché $\beta$ è banale $\displaystyle 0=\int_{\beta}f(z)\diff z=\int_{\gamma} f(z)\diff z+\sum_{i=1}^k \left(\int_{\bar{l}_i} f(z)\diff z+\int_{\bar{\alpha}_i}f(z)\diff z+\int_{l_i}f(z)\diff z\right)=\int_{\gamma} f(z)\diff z+\sum_{i=1}^k \int_{\bar{\alpha}_i}f(z)\diff z$.
  Dunque $\displaystyle \int_{\gamma} f(z)\diff z=-\sum_{i=1}^k \int_{\bar{\alpha}_i}f(z)\diff z=$\\
  $\displaystyle =\sum_{i=1}^k \int_{\alpha_i}f(z)\diff z=2\pi i \sum_{i=1}^k \Res(f, z_i)$.
\end{proof}

Vediamo ora come effettuare il calcolo dei residui. Se $z_0$ è un polo semplice di $f$, allora $\displaystyle f(z)=\frac{a_{-1}}{z-z_0}+\sum_{n \ge 0} a_n(z-z_0)^n$, per cui $\displaystyle (z-z_0)f(z)=a_{-1}+\sum_{n \ge 0} a_n(z-z_0)^{n+1}$ e $\Res(f,z_0)=a_{-1}=\displaystyle \lim_{z \longrightarrow z_0} (z-z_0)f(z)$.
Dunque, se $f=p/q$, $p, q$ olomorfe, $p(z_0)\not=0$ e $z_0$ è uno zero semplice di $q$, si ha $\displaystyle \Res(f,z_0)=\lim_{z\longrightarrow z_0} \frac{(z-z_0)p(z)}{q(z)}=\frac{p(z_0)}{q'(z_0)}$ in quanto $q(z)=q(z)-q(z_0)$.

\begin{ex}
  $f(z)=\dfrac{e^{iz}}{z^2+1}$ ha singolarità in $z=\pm i$, che sono poli semplici in quanto zeri semplici di $z^2+1$. Dunque $\Res(f,i)=\left.\dfrac{e^{iz}}{2z}\right|_{z=i}=-\dfrac{i}{2e}$, $\Res(f,-i)=\left.\dfrac{e^{iz}}{2z}\right|_{z=-i}=\dfrac{ie}{2}$.
\end{ex}

Per i poli di ordine superiore una strategia possibile è la seguente: se $f$ ha un polo di ordine $k$ in $z_0$, si pone $g(z)=(z-z_0)^kf(z)$, che si estende a una funzione olomorfa anche in $z_0$.
$f(z)=a_{-k}(z-z_0)^{-k}+\dots+a_{-1}(z-z_0)^{-1}+\sum_{n \ge 0} a_n(z-z_0)^n$ $\implies$ $g(z)=a_{-k}+a_{-k+1}(z-z_0)+a_{-k+2}(z-z_0)^2+\dots+a_{-1}(z-z_0)^{k-1}+\dots$ $\implies$ $\Res(f,z_0)=a_{-1}=\dfrac{g^{(k-1)}(z_0)}{(k-1)!}$.

\begin{ex}
  $f(z)=\dfrac{e^{iz}}{z(z^2+1)^2}$ ha un polo semplice in $0$ e due poli doppi in $\pm i$. Calcoliamo il residuo in $i$: $g(z)=(z-i)^2f(z)=\dfrac{e^{iz}}{z(z+i)^2}$. $g^{(k-1)}(z)=g'(z)=\dfrac{ize^{iz}(z+i)-e^{iz}((z+i)+2z)}{z^3(z+i)^3}$, dunque $\Res(f,i)=\dfrac{g'(i)}{1!}=-\dfrac{3}{4e}$.
\end{ex}

Vediamo ora alcune applicazioni utili del teorema dei residui. Verrà riportata la prima come esempio e le altre saranno lasciate come esercizio (eventualmente con qualche aiuto se ci sono degli inghippi a cui stare attenti) perché l'approccio è sostanzialmente identico.

\begin{ex}
  Calcolare $\displaystyle \int_{-\infty}^{+\infty} \frac{\cos{x}}{x^2+1}\diff x=\lim_{R \longrightarrow +\infty} \int_{-R}^R \dfrac{\cos{x}}{x^2+1}\diff x$. Per $x \in \mathbb{R}$, $\dfrac{\cos{x}}{x^2+1}=\mathfrak{Re}\left(\dfrac{e^{ix}}{x^2+1}\right)$.
  La funzione $\dfrac{e^{iz}}{z^2+1}$ ha poli semplici in $\pm i$. Sia $\Omega_R=\{z \mid |z| \le R, \mathfrak{Im}(z) \ge 0\}$, $R>1$. Una parametrizzazione di $\partial\Omega_R$ è $\gamma_R=\alpha_R*\beta_R$ dove $\alpha_R:[-R, R] \longrightarrow \mathbb{C}, \alpha_R(t)=t$ e $\beta_R:[0, \pi] \longrightarrow \mathbb{C}, \beta_R(t)=Re^{it}$.
  Per il teorema dei residui, se $f(z)=\dfrac{e^{iz}}{z^2+1}$, $\displaystyle \int_{\alpha_R} f(z)\diff z+\int_{\beta_R}f(z)\diff z=\int_{\partial \Omega_R} f(z)\diff z=2\pi i\Res(f, i)=2\pi i\left(-\frac{i}{2e}\right)=\frac{\pi}{e}$.
  Osserviamo che $\displaystyle \int_{\alpha_R} f(z)\diff z=\int_{-R}^R f(\alpha(t))\alpha'(t)\diff t=\int_{-R}^R f(t)\diff t=\int_{-R}^R \frac{\cos{t}+i\sin{t}}{t^2+1}\diff t$, la cui parte reale è l'integrale richiesto (prima di passare al limite).
  Dunque $\displaystyle \int_{-R}^R \dfrac{\cos{t}}{1+t^2}\diff t=\mathfrak{Re}\left(\int_{\alpha_R} f(z)\diff z\right)=\mathfrak{Re}\left(\int_{\partial \Omega_R} f(z)\diff z-\int_{\beta_R}f(z)\diff z\right)=\frac{\pi}{e}-\mathfrak{Re}\left(\int_{\beta_R} f(z) \diff z\right)$.
  Adesso mostriamo che $\displaystyle \lim_{R \longrightarrow +\infty} \int_{\beta_R} f(z)\diff z=0$, da cui $\displaystyle \int_{-\infty}^{+\infty} \frac{\cos{t}}{1+t^2}\diff t=\frac{\pi}{e}$.
  $\beta_R(t)=Re^{it}=R(\cos{t}+i\sin{t})$, per cui $e^{i\beta_R(t)}=e^{iR(\cos{t}+i\sin{t})}=e^{iR\cos{t}}\cdot e^{-R\sin{t}}$, il cui modulo è $e^{-R\sin{t}}$, dunque $| f(\beta_R(t))\beta_R'(t)|=\left|\dfrac{e^{i\beta_R(t)}}{\beta_R(t)^2+1}iRe^{it}\right|=\dfrac{Re^{-R\sin{t}}}{|R^2e^{2it}+1|} \le \dfrac{R}{|R^2e^{2it}|-1}=\dfrac{R}{R^2-1}$, per cui
  $\displaystyle \left|\int_{\beta_R}f(z)\diff z\right|=\left|\int_0^\pi f(\beta_R(t))\beta_R'(t)\diff t\right| \le \int_0^\pi \frac{R}{R^2-1}\diff t=\dfrac{R\pi}{R^2-1} \longrightarrow 0$ per $R \longrightarrow +\infty$, come voluto.
\end{ex}

Attenzione: per "complessificare" $\dfrac{\cos{x}}{x^2+1}$ si sarebbe potuto prendere $\dfrac{\cos{z}}{z^2+1}$.
Il problema è che $\left| \dfrac{\cos{z}}{z^2+1}\right|$ non è stimabile facilmente lungo $\beta_R$, per cui $\displaystyle \int_{\beta_R} \frac{\cos{z}}{z^2+1}\diff z \not\rightarrow 0$ per $R \longrightarrow +\infty$ ($\cos(it)=\cosh{t}$). Dunque meglio la scelta fatta di prendere $\cos{t}=\mathfrak{Re}(e^{it})$. \marginpar\warningsign

In modo analogo si calcola $\displaystyle \int_{-\infty}^{+\infty} \frac{P(x)}{Q(x)}\diff x$ dove $P, Q$ sono polinomi, $Q(x) \not=0$ per ogni $x \in \mathbb{R}$ e $\deg{Q} \ge \deg{P}+2$.

\begin{exc}
  Calcolare $\displaystyle \int_{-\infty}^{+\infty} \frac{1}{1+x^4}\diff x$.
\end{exc}

\begin{exc}
  Calcolare $\displaystyle \int_0^{+\infty} \frac{1}{1+x^n} \diff x$, $n \ge 2$. Hint: se $n$ è pari si può svolgere come l'esercizio precedente, se $n$ è dispari c'è un polo in $z_0=-1$ che dà fastidio. Considerare allora $D_R=\{z \in \mathbb{C} \mid |z| \le R, 0 \le \arg{z} \le 2\pi/n\}$. In questo caso, l'unico polo di cui preoccuparsi è $z_0=e^{i\frac{\pi}{n}}$.
\end{exc}

Per svolgere il prossimo esercizio servono i seguenti due lemmi.

\begin{lm}
  Sia $f$ meromorfa con polo semplice in $0$ e sia $\gamma_{\epsilon}:[0,\pi] \longrightarrow \mathbb{C}\setminus\{0\}$, $\gamma_{\epsilon}(t)=\epsilon\cdot e^{it}$, $\epsilon>0$. Allora $\displaystyle \lim_{\epsilon \longrightarrow 0} \int_{\gamma_\epsilon} f(z)\diff z=\pi i \Res(f,0)$.
\end{lm}

\begin{proof}
  $f(z)=\dfrac{a_{-1}}{z}+g(z)$, $g$ olomorfa in un intorno di $0$ con primitiva $G$. Allora $\displaystyle \int_{\gamma_\epsilon} f(z)\diff z=\int_{\gamma_\epsilon} \dfrac{a_{-1}}{z}\diff z+\int_{\gamma_\epsilon} g(z)\diff z=\pi i a_{-1}+G(\gamma_\epsilon(\pi))-G(\gamma_\epsilon(0))$.
  È chiaro che $G(\gamma_\epsilon(\pi))-G(\gamma_\epsilon(0)) \longrightarrow G(0)-G(0)=0$ per $\epsilon \longrightarrow 0$, da cui segue la tesi.
\end{proof}

\begin{lm}
  Sia $f$ meromorfa su $\mathbb{C}$ con la seguente proprietà: detto $M(R)=\max\{|f(z)| \mid |z|=R, \mathfrak{Im}(z) \ge 0\}$. Allora $\displaystyle \lim_{R \longrightarrow +\infty} M(R)=0$. Allora, se $\gamma_R(y)=Re^{it}$, $t \in [0,\pi]$, si ha $\displaystyle \lim_{R \longrightarrow +\infty} \int_{\gamma_R} f(z)e^{iz}\diff z=0$.
\end{lm}

\begin{proof}
  Se $z=\gamma_R(t)=R\cos{t}+iR\sin{t}$, $e^{iz}=e^{iR\cos{t}}\cdot e^{i^2R\sin{t}}=e^{iR\sin{t}}\cdot e^{-R\sin{t}}$, da cui $|e^{iz}|=e^{-R\sin{t}}$.
  Dunque $\displaystyle \left| \int_{\gamma_R} f(z)e^{iz}\diff z\right| \le \int_0^\pi | f(\gamma_R(t))\cdot e^{i\gamma_R(t)}\cdot \gamma_R'(t)|\diff t=\int_0^\pi |f(\gamma_R(t))|\cdot |e^{i\gamma_R(t)}|\cdot |Ri e^{it}| \diff t \le \int_0^\pi M(R) \cdot e^{-R\sin{t}} \cdot R \diff t=M(R)\int_0^\pi e^{-R\sin{t}}\cdot R \diff t$.
  Per concludere basta vedere che l'ultimo integrale è limitato indipendentemente da $R$. Ora, su $[0,\pi/2]$ la funzione $\dfrac{\sin{t}}{t}$ è limitata dal basso da una cosante positiva $K$ (in quanto si estende con continuità in $0$ a una funzione strettamente positiva e $[0,\pi/2]$ è compatto. Allora $\displaystyle \int_0^\pi Re^{-R\sin{t}} \diff t=2 \int_0^{\pi/2} Re^{-R\sin{t}}\diff t=2R \int_0^{\pi/2} e^{-R\frac{\sin{t}}{t}\cdot t} \diff t \le 2R \int_0^{\pi/2} e^{-RKt}\diff t=2R\left[-\frac{e^{-RKt}}{RK}\right]_0^{\pi/2}=\frac{2}{K}[-e^{-RK\pi/2}+1] \le 2/K$.
\end{proof}

Attenzione: se non ci fosse il termine $e^{iz}$ servirebbe $\displaystyle \lim_{R \longrightarrow +\infty} RM(R)=0$; quel fattore oscillante rende sufficiente la condizione più debole. \marginpar\warningsign

\begin{exc}
  Calcolare $\displaystyle \int_0^{+\infty} \frac{\sin{x}}{x} \diff x$ dove estendiamo $\dfrac{\sin{x}}{x}$ in $0$ ponendola uguale a $1$. Hint: integrare lungo il bordo della semicorona circolare $C_R=\{z \mid \mathfrak{Im}(z) \ge 0, 1/R \le |z| \le R\}$.
\end{exc}
