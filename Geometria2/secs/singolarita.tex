\begin{thm}
    Sia $D\subseteq\C$ aperto, $f:D\setminus\{z_0\}\rightarrow\C$ olomorfa, con
    $z_0\in\C$. Allora $f$ si estende a una funzione olomorfa $g:D\rightarrow\C$
    se e solo se $f$ \`e limitata in un intorno di $z_0$.
\end{thm}
\begin{proof}
    il $(\Rightarrow)$ \`e facile, infatti se $g$ \`e olomorfa, \`e continua,
    dunque localmente limitata. Per l'altra freccia, sia $\epsilon$ tale che 
    $B = B(z_0,\epsilon)\subseteq D$ e $|f(z)|\leq M$ per $z\in B\setminus
    \{z_0\}$. Sappiamo che $f$ ha uno sviluppo in serie di Laurent su
    $B\setminus \{z_0\}$:
    \[
        f(z) = \sum_{-\infty}^\infty a_n (z-z_0)^n
    \]
    con $|a_n|\leq\frac{M}{R*n}$, $R<\epsilon$, $M=sup\{|f(z)|\ t.c.\
    |z-z_0|=R\}$.

    Se $n<0$, $\lim_{R\rightarrow 0^+} M/R^n = 0$, per cui $a_n = 0$. Allora la
    serie di Laurent ha solo termini con esponente non negativo, quindi $f$ si
    estende a tale funzione olomorfa.
\end{proof}

\begin{defn}
    Nel caso descritto nel teorema precedente, $z_0$ s chiama singolarit\`a
    eliminabile o rimuovibile di $f$. In caso contrario, si chiama singolarit\`a
    isolata.
\end{defn}

\begin{defn}
    Sia $z_0$ singolarit\`a isolata di $f:D\setminus\{z_0\}\rightarrow\C$
    olomorfa con sviluppo di Laurent $\sum a_n (z-z_0)^n$ in un piccolo disco
    puntato centrato in $z_0$. Allora $z_0$ si dice:
    \begin{nlist}
        \item   
            Polo di ordine $n_0$ se $X=\{a_n, n<0, a_n\neq 0\}$ \`e finito e
            $n_0 = -min(X)$.
        \item
            Singolarit\`a essenziale se $X$ \`e infinito.
    \end{nlist}
\end{defn}

\begin{oss}
    Il fatto che se $a_n\neq 0 $ per qualche $n<0$ allora la singolarit\`a non
    \`e eliminabile pu\`o essere vista nel seguente modo: sia $g(z) =
    f(z)(z-z_0)^{|n_0|-1}$, $a_{n_0}\neq 0$, $n_0<0$. Se $f$ fosse estendibile,
    lo sarebbe anche $g$. Se $b_n$ \`e il coefficiente $n$-esimo dello sviluppo
    di laurent di $g$, avremmo $b_{-1} = a_{n_0} \neq 0$.
    Ma 
    \[  
        b_{-1} = \frac{1}{2\pi i} \int_\gamma \frac{g(w)}{(w-z_0)^{-1+1}} = 0
    \]
    poich\'e $g$ olomorfa anche in $z_0$.
\end{oss}

\begin{prop}
    Sia $z_0$ una singolarit\`a isolata di $f:D\setminus\{z_0\}\rightarrow\C$,
    con $D$ aperto, $z_0\in D$, $f$ olomorfa. Sono fatti equivalenti:
    \begin{nlist}
    \item  
        $z_0$ \`e un polo di ordine $n_0$ di $f$;
    \item
        $f(z) = g(z)/(z-z_0)^{n_0}$, con $g$ olomorfa su $D$ e $g(z_0)\neq 0$.
    \item
        $1/f$ si estende a una funzione olomorfa $k:U\rightarrow\C$ in un
        intorno di $z_0$, con $z_0$ zero di ordine $n_0$ per $k$.
    \end{nlist}
\end{prop}
\begin{proof}
    ($i\Rightarrow ii)$. Si ha che $f_z = \sum_{n\geq n_0} a_n(z-z_0)^n$ e
    $a_{-n_0} \neq 0$. Alora semplicemente si raccolglie il $(z-z_0)^{-n_0}$ e
    quello che rimane \`e una serie di potenze, cio\`e la nostra funzione
    olomorfa $g$.

    ($ii\Rightarrow iii)$. In un intorno di $z_0$, $g$ non si annulla, quindi
    \`e ben definita e olomorfa $k=1/f = (z-z_0)^{n_0}/g(z)$, che ha in $z_0$
    uno zero di ordine $n_0$.

    ($iii\Rightarrow i)$. Se $k(z) = h(z)(z-z_0)^{n_0}$, $h(z_0)\neq 0$ in una
    palla bucata centrata in $z_0$, si ha
    \[
        f(z) = \frac{(z-z_0)^{-n_0}}{h(z)} = \sum_{n\geq -n_0}
        a_{n+n_0}(z-z_0)^n.
    \]
    dove $1/h(z)$ \`e analitica.
\end{proof}

\begin{defn}
    Sia $D\subseteq\C$ aperto. Una funzione meromorfa su $D$ \`e una funzione
    olomorfa $f:D\setminus S\rightarrow\C$, dove $S$ \`e un insieme discreto (e
    chiuso in $D$) e per ogni $z_0\in S$, $f$ ha un polo in $z_0$.
\end{defn}

\begin{exc}
    Rapporti di funzioni olomorfe sono meromorfi.
\end{exc}

\begin{thm}
    (di Weierstrass). Sia $z_0$ una singolarit\`a essenziale per
    $f:D\setminus\{z_0\}\rightarrow\C$. Allora per ogni intorno $U\subseteq D$
    di $z_0$, l'insieme $f(U\setminus\{z_0\})$ \`e denso in $\C$.
\end{thm}
\begin{proof}
    Per assurdo, se $f(U\setminus\{z_0\})$ non \`e denso, esiste un complesso
    $a$ con $B(a,R)\cap f(U\setminus\{z_0\})=\emptyset$, $R>0$.
    
    Sia $g:U\setminus\{z_0\}\rightarrow\C$, $g(z) = \frac{1}{f(z)-a}$. Si ha
    dunque $|g(z)| \leq 1/R$, per $z\in U$. Allora $z_0$ \`e una singolarit\`a
    eliminabile per $g$, che si estende dunque ad una funzione olomorfa (che
    denotiamo comunque con $g$) su $U$. Ma allora $f$ avrebbe un polo in $z_0$,
    che \`e contro le ipotesi.
\end{proof}

\begin{oss}
    Se una funzione ha una singolarit\`a essenziale in $z_0$, allora non ammette
    limite per $z\rightarrow z_0$.
\end{oss}

\subsubsection{Teorema dei Residui}

\begin{defn}
    Sia $D\subseteq\C$ chiuso con parte interna non vuota, e sia
    $\Gamma=\partial D$. Diciamo che $D$ ha bordo $C^1$ a tratti se per ogni
    componente connessa $\Gamma_i$ di $\Gamma$, esiste una curva in $\Gamma_i$
    che sia $C^1$ a tratti, surgettiva e iniettiva eccetto al pi\`u che negli
    estremi.

    Le componenti di bordo di un tale $D$ si possono orientare canonicamente
    come segue: diciamo che $\gamma:J\rightarrow \Gamma_i$ con $J$ intervallo
    reale \`e positiva se per tutti i $t_0$ per cui $\gamma'(t_0)$ \`e definita,
    il vettore $-i\gamma'(t_0)$ punta all'esterno di $D$ in $\gamma(t_0)$.

    Il significato di "punta all'esterno" a livello intuitivo \`e chiaro, a
    livello formale un vettore $v$ punta all'esterno di $D$ in $z_0\in\partial D$ se
    esiste un$\epsilon>0$ tale che $\{t\in(-\epsilon,\epsilon) \ :\ z_0+tv\in D
    \} = (\epsilon,0]$.

    In questi casi, se $R\subseteq D$ \`e un dominio con bordo $C^1$, $\Gamma =
    \partial R  \Gamma_1 \cup \dots \cup \Gamma_k$, con parametrizzazioni
    positive $\gamma_i:J_i\rightarrow \Gamma_i$, e $\omega$ \`e una 1-forma su
    $D$, poniamo
    \[
        \int_{\partial R}\omega = \sum_{i=1}^k \int_{\gamma_i} \omega
    \]
\end{defn}

\begin{thm}
    Sia $D\subseteq \C$ aperto, $f:D\setminus S\rightarrow \C$ olomorfa con $S$
    chiuso e discreto in $D$. Sia $R\subseteq D$ compatto e con bordo $C^1$
    atratti, per cui $R\cap S = \{z_1,\dots,z_k\}$ \`e finito, $\partial R \cap
    S \neq \emptyset$. Allora
    \[
        \int_{\partial R} f(z)dz = 2 \pi i \sum_{i=1}^k \Res(f,z_i)
    \]
\end{thm}

