D'ora in poi tutti gli spazi saranno localmente connessi per archi.

\begin{prop}
  Sia $p:E \longrightarrow X$ rivestimento con $E$ connesso e siano $\tilde{x}_0 \in p^{-1}(x_0)=F$ e $\psi:\pi_1(X, x_0) \longrightarrow F$, $\psi(\alpha)=\tilde{x}_0 \cdot \alpha$.
  $\psi$ induce una bigezione $\mfaktor{p_{\star}(\pi_1(E, \tilde{x}_0))}{\pi_1(X, x_0)} \cong F$.
  In particolare, $stab(\tilde{x}_0)=p_{\star}(\pi_1(E, \tilde{x}_0))$ e, al variare di $\tilde{x} \in F$, i gruppi $p_{\star}(\pi_1(E, \tilde{x}))=stab(\tilde{x})$ sono tutti e soli i coniugati di $p_{\star}(\pi_1(E, \tilde{x}_0))=stab(\tilde{x}_0)$.
\end{prop}

\begin{proof}
  Poiché $E$ è connesso, l'azione di monodronia è transitiva, per cui $\psi$ è suriettiva.
  $\alpha=[\gamma] \in stab(\tilde{x}_0) \iff \tilde{\gamma}_{\tilde{x}_0}(1)=\tilde{x}_0 \iff \tilde{\gamma}_{\tilde{x}_0}$ è un loop in $E$ $\iff$ $[\gamma] \in p_{\star}(\pi_1(E, \tilde{x}_0))$.
  Infatti, $\gamma$ si solleva a un laccio chiuso se e solo se $\tilde{\gamma}_{\tilde{x}_0}$ definisce un elemento di $\pi_1(E, \tilde{x}_0)$.
  Da ciò segue $\psi(\alpha)=\psi(\beta) \iff \tilde{x}_0 \cdot \alpha =\tilde{x}_0 \cdot \beta \iff \tilde{x}_0 \cdot (\alpha \beta^{-1})=\tilde{x}_0 \iff \alpha\beta^{-1} \in stab(\tilde{x}_0) \iff [\alpha]=[\beta]$
  in $\mfaktor{stab(\tilde{x}_0)}{\pi_1(X, x_0)}$, dunque $\psi$ induce la bigezione richiesta.

  Usando il fatto che l'azione è transitiva è facile vedere che $stab(\tilde{x}), \tilde{x} \in F$ sono tutti e solo i coniugati di $stab(\tilde{x}_0)$ (se $\tilde{x} \in F$, $\tilde{x}=\tilde{x}_0 \cdot \eta$ e $stab(\tilde{x})=\eta^{-1}\cdot stab(\tilde{x}_0)\cdot\eta$).
\end{proof}

\begin{thm}
  (di sollevamento) Sia $p:E \longrightarrow X$ un rivestimento connesso (cioè $E$ è connesso), $x_0 \in X, \tilde{x}_0 \in p^{-1}(x_0)$ e sia $f:Y \longrightarrow X$ continua, $Y$ connesso, $y_0 \in Y$ t.c. $f(y_0)=x_0$.
  Allora esiste un sollevamento $\tilde{f}:Y \longrightarrow E$ di $f$ con $\tilde{f}(y_0)=\tilde{x}_0$ $\iff$ $f_{\star}(\pi_1(Y, y_0)) \subseteq p_{\star} (\pi_1(E, \tilde{x}_0))$.
\end{thm}

\begin{proof}
  Consideriamo il seguente diagramma:
  \begin{center}
    \begin{tikzcd}
            & E \arrow[d, "p"]\\
            Y \arrow[ru, dashed, "\tilde{f}"] \arrow[r, "f"] & X
     \end{tikzcd}
  \end{center}
  ($\implies$) Se $f=p \circ \tilde{f}$, $f_{\star}=p_{\star} \circ \tilde{f}_{\star}$ (sui $\pi_1$ con i punti base scelti), dunque $\Ima{f_{\star}} \subseteq \Ima{p_{\star}}$.

  ($\Leftarrow$) Definiamo $\tilde{f}$ come segue: se $y \in Y$, scegliamo un cammino $\gamma$ in $Y$ con $\gamma(0)=y_0, \gamma(1)=y$. Poniamo $\tilde{f}(y)=\widetilde{(f \circ \gamma)}_{\tilde{x}_0}(1)$.
  $p(\tilde{f}(y))=p(\widetilde{(f \circ \gamma)}_{\tilde{x}_0}(1))=(f \circ \gamma)_{\tilde{x}_0}(1)=f(y)$ $\implies$ $p \circ \tilde{f}=f$.
  Verifichiamo che $\tilde{f}$ sia ben definita: se $\beta$ è un altro cammino da $y_0$ a $y$, $\gamma \sim \gamma * \bar{\beta}*\beta$ $\implies$ $f \circ \gamma \sim f \circ (\gamma * \bar{\beta}*\beta)=(f \circ (\gamma * \bar{\beta}))*(f \circ \beta)$,
  ma adesso notiamo che $\gamma *\bar{\beta}=\alpha$ è un loop in $y_0$, quindi $f \circ \alpha$ è un loop in $x_0$ e $[f \circ \alpha] \in f_{\star}(\pi_1(Y, y_0)) \subseteq p_{\star}(\pi_1(E, \tilde{x}_0))$, dunque $f \circ \alpha$ si solleva a un loop in $E$ a partire da $\tilde{x}_0$.
  Perciò $\widetilde{(f \circ \gamma)}_{\tilde{x}_0}(1)=\widetilde{((f \circ \alpha)*(f \circ \beta))}_{\tilde{x}_0}(1)=\widetilde{(f \circ \alpha)}_{\tilde{x}_0}*\widetilde{(f \circ \beta)}_{\widetilde{(f \circ \alpha)}_{\tilde{x}_0}(1)}(1)=\widetilde{(f \circ \alpha)}_{\tilde{x}_0}*\widetilde{(f \circ \beta)}_{\tilde{x}_0}(1)=\widetilde{(f \circ \beta)}_{\tilde{x}_0}(1)$
  $\implies$ $\tilde{f}$ è ben definita.

  Continuità di $\tilde{f}$: dato $y \in Y$, sia $U$ un intorno aperto connesso per archi e ben rivestito di $f(y)$ in $X$ e sia $W=f^{-1}(U)$, che è aperto. $\displaystyle p^{-1}(U)=\bigsqcup_{i \in I} V_i$ con i $V_i$ aperti in $E$. Scegliamo $i_0$ t.c. $\tilde{f}(y) \in V_{i_0}$. Sia anche $s:U \longrightarrow V_{i_0}$ l'inversa continua di $p\restrict{V_{i_0}}$.
  Basta vedere che $\tilde{f}\restrict{W}=s \circ f\restrict{W}$, così avremmo $\tilde{f}\restrict{W}$ continua, dunque $\tilde{f}$ continua. Sia $\gamma \in \Omega(y_0, y)$ e per ogni $z \in W$ sia $\gamma_z \in \Omega(y, z)$. Scegliamo $\alpha_z=\gamma*\gamma_z$.
  $\tilde{f}(z)=\widetilde{(f \circ \alpha_z)}_{\tilde{x}_0}(1)=\widetilde{((f \circ \gamma)*(f \circ \gamma_z))}_{\tilde{x}_0}(1)=\widetilde{(f \circ \gamma)}_{\tilde{x}_0}*\widetilde{(f \circ \gamma_z)}_{\widetilde{(f \circ \gamma)}_{\tilde{x}_0}(1)}(1)=\widetilde{(f \circ \gamma)}_{\tilde{x}_0}*\widetilde{(f \circ \gamma_z)}_{\tilde{f}(y)}(1)=\widetilde{(f \circ \gamma_z)}_{\tilde{f}(y)}(1)$.
  Adesso notiamo che $s \circ f \circ \gamma_z$ solleva $f \circ \gamma_z$ a partire da $\tilde{f}(y)$, dunque per unicità del sollevamento dev'essere uguale a $\widetilde{(f \circ \gamma_z)}_{\tilde{f}(y)}$, perciò $\widetilde{(f \circ \gamma_z)}_{\tilde{f}(y)}(1)=(s \circ f \circ \gamma_z)(1)=s(f(z))$.
\end{proof}

\begin{cor} \label{soll_conn}
  Siano $p:E \longrightarrow X$ rivestimento connesso, $x_0 \in X$, $\tilde{x}_0 \in p^{-1}(x_0)$, $f:Y \longrightarrow X$ con $f(y_0)=x_0$. Se $Y$ è semplicemente connesso, esiste un unico sollevamento $\tilde{f}:Y \longrightarrow E$ di $f$ con $\tilde{f}(y_0)=\tilde{x}_0$.
\end{cor}

\begin{thm}
  (Borsuk-Ulam) Non esistono mappe continue $f:S^2 \longrightarrow S^1$ t.c. $f(-x)=-f(x)$.
\end{thm}

\begin{proof}
  Per assurdo, sia $f:S^2 \longrightarrow S^1$ continua t.c. $f(-x)=-f(x)$. Siano $p:S^2 \longrightarrow \mathbb{P}^2(\mathbb{R}), q:S^1 \longrightarrow \mathbb{P}^1(\mathbb{R})$ le proiezioni. Abbiamo un diagramma commutativo di funzioni continue
  \begin{center}
    \begin{tikzcd}
            S^2 \arrow[d, "p"] \arrow[r, "f"] & S^1 \arrow[d, "q"]\\
            \mathbb{P}^2(\mathbb{R}) \arrow[r, "\bar{f}"] & \mathbb{P}^1(\mathbb{R})
     \end{tikzcd}
   \end{center}
  Si noti che il diagramma commuta perché $q(f(x_0))=q(f(-x_0))$. Sappiamo che $\pi_1(\mathbb{P}^2(\mathbb{R}))=\mathbb{Z}_2$ e dato che $\mathbb{P}^1(\mathbb{R}) \cong S^1$ abbiamo che $\pi_1(\mathbb{P}^1(\mathbb{R}))=\mathbb{Z}$.
  Allora $\bar{f}_{\star}:\pi_1(\mathbb{P}^2(\mathbb{R})) \longrightarrow \pi_1(\mathbb{P}^1(\mathbb{R}))$ è la mappa banale.
  \begin{center}
    \begin{tikzcd}
            & & S^1 \arrow[d, "q"]\\
            S^2 \arrow[rru, bend left, "f"] \arrow[r, "p"] & \mathbb{P}^2(\mathbb{R}) \arrow[ru, "h"] \arrow[r, "\bar{f}"] & \mathbb{P}^1(\mathbb{R})
    \end{tikzcd}
  \end{center}
  Per il teorema di sollevamento, esiste $h:\mathbb{P}^2(\mathbb{R}) \longrightarrow S^1$ con $q \circ h=\bar{f}$ (attenzione: non sappiamo che $h \circ p=f$). Scegliamo $x_0 \in S^2$. Sappiamo che $q(h(p(x_0)))=\bar{f}(p(x_0))=q(f(x_0))$, dunque $h \circ p$ e $f$ sono entrambi sollevamenti di $\bar{f} \circ p$.
  Dato $x_0 \in S^2$, sia $z_0 \in \mathbb{P}^1(\mathbb{R}), z_0=\bar{f}(p(x_0))$. $q^{-1}(z_0)=\{y_0, -y_0\}$. Dunque $f(x_0) \in \{y_0, -y_0\}$. Supponiamo $f(x_0)=y_0$. Allora $f(-x_0)=-f(x_0)=-y_0$.
  Anche $h(p(x_0)), h(p(-x_0)) \in \{y_0, -y_0\}$, ma $p(x_0)=p(-x_0) \implies h(p(x_0))=h(p(-x_0))$. Dunque o $h(p(x_0))=y_0=f(x_0)$ o $h(p(-x_0))=-y_0=f(-x_0)$. Perciò $f$ e $h \circ p$ sono sollevamenti che coincidono in un punto e per unicità (e connessione di $S^2$) $f=h \circ p$, assurdo perché $f(x_0) \not=f(-x_0)$ ma $h(p(x_0))=h(p(-x_0))$.
\end{proof}

\begin{cor}
  Sia $f: S^2 \longrightarrow \mathbb{R}^2$ continua, allora esiste $x \in S^2$ t.c. $f(x)=f(-x)$.
\end{cor}

\begin{proof}
  Se $f(x) \not=f(-x)$ per ogni $x \in S^2$, $g:S^2 \longrightarrow S^1, g(x)=\dfrac{f(x)-f(-x)}{\|f(x)-f(-x)\|}$ è ben definita e continua e t.c. $g(x)=-g(-x)$, assurdo.
\end{proof}

\begin{cor}
  In ogni dato istante, esistono due punti antipodali sulla superficie della Terra che hanno uguali temperatura e pressione.
\end{cor}

\begin{proof}
  Assumiamo che la superficie terrestre sia una sfera e che temperatura e pressione siano funzioni continue del punto in cui vengono calcolate. Allora basta applicare il corollario precedente alla funzione che associa a ogni punto sulla superficie la coppia (temperatura, pressione).
\end{proof}

\begin{defn}
  Dati $p_1:E_1 \longrightarrow X, p_2:E_2 \longrightarrow X$ rivestimenti, un \textsc{morfismo} tra $p_1$ e $p_2$ è una mappa continua $\varphi:E_1 \longrightarrow E_2$ t.c. $p_2 \circ \varphi=p_1$. Il seguente diagramma commuta:
  \begin{center}
    \begin{tikzcd}
      E_1 \arrow[rr, "\varphi"] \arrow[rd, "p_1"] & & E_2 \arrow[ld, "p_2"]\\
      & X
    \end{tikzcd}
  \end{center}
  $\varphi$ è un isomorfismo se e solo se esiste $\psi:E_2 \longrightarrow E_1$ continua con $\varphi \circ \psi=\id_{E_2}, \psi \circ \varphi=\id_{E_1}$.
\end{defn}

Composizione di morfismi è un morfismo e l'identità è un morfismo, per cui $\text{Aut}(E)=\text{Aut}(E, p)=\{\varphi:E \longrightarrow E \mid \varphi \text{ è un isomorfismo}\}$ è un gruppo con la composizione.

Fissiamo un rivestimento connesso $p:E \longrightarrow X$.

\begin{thm}
  \begin{nlist}
    \item $\text{Aut}(E)$ agiscono su $E$ in maniera propriamente discontinua (in particolare, libera);
    \item $\text{Aut}(E)$ agisce sulle fibre di $E$;
    \item se $F$ è una fibra di $E$ e $\tilde{x}_0, \tilde{x}_1 \in F$, esiste $\varphi \in \text{Aut}(E)$ con $\varphi(\tilde{x}_0)=\tilde{x}_1$ $\iff$ $p_{\star}(\pi_1(E, \tilde{x}_0))=p_{\star}(\pi_1(E, \tilde{x}_1))$
    (entrambi sottogruppi di $\pi_1(X, p(\tilde{x}_0))=\pi_1(X, p(\tilde{x}_1))$).
  \end{nlist}
\end{thm}

\begin{proof}
  \begin{nlist}
    \item Dato $\tilde{x} \in E$, sia $U$ un intorno ben rivestito connesso per archi di $p(\tilde{x}) \in X$. $\displaystyle p^{-1}(U)=\bigsqcup_{i \in I} V_i$ e sia $i_0$ t.c. $\tilde{x} \in V_{i_0}$.
    Basta vedere che se $\varphi \in \text{Aut}(E)$ e $\varphi(V_{i_0})\cap V_{i_0} \not=\emptyset$, allora $\varphi=\id$. Se $\varphi(V_{i_0})\cap V_{i_0} \not=\emptyset$, esiste $z \in V_{i_0}$ t.c. $\varphi(z) \in V_{i_0}$.
    Poiché $p \circ \varphi=p$, $p(\varphi(z))=p(z)$, ma $p\restrict{V_{i_0}}$ è iniettiva $\implies$ $\varphi(z)=z$. Ora, sia $\varphi$ sia $\id$ sono sollevamenti di $p$ (a partire da $E$) e coincidono in $z$. Per unicità del sollevamento, $\varphi=\id$.
    \item Se $F=p^{-1}(x_0)$ e $\tilde{x} \in F$, $p(\varphi(\tilde{x}))=p(\tilde{x})=x_0$ $\implies$ $\varphi(\tilde{x}) \in F$.
    \item ($\implies$) Se $\varphi \in \text{Aut}(E)$ verifica $\varphi(\tilde{x}_0)=\tilde{x}_1$, poiché $p \circ \varphi=p$, $p_{\star} \circ \varphi_{\star}=p_{\star}$ come mappe da $\pi_1(E, \tilde{x}_0)$ in $\pi_1(X, x_0)$.
    In particolare, $p_{\star}(\varphi_{\star}(\pi_1(E, \tilde{x}_0)))=p_{\star}(\pi_1(E, \tilde{x}_0))$, ma $\varphi_{\star}(\pi_1(E, \tilde{x}_0))=\pi_1(E, \tilde{x}_1)$ perché $\varphi:E \longrightarrow E$ è un omeomorfismo, dunque $\varphi_{\star}:\pi_1(E, \tilde{x}_0) \longrightarrow \pi_1(E, \tilde{x}_1)$ è un isomorfismo.

    ($\Leftarrow$) Se $p_{\star}(\pi_1(E, \tilde{x}_0))=p_{\star}(\pi_1(E, \tilde{x}_1))$, applichiamo il teorema di sollevamento al seguente diagramma:
    \begin{center}
      \begin{tikzcd}
        & (E, \tilde{x}_1) \arrow[d, "p"]\\
        (E, \tilde{x}_0) \arrow[r, "p"] & (X, x_0)
      \end{tikzcd}
    \end{center}
    Applichiamolo prima considerando come rivestimento quello a partire da $(E, \tilde{x}_1)$ per ottenere $\varphi:E \longrightarrow E$ con $\varphi(\tilde{x}_0)=\tilde{x}_1$ e $p \circ \varphi=p$. Analogamente, considerando invece $(E, \tilde{x}_0)$, troviamo $\psi:E \longrightarrow E$ con $\psi(\tilde{x}_1)=\tilde{x}_0$ e $p \circ \psi=p$.
    Ora $\varphi \circ \psi$ e $\psi \circ \varphi$ sollevano $p$, così come l'identità, e sono t.c. $\varphi(\psi(\tilde{x}_1))=\tilde{x}_1$ e $\psi(\varphi(\tilde{x}_0))=\tilde{x}_0$, per cui per unicità dei sollevamenti $\varphi \circ \psi=\psi \circ \varphi=\id_E$.
  \end{nlist}
\end{proof}

\begin{thm} \label{mon_comm}
  Sia $F=p^{-1}(x_0)$. Per ogni $\varphi \in \text{Aut}(E), \alpha \in \pi_1(X, x_0), \tilde{x} \in F$ si ha $\varphi(\tilde{x} \cdot \alpha)=\varphi(\tilde{x}) \cdot \alpha$.
\end{thm}

\begin{proof}
  Sia $\alpha=[\gamma]$.
  Allora $\varphi \circ \tilde{\gamma}_{\tilde{x}}$ è un sollevamento di $\gamma$ ($p \circ \varphi \circ \tilde{\gamma}_{\tilde{x}}=p \circ \tilde{\gamma}_{\tilde{x}}=\gamma$) e ha come punto iniziale $\varphi(\tilde{\gamma}_{\tilde{x}}(0))=\varphi(\tilde{x})$.
  Dunque $\varphi \circ \tilde{\gamma}_{\tilde{x}}=\tilde{\gamma}_{\varphi(\tilde{x})}$.
  Quindi $\varphi(\tilde{x}) \cdot \alpha=\tilde{\gamma}_{\varphi(\tilde{x})}(1)=(\varphi \circ \tilde{\gamma}_{\tilde{x}})(1)=\varphi(\tilde{\gamma}_{\tilde{x}}(1))=\varphi(\tilde{x} \cdot \alpha)$.
\end{proof}

\begin{defn}
  $p:E \longrightarrow X$ rivestimento connesso si dice \textsc{regolare} se per ogni fibra $F \subseteq E$ l'azione di $\text{Aut}(E)$ su $F$ è transitiva.
\end{defn}

\begin{thm}
  Sono fatti equivalenti:
  \begin{nlist}
    \item $p$ è regolare;
    \item esiste una fibra $F \subseteq E$ t.c. l'azione di $\text{Aut}(E)$ su $F$ sia transitiva;
    \item per ogni $\tilde{x} \in E$, $p_{\star}(\pi_1(E, \tilde{x})) \vartriangleleft \pi_1(X, p(\tilde{x}))$;
    \item esiste $\tilde{x} \in E$ t.c. $p_{\star}(\pi_1(E, \tilde{x})) \vartriangleleft \pi_1(X, p(\tilde{x}))$.
  \end{nlist}
\end{thm}

\begin{proof}
  (iii) $\implies$ (iv) è ovvia.

  (iv) $\implies$ (iii). Supponiamo che valga per $\tilde{x} \in E$ e sia $\tilde{y} \in E, \tilde{\gamma} \in \Omega(\tilde{x}, \tilde{y})$. Sia $\gamma=p \circ \tilde{\gamma}$, $\gamma \in \Omega(x, y)$, $x=p(\tilde{x}), y=p(\tilde{y})$.
  Abbiamo due isomorfismi $\tilde{\gamma}_{\sharp}: \pi_1(E, \tilde{x}) \longrightarrow \pi_1(E, \tilde{y})$ e $\gamma_{\sharp}:\pi_1(X, x) \longrightarrow \pi_1(X, y)$. Si vede che il seguente diagramma commuta:
  \begin{center}
    \begin{tikzcd}
      \pi_1(E, \tilde{x}) \arrow[r, "\tilde{\gamma}_{\sharp}"] \arrow[d, "p_{\star}"] & \pi_1(E, \tilde{y}) \arrow[d, "p_{\star}"]\\
      \pi_1(X, x) \arrow[r, "\gamma_{\sharp}"] & \pi_1(X, y)
    \end{tikzcd}
  \end{center}
  Dunque $p_{\star}(\pi_1(E, \tilde{x})) \vartriangleleft \pi_1(X, x) \iff p_{\star}(\pi_1(E, \tilde{y})) \vartriangleleft \pi_1(X, y)$.

  Per concludere, basta vedere che se $F=p^{-1}(x)$ è una fibra e $\tilde{x} \in F$, allora l'azione di $\text{Aut}(E)$ su $F$ è transitiva $\iff$ $p_{\star}(\pi_1(E, \tilde{x})) \vartriangleleft \pi_1(X, x)$.
  Dati $\tilde{x}, \tilde{y} \in F$, abbiamo visto che esiste $\varphi \in \text{Aut}(E)$ con $\varphi(\tilde{x})=\tilde{y}$ $\iff$ $p_{\star}(\pi_1(E, \tilde{x}))=p_{\star}(\pi_1(E, \tilde{y}))$.
  Abbiamo anche visto che al variare di $\tilde{y} \in F$ i gruppi $p_{\star}(\pi_1(E, \tilde{y}))$ sono tutti e soli i coniugati di $p_{\star}(\pi_1(E, \tilde{x}))$. Dunque l'azione su $F$ è transitiva $\iff$ $p_{\star}(\pi_1(E, \tilde{x}))$ coincide con tutti i suoi coniugati, cioè se è normale.
\end{proof}

\begin{thm}
  Sia $p:E \longrightarrow X$ regolare. Allora
  $$\text{Aut}(E) \cong \faktor{\pi_1(X, x)}{p_{\star}(\pi_1(E, \tilde{x}))}$$
  dove l'isomorfismo è un isomorfismo di gruppi e $x=p(\tilde{x})$. Ciò vale per ogni $\tilde{x} \in E$.
\end{thm}

\begin{proof}
  Ricordiamo che $p_{\star}(\pi_1(E, \tilde{x})) \vartriangleleft \pi_1(X, x)$, dunque in effetti il quoziente è un gruppo.
  Sia $\vartheta:\pi_1(X, x) \longrightarrow \text{Aut}(E)$ t.c. per ogni $\alpha \in \pi_1(X, x)$ $\vartheta(\alpha)$ sia l'unico elemento di $\text{Aut}(E)$ t.c. $\vartheta(\alpha)(\tilde{x})=\tilde{x} \cdot \alpha$.
  Tale $\vartheta(\alpha)$ esiste perché il rivestimento è regolare ed è unico perché $\text{Aut}(E)$ agisce liberamente su $E$.
  \begin{nlist}
    \item $\vartheta$ è un omomorfismo: ($\vartheta(\alpha_1) \circ \vartheta(\alpha_2))(\tilde{x})=\vartheta(\alpha_1)(\vartheta(\alpha_2)(\tilde{x}))=\vartheta(\alpha_1)(\tilde{x} \cdot \alpha_2)$,
    ma per il teorema \ref{mon_comm} $\vartheta(\alpha_1)(\tilde{x} \cdot \alpha_2)=(\vartheta(\alpha_1)(\tilde{x}))\cdot\alpha_2=(\tilde{x}\cdot\alpha_1)\cdot\alpha_2)=\tilde{x}\cdot(\alpha_1\cdot\alpha_2)=\vartheta(\alpha_1 \cdot \alpha_2)(\tilde{x}))$
    $\implies$ $\vartheta(\alpha_1)\circ\vartheta(\alpha_2)=\vartheta(\alpha_1 \cdot \alpha_2)$.
    \item $\vartheta$ è suriettivo. Se $\varphi \in \text{Aut}(E)$, poiché l'azione di monodromia è transitiva, esiste $\alpha \in \pi_1(X, x)$ con $\tilde{x} \cdot \alpha=\varphi(\tilde{x})$. Dunque $\vartheta(\alpha)=\varphi$ in quanto $\vartheta(\alpha)$ e $\varphi$ coincidono su $\tilde{x}$ e sono perciò uguali (sempre perché $\text{Aut}(E)$ agiscono liberamente).
    \item $\ker{\vartheta}=p_{\star}(\pi_1(E, \tilde{x}))$. Infatti, poiché $\text{Aut}(E)$ agisce liberamente, $\vartheta(\alpha)=\id \iff \vartheta(\alpha)(\tilde{x})=\tilde{x} \iff \tilde{x} \cdot \alpha=x \iff \alpha \in p_{\star}(\pi_1(E, \tilde{x}))$.
  \end{nlist}
\end{proof}

\begin{cor}
  $p:E \longrightarrow X$ regolare $\implies$ $X \cong \faktor{E}{\text{Aut}(E)}$ (cioè sono omeomorfi).
\end{cor}

\begin{proof}
  Essendo aperta e suriettiva, $p:E \longrightarrow X$ è un'identificazione, per cui $X \cong \faktor{E}{\sim}$ dove $\tilde{x} \sim \tilde{y} \iff p(\tilde{x})=p(\tilde{y}) \iff \tilde{x}=\varphi(\tilde{y})$ per qualche $\varphi \in \text{Aut}(E)$, in quanto $\text{Aut}(E)$ agiscono transitivamente sulle fibre.
\end{proof}

Vale una sorta di viceversa.

\begin{prop}
  Se $G$ agisce in maniera propriamente discontinua su uno spazio connesso $E$, allora la proiezione $p:E \longrightarrow \faktor{E}{G}$ è un rivestimento regolare e $G=\text{Aut}(E, p)$.
\end{prop}

\begin{proof}
  Dato $x \in \faktor{E}{G}$, dobbiamo costruire un intorno $U \ni x$ ben rivestito. Scegliamo $\tilde{x} \in E$ con $p(\tilde{x})=x$; per definizione di azione propriamente discontinua, esiste un aperto $V$ di $E$ t.c. $\tilde{x} \in V$ e $\gamma(V)\cap V=\emptyset$ per ogni $\gamma \in G\setminus\{\id\}$. Poniamo $U=p(V)$. Le proiezioni al quoziente per azioni di gruppo sono aperte, per cui $U$ è aperto.
  Per costruzione, $\displaystyle p^{-1}(U)=\bigcup_{\gamma \in G} \gamma(V)$.
  Ogni $\gamma(V)$ è aperto (i $\gamma$ sono omeomorfismi) e l'unione è disgiunta poiché $\gamma_1(V) \cap \gamma_2(V) \not=\emptyset \implies V \cap (\gamma_1^{-1}\gamma_2)(V) \not=\emptyset \implies \gamma_1^{-1}\gamma_2=\id \implies \gamma_1=\gamma_2$.
  Infine, $p\restrict{\gamma(V)}$ è un omeomorfismo su $U$ per ogni $\gamma \in G$ (è continua, aperta e suriettiva; l'iniettività segue ancora dal fatto che $V \cap \gamma(V)=\emptyset$ per ogni $\gamma\not=\id$).
\end{proof}

\begin{oss}
  Un rivestimento universale è regolare. Infatti, $p_{\star}(\{1\})=\{1\} \vartriangleleft \pi_1(X, x)$.
\end{oss}

\begin{cor}
  Se $p:E \longrightarrow X$ è un rivestimento universale, $\text{Aut}(E)\cong\pi_1(X, x)$.
\end{cor}

\begin{cor}
  Se $E$ è semplicemente connesso e $G$ agisce su $E$ in maniera propriamente discontinua, allora $\pi_1(\faktor{E}{G}) \cong G$.
\end{cor}

\begin{ex}
  \begin{nlist}
    \item $S^1=\faktor{\mathbb{R}}{\mathbb{Z}}$, $\mathbb{R}$ semplicemente connesso $\implies$ $\pi_1(S^1)=\mathbb{Z}$.
    \item $\underbrace{S^1 \times S^1 \times \dots \times S^1}_{n \text{ volte}}=\faktor{\mathbb{R}^n}{\mathbb{Z}^n}$, $\mathbb{R}^n$ semplicemente connesso $\implies$ $\pi_1((S^1)^n)=\mathbb{Z}^n$.
    \item $\mathbb{P}^n(\mathbb{R})=\faktor{S^n}{G}, G=\{\pm\id\}$. Se $n \ge 2$, $S^n$ semplicemente connesso $\implies$ $\pi_1(\mathbb{P}^n(\mathbb{R})) \cong G=\mathbb{Z}_2$.
  \end{nlist}
\end{ex}
