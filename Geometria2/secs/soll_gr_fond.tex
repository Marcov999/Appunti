D'ora in poi tutti gli spazi saranno localmente connessi per archi.

\begin{prop}
  Sia $p:E \rightarrow X$ rivestimento con $E$ connesso e siano $\tilde{x}_0 \in p^{-1}(x_0)=F$ e $\psi:\pi_1(X, x_0) \rightarrow F$, $\psi(\alpha)=\tilde{x}_0 \cdot \alpha$.
  $\psi$ induce una bigezione $\mfaktor{p_{\star}(\pi_1(E, \tilde{x}_0))}{\pi_1(X, x_0)} \cong F$.
  In particolare, $stab(\tilde{x}_0)=p_{\star}(\pi_1(E, \tilde{x}_0))$ e, al variare di $\tilde{x} \in F$, i gruppi $p_{\star}(\pi_1(E, \tilde{x}))=stab(\tilde{x})$ sono tutti e soli i coniugati di $p_{\star}(\pi_1(E, \tilde{x}_0))=stab(\tilde{x}_0)$.
\end{prop}

\begin{proof}
  Poiché $E$ è connesso, l'azione di monodronia è transitiva, per cui $\psi$ è suriettiva.
  $\alpha=[\gamma] \in stab{\tilde{x}_0} \iff \tilde{\gamma}_{\tilde{x}_0}(1)=\tilde{x}_0 \iff \tilde{\gamma}_{\tilde{x}_0}$ è un loop in $E$ $\iff$ $[\gamma] \in p_{\star}(\pi_1(E, \tilde{x}_0))$.
  Infatti, $\gamma$ si solleva a un laccio chiuso se e solo se $\tilde{\gamma}_{\tilde{x}_0}$ definisce un elemento di $\pi_1(E, \tilde{x}_0)$.
  Da ciò segue $\psi(\alpha)=\psi(\beta) \iff \tilde{x}_0 \cdot \alpha =\tilde{x}_0 \cdot \beta \iff \tilde{x}_0 \cdot (\alpha \beta^{-1})=\tilde{x}_0 \iff \alpha\beta^{-1} \in stab(\tilde{x}_0) \iff [\alpha]=[\beta]$
  in $\mfaktor{stab(\tilde{x}_0)}{\pi_1(X, x_0)}$, dunque $\psi$ induce la bigezione richiesta.

  Usando il fatto che l'azione è transitiva è facile vedere che $stab(\tilde{x}), \tilde{x} \in F$ sono tutti e solo i coniugati di $stab(\tilde{x}_0)$ (se $\tilde{x} \in F$, $\tilde{x}=\tilde{x}_0 \cdot \eta$ e $stab(\tilde{x})=\eta^{-1}\cdot stab(\tilde{x}_0)\cdot\eta$).
\end{proof}

\begin{thm}
  Teorema di sollevamento. Sia $p:E \rightarrow X$ un rivestimento connesso (cioè $E$ è connesso), $x_0 \in X, \tilde{x}_0 \in p^{-1}(x_0)$ e sia $f:Y \rightarrow X$ continua, $Y$ connesso, $y_0 \in Y$.
  Allora esiste un sollevamento $\tilde{f}:Y \rightarrow E$ di $f$ con $\tilde{f}(y_0)=\tilde{x}_0$ $\iff$ $f_{\star}(\pi_1(Y, y_0)) \subseteq p_{\star} (\pi_1(E, \tilde{x}_0))$.
\end{thm}

\begin{proof}
  Consideriamo il seguente diagramma:
  \begin{center}
    \begin{tikzcd}
            & E \arrow[d, "p"]\\
            Y \arrow[ru, dashed, "\tilde{f}"] \arrow[r, "f"] & X
     \end{tikzcd}
  \end{center}
  ($\implies$) Se $f=p \circ \tilde{f}$, $f_{\star}=p_{\star}=p_{\star} \circ \tilde{f}_{\star}$ (sui $\pi_1$ con i punti base scelti), dunque $\Ima{f_{\star}} \subseteq \Ima{p_{\star}}$.

  ($\Leftarrow$) Definiamo $\tilde{f}$ come segue: se $t \in Y$, scegliamo un cammino $\gamma$ in $Y$ con $\gamma(0)=y_0, \gamma(1)=y$. Poniamo $\tilde{f}(y)=\widetilde{(f \circ \gamma)}_{\tilde{x}_0}(1)$.
  $p(\tilde{f}(y))=p(\widetilde{(f \circ \gamma)}_{\tilde{x}_0}(1))=(f \circ \gamma)_{\tilde{x}_0}(1)=f(y)$ $\implies$ $p \circ \tilde{f}=f$.
  Verifichiamo che $\tilde{f}$ sia ben definita: se $\beta$ è un altro cammino da $y_0$ a $y$, $\gamma \sim \gamma * \bar{\beta}*\beta$ $\implies$ $f \circ \gamma \sim f \circ (\gamma * \bar{\beta}*\beta)=(f \circ (\gamma * \bar{\beta}))*(f \circ \beta)$,
  ma adesso notiamo che $\gamma *\bar{\beta}=\alpha$ è un loop in $y_0$, quindi $f \circ \alpha$ è un loop in $x_0$ e $[f \circ \alpha] \in f_{\star}(\pi_1(Y, y_0)) \subseteq p_{\star}(\pi_1(E, \tilde{x}_0))$, dunque $f \circ \alpha$ si solleva a un loop in $E$ a partire da $\tilde{x}_0$.
  Perciò $\widetilde{(f \circ \gamma)}_{\tilde{x}_0}(1)=\widetilde{((f \circ \alpha)*(f \circ \beta))}_{\tilde{x}_0}(1)=\widetilde{(f \circ \alpha)}_{\tilde{x}_0}*\widetilde{(f \circ \beta)}_{\widetilde{(f \circ \alpha)}_{\tilde{x}_0}(1)}(1)=\widetilde{(f \circ \alpha)}_{\tilde{x}_0}*\widetilde{(f \circ \beta)}_{\tilde{x}_0}(1)=\widetilde{(f \circ \beta)}_{\tilde{x}_0}(1)$
  $\implies$ $\tilde{f}$ è ben definita.

  Continuità di $\tilde{f}$: dato $y \in Y$, sia $U$ un intorno aperto connesso per archi e ben rivestito di $f(y)$ in $X$ e sia $W=f^{-1}(U)$, che è aperto. $\displaystyle p^{-1}=\bigsqcup_{i \in I} V_i$ con i $V_i$ aperti in $E$. Scegliamo $i_0$ t.c. $\tilde{f}(y) \in V_{i_0}$. Sia anche $s:U \rightarrow V_{i_0}$ l'inversa continua di $p\restrict{V_{i_0}}$.
  Basta vedere che $\tilde{f}\restrict{W}=s \circ f\restrict{W}$, così avremmo $\tilde{f}\restrict{W}$ continua, dunque $\tilde{f}$ continua. Sia $\gamma \in \Omega(y_0, y)$ e per ogni $z \in U$ sia $\gamma_z \in \Omega(y, z)$. Scegliamo $\alpha_z=\gamma*\gamma_z$.
  $\tilde{f}(z)=\widetilde{(f \circ \alpha_z)}_{\tilde{x}_0}(1)=\widetilde{((f \circ \gamma)*(f \circ \gamma_z))}_{\tilde{x}_0}(1)=\widetilde{(f \circ \gamma)}_{\tilde{x}_0}*\widetilde{(f \circ \gamma_z)}_{\widetilde{(f \circ \gamma)}_{\tilde{x}_0}(1)}(1)=\widetilde{(f \circ \gamma)}_{\tilde{x}_0}*\widetilde{(f \circ \gamma_z)}_{\tilde{f}(y)}(1)=\widetilde{(f \circ \gamma_z)}_{\tilde{f}(y)}(1)$.
  Adesso notiamo che $s \circ f \circ \gamma_z$ solleva $f \circ \gamma_z$ a partire da $\tilde{f}(y)$, dunque per unicità del sollevamento dev'essere uguale a $\widetilde{(f \circ \gamma_z)}_{\tilde{f}(y)}$, perciò $\widetilde{(f \circ \gamma_z)}_{\tilde{f}(y)}(1)=(s \circ f \circ \gamma_z)(1)=s(f(z))$.
\end{proof}

\begin{cor}
  Siano $p:E \rightarrow X$ rivestimento connesso, $x_0 \in X$, $\tilde{x}_0 \in p^{-1}(x_0)$, $f:Y \rightarrow X$ con $f(y_0)=x_0$. Se $Y$ è semplicemente connesso, esiste un unico sollevamente $\tilde{f}:Y \rightarrow E$ di $f$ con $\tilde{f}(y_0)=\tilde{x}_0$.
\end{cor}

\begin{thm}
  (Borsuk-Ulam) Non esistono mappe continue $f:S^2 \rightarrow S^1$ t.c. $f(-x)=-f(x)$.
\end{thm}

\begin{proof}
  Per assurdo, sia $f:S^2 \rightarrow S^1$ continua t.c. $f(-x)=-f(x)$. Siano $p:S^2 \rightarrow \mathbb{P}^2(\mathbb{R}), q:S^1 \rightarrow \mathbb{P}^1(\mathbb{R})$ le proiezioni. Abbiamo un diagramma commutativo di funzioni continue
  \begin{center}
    \begin{tikzcd}
            S^2 \arrow[d, "p"] \arrow[r, "f"] & S^1 \arrow[d, "q"]\\
            \mathbb{P}^2(\mathbb{R}) \arrow[r, "\bar{f}"] & \mathbb{P}^1(\mathbb{R})
     \end{tikzcd}
   \end{center}
  Si noti che il diagramma commuta perché $q(f(x_0))=q(f(-x_0))$. Sappiamo che $\pi_1(\mathbb{P}^2(\mathbb{R}))=\mathbb{Z}_2$ e dato che $\mathbb{P}^1(\mathbb{R}) \cong S^1$ abbiamo che $\pi_1(\mathbb{P}^1(\mathbb{R}))=\mathbb{Z}$.
  Allora $\bar{f}_{\star}:\pi_1(\mathbb{P}^2(\mathbb{R})) \rightarrow \pi_1(\mathbb{P}^1(\mathbb{R}))$ è la mappa banale.
  \begin{center}
    \begin{tikzcd}
            & & S^1 \arrow[d, "q"]\\
            S^2 \arrow[rru, bend left, "f"] \arrow[r, "p"] & \mathbb{P}^2(\mathbb{R}) \arrow[ru, "h"] \arrow[r, "\bar{f}"] & \mathbb{P}^1(\mathbb{R})
    \end{tikzcd}
  \end{center}
  Per il teorema di sollevamento, esiste $h:\mathbb{P}^2(\mathbb{R}) \rightarrow S^1$ con $q \circ h=\bar{f}$ (attenzione: non sappiamo che $h \circ p=f$). Scegliamo $x_0 \in S^2$. Sappiamo che $q(h(p(x_0)))=\bar{f}(p(x_0))=q(f(x_0))$, dunque $h \circ p$ e $f$ sono entrambi sollevamenti di $\bar{f} \circ p$.
  Dato $x_0 \in S^2$, sia $z_0 \in \mathbb{P}^1(\mathbb{R}), z_0=\bar{f}(p(x_0))$. $q^{-1}(z_0)=\{y_0, -y_0\}$. Dunque $f(x_0) \in \{y_0, -y_0\}$. Supponiamo $f(x_0)=y_0$. Allora $f(-x_0)=-f(x_0)=-y_0$.
  Anche $h(p(x_0)), h(p(-x_0)) \in \{y_0, -y_0\}$, ma $p(x_0)=p(-x_0) \implies h(p(x_0))=h(p(-x_0))$. Dunque o $h(p(x_0))=y_0=f(x_0)$ o $h(p(-x_0))=-y_0=f(-x_0)$. Perciò $f$ e $h \circ p$ sono sollevamenti che coincidono in un punto e per unicità (e connessione di $S^2$) $f=h \circ p$, assurdo perché $f(x_0) \not=f(-x_0)$ ma $h(p(x_0))=h(p(-x_0))$.
\end{proof}

\begin{cor}
  Sia $f: S^2 \rightarrow \mathbb{R}^2$ continua, allora esiste $x \in S^2$ t.c. $f(x)=f(-x)$.
\end{cor}

\begin{proof}
  Se $f(x) \not=f(-x)$ per ogni $x \in S^2$, $g:S^2 \rightarrow S^1, g(x)=\dfrac{f(x)-f(-x)}{\|f(x)-f(-x)\|}$ è ben definita e continua e t.c. $g(x)=-g(-x)$, assurdo.
\end{proof}

\begin{cor}
  In ogni dato istante, esistono due punti antipodali sulla superficie della Terra che hanno uguali temperatura e pressione.
\end{cor}

\begin{proof}
  Assumiamo che la superficie terrestre sia una sfera e che temperatura e pressione siano funzioni continue del punto in cui vengono calcolate. Allora basta applicare il corollario precedente la funzione che associa a ogni punto sulla superficie la coppia (temperatura, pressione).
\end{proof}
