\begin{defn} 
	Sia $\{X_\alpha\}_{\alpha \in A}$ una famiglia di spazi topologici. Allora
	il \textsc{prodotto cartesiano} della famiglia è:
	$$X:=\prod_{\alpha \in A} X_\alpha = \{f:A \longrightarrow \bigcup _{\alpha
	\in A} X_\alpha  \mid f(\alpha)\in X_\alpha, \forall \alpha \in A\} $$
	Se $A=\{1, \dots ,n\}$, allora:
	$$\prod_{i=1}^n X_i=X_1 \times \cdots \times X_n \ni (x_1, \dots ,x_n)
	\qquad x_i:=f(i)$$
\end{defn}
\begin{defn}
	La \textsc{topologia prodotto} su $X=\begin{matrix} \prod_{i\alpha \in A}
	X_\alpha \end{matrix}	$ è la topologia meno fine su $X$ che rende
	continue tutte le proiezioni $p_\alpha : X \longrightarrow X_\alpha$.
\end{defn}
\begin{oss}
	$X$ viene naturalmente con proiezioni:
	$$p_\alpha :X \longrightarrow X_\alpha \qquad f \longmapsto f(\alpha) \qquad
	\forall \alpha \in A$$
	Inoltre questa topologia è ben definita poiché l'intersezione di topologie è
	una topologia.
\end{oss}
\begin{prop}
	Una base per la topologia prodotto su $X$ è data da:
	$$\mathcal{B}=\left \{\prod_{\alpha \in A} U_\alpha \mid U_\alpha \subset
	X_\alpha \text{ aperto }, U_\alpha =X_\alpha \text{ tranne al più 		un
	numero finito di }\alpha \right \}$$
\end{prop}
\begin{proof}
	Sia $U_\alpha \subset X_\alpha$ aperto. Allora:
	$$p_\alpha ^{-1} (U_\alpha)=\prod_{\beta \in A} V_\beta \qquad
	\text{dove}\qquad V_\beta =\begin{cases}U_\alpha & \text{se }\beta \alpha \\
	X_\beta & \text{se } \beta \ne \alpha \end{cases}$$
	Dunque le proiezioni $p_\alpha$ sono continue se e solo se tutte le
	preimmagini di tale forma sono aperte nella topologia prodotto.
    Quindi:
	$$p_\alpha ^{-1} (U_\alpha) \subset X \text{ aperto } \qquad \forall \text{
	aperto } U_\alpha \subset X_\alpha$$
	Siano $U_{\alpha _1} \subset X_{\alpha _1} , \dots , U_{\alpha _n} \subset
	X_{\alpha _n}$ aperti, con $\alpha _1, \dots, \alpha _n \in A$. Allora:
	$$X \supset \bigcap _{i=1}^n p_{\alpha _I}^{-1}(U_{\alpha _i}) = \prod
	_{\alpha \in A} V_\alpha \qquad \text{dove}\qquad V_\alpha =
	\begin{cases}U_{\alpha _i} & \text{se }\alpha =\alpha  _i \\ X_\alpha &
	\text{se } \alpha \ne \alpha _i \end{cases}$$
	Dunque ogni elemento di $\mathcal{B}$ è aperto nella topologia prodotto su
	$X$ (infatti l'intersezione di aperti è un aperto). Se $ \mathcal{B}$ è
	base di una topologia abbiamo finito, in quanto per definizione la topologia
	prodotto è la meno fine che rende continue tutte le $p_\alpha$.
	Altrimenti:
	$$\left( \prod_{\alpha \in A} U_\alpha \right) \bigcap \left( \prod_{\alpha
	\in A} V_\alpha \right)=\prod _{\alpha \in A} (U_\alpha \cap V_\alpha)$$
	Dunque le intersezioni di elementi di $\mathcal{B}$ sono ancora in
	$\mathcal{B}$, che quindi è chiuso rispetto alle intersezioni finite, e
	perciò è una base.
\end{proof}
\begin{oss}
	Supponiamo $\mathcal{B}_\alpha$ base della topologia di $X_\alpha$.
	Definiamo:
	$$\mathcal{B}'=\left \{ \prod_{\alpha \in A} B_\alpha \mid B_\alpha \in
	\mathcal{B}_\alpha, B_\alpha =X_\alpha \text{ tranne al più un numero
	finito di } \alpha \right \}$$
	Allora $\mathcal{B}'$ è una base per la topologia prodotto in $X$.
\end{oss}
\begin{cor}
	Supponiamo $A$ numerabile e $\mathcal{B}_\alpha$ base numerabile per
	$X_\alpha$ per ogni $\alpha$. Allora $\mathcal{B}'$ è una base numerabile
	per $X$.
\end{cor}
\begin{proof}
	Per esercizio.
\end{proof}
\begin{cor}
	(analogo al precedente) Supponiamo $A$ numerabile e che tutti gli $X_\alpha$
	siano primo-numerabili. Allora la topologia prodotto è	primo-numerabile.
\end{cor}
\begin{oss}
	In generale, se $A$ non è numerabile i due corollari sono falsi.
\end{oss}
\begin{ex}
	Prendiamo $\mathbb{R}$ con la topologia euclidea, e consideriamo:
	$$\mathbb{R}^n=\underbrace{\mathbb{R} \times \cdots \times 	\mathbb{R}}_
	{n\text{ volte}}$$
	Esso eredita una topologia prodotto che coincide con la topologia euclidea
	su $\mathbb{R}^n$. In particolare, la topologia euclidea ha per base le
	palle aperte. La topologia prodotto ha per base i parallelepipedi retti
	aperti:
	$$(a_1,b_1)\times \cdots \times (a_n,b_n)$$
	In effetti tali parallelepipedi sono aperti nella topologia euclidea (e
	viceversa).
\end{ex}
