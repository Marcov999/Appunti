\begin{defn}
    $f: (X, \tau) \rightarrow (Y, \tau')$ si dice \textsc{continua} se
    ${f^{-1}(A) \in \tau,}\; {\forall A \in \tau'}$, cioè una funzione \`e
    continua se la controimmagine di aperti \`e aperta.
\end{defn}

Notiamo che per il teorema \ref{thm:cont_inv}, le due
definizioni di funzione continua sono equivalenti per uno spazio metrico se si
considera la topologia indotta dalla metrica.

\begin{thm}
    \begin{nlist}
        \item L'identità è una funzione continua;
        \item composizione di funzioni continue è una funzione continua.
    \end{nlist}
\end{thm}

\begin{defn}
    Una funzione $f: X \rightarrow Y$ si dice \textsc{omeomorfismo} se $f$ è
    continua e esiste una funzione $g:Y\rightarrow X$ continua tale che $f \circ
    g = Id_Y$ e $g \circ f = Id_x$. Cio\`e $f$ \`e continua, bigettiva e con
    inversa continua.
\end{defn}

\begin{oss}
	\begin{nlist}
	\item Composizione di omeomorfismi è un omeomorfismo; due spazi legati da un
	omeomorfismo si dicomo \textsc{omeomorfi} e essere omeomorfi è una relazione
	di equivalenza;
	\item l'insieme degli omeomorfismi da $(X, \tau)$ in sé è un gruppo;
	\item se $f:X \rightarrow Y$ è continua e bigettiva non è detto che sia un
	omeomorfismo, cioè $f^{-1}$ può non essere continua.
    \marginpar{\warningsign}
\end{nlist}
\end{oss}

\begin{ex}
	Siano $\tau_E$ la topologia euclidea, $\tau_C$ la cofinita, $\tau_D$ la
	discreta e $\tau_I$ l'indiscreta. Le seguenti mappe sono dunque continue:
    \begin{itemize}
	\item $Id:(\mathbb{R}, \tau_D) \rightarrow (\mathbb{R}, \tau_E)$,
    \item $Id:(\mathbb{R}, \tau_E) \rightarrow (\mathbb{R}, \tau_C)$,
    \item $Id:(\mathbb{R}, \tau_C) \rightarrow (\mathbb{R}, \tau_I)$.
    \end{itemize}
    Nessuna delle inverse è però continua. Più in generale, $Id: (X, \tau)
    \rightarrow (X, \sigma)$ con $\sigma \subsetneq \tau$ è continua ma
    l'inversa no.
\end{ex}

\begin{exc}
	Il seguente esercizio è frutto di una domanda fatta da uno studente a
	lezione e potrebbe essere più difficile di altri esercizi del corso.

    \marginpar{\warningsign}
	Trovare un esempio (o dimostrare che non esiste) di uno spazio topologico
	$X$ e una funzione $f: (X, \tau) \rightarrow (X, \tau)$ continua e bigettiva
	con inversa non continua.

    Stando a quanto dice Frigerio, probabilmente tale funzione esiste,
    euristicamente perché non c'è un modo facile di dimostrare il contrario.
\end{exc}

\begin{sol}
	La seguente soluzione è un esempio mostrato a lezione da Gandini. Vedi te a
	volte il caso.

	Mettiamo su $\mathbb{Z}$ la seguente topologia: $A \subseteq \mathbb{Z}$ è
	aperto se $A=\mathbb{Z}$ o $A \subseteq \mathbb{N}$. È semplice verificare
	che è una topologia. Allora la funzione $f: \mathbb{Z} \rightarrow
	\mathbb{Z}$ t.c. $f(n)=n-1$ è l'esempio cercato. Banalmente è bigettiva, è
	semplice verificare che è continua ma l'inversa no.
\end{sol}

Introduciamo adesso un concetto che permetterà di caratterizzare la continuità
in spazi topologici in modo analogo a quanto fatto per gli spazi metrici.

\begin{defn}
	Sia $(X, \tau)$ uno spazio topologico fissato e $x_0 \in X$. Un insieme $U
	\subseteq X$ è un \textsc{intorno} di $x_0$ se $x_0 \in
	U^{\circ}$, o equivalentemente se esiste $V$ aperto con $x_0 \in V
	\subseteq U$. L'insieme degli intorni di $x_0$ si denota con
	$\mathcal{I}(x_0)$.
\end{defn}

\begin{defn}
	$f:(X, \tau) \rightarrow (Y, \tau')$ è detta \textit{continua in $x_0$} se
	per ogni intorno $U$ di $f(x_0)$ esiste un intorno $V$ di $x_0$ t.c. $f(V)
	\subseteq U$.
\end{defn}

Appare dunque intuitivo il seguente risultato.

\begin{thm}
	$f$ è continua $\Leftrightarrow$ è continua in ogni $x_0 \in X$.
\end{thm}

\begin{proof}
	($\implies$) Supponiamo $f$ continua e $x_0 \in X$, sia inoltre $U$ un
	intorno di $f(x_0)$. Per definizione di intorno esiste un aperto $A$ con
	$f(x_0) \in A \subseteq U$, perciò $x_0 \in f^{-1}(A) \subseteq f^{-1}(U)$,
	ma poiché $f$ è continua abbiamo che $f^{-1}(A)$ è ancora un aperto, perciò
	ponendo $V=f^{-1}(U)$ abbiamo che $V$ è un intorno di $x_0$ t.c. $f(V)
	\subseteq U$, che è quello che volevamo.

	($\Leftarrow$) Supponiamo $f$ continua in ogni punto di $X$ e sia $A$ un
	insieme aperto in $Y$. Per ogni $x \in f^{-1}(A)$, $A$ è un intorno di
	$f(x)$. Ma dato che $f$ è continua in ogni punto di $X$, esiste un intorno
	$V_x$ di $x$ t.c. $x \in V_x \subseteq f^{-1}(A)$. Per definizione di
	intorno, ciò significa che esiste un aperto $A_x$ di $X$ t.c. $x \in A_x
	\subseteq V \subseteq f^{-1}(A)$. Dunque dev'essere $\displaystyle
	f^{-1}(A)= \bigcup_{x \in f^{-1}(A)} A_x$,
	quindi $f^{-1}(A)$ è un aperto in $X$ per ogni $A$ aperto in $Y$, il che
	equivale a dire che $f$ è continua.
\end{proof}
