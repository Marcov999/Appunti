Prima di proseguire verso il teorema di Tychonoff è necessario introdurre alcuni concetti basilari della teoria degli insiemi e un risultato molto importante, il lemma di Zorn.

\begin{defn}
  Un insieme $A$ si dice \textit{(parzialmente) ordinato} se esiste su di esso una relazione d'ordine $\le$ tale che sia:
  \begin{nlist}
    \item riflessiva, cioè $a \le a$ per ogni $a \in A$;
    \item antisimmetrica, cioè $a \le b \land b \le a \implies a=b$;
    \item transitiva, cioè $a \le b \land b \le c \implies a \le c$.
  \end{nlist}
  Talvolta, per indicare $a \le b$, scriveremo anche $b \ge a$.
  Un insieme ordinato $A$ si dice \textit{totalmente ordinato} se per ogni $a, b \in A$ vale $a \le b$ o $b \le a$.
\end{defn}

\begin{defn}
  Sia $A$ un insieme parzialmente ordinato, un sottoinsieme $C \subseteq A$ è detto \textit{catena} se $C$ è un insieme totalmente ordinato con la relazione d'ordine indotta da $A$.
\end{defn}

\begin{defn}
  Sia $B \subseteq A$, un elemento $m \in A$ si dice \textit{maggiorante} di $B$ se $b \le m$ per ogni $b \in B$.
\end{defn}

\begin{defn}
  Sia $A$ un insieme parzialmente ordinato. Un elemento $m \in A$ si dice maggiorante se, per ogni $a \in A$, $m \le a \implies a=m$.
\end{defn}

Tra gli assiomi dai quali è definita tutta la matematica è comunemente accettato l'inserimento dell'assioma della scelta, che informalmente dice che il prodotto di un insieme di insiemi, indicizzati secondo un altro insieme, è non vuoto. Formalmente, esiste una funzione da un insieme di indici all'unione di insiemi indicizzati da quell'insieme tale che l'immagine di ogni indice appartiene all'insieme indicizzato dall'indice stesso. L'assioma della scelta implica (anzi, è equivalente a) il seguente lemma.

\begin{lm}
  Lemma di Zorn. Sia $A$ un insieme parzialmente ordinato tale che ogni sua catena possiede un maggiorante. Allora $A$ ha almeno un elemento massimale.
\end{lm}

Siamo adesso pronti per tornare alla topologia.

\begin{thm} \label{alexander}
  Teorema di Alexander. Sia $X$ spazio topologico, $\mathcal{P}$ una sua prebase. Se da ogni ricorpimento di $X$ con elementi di $\mathcal{P}$ si può estrarre un sottoricoprimento finito, allora $X$ è compatto.
\end{thm}

\begin{proof}
  Supponiamo per assurdo che $X$ non sia compatto e sia $\Omega$ l'insieme dei ricoprimenti aperti da cui non è possibile estrarre sottoricoprimenti finiti, ordinato con l'inclusione tra insiemi. L'assurdo implica $\Omega \not= \emptyset$. Mostriamo che $\Omega$ soddisfa le ipotesi del lemma di Zorn.
  Sia $\Gamma \subseteq \Omega$ una catena.
  Consideriamo l'insieme $\displaystyle \mathcal{M}=\bigcup_{\mathcal{C} \in \Gamma} \mathcal{C}$, vogliamo dire che $\mathcal{M}$ è il maggiorante di $\Gamma$ cercato. Per definizione $\mathcal{M} \supseteq \mathcal{C}$ per ogni $\mathcal{C} \in \Gamma$. Ci manca da mostrare che $\mathcal{M} \in \Omega$.
  Supponiamo per assurdo che $\mathcal{M} \not\in \Omega$. Ovviamente $\mathcal{M}$ è un ricoprimento aperto di $X$ in quanto unione di ricoprimenti aperti. Se assumiamo l'assurdo, allora esiste un sottoricoprimento finito $\mathcal{A}=\{A_1, \dots, A_n\} \subseteq \mathcal{M}$ con $A_i \in \mathcal{C}_i$ per $i=1, \dots, n$.
  Ma dato che i $\mathcal{C}_i$ stanno in $\Gamma$, che è una catena, esiste $i_0$ t.c. $\mathcal{C}_{i_0} \supseteq \mathcal{C}_i$ per ogni $i$. Allora $\mathcal{A}$ è un sottoricoprimento finito di $\mathcal{C}_{i_0} \in \Omega$, assurdo. Allora esiste $\mathcal{Z} \in \Omega$ massimale.

  Mostriamo che $\mathcal{P} \cap \mathcal{Z}$ è un ricoprimento aperto di $X$. Sia $x \in X$, allora per definizione di ricoprimento e di prebase, se $\mathcal{Z}=\{Z_i\}_{i \in I}$, allora esiste $i_0 \in I$ con $x \in Z_{i_0}$ e $B$ appartenente alla base associata a $\mathcal{P}$ con $x \in B \subseteq Z_{i_0}$,
  cioè esistono $P_1, \dots, P_n \in \mathcal{P}$ t.c. $x \in P_1 \cap \dots \cap P_n \subseteq Z_{i_0}$. Vediamo che esiste $j$ con $P_j \in \mathcal{Z}$. Se per assurdo così non fosse, avremmo $\mathcal{Z} \cup \{P_j\} \not\in \Omega$ per massimalità di $\mathcal{Z}$,
  perciò esiste un sottoinsieme finito $I_j \subseteq I$ con $\displaystyle X=P_j \cup \bigcup_{i \in I_j} Z_i$. Poiché per assurdo questo deve valere per ogni $j=1, \dots, n$, allora $\displaystyle X=\left(\bigcap_{j=1}^n P_j\right) \cup \left(\bigcup_{j=1}^n \bigcup_{i \in I_j} Z_i\right)$,
  ma $\displaystyle \bigcap_{j=1}^n P_j \subseteq Z_{i_0}$, allora $\{Z_{i_0}\} \cup \{Z_i\}_{i \in I_j, j=1, \dots, n}$ è un sottoricoprimento finito estratto da $\mathcal{Z}$, assurdo.

  Dunque $\mathcal{P} \cap \mathcal{Z}$ è un ricoprimento aperto di $X$, ma allora è un ricoprimento fatto con elementi della prebase, dal quale possiamo estrarre un sottoricoprimento finito per ipotesi, ma questo sarebbe un sottoricoprimento finito fatto con elementi di $\mathcal{Z} \in \Omega$, assurdo.
\end{proof}

\begin{thm}
  Teorema di Tychonoff. Sia $X_i, i \in I$ un insieme di spazi topologici compatti, allora $\displaystyle \prod_{i \in I} X_i$ è compatto.
\end{thm}

\begin{proof}
  Per il teorema di Alexander \ref{alexander}, possiamo restringerci a un ricoprimento $\mathcal{U}$ fatto con aperti della prebase. Scegliamo la prebase canonica, allora si ha $\displaystyle \mathcal{U}=\bigcup_{i \in I} \mathcal{A}_i, \mathcal{A}_i=\{\pi_i^{-1}(D), D \in \mathcal{D}_i\}$ dove $\mathcal{D}_i$ è una famiglia di aperti di $X_i$ per ogni $i \in I$.
  Mostriamo che esiste $i_0 \in I$ per cui $\mathcal{D}_{i_0}$ è un ricoprimento di $X_{i_0}$. Se per assurdo così non fosse, per ogni $i \in I$ esisterebbe $x_i \in X_i$ t.c. $\displaystyle x_i \not \in \bigcup_{D \in \mathcal{D}_i} D$, allora l'elemento $(x_i)_{i \in I}$ non apparterrebbe ad alcun elemento di $\mathcal{U}$, assurdo.
  Ma allora, dato che $X_{i_0}$ è compatto, $\mathcal{D}_{i_0}$ ammette, in quanto ricoprimento aperto, un sottoricprimento finito $D_1, \dots, D_n$ e $\{\pi_{i_0}^{-1}(D_1), \dots, \pi_{i_0}^{-1}(D_n)\}$ è il sottoricoprimento finito di $\mathcal{U}$ cercato.
\end{proof}

\begin{ex}
  Sia $X$ un insieme, $A \subseteq \mathbb{R}$, $Fun(X, A)=\{f:X \rightarrow A\}=A^X$. Dotando $A^X$ della topologia prodotto ($A$ ha la topologia euclidea) si ottiene la topologia detta della convergenza puntuale.
\end{ex}

\begin{lm}
  Siano $A \subseteq \mathbb{R}, \{f_n\}_{n \in \mathbb{N}}, f_n:X \rightarrow A$, allora $f_n \rightarrow f$ puntualmente $\Leftrightarrow f_b \rightarrow f$ in $A^X$.
\end{lm}

\begin{proof}
  La dimostrazione è molto semplice, ma per comprenderla al volo è necessaria un po' dimestichezza con basi, prebasi, topologia prodotto e "traduzioni in matematichese". Sono tutti dei se e solo se e i vari statement sono ridotti il più possibile a stringhe di simboli, quindi se non capite bene un enunciato o un passaggio non disperatevi: è tutto scritto in modo poco comprensibile, prendetelo come un buon esercizio per applicare ciò che avete imparato. Scusate.

  $f_n \rightarrow f$ puntualmente $\Leftrightarrow \forall x \in X, f_n(x) \rightarrow f(x) \Leftrightarrow \forall x \in X \forall \epsilon>0, \exists n_{x, \epsilon} | \forall n \ge n_{x, \epsilon}, |f_n(x)-f(x)|<\epsilon \Leftrightarrow \forall x \in X, B \in \mathcal{B}$
  base della topologia euclidea su $A$ t.c. $f \in \pi_x^{-1}(B), \exists n_{x, B} | \forall n \ge n_{x, B}, f_n \in \pi_x^{-1}(B) \Leftrightarrow \forall x \in X, B \subseteq A$ aperto t.c.
  $f \in \pi_x^{-1}(B), \exists n_{x, B} | \forall n \ge n_{x, B}, f_n \in \pi_x^{-1}(B) \Leftrightarrow \forall P \in \mathcal{P}$ prebase della topologia prodotto sua $A^X$ t.c.
  $f \in P, \exists n_P | \forall n \ge n_P, f_n \in P \Leftrightarrow \forall U \subseteq A^X | f \in U^{\circ}, \exists n_U | \forall n \ge n_U, f_n \in U \Leftrightarrow f_n \rightarrow f$ in $A^X$.
\end{proof}
