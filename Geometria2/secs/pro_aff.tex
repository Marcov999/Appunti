{
\newcommand{\pro}{\P^n(\K)}
\renewcommand{\dim}{\text{dim}}

Lavoriamo in $\P^n(\K)$. Consideriamo la funzione $j_i:\K^n \rightarrow
\P^n(\K)$ tal che $j_i(x_1,\dots, x_n)=[x_1:\dots:1:\dots:x_n]$, dove 1 compare
in posizione $i$-esima. $j_i$ \`e la $i$-esima carta affine, che stabilisce una
bigezione tra $\K^n$ e $U_i = j_i(\K^n)=\{x_i\neq 0\}\subseteq \P^n(\K)$.

Indicheremo con $H_i$ l'iperpiano complementare a $U_i$. Notiamo anche che
l'inversa di di $j_i$ \`e
\[
    j_i^{-1}([x_0:\dots:x_n])=(\frac{x_0}{x_i},\dots,\frac{x_{i-1}}{x_i},
    \frac{x_{i+1}}{x_i}, \dots, \frac{x_n}{x_i})
\]

\begin{prop}
    Se $W\subseteq \K^n$ \`e un sottospazio affine, esiste unico
    $\overline{W}\subseteq \P^n(\K)$ sottospazio proiettivo con $j_0(W) =
    \overline{W} \cap U_0$. Allora $\overline{W}$ si chiama chiusura proiettiva
    di $W$ e $dim \overline{W} = dim W$.

    Viceversa, se $K\subseteq \P^n(\K)$ \`e un sottospazio proiettivo non
    contenuto in $H_0$, allora $K\cap U_0 = j_0(K')$ con $K'$ sottospazio affine.
    Dunque c'\`e una corrispondenza biunivoca tra sottospazi affini di
    dimensione $k$ e sottospazi proiettivi di dimensione $k$ non contenuti in
    $H_0$.
\end{prop}
\begin{proof}
    Sia $K$ un sottospazio proiettivo di $\pro$ di dimensione $k$, con $K
    \not\subseteq H_0$. Sia $h=n-k$.  Allora $K$ \`e definito da un sistema di
    equazioni
    \[
    \begin{cases}
        a_{1,0}x_0+&\dots+a_{1,n}x_n=0 \\
        &\dots\\
        a_{h,0}x_0+&\dots+a_{h,n}x_n=0 \\
    \end{cases}
    \]
    che ha rango $h$. Vale che $j_0(x_1,\dots,x_n)\in K$ se e solo se
    $[1:x_1:\dots:x_n]\in K$ se e solo se vale il sistema di equazioni sopra.
    Sostituendo gli $x_0$ si ritrova un sistema non omogeneo. Ora basta guardare
    il rango di questo sistema per vedere che l'insieme delle soluzioni \`e non
    vuoto.

    Per il secondo punto, si parte dal sistema non omogeneo e si ritrova il
    sistema omogeneo introducendo la variabile $x_0$.
\end{proof}

\begin{defn}
    I punti di $\overline{W}\setminus W$ (con le notazioni appena introdotte) si
    dicono punti impropri o punti all'inifinito di $W$. Si identificano $\K^n$
    con $U_0$ e $W$ con $j_0(W)\subseteq U_0$. $H_0$ viene chiamato iperpiano
    all'infinito di $\pro$. 
\end{defn}
}
