All'interno di tutto questo paragrafo, $(X, d)$ indica una spazio metrico.

\begin{defn}
  $\{x_n\} \subseteq X$ è \textsc{di Cauchy} se per ogni $\epsilon>0$ esiste $n_0$ t.c. $d(x_n, x_m) \le \epsilon$ per ogni $n, m \ge n_0$.
\end{defn}

\begin{lm} \label{conv->cauchy}
  Se $\{x_n\}$ è convergente, allora è di Cauchy.
\end{lm}

\begin{proof}
  Sia $x \in X$ il limite a cui converge $\{x_n\}$ e sia $\epsilon>0$, allora esiste $n_0$ t.c. per ogni $n \ge n_0$ $d(x_n, x) < \epsilon/2$. Allora per la disuguaglianza triangolare si ha che per ogni $m, n \ge n_0$ vale che $d(x_n, x_m) \le d(x_n, x)+d(x, x_m)<\epsilon/2+\epsilon/2=\epsilon$, da cui la tesi.
\end{proof}

\begin{defn}
  $(X, d)$ è \textsc{completo} se ogni successione di Cauchy in $X$ converge.
\end{defn}

\begin{ex}
  $\mathbb{R}^n$ è completo.
\end{ex}

\begin{ex}
  $\mathbb{Q}$ non è completo.
\end{ex}

\begin{lm}
  Sia $\{x_n\}$ di Cauchy, se ha una sottosuccessione convergente allora è convergente.
\end{lm}

\begin{proof}
  Sia $\{x_{n_i}\}$ la sottosuccessione che converge al limite $x \in X$ e sia $\epsilon>0$. Allora esiste $i_0$ t.c. per ogni $i \ge i_0$ $d(x_{n_i}, x) < \epsilon/2$. Inoltre, esiste $\bar{n}$ t.c. per ogni $n, m \ge \bar{n}$ $d(x_n, x_m) \le \epsilon/2$.
  Poniamo $n^*=\max{\{n_{\bar{n}}, n_{i_0}\}}$ dove abbiamo usato $n_{\bar{n}} \ge \bar{n}$ al posto di $\bar{n}$ stesso per assicurarci che $n^*$ faccia parte degli indici della sottosuccessione convergente.
  Si ha allora che per ogni $n \ge n^*$ $d(x_n, x) \le d(x_n, x_{n^*})+d(x_{n^*} , x)<\epsilon/2+\epsilon/2=\epsilon$, da cui la tesi.
\end{proof}

\begin{cor} \label{cs->compl}
  $(X, d)$ compatto per successioni $\implies$ $(X, d)$ completo.
\end{cor}

\begin{lm}
  Siano $(X, d)$ spazio metrico completo, $Y \subseteq X$ con la metrica indotta, allora $Y$ è completo $\Leftrightarrow$ $Y$ è chiuso in $X$.
\end{lm}

\begin{proof}
  Per il fatto \ref{metr-N1} e la proposizione \ref{chiusoxsucc}, possiamo fare la dimostrazione supponendo o dimostrando (a seconda della freccia) $Y$ chiuso per successioni.

  ($\implies$) Sia $\{y_n\}$ una successione in $Y$ convergente a un elemento di $X$. Per il lemma \ref{conv->cauchy} sappiamo che $\{y_n\}$ è di Cauchy, ma stiamo supponendo $Y$ completo, dunque $\{y_n\}$ converge in $Y$, dunque $Y$ è chiuso per successioni. Notiamo che per quest'implicazione non era necessaria l'ipotesi $(X, d)$ completo.

  ($\Leftarrow$) Sia $\{y_n\}$ una successione di Cauchy in $Y$. Dato che $X$ è completo, questa converge a un elemento di $X$, ma $Y$ è chiuso per successioni, dunque l'elemento appartiene anche a $Y$ e la successione converge in $Y$.
\end{proof}

\begin{defn}
  $(X, d)$ è \textsc{totalmente limitato} se per ogni $\epsilon>0$ esiste un ricoprimento finito di $X$ fatto con palle di raggio $\epsilon$.
\end{defn}

\begin{lm}
  $(X, d)$ totalmente limitato implica $(X, d)$ limitato.
\end{lm}

\begin{proof}
  Prendiamo $\epsilon=1$, fissiamo il centro $x_0$ di una delle palle e sia $D$ la massima distanza tra $x_0$ e uno qualsiasi degli altri centri, che esiste perché le palle, dunque anche i loro centri, sono in numero finito. Allora è una banale applicazione della disuguaglianza triangolare verificare che la palla di centro $x_0$ e raggio $D+1$ ricopre tutto $X$.
\end{proof}

\begin{ex}
  Consideriamo su $\mathbb{R}$ la distanza $d(x, y)=\min{\{|x-y|, 1\}}$, allora $(\mathbb{R}, d)$ è limitato ma non totalmente limitato.
\end{ex}

\begin{prop} \label{tl->bn}
  $(X, d)$ totalmente limitato implica $(X, d)$ a base numerabile.
\end{prop}

\begin{proof}
  Per la proposizione \ref{metr-num} basta vedere che $X$ è separabile. Per la totale limitatezza, per ogni $n \ge 1$ esiste $F_n \subseteq X$ finito con $\displaystyle X=\bigcup_{p \in F_n} B(p, 1/n)$. Allora $\displaystyle \bigcup_{n \ge 1} F_n$ è numerabile, mostriamo dunque che è denso. Sia dunque $U \subseteq X$ aperto non vuoto, allora esiste $n_0 \ge 1, z \in U$ con $B(z, 1/n_0) \subseteq U$.
  Abbiamo che $z \in B(p, 1/n_0)$ per qualche $p \in F_{n_0} \implies d(p, z)<1/n_0 \implies p \in B(z, 1/n_0) \implies p \in U$, da cui la tesi.
\end{proof}

\begin{thm}
  Sia $(X, d)$ uno spazio metrico. Allora le seguenti sono equivalenti:
  \begin{nlist}
    \item $X$ è compatto;
    \item $X$ è compatto per successioni;
    \item $X$ è totalmente limitato e completo.
  \end{nlist}
\end{thm}

\begin{proof}
  ((i) $\implies$ (ii)) $(X, d)$ è metrico, dunque per il fatto \ref{metr-N1} è primo numerabile, allora per la proposizione \ref{N1->(c->cs)} otteniamo che (i) $\implies$ (ii).

  ((ii) $\implies$ (iii)) Per il corollario \ref{cs->compl} abbiamo che $X$ è completo. Supponiamo per assurdo che non sia totalmente limitato. Allora esiste $\epsilon>0$ tale che non è possibile ricoprire $X$ con un numero finito di palle di raggio $\epsilon$. Fissiamo $x_0 \in X$ e definiamo $\{x_n\}$ induttivamente fissando $x_{n+1} \in X \setminus (B(x_0, \epsilon) \cup B(x_1, \epsilon) \cup \dots \cup B(x_n, \epsilon))$, che esiste per l'ipotesi assurda.
  Allora dati due naturali qualsiasi $m \not= n$ abbiamo che $d(x_n, x_m) \ge \epsilon$, quindi non si possono estrarre da $\{x_n\}$ sottosuccessioni di Cauchy, in particolare non si possono estrarre sottosuccessoni convergenti, assurdo perché siamo sotto l'ipotesi (ii).

  ((iii) $\implies$ (i)) Per la proposizione \ref{tl->bn} e per il teorema \ref{bn->(c_sse_cs)}, sotto l'ipotesi (iii) abbiamo che (i) $\Leftrightarrow$ (ii). Dimostriamo allora (ii). Sia $\{x_n\}$ una successione in $X$. $X$ è completo per ipotesi, quindi basta trovare una sottosuccessione di Cauchy.
  Per totale limitatezza eiste un ricoprimento con un numero finito di palle di raggio $2^{-0}$. Allora esiste una di queste palle che contiene infiniti $\{x_n\}$. Fissiamo $n_0$ come il minimo indice tale che $x_{n_0}$ sta in questa palla e procediamo induttivamente.
  Abbiamo $x_{n_i}$ che appartiene a una palla di raggio $2^{-i}$ contenente infiniti $x_n$ (in particolare, infiniti di indice maggiore di $n_i$). Ricopriamo $X$ con un numero finito di palle di raggio $2^{-(i+1)}$. Allora ce ne dev'essere una che contiene infiniti di quegli $x_n$ che stavano nella palla di raggio $2^{-i}$ contenete $x_{n_i}$. Prendiamo dunque $n_{i+1}$ come il minimo indice maggiore di $n_i$ tale che $x_{n_{i+1}}$ sta nella palla di raggio $2^{-(i+1)}$ trovata. È una semplice verifica dimostrare che la sottosuccessione $\{x_{n_i}\}$ è di Cauchy.
\end{proof}

\begin{thm}(\emph{del numero di Lebesgue}) Sia $(X,d)$ spazio metrico compatto, e sia $\mathcal{U}$ ricoprimento aperto di $X$. Allora esiste $\varepsilon >0$ tale che $\forall x \in X$, $\exists U \in \mathcal{U}$, con $B(x, \varepsilon) \subset U$.
\end{thm}
\begin{proof}
Sia $X_n=\{ x \in X \mid \exists U \in \mathcal{U} \text{ con } B(x, 2^{-n}) \subset U \}$. Vogliamo mostrare che esiste $n$ tale che $X=X_n$. Vediamo le proprietà degli $X_n$:
\begin{nlist}
\item $\displaystyle X=\bigcup _{n \in \mathbb{N}} X_n$
\item $X_n \subset X_{n+1}$
\end{nlist}
Vediamo che vale che $X_n \subset {\mathop X\limits^ \circ} _{n+1}$:\\
$\displaystyle X=\bigcup _{n \in \mathbb{N}} {\mathop X\limits^ \circ}_n \Longrightarrow X={\mathop X\limits^ \circ}_n$ se $n \gg 0$ in quanto per compattezza posso estrarre un sottoricoprimento finito. Sia $x \in X_n$, dunque $\exists U \in \mathcal{U}$ tale che $B(x, 2^{-n}) \subset U$. Vediamo allora che $B(x, 2^{-(n+1)}) \subset X_{n+1} \Longrightarrow X_n \subset {\mathop X\limits^ \circ}_{n+1}$. Sia $y \in B(x, 2^{-(n+1)})$, allora mostro che $B(y, 2^{-(n+1)}) \subset U$ (e che dunque $y \in X_{n+1}$ per definizione): data
$$d(y,x)< \dfrac{1}{2^{n+1}}$$
prendiamo $z \in B(y,2^{-(n+1)})$. Allora:
$$d(z,y)<\dfrac{1}{2^{n+1}} \qquad d(z,x)<\dfrac{2}{2^{n+1}}=\dfrac{1}{2^n} \Longrightarrow z \in B(x,2^{-n}) \subset U$$
\end{proof}

\begin{defn}
Dati $(X,d_X)$, $(Y,d_Y)$ spazi metrici e $f:X \longrightarrow Y$, allora $f$ è \textsc{uniformemente continua} se $\forall \varepsilon >0$ esiste $\delta >0$ tale che $d_Y(f(x),f(x'))<\varepsilon$ se $d_X(x,x')<\delta$.
\end{defn}

\begin{oss}
Uniformemente continua $\Longrightarrow$ continua.
\end{oss}

\begin{thm}(\emph{di Heine-Cantor})
Siano $(X,d_X)$ e $(Y,d_Y)$ spazi metrici con $X$ compatto. Allora se $f:X \longrightarrow Y$ è continua, è anche uniformemente continua.
\end{thm}
\begin{proof}
Sia $\varepsilon>0$. Cerchiamo $\delta$ tale che $f(B(x,\delta)) \subset B(f(x), \varepsilon)$ per ogni $x \in X$. Vale a dire che $B(x,\delta) \subset f^{-1}(B(f(x),\varepsilon))$. Applichiamo il teorema del numero di Lebesgue al ricoprimento aperto
$$\mathcal{U}=\{f^{-1}(B(f(x),\varepsilon))\}$$
Quindi $\exists \delta >0$ tale che $\exists U \in \mathcal{U}: B(x,\delta) \subset U$. Dunque $\forall x \in X, \exists x' \in X$ con $B(x,\delta) \subset f^{-1}(B(f(x'),\varepsilon))$. Sia $y \in B(x, \delta)$, allora 
$$f(y) \in B(f(x'),\varepsilon) \ni f(x) \Longrightarrow d_Y(f(x),f(y))<2\varepsilon$$
\end{proof}

Vediamo quindi alcuni esempi di spazi metrici completi.

\begin{ex}
Sia $X$ insieme arbitrario e $Y$ spazio metrico completo, e sia:
$$\mathcal{B}(X,Y)=\{f:X \longrightarrow Y \mid f \text{ è limitata}\}$$
Allora $\mathcal{B}(X,Y)$ è uno spazio metrico completo con la distanza
$$d_\infty (f,g)=\sup _{x \in X} d_Y(f(x),g(x))$$
Vediamo quindi che $(\mathcal{B}(X,Y),d_\infty)$ è completo.\\
Sia $\{f_n\} \subset \mathcal{B}(X,Y)$ di Cauchy, allora $\displaystyle \lim _{n \rightarrow \infty} f_n=f \in \mathcal{B}(X,Y)$. Definiamo $f(x)$ per ogni $x \in X$. Abbiamo che $\{f_n\}_N \in \mathbb{N}$ di Cauchy in $\mathcal{B}(X,Y)$ implica che $\forall x \in X, \{f_n(x)\} \subset Y \Longrightarrow f_n(x)$ ammette limite in $Y$ (infatti $Y$ è completo), e definiamo allora $\displaystyle f(x):=\lim _{n \rightarrow \infty} f_n(x)$. Rimane da vedere che:
\begin{nlist}
\item $f$ è limitata
\item $f_n \longrightarrow f$ nella distanza $d_\infty$
\end{nlist}
i) $d_Y(f(x),f(x')) \le d_Y(f(x),f_n(x))+d_Y(f_n(x),f_n(x'))+d_Y(f_n(x'),f(x'))$\\
ii) $\displaystyle d_Y(f(x),f_n(x))=d_Y(\lim _{m \rightarrow \infty} f_m(x), f_n(x))=\lim _{m \rightarrow \infty} d_Y(f_m(x),f_n(x))<\varepsilon$
\end{ex}

\begin{ex}
Supponiamo adesso $X$ spazio topologico, e poniamo:
$$\mathcal{C}_{\mathcal{B}}(X,Y)=\{f \in \mathcal{B}(X,Y) \mid f \text{ è continua}\} \subset \mathcal{B}(X,Y)$$
Vediamo che $\mathcal{C}_\mathcal{B}(X,Y) \subset \mathcal{B}(X,Y)$ è chiuso, dunque $\mathcal{C}_\mathcal{B}(X,Y)$ è completo con $d_\infty$.
\end{ex}
