\begin{defn}
    Siano $X$ uno spazio topologico e $\sim$ una relazione di equivalenza
    definita su $X$. Vengono allora indotti un insieme $Y = \faktor{X}{\sim}$ e
    una proiezione $\pi\colon X\longrightarrow Y$ che manda ogni punto nella sua
    classe di equivalenza. La topologia quoziente definita su $Y$ \`e la
    topologia pi\`u fine che rende $\pi$ continua.
\end{defn}

\begin{defn}
    Data la proiezione $\pi$ come sopra, un sottoinsieme $Z \subseteq X$ si dice
    saturo se \`e unione di classi di equivalenza per $\sim$. Cio\`e ${Z =
    \pi^{-1}\pi(Z)}$.
\end{defn}

\begin{oss}
    Nel setting di prima, detta $\tau_Y$ la topologia quoziente, si ha che
    \[
    A\in \tau_Y \Longleftrightarrow
    \pi^{-1}(A) \text{\ aperto} \Longleftrightarrow
    A = \pi(B) \text{\ con $B$ aperto saturo}.
    \]
\end{oss}

\begin{thm}
    \emph{Propriet\`a universale del quoziente.} Siano $X$ un insieme con una relazione di equivalenza $\sim$ e una proiezione $\pi$. Allora la topologia quoziente \`e l'unica topologia per cui per ogni spazio topologico $Z$ e funzione ${f\colon \faktor{X}{\sim} \longrightarrow Z}$, $f$ \`e continua se e solo se $f \circ \pi$ \`e continua.
\end{thm}
\begin{proof}
    Il seguente diagramma \`e commutativo.

    \begin{center}\begin{tikzcd}
        X \arrow[dd, "\pi", two heads] \arrow[rd, "f\circ \pi"] &   \\
        & Z \\
            \faktor{X}{\sim} \arrow[ru, "f"] &
    \end{tikzcd}\end{center}

    Come prima cosa verifichiamo che la topologia quoziente ha la propriet\`a universale. Un'implicazione \`e facile: se $f$ e $\pi$ sono continue, anche la loro composizione lo \`e. Viceversa sia $f\circ \pi$ continua, e si prenda un aperto in Z. Allora $(f \circ \pi)^{-1}(A) = (\pi^{-1} \circ f^{-1})(A)$ \`e aperto, e dunque anche $f^{-1}(A)$ \`e aperto.
    % TODO: forse qui bisogna spiegarlo meglio.

    Mostriamo ora che se una certa topologia $\sigma$ gode della propriet\`a universale, allora \`e uguale alla topologia quoziente $\tau$.

    Si prenda $Z = (\faktor{X}{\sim}, \tau)$ e $f$ l'identit\`a. Allora poich\'e $\pi$ \`e continua con la topologia $\tau$, si ha che
    \[
        \id\colon \faktor{X}{\sim}, \sigma \longrightarrow
        \faktor{X}{\sim}, \tau
    \]
    \`e dunque $\sigma$ \`e pi\`u fine di $\tau$.

    Viceversa, prendendo per $Z$ lo stesso insieme ma dotato della topologia $\sigma$ e $f = \id$, si ottiene che $\pi$ \`e continua anche con $\sigma$. Per definizione di topologia quoziente, allora $\tau$ \`e pi\`u fine di $\sigma$.
\end{proof}

\begin{defn}
    Una funzione $f\colon X \longrightarrow Y$ continua e surgettiva si chiama identificazione quando per ogni $A \subseteq Y$, $A$ \`e aperto se e solo se $f^{-1}(A)$ \`e aperto.
\end{defn}

\begin{oss}
    Sia $f$ una bigezione continua. Sono equivalenti:
    \begin{nlist}
        \item $f$ \`e immersione;
        \item $f$ \`e omeomorfismo;
        \item $f$ \`e identificazione.
    \end{nlist}
\end{oss}

\begin{oss}
    Sia $f$ continua surgettiva. Allora:
    \begin{nlist}
        \item Se $f$ \`e aperta, \`e identificazione.
        \item Se $f$ \`e chiusa, \`e identificazione.
        \item Ogni altra implicazione \`e abusiva.
    \end{nlist}
\end{oss}

\begin{prop}
    Sia $f\colon X \longrightarrow Y$ un'identificazione. Allora la funzione $\bar{f}$ definita dal seguente diagramma commutativo \`e un omeomorfismo.

    \begin{center}
        \begin{tikzcd}
            X \arrow[d, "\pi"'] \arrow[r, "f"]     & Y \\
            \faktor{X}{\sim} \arrow[ru, "\bar{f}"'] &
        \end{tikzcd}
    \end{center}
\end{prop}
\begin{proof}
    Per come \`e definita $\bar{f}$, essa \`e ben definita e bigettiva. Poich\'e $f$ \`e continua, per la propriet\`a universale, anche $\bar{f}$ \`e continua. Voglio mostrare che $\bar{f}$ \`e aperta. Sia allora $A$ un aperto di $\faktor{X}{\sim}$. Si ha che $\bar{f}(A)=(f\circ \pi^{-1})(A)$ \`e aperto se e solo se ${ (f^{-1} \circ f \circ \pi ^{-1})(A)}$ \`e aperto perch\'e $f$ \`e identificazione. Poich\'e $\pi$ \`e la proiezione sulla topologia quoziente, $A$ \`e aperto.
\end{proof}

\begin{oss}
    Si ha che $\bar{f}$ \`e un omeomorfismo se e solo se $f$ \`e un'identificazione.
\end{oss}

\begin{defn}
    Sia $A \subseteq X$. Si definisce il quoziente
    \[
    \faktor{X}{A} = \faktor{X}{\sim}
    \]
    dove $x\sim x'$ se e solo se $x = x'$ oppure $x,x'\in A$. Cio\`e in questo quoziente $A$ ``collassa'' in un punto.
\end{defn}

\begin{ex}
    Il disco quozientato sulla sua frontiera \`e isomorfo alla sfera. Cio\`e
    $D^n/\partial D^n\cong S^n$.
\end{ex}
\begin{proof}
    Basta verificare che la funzione
    \begin{align*}
        f\colon D^n/\partial D^n & \longrightarrow S^n\\
        x & \longrightarrow (2 \norm{x}^2-1,\; x \sqrt{1-\norm{x}^2})
    \end{align*}
    sia un omeomorfismo. Sicuramente si ha la continuit\`a. Per l'iniettivit\`a
    si noti che il primo termine distingue i vettori con norma diversa, mentre
    il secondo distingue i vettori con direzione diversa, con l'eccezione dei
    punti di norma $1$, che per\`o sono identificati. Infine la funzione \`e
    surgettiva, in quanto l'immagine \`e composta effettivamente di vettori di
    lunghezza unitaria.
\end{proof}

\begin{ex}
    Sia
    \[
        X = \{x \geq 0\} \cup \{y = 0\} \subseteq \mathbb{R}^2
    \]
    e sia $\pi\colon X \longrightarrow \mathbb{R}$ la proiezione sul primo
    elemento. Allora $\pi$ \`e un'identificazione ma non \`e chiusa n\'e aperta.
\end{ex}

\begin{ex}
    Sia
    \[
        A = \{(x, 0)\} \subseteq \mathbb{R}^2
    \]
    e sia $X = \mathbb{R}^2/A$. Allora $X$ non \`e primo numerabile.
\end{ex}
