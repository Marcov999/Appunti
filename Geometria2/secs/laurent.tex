\begin{defn}
    Una serie di Laurent \`e un'espressione del tipo
    \[
        \sum_{n\in \mathbb{Z}}a_n z^n.
    \]
    Ad essa possiamo associare due serie di potenze, $\sum_{n\geq 0}a_n z^n$ e
    $\sum_{n< 0}a_n z^{-n}$. Assumiamole assolutamente convergenti con raggio di
    convergenza finito e non nullo.
    Si definiscono $\rho_1$ come il raggio di convergenza della prima serie e
    $\rho_2$ come il reciproco del raggio di convergenza della seconda.
    Consideriamo anche $f_1(z)=\sum_{n\geq 0}a_n z^n$, $f_1(z)=\sum_{n< 0}a_n
    z^n$.
\end{defn}
$f_1$ \`e chiaramente analitca nella palla di raggio $\rho_1$. Invece $f_2$
converge assolutamente per $|z|>\rho_2$.

\begin{prop} $f_2$ \`e olomorfa in $|z|>\rho_2$.
\end{prop}
\begin{proof}
    Sia $z=1/u$, $g(u) = f_2(1/u) = \sum_{k>0}a_{-k}u^k$, che converge
    assolutamente per $|u|<1/\rho_2$. Inoltre 
    \[
        f'_2(z) = -g'(\frac{1}{z})\frac{1}{x^2} = \sum_{n<0}na_nz^{n-1}.
    \]
    Poich\'e $f'_2$ esiste $f_2$ \`e olomorfa.
\end{proof}

\begin{prop}
    Sia $f(z)=\sum_{n\in\mathbb{Z}}a_nz^n$ una serie di Laurent. Supponiamo che:
    \begin{nlist}
        \item $\sum_{n\geq 0}a_nz^n$ e $\sum_{n<0}a_nz^{-n}$ assolutamente
            convergenti
        \item $\rho_2<\rho_1$
    \end{nlist}
    Allora la somma $f(z)$ \`e olomorfa per per $\rho_2<|z|<\rho_1$ e la serie
    converge normalmente in $r_2\leq |z| \leq r_1$ con $\rho_2<r_2<r_1<\rho_1$.
\end{prop}

\begin{defn}
    Diciamo che una funzione $f(z)$ definita sulla corona circolare
    $\rho_2<r_2<r_1<\rho_1$ ha un'espansione di Laurent se esiste una serie di
    Laurent che converge in tale corona e che la sua somma sia uguale a $f$ in
    ogni suo punto.
\end{defn}

\begin{oss}    
    Se $f$ ha un'espansione di Laurent, $f$ \`e olomorfa nella corona.
\end{oss}    

\begin{thm}
    Sia $a\in\mathbb{C}$ e sia $f$ olomorfa nella corona circolare
    \[
        A = \{z\in\mathbb{C}:0<\rho_2<|z-a|<\rho_1<\infty\}.
    \]
    Allora $f$ ha un'espansione di Laurent, cio\`e esplicitamente per $z\in A$:
    \[
        f(z) = \sum_{n\in\mathbb{Z}}(z-a)^n \text{\ con\ }
        a_n = \frac{1}{2\pi i} \int_\gamma \frac{f(w)}{(w-a)^{n+1}}dw
    \]
\end{thm}
\begin{proof}
    Senza perdita di generalit\`a, prendiamo $a=0$. Scegliamo $\rho_2 <
    r'_2<r_2<r_1<\rho_1$, $\gamma_1(t)=r_1e^{2\pi i t}$,$\gamma_2(t)=r_2e^{2\pi
    i t}$.
    Sia $B=\{z: r'_2\leq |z| \leq r'_1\}$, $r_2\leq |z| \leq r_1$, e $r>0$ tale
    che $\overline{B(z,r)}\subseteq \{w: r'_2<|w|<r'_1\}$. Si prenda $\alpha$
    parametrizzazione del bordo di $\overline{B(z,r)}$, in senso antiorario.
    Si ha per ogni forma chiusa in $A\setminus B(z,r)$ $\omega$ che
    \[
        \int_{\gamma_1}\omega-\int_{\gamma_2}\omega-\int_{\alpha}\omega=0
    \]
    Questo segue dal fatto che il loop $l_2 * \alpha_1 * l_1 * \gamma_2 *
    \overline{l_1} * \alpha_2 * \overline{l_2} * \overline{\gamma_1}$ \`e banale
    in $A\setminus\{0\}$.

    Lungo $\gamma_1$ si ha ha che $|\gamma_1(t)| = r_1 > |z|$, per cui se $w =
    \gamma_1(t)$, si ha che:
    \[
        \frac{f(w)}{w-z} = \frac{f(w)}{w(1-z/w)} = \frac{f(w)}{w}\frac{1}{1-z/w}
        = \frac{f(w)}{w} \sum_{n=0}^\infty \left(\frac{z}{w}\right)^n = 
        \sum_{n=0}^\infty \frac{z^nf(w)}{w^{n+1}}
    \]

    La convergenza di tale serie nella corona tra $r_2$ e $r_1$ \`e normale,
    dunque uniforme, e si pu\`o scambiare serie e integrale in:
    \[
        \int_{\gamma_1} \frac{f(w)dw}{z-w} =
        \int_{\gamma_1} (\sum_{n\geq 0}\frac{z^nf(w)}{w^{n+1}})dw = 
        \sum_{n\geq 0}z^n\int_{\gamma_1} \frac{f(w)}{w^{n+1}}dw = 
        \sum_{n\geq 0}b_n z^n
    \]
    con $b_n=\int_{\gamma_1} \frac{f(w)}{w^{n+1}}dw$.

    Poich\`e essere liberamente omotopi su $A$ \`e equivalente a esserlo su
    $\mathbb{C}\setminus\{0\}$, allora l'integrale di $b_n$ si pu\`o anche fare
    su un qualunque loop $\gamma$ liberamente omotopo a $\gamma_1$.

    Ripetendo il ragionemento su $\gamma_2$ (occhio a un segno nella serie
    geometrica!), si ottiene che
    \[
        -\int_{\gamma_2} \frac{f(w)dw}{z-w} =
        \sum_{n< 0}b_n z^n
    \]
    con $b_n =\int_{\gamma_2} \frac{f(w)}{w^{n+1}}dw = \int_\gamma
    \frac{f(w)}{w^{n+1}}dw$.

    Allora per la  formula di Cauchy, si ha che
    \[
        f(z) = \frac{1}{2\pi i} \int_\alpha \frac{f(w)}{w-z} dw = 
        \frac{1}{2\pi i}( \int_{\gamma_1} \frac{f(w)}{w-z} dw - 
        \int_{\gamma_2} \frac{f(w)}{w-z} dw) = \frac{1}{2\pi i}
        \sum_{n=-\infty}^\infty b_n z^n.
    \]
\end{proof}

\begin{cor}
    Sia $A = \{\rho_2<|z|<\rho_1\}$, $f:A\rightarrow \mathbb{C}$ olomorfa.
    Allora $f = f_1 + f_2$, con $f_1:B(0,\rho_1)\rightarrow\mathbb{C}$ e $f_2:
    \mathbb{C}\setminus \overline{B(0,\rho_2)}\rightarrow\mathbb{C}$ e sono
    olomorfe. Richiedendo che $\lim_{|z|\rightarrow +\infty} |f_2(z)| = 0$,
    tale scrittura \`e unica.
\end{cor}
\begin{proof}
    Per l'esistenza, basta scrivere $f$ come serie di Laurent e assegnare a
    $f_1$ e $f_2$ le somme sugli indici negativi e non negativi.
    Supponiamo che tale scomposizione si possa fare in due modi, con $f_1$ e
    $f_2$, e dall'altra parte $g_1$ e $g_2$.
    Definisco la funzione olomorfa $F:\mathbb{C}\rightarrow\mathbb{C}$ come 
    $f_1-g_1$ dentro la palla di raggio $\rho_1$ e come $g_2-f_2$ fuori dalla
    palla di raggio $\rho_2$. Tale funzione \`e ben definita in quanto le
    differenze sono uguali nella corona. Inoltre vale che $\lim_{|z|\rightarrow
    \infty} F(z) = 0$. Dunque $F$ \`e limitata e per Liouville \`e costante.
    Dunque $f_1=g_1$ e $f_2 = g_2$.
\end{proof}

\begin{thm}
    la serie di Laurent \`e unica.
\end{thm}
\begin{proof}
    Per semplicit\`a sia $z_0=0$, $f:D\rightarrow\C$ olomorfa con $D = \{z,
    \rho_2<|z|<\rho_1\}$ con sviluppo $\sum_{-\infty}^\infty a_nz^n$.

    Fissiamo $k\in\mathbb{Z}$. Sia $\gamma$ un loop in $D$ con $I(\gamma,0)=1$ e
    calcoliamo
    \[
        \int_\gamma \frac{f(z)}{z^{k+1}}dz = 
        \int_\gamma \frac{\sum a_nz^n}{z^{k+1}}dz = 
        \int_\gamma (\sum a_nz^{n-k-1})dz = 
        \sum a_n \int_\gamma z^{n-k-1}dz.
    \]
    dove l'ultima uguaglianza \`e giustificata dalla convergenza uniforme della
    serie.
    Ora, $z^jdz$ \`e esatta su $\C\setminus\{0\}$ per $j\neq -1$, per cui se
    $n-k-1\neq -1$, si ha che
    \[
        \int_\gamma z^{n-k-1}dz = 0 \qquad \text{per $n\neq k$}.
    \]
    Per cui
    \[
        \int\frac{f(z)}{z^{k+1}}dz = a_k \int_\gamma\frac{1}{z}dz = 2\pi i a_k.
    \]
    Per cui gli $a_k$ sono univocamente determinati da $f$.
\end{proof}

\begin{defn}
    Data una palla $B$ centrata in $z_0$, $f:B\setminus\{z_0\}\rightarrow\C$
    olomorfa con sviluppo di Laurent con coefficienti $a_n$, il residuo di $f$
    in $z_0$ \`e
    \[
        \Res(f,z_0)=a_{-1}
    \]
\end{defn}

