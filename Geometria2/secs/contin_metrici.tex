\begin{defn}
    Dati $p \in X, R>0$, ${B(p, R):=\{ x \in X \mid d(p, x)<R\}}$ è detta la
    \textsc{palla aperta} di centro $p$ e raggio $R$.
\end{defn}

\begin{defn}
    ${f:(X, d) \rightarrow (Y, d')}$ è \textit{continua in $x_0$} se ${\forall
    \epsilon > 0 \; \exists \; \delta >0}\\\ \tc {f(B(x_0, \delta)) \subseteq
    B(f(x_0), \epsilon)}$, cioè ${f^{-1}(B(f(x_0), \epsilon)) \supseteq B(x_0,
    \delta)}$.
\end{defn}

\begin{defn}
    ${f(X, d) \rightarrow (Y, d')}$ è detta \textsc{continua} se è continua in
    ogni ${x_0 \in X}$.
\end{defn}

\begin{oss}
    Gli embedding isometrici sono continui.
\end{oss}

\begin{defn}
    Sia $X$ uno spazio metrico, $A \subseteq X$ è detto \textsc{aperto} se per
    ogni $x \in A$ esiste ${R>0 \tc B(x, R) \subseteq A}$.
\end{defn}

\begin{ftt}
    Le palle aperte sono aperti. Per dimostrarlo, sfruttare la disuguaglianza
    triangolare.
\end{ftt}

\begin{thm} \label{thm:cont_inv}
    $f(X, d) \rightarrow (Y, d')$ è continua se e solo se per ogni aperto $A$ di
    $Y$ $f^{-1}(A)$ è aperto di $X$.
\end{thm}

\begin{proof}
    Supponiamo $f$ continua. Prendiamo $x \in f^{-1}(A)$, allora si ha che $f(x)
    \in A$, ma dato che $A$ è aperto esiste una palla aperta $B_Y$ di centro
    $f(x)$
     t.c. $B_Y \subseteq A$, da cui $f^{-1}(B_Y) \subseteq f^{-1}(A)$.
    Usando la definizione di continuità, detto $\epsilon$ il raggio di $B_Y$,
    scegliendo il $\delta$ corrispondente come raggio di una palla di centro
    $x$, sia essa $B_X$, si ha che $B_X \subseteq f^{-1}(B_Y) \subseteq
    f^{-1}(A)$, perciò abbiamo trovato una palla centrata in $x$ tutta contenuta
    in $f^{-1}(A)$ e questo prova che è un aperto.

    Viceversa, supponiamo che le controimmagini di aperti siano a loro volta
    aperti. Dati $x_0 \in X$ e $\epsilon >0$ si ha che $B(f(x_0), \epsilon)$ è
    un aperto di $Y$, perciò la sua controimmagine è un aperto, ma allora per
    definizione di aperto esiste $\delta>0$ tale che $B(x_0, \delta) \subseteq
    f^{-1}(B(f(x_0), \epsilon))$ e questo prova che $f$ è continua.
\end{proof}

\begin{oss}
    La continuità di una funzione dipende quindi solo indirettamente dalla
    metrica, mentre è direttamente collegata agli aperti generati dalla metrica
    stessa. Segue facilmente che due metriche che generano gli stessi aperti
    portano anche alla stessa famiglia di funzioni continue. Diamo dunque la
    seguente definizione.
\end{oss}

\begin{defn}
    Due distanze $d, d'$ su un insieme $X$ si dicono \textsc{topologicamente
    equivalenti} se inducono la stessa famiglia di aperti.
\end{defn}

\begin{lm}
    Siano $d, d'$ distanze su $X$ t.c. esiste $k \ge 1$ t.c. per ogni $x, y \in
    X$ valga $d(x, y)/k \le d'(x, y) \le k \cdot d(x, y)$. Allora $d$ e $d'$
    sono topologicamente equivalenti.
\end{lm}

\begin{proof}
    Sia $A$ un aperto indotto da $d'$ e $x_0 \in A$. Per definizione di aperto
    esiste $R>0$ tale che $B_{d'}(x_0, R) \subseteq A$. Considero $B_d(x_0,
    R/k)$. Dato un elemento $x \in B_d(x_0, R/k)$ ho che $d'(x_0, x) \le k \cdot
    d(x_0, x)<k \cdot R/k=R$, dunque $x \in B_{d'}(x_0, R)$. Allora $B_d(x_0,
    R/k) \subseteq B_{d'}(x_0, R/k) \subseteq A$ e dunque $A$ è anche un aperto
    indotto da $d$. Per la simmetria delle disuguaglianze nelle ipotesi si
    dimostra anche l'opposto, perciò gli aperti di $d$ e $d'$ sono gli stessi,
    come voluto.
\end{proof}

\begin{cor}
    $d_1, d_2, d_{\infty}$ sono topologicamente equivalenti su $\mathbb{R}^n$.
\end{cor}

\begin{center}
\pagestyle{empty}
\begin{tikzpicture}[line cap=round,line join=round,
    >=triangle 45,x=2.0cm,y=2.0cm]
    \draw[->,color=black] (-1.72,0) -- (1.78,0);
    \foreach \x in {-1,1}
    \draw[shift={(\x,0)},color=black] (0pt,2pt) -- (0pt,-2pt);
    \draw[->,color=black] (0,-1.4) -- (0,1.44);
    \foreach \y in {-1,1}
    \draw[shift={(0,\y)},color=black] (2pt,0pt) -- (-2pt,0pt);
    \clip(-1.72,-1.2) rectangle (1.78,1.24);
    \draw(0,0) circle (2cm);
    \draw (-1,0)-- (0,1);
    \draw (0,1)-- (1,0);
    \draw (1,0)-- (0,-1);
    \draw (0,-1)-- (-1,0);
    \draw (-1,1)-- (1,1);
    \draw (1,1)-- (1,-1);
    \draw (1,-1)-- (-1,-1);
    \draw (-1,1)-- (-1,-1);
\end{tikzpicture}

Nell'immagine sono rappresentate, nell'ordine dalla più interna alla più
esterna, le palle aperte centrate nell'origine e di raggio $1$ rispettivamente
nelle metriche $d_1, d_2, d_{\infty}$.
\end{center}

\begin{proof}
    Per AM-QM si ha che $$\displaystyle \frac{\sum_{i=1}^n |x_i-y_i|}{n} \le
    \sqrt{\frac{\sum_{i=1}^n (x_i-y_i)^2}{n}},$$ da cui $d_1(x, y) \le \sqrt{n}
    \cdot d_2(x, y)$. Stimando tutti i termini con il massimo otteniamo che
    $$\displaystyle \sqrt{\sum_{i=1}^n (x_i-y_i)^2} \le \sqrt{\sum_{i=1}^n
    \sup_{j=1, \dots, n} \{(x_j-y_j)^2\}}=\sqrt{n} \cdot \sup_{j=1, \dots, n} \{
    |x_j-y_j| \},$$ da cui $d_2(x, y) \le \sqrt{n} \cdot d_{\infty} (x, y).$
    Infine è ovvio che $$\displaystyle \sup_{i=1, \dots, n} |x_i-y_i| \le
    \sum_{i=1}^n |x_i-y_i|$$, da cui $d_{\infty}(x, y) \le d_1(x, y).$
    Scegliendo $k=\sqrt{n}$ è ora sufficiente applicare il lemma.
\end{proof}

Nella dimostrazione del corollario era importante che lo spazio fosse di
dimensione finita. Nello spazio delle funzioni continue da $[0, 1]$ a
$\mathbb{R}$ le tre distanze non sono topologicamente equivalenti.
